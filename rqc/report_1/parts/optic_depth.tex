\subsection{Оптическая глубина}


Интенсивность слабого одиночного луча, проходящего через ячейку описывается законом Бэра:
\begin{equation*}
    d I / d x = - \alpha I,
    \hspace{10 mm} 
    \alpha = \alpha (\nu).
\end{equation*}
В хорошем приближение\footnote{
    Для слабого луча [1].
}  $\alpha \neq \alpha(x)$ . 
Введем оптическую длину $\tau(\nu) = l \alpha(\nu)$, тогда
\begin{equation*}
    \sub{I}{out} = \sub{I}{in} e^{- \alpha(\nu) l} = \sub{I}{in} e^{- \tau(\nu)}.
\end{equation*}
Вклад от группы атомов $(v,\, v + d v)$ в $\tau(\nu)$ можем быть записан, как
\begin{equation*}
    d \alpha (\nu, v) = \sigma(\nu, v) \, dn(v),
    \hspace{0.5cm} \Rightarrow \hspace{0.5cm}    
    d \tau(\nu, v) = l \sigma(\nu, v) \, dn(v).
\end{equation*}
Коэффициент поглощения $\sigma(\nu, v)$ имеет Лоренцовский профиль с натуральной шириной $\Gamma$ (?) и смещенной по Допплеру резонансной частотой
\begin{equation}
    \label{1_4}
    \sigma(\nu, v) = \sigma_0 \frac{\Gamma^2/4}{(\nu - \nu_0(1 - v / c))^2 + \Gamma^2/4},
\end{equation}
где $\sigma_0$ -- резонансное сечение поглощения\footnote{
    [1], problem 1: $\sigma_0 \sim n$ атомов в ячейке.
} , зависящее от вида дипольного перехода и поляризации падающего света [1$\Rightarrow$4]. 

Часть атомов $d n (v)$ с определенной скорости можем найти из распределения Больцмана
\begin{equation*}
    d n(v) = n_0 \sqrt{\frac{m}{2 \pi \sub{k}{B} T}} \exp\left(
        - \frac{m v^2}{2 \sub{k}{B} T} 
    \right) \d v,
\end{equation*} 
где $n_0 = N/V$ -- концентрация атомов в ячейке. 

Собирая все вместе (?) приходим к выражению
\begin{equation}
    \label{1_6}
    d \tau (\nu, v) = \frac{2}{\pi} \frac{\tau_0}{\sigma_0 \Gamma} \frac{\nu_0}{c} 
     \sigma(\nu, v) \exp\left(
        - \frac{m v^2}{2 \sub{k}{B} T}
     \right) \d v,
\end{equation}
где $\tau_0$  -- сответствующая нормировка такая, что для резонанса $\tau_0 = \int_v \d \tau (\nu_0,\, v)$.



Для насышенной спектроскопии нужно учесть эффект от дополнительного насыщающего лазерного луча. Из-за него значительная часть атомов в ячейке будут в возбужденном состоянии. Так как атомы могут поглощать свет только когда они в невозможденном состоянии, к \eqref{1_6} добаваить фактор $(\ng-\ne)/N$, описывающей разницу 
между количеством атомов в возбужденном состоянии $\ne$ и невозбужденном $\ng$. 

