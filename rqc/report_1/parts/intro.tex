% \textbf{Цель работы}. 
% Создание установки для проведения экспериментов и изучения насыщенной спектроскопии. 

% \phantom{42}

% \textbf{Задачи работы}. 
% Создание лазерной оптической системы с газовой ячейкой ${}^7$Li для наблюдения линий поглощения с помощью насыщенной спектроскопии.
% Оценка максимально допустимой контрастности спектроскопии.


\section{Введение}

В работе изучалась насыщенная спектроскопия ${}^7$Li. Упор был сделан на получение узких высококонтрастных линий спектра. 


Наблюдение контрастных линий спектра атомов может быть полезно для стабилизации лазера, путём запирания его частоты по атомному переходу. Данный метод нашел широкое применение в лазерном охлаждение атомов \cite{phd_mahalov,Rieger_doppler-freesaturation}. Добиваясь более узких и контрастных линий можно получить более чистую стабилизацию. 

В рамках данной работы была собрана установка для наблюдения насыщенной спектроскопии, оценена максимальная контрастность при заданной мощности лазера в приближении двухуровневой системы.



\section{Оборудование}

В работе использовался лазер типа ECDL-6725R с настраиваемым высококогерентным излучением с центральной длиной волны 671 нм на основе лазерного диода; фотодатчик фирмы ThorLabs модели DET36A/M; ячейка ${}^7$Li с контролируемой температурой.


На рисунке \ref{fig:sheme} приведена схема установки. Сигнальный луч дополнительно ослаблялся полупрозрачными зеркалами.

\begin{figure}[h]
    \centering
    \raisebox{0.25\height}{\incfig{scheme2}}
    \incfig{scheme_photo}
    \caption{Схема установки и фото установки}
    \label{fig:sheme}
\end{figure}

Также на \ref{fig:sheme} приведено фото установки. Данные с фотодетектора снимались осциллографом PicoScope. Красной пунктирной линией показан ход насыщающего пучка, желтой показан ход сигнального луча. 

Насыщающий луч переводит часть атомов в возбужденное состояние, соответственно уменьшается поглощение для сигнального пучка. Так как пучки встречные, то одновременно в резонансе они будут только при $\omega = \sub{\omega}{рез}$, взаимодействуя с группой атомов с нулевой скоростью, соответственно наблюдается линия спектра без доплеровского уширения. Исследуем подробнее механизм насыщения и оценим максимально возможное насыщение.

