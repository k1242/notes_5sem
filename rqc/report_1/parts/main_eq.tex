\subsection{Итоговая картина для двухуровневой системы}


Собираем всё вместе, в зависимости от мощности насыщающего лазера некоторое количество атомов будет находиться в возбужденном состоянии:
\begin{equation*}
    \frac{\ne}{N} = \frac{s/2}{1 + s + 4 (\Delta_+ \nu)^2/\Gamma^2},
    \hspace{5 mm} 
    \frac{\sigma(\nu, v) }{\sigma_0} = \frac{1}{4 (\Delta_{-}\nu)^2 / \Gamma^2 + 1},
    \hspace{5 mm} 
    \Delta_{\pm} \nu = \nu - \nu_0(1\pm  v /c).
\end{equation*}
Тогда
\begin{equation}
    \frac{\sub{I}{out}}{\sub{I}{int}} = \exp\left[
        - \kappa
        \int_{-\infty}^{\infty} 
        \left(
            1 - 2 \frac{N_e (\nu, v)}{N}
        \right)
        \frac{\sigma(\nu, v)}{\sigma_0}  \exp\left(
        - \frac{m v^2}{2 \sub{k}{B} T} 
    \right) \d v
    \right]
    , 
\end{equation}
где
\begin{equation*}
    s = \Phi/\sub{\Phi}{sat},
    \hspace{5 mm} 
    \sub{\Phi}{sat} = \Gamma/2 \sigma_0,
    \hspace{5 mm}
    \kappa = \sigma_0 n l \sqrt{\frac{m}{2 \pi \sub{k}{B} T}},
\end{equation*}
можно подставить, но пока не нужно:
\begin{equation}
    \frac{\sub{I}{out}}{\sub{I}{int}} = \exp\left[
        - \kappa
        \int_{-\infty}^{\infty} 
        \left(
            1 - \frac{s}{1 + s + 4 (\Delta_+ \nu)^2/\Gamma^2}
        \right)
        \frac{1}{4 (\Delta_{-}\nu)^2 / \Gamma^2 + 1}  \exp\left(
        - \frac{m v^2}{2 \sub{k}{B} T} 
    \right) \d v
    \right] = \exp\left[
        - \kappa F(s, \nu)
    \right]
    .
    \label{maineq}
\end{equation}

