\section{Тонкая и сверхтонкая структура \texorpdfstring{${}^7$Li}{7Li}}

В работе изучелись переходы D${}_1$ и D${}_2$ ${}^7$Li (рис. \ref{fig:li_levels}). 
Подробнее про тонкое и сверхтонкое расщепление лития можно прочитать в \cite{gold}.
Для D${}_1$ и D${}_2$ линий наблюдалось расщепление уровня \term{2}{2}{S}{1/2}, для D${}_1$ также наблюдалось расщепление \term{2}{2}{P}{1/2}.

\begin{figure}[h]
    \centering
    \incfig{Li_levels}
    \caption{Термы ${}^7$Li}
    \label{fig:li_levels}
\end{figure}


Соответствующие переходы помимо естественного уширения ($\sim 6$ МГц) также уширены из-за неколлинеарности пучков \cite{cross_beams}, уширены по Допплеру \cite{demtreder,gold} ($\sim 3$ ГГц). Для насыщенной спктроскопии становится критичным уширение из-за мощности насыщающего пучка \cite{gold}, которое подробнее будет рассмотрено в следующем разделе.