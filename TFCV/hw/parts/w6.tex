\subsection*{\S11, №3(4)}

Разложим в ряд Лорана по степеням $z$ в кольце $1 < |z| < 2$, функцию
\begin{equation*}
    f(z) = \frac{1}{(z-1)62 (z+2)} = 
    \frac{Az + B}{(z-1)^2} + \frac{C}{z+2},
    \hspace{5 mm} 
    \left\{\begin{aligned}
        2 B + C &= 1 \\
        C + A &= 0 \\
        2 A + B &= 2C \\
    \end{aligned}\right.
    \hspace{0.5cm} \Rightarrow \hspace{0.5cm}
    f(z) = - \frac{z - 4}{9 (z-1)^2} + \frac{1}{9 (z+2)},
\end{equation*}
где, выделяя удобные слагаемые, находим
\begin{equation*}
    f(z) = \frac{1}{3(1-z)^2} + \frac{1}{9} \left(
        \frac{1}{2 (1 + z/2)} + \frac{1}{1-z}
    \right),
    \hspace{5 mm} 
    \frac{1}{z + 2} = \sum_{n=0}^{\infty} \frac{(-1)^n}{2^{n+1}} z^n,
    \hspace{5 mm} 
    \frac{1}{(1-z)^2} = \sum_{n=1}^{\infty} (n-1) \frac{1}{z^n},
\end{equation*}
таким образом искомое разложение:
\begin{equation*}
    f(z) = \frac{1}{9} \sum_{n=0}^{\infty} \frac{(-1)^n}{2^{n+1}} z^n + 
    \sum_{n=1}^{\infty} \frac{-3 n - 4}{9} \frac{1}{z^n}. 
\end{equation*}




\subsection*{\S11, №4(6)}

Разложим в ряд Лорана по степеням $z-a$ в кольце $D$ функцию, вида
\begin{equation*}
    f(z) = \frac{2 z}{z^2 - 2 i},
    \hspace{5 mm} 
    a = 1, 
    \hspace{5 mm} 
    -1 \in D.
\end{equation*}
Для начала подставим $t = z - 1$:
\begin{equation*}
    f(t) = \frac{1}{2 + i} \frac{1}{1 + \frac{t}{2+i}} + \frac{i}{1 - \frac{t}{i}},
    \hspace{5 mm} 
    \frac{i}{1 + t/i} = \sum_{n=1}^{\infty}  \frac{i^{n+1}}{(z-1)^n},
    \hspace{5 mm} 
    \frac{1}{1 + \frac{t}{2+i}} = \sum_{n=0}^{\infty}  \frac{(-1)^n}{(2+i)^n} t^n.
\end{equation*}
Собирая все вместе, находим
\begin{equation*}
    f(z) = \sum_{n=0}^{\infty}  \frac{(-1)^n}{(2 + i)^{n+1}} (z-1)^n + 
    \sum_{n=1}^{\infty} i^{n-1} (z-1)^n.
\end{equation*}





\subsection*{\S11, №7(3)}




Разложим в ряд Лорана по степеням $z-a$ в кольце $D$ функцию, вида
\begin{equation*}
    f(z) = \frac{5 - 4z}{(z+1) (z^2 -1)^2}, 
    \hspace{5 mm} 
    a= 1, 
    \hspace{5 mm} 
    z_0 = 0.
\end{equation*}
Сделаем подстановку $t = z-1$, тогда
\begin{equation*}
    f(t) = \frac{1-4 t}{t^2 (t+2)^3} = \frac{-\frac{11}{16} t + \frac{1}{8}}{t^2} + 
    \frac{\frac{11}{16} t^2 + 4 t + \frac{15}{2}}{(t+2)^3},
\end{equation*}
где разложение было получено, как решение
\begin{equation*}
    A + C = 0, 
    \hspace{5 mm} 
    6 A + D + B = 0,
    \hspace{5 mm} 
    8 A + 12 B = -4,
    \hspace{5 mm} 
    8 B = 1,
\end{equation*}
решая которую, нашли, что
\begin{equation*}
    A = - \frac{11}{16}, \hspace{5 mm} 
    B = \frac{1}{8}, \hspace{5 mm} 
    C = \frac{11}{16},     \hspace{5 mm} 
    D = 4, \hspace{5 mm} 
    E = \frac{15}{2}.
\end{equation*}
Раскрывая, до удобного вида, находим
\begin{equation*}
    f(t) = - \frac{11}{16 t} + \frac{1}{8 t^2} + \frac{1}{(t+2)^3} \left(
        \frac{11}{16} t^2 + 4 4t + \frac{15}{2}
    \right).
    % +  \frac{1}{8} \frac{1}{(1 + t/2)^3} \left(\frac{11}{16} t^2 + 4 t + \frac{15}{2}\right).
\end{equation*}
Вспоним, что
\begin{equation*}
    \frac{1}{1 + t/2} = \sum_{n=0}^{\infty} (-1)^n \left(\frac{t}{2}\right)^n,
    \hspace{5 mm} 
    \bigg| \frac{t}{2} \bigg| < 1,
    \hspace{10 mm} 
    \frac{1}{2} \frac{1}{(1 + t/2)^3} = \sum_{n=2}^{\infty} \frac{(-1)^n}{2^n} n (n-1) t^{n-2}.
\end{equation*}
Собирая всё вместе находим, что
\begin{equation*}
    f(t) = \frac{1}{8 t^2} - \frac{11}{16 t} + \frac{1}{64} \sum_{n=0}^{\infty} 
    (-1)^n \frac{t^n}{2^n} (3+n) (20 + 9n),
    \hspace{5 mm} 
    0 < |t| < 2,
\end{equation*}
или, возвращаясь к $z$:
\begin{equation*}
    f(z) = \frac{1}{8 (z-1)^2} - \frac{11}{16 (z-1)} + \frac{1}{64} \sum_{n=0}^{\infty} 
    (-1)^n \frac{(z-1)^n}{2^n} (3+n) (20 + 9n),
    \hspace{5 mm} 
    0 < |z-1| < 2.
\end{equation*}



\subsection*{\S11, №10(6)}

Разложим функцию $f(z)$ в ряд Лорана по степеням $z-2i$, считая, что $0$ принадлежит кольцу, где $f(z)$ вида
\begin{equation*}
    f(z) = \frac{z-1- 5i}{z^2 -2z + 2} + \frac{3z - 1 - 3i}{z^2 - z(1 + 2i) - 1 + 1}.
\end{equation*}
Для начла перейдём к переменной $t = z - 2i$:
\begin{equation*}
    f(t) = \frac{t-1-3i}{(t-1+i)(t-1+3i)} + \frac{3t - 1 + 3i}{(t+i)(t-1+i)} = 
    \underbrace{\frac{1}{i + t}}_{\text{по Лорану}} + \underbrace{\frac{3}{t + (1 + 3i)}}_{\text{по Тейлору}}.
\end{equation*}
В частности:
\begin{equation*}
    \frac{1}{t(1 + i/t)} = \sum_{n=1}^{\infty}  (-1)^{n-1} i^{n-1} t^n,
    \hspace{10 mm} 
    \frac{3}{- \left(\frac{t}{1-3i}\right) + 1} = 3 \sum_{n=0}^{\infty} \frac{1}{(1-3i)^{n+1}} t^n,
\end{equation*}
собирая всё вместе, находим
\begin{equation*}
    f(t) = \sum_{n=1}^{\infty} (-i)^{n-1} \frac{1}{t^n} -3 \sum_{n=0}^{\infty} \frac{1}{(1-3i)^{n+1}} t^n,
    \hspace{5 mm} 
    1 < |t| < \sqrt{10},
\end{equation*}
где ограничения продиктованы сходимостью указанных разложений по Лорану и по Тейлору. Возвращаясь к $z$:
\begin{equation*}
    f(t) = \sum_{n=1}^{\infty} (-i)^{n-1} \frac{1}{(z-2i)^n} -3 \sum_{n=0}^{\infty} \frac{1}{(1-3i)^{n+1}} (z-2i)^n,
    \hspace{5 mm} 
    1 < |z - 2i| < \sqrt{10}.
\end{equation*}


\subsection*{\S12, №2(7)}

Покажем, что $z = \pi i $ -- полюс функции
\begin{equation*}
    f(z) = \frac{z}{(e^z + 1)^2}.
\end{equation*}
Действительно,
\begin{equation*}
    \lim_{z \to \pi i} f(z) (z - \pi i)^2 = \pi i \lim_{\varepsilon \to 0}
    \frac{\varepsilon^2}{(\cos (\pi + \varepsilon) + 1  + i \sin(\pi + \varepsilon))^2} = 
    \pi i \lim_{\varepsilon \to 0} \frac{\varepsilon^2}{(- i \varepsilon)^2} = \pi i,
\end{equation*}
а значит $z = \pi i $ -- полюс второго порядка.



\subsection*{\S12, №8}


\textbf{3)}. Найдём все изолированные особые точки однозначного характера для $f(z)$, вида
\begin{equation*}
    f(z) = z^2 \sin \frac{z}{z+1}.
\end{equation*}
Для начала заметим, что $z = -1$: $\not \exists \lim$, ($\lim_{z \to -1+0} \neq \lim_{z \to -1-0}$), а значит $z = -1$ -- существенная особая точка. 

Также можем найти, что
\begin{equation*}
    \lim_{z \to \infty} \frac{f(z)}{z^2} = \sin 1,
\end{equation*}
а значит $z = \infty$ -- полюс II порядка. 


\textbf{7)}. Теперь рассмотрим функцию, вида
\begin{equation*}
    f(z) = e^{\ctg \pi/z}. 
\end{equation*}
Во-первых, $z = 1/ k$, $k \in \mathbb{Z}\ \{0\}$ -- СОТ, т.к.
\begin{equation*}
    \underbrace{\lim_{\varepsilon \to 0} f\left(\frac{\pi}{\pi + \varepsilon}\right)}_{- \infty} \neq \underbrace{\lim_{\varepsilon \to 0} f\left(\frac{\pi}{\pi-\varepsilon}\right)}_{+ \infty}.
\end{equation*}
Точка $z = 0$ не является изолированной, а $z = \infty$ -- СОТ, т.к. $\ctg t$ испытывает разрыв в $t = 0$. 




\subsection*{\S12, №17(9)}

Рассмотрим функцию, вида
\begin{equation*}
    f(z) = \frac{e^{1/(z-2i)}}{1 - \cos i \pi z}.
\end{equation*}
Для начала заметим, что $z = 2 i$ -- СОТ ($1/x$ разрывна в 0). 

Рассмотрим точки такие, что $\cos i \pi z = 1$:
\begin{equation*}
    \lim_{z \to 2 i k} \frac{(z - 2 i k)^2}{1 - \cos (i \pi z)} = 
    \lim_{\varepsilon \to 0} \frac{\varepsilon^2}{1 - \cos \varepsilon} = 2 = \const \neq 0,
\end{equation*}
а значит $z = 2 i k$, где $k \in \mathbb{Z}$ -- полюса II порядка. Точка $z = \infty$ не является изолированной. 




\subsection*{\S12, №20(5)}

Повторим указанную выше процедуру для функции, вида
\begin{equation*}
    f(z) = \frac{\sin \pi z - \ch \pi/z}{(i - e^{\pi/z})^2}.
\end{equation*}
Стоит обратить внимание на точки $e^{\pi/z} = i \ \Leftrightarrow \ \frac{\pi}{z} = \frac{\pi}{2} + 2 \pi k$, $z = 0$ и $z = \infty$, где $k \in \mathbb{Z}$. 

Для начала заметим, что
\begin{equation*}
    z_k = - i\left(\frac{1}{2}  + 2k\right)^{-1},
    \hspace{0.5cm} \Rightarrow \hspace{0.5cm}
    \lim_{\varepsilon \to 0} f(z_k + \varepsilon) \varepsilon^2 = 
    \left(\sin \pi z_k - \ch \pi/z_k \right) \cdot \lim_{\varepsilon \to 0} \frac{\varepsilon^2}{- \frac{1}{16} (1 + 4k)^4 \pi^2 \varepsilon^2 + o(\varepsilon^2)} = \const \neq 0,
\end{equation*}
а значит $z_k$ -- полюса II порядка. 
Сразу получаем, что $z = 0$ не является изолированным.  

Осталось показать, что $z = \infty$ -- СОТ. Действительно, при $z \to \infty$ из $f(z)$ можно выделить $\sin z$, который является целой трансцендентной функцией, а значит $z = \infty$ -- СОТ. 




\Tsec{T29}

Пусть функция $f$ голоморфна в проколотом единичном круге $0 < |z| < 1$, и при некоторых $A > 0$и $\alpha \in [0, 1]$ выполняется неравенство
\begin{equation*}
    |f(z) | \leq \frac{A}{|z|^\alpha}.
\end{equation*}
Логично выделить случаи $\alpha = 1$ и $\alpha \in [0, 1)$. 

При $\alpha = 1$, $a$ -- устранимая особая точка или полюс первого порядка, т.к. 
\begin{equation*}
    \lim_{z \to 0} f(z) \cdot z^{1} \leq A, 
\end{equation*}
а значит в разложении $c_{k\leq -2} =0$, и $c_{-1}$ может быть не равно 0.


При $\alpha \in [0, 1)$ верно, что $c_{k \leq -1} = 0$, а значит $z = 0$  не более, чем УОТ.




\Tsec{T30}

Найдём главную часть ряда Лорана, функции
\begin{equation*}
    f(z) = \frac{1}{\sin z}.
\end{equation*}
Представим функцию, виде
\begin{equation*}
    f(z) = \frac{1}{z} \frac{z}{\sin z},
\end{equation*}
и воспользуемся методом неопределенных коэффициентов для нахождения разложения ${z}/{\sin z}$:
\begin{equation*}
    \underbrace{\left(\sum_{k=0}^{\infty} a_k z^k \right)}_{\sin z} \cdot
    \underbrace{\left(\sum_{k=0}^{\infty} b_k z^k\right)}_{z/\sin z} = 
    \underbrace{\sum_{k=0}^{\infty} c_k z^k}_{\equiv z}.
\end{equation*}
Само собой, выбираем кольцо, не содержащее точек, вида $z = \pi k$, $k \in \mathbb{Z}$. 

Вспоним, что
\begin{equation*}
    \sin z = \sum_{k=0}^{\infty}  \frac{(-i)^k}{(2k+1)!} z^{2k + 1},
\end{equation*}
тогда получаем набор уравнений на $b_k$, из которых нас интересует только $b_0$ (ищем главную часть $f(z)$):
\begin{equation*}
    \left\{\begin{aligned}
        a_0 b_0 &= c_0
        a_1 b_0 + a_0 b_1 &= c_1
    \end{aligned}\right.
    \hspace{0.5cm} \Rightarrow \hspace{0.5cm}
    b_0 = 1,
\end{equation*}
т.к. $a_0 = 0$, $a_1 = 1$ и $c_0 = 0$, $c_1 = 1$. А значит искомая главная часть $f(z)$:
\begin{equation*}
    f(z) = \boxed{\frac{1}{z}} + \text{TheilorSeries}[f](z).
\end{equation*}

