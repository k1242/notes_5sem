\subsection*{\S7, №5}

Разложим в ряд Тейлора функцию
\begin{equation*}
    f(z) = \frac{z^2 + 2 z}{(z+1)^3},
\end{equation*}
в окрестности точки $z = 1$. 

Заметим, что
\begin{equation*}
    \left(\frac{1}{z+1}\right)' = - \frac{1}{(z+1)^2},
    \hspace{5 mm} 
    \left(\frac{1}{(z+1)^2}\right)' = - \frac{2}{(z+1)^3},
    \hspace{0.5cm} \Rightarrow \hspace{0.5cm}
    \frac{1}{2} \left(\frac{1}{z+1}\right)'' = \frac{1}{(z+1)^3}.
\end{equation*}
Тогда
\begin{equation*}
    f(z) = \frac{(z+1)^2 - 1}{(z+1)^3} = \frac{1}{z + 1} - \frac{1}{2} \left(\frac{1}{z + 1}\right)''.
\end{equation*}
Вспомним, что
\begin{equation*}
    \frac{1}{1 + z} = \sum_{n=0}^{\infty} (-z)^n,
    \hspace{5 mm} 
    \frac{1}{1+z} = \frac{1}{2} \frac{1}{\frac{z-1}{2}+1} = \frac{1}{2} \sum_{n=0}^{\infty} (-1)^n \frac{(z-1)^n}{2^n}.
\end{equation*}
Теперь можем найти
\begin{equation*}
    f(z) = \frac{1}{2} \sum_{n=0}^{\infty}  (-1)^n \frac{(z-1)^n}{2^n} - 
    \frac{1}{4} \sum_{n=2}^{\infty}  (-1)^n n (n-1) \frac{(z-1)^{n-2}}{2^n} = 
    \frac{3}{8} - \frac{1}{16}(z-1) + 
    \sum_{n=2}^{\infty} \frac{(-1)^n}{2^{n+1}} (z-1)^n \left(
        1 - \frac{1}{8}(n+2) (n+1)
    \right),
\end{equation*}
то есть получили разложение по Тейлору. 




\subsection*{\S7, №6(2)}

Разложим в ряд Тейлора в окрестности точки $z= 0$ функции:
\begin{equation*}
    f(z) = \cos^3 z = \frac{1}{4} \cos(3x) + \frac{3}{4} \cos(x) = 
    \sum_{n=0}^{\infty} z^{2n} \frac{(-1)^n}{4 (2n)!} \left(3^{2n} + 3\right).
\end{equation*}


\subsection*{\S7, №11}

Определим порядок нуля $m$ функции $f(z)$ в точке $a$.

\textbf{2)}. Во-первых рассмотрим
\begin{equation*}
    f(z) = (z^2 - \pi^2)^4 \sin^3 z, \hspace{5 mm} 
    a = \pi.
\end{equation*}
Можем представить функции в виде
\begin{align*}
    z^2 - \pi^2 &= (z-\pi) h_1 (z), \hspace{5 mm} h_1 (z) \neq 0, \\
    \sin z &= (z-\pi) h_2 (z), \hspace{5 mm} h_2 (z) \neq 0,
\end{align*}
где $h$ -- голоморфная функция. Тогда
\begin{equation*}
    f(z) = (z-\pi)^{3+3} h_1(z)^3 h_2^3(z),
\end{equation*}
таким образом нашли, что $m=6$.



\textbf{3)}. Во-вторых рассмотрим
\begin{equation*}
    f(z) = (z^2 + \pi^2)^2 (e^{2z} -1)^4, \hspace{5 mm} 
    a = \pi i.
\end{equation*}
Аналогично раскрываем
\begin{align*}
    z^2 + \pi^2 &= (z-i\pi) h_1 (z), \hspace{5 mm} h_1 (z) \neq 0, \\
    e^{2z}-1 &= (z - i \pi) h_2 (z), \hspace{5 mm} h_2 (z) \neq 0,
\end{align*}
и снова получаем $m=6$.





\subsection*{\S7, №12 (3)}

Найдем, какая же функция является решением для гармонического осциллятора, вида
\begin{equation*}
    f''(z) + \lambda^2 f(z) = 0,
    \hspace{10 mm} f(0) = 0, \hspace{5 mm} 
    f'(0) = \lambda.
\end{equation*}
Считая $f(z)$ голоморфной, запишем
\begin{equation*}
    f(z) = \sum_{k=0}^{\infty}  c_k x^k,
    \hspace{10 mm} 
    f''(z) = \sum_{k=2}^{\infty}  
    k (k-1) c_k x^{k-2} = 
    \sum_{k=0}^{\infty} c_{k+2} (k+1) (k+2) x^k.
\end{equation*}
Подставляя в уравнение, находим
\begin{equation*}
    (c_0 \lambda^2 + 2 c_2) z^0 + 
    (c_1 \lambda^2 + 6 c_3) z^1 + 
    \sum_{k=2}^{\infty} \left(
        c_{k+2} (k+2)(k+1) + \lambda^2 c_k
    \right) z^k = 0.
\end{equation*}
Приравнивая к $0$ коэффициенты при всех степенях, находим
\begin{equation*}
    c_0 = c_2 = c_{2k} =  0,
    \hspace{5 mm} 
    c_1 = \lambda,
    \hspace{5 mm} 
    c_3 = - \frac{\lambda^3}{6},
    \hspace{5 mm} 
    c_5 = \frac{\lambda^5}{5!}, \hspace{5 mm} \ldots
    \hspace{0.5cm} \Rightarrow \hspace{0.5cm}
    f(z) = \sin (\lambda z).
\end{equation*}




\subsection*{\S9, №2}


\textbf{1)}. Покажем, что не существует голоморфной функции такой, что
\begin{equation*}
    f\left(\frac{1}{n}\right) = \sin \frac{\pi n}{2}.
\end{equation*}
Действительно, рассмотрим $g = f - f(0)$, где $f(0) \in [-1, 1]$, тогда $g(0) = 0$ -- не изолированный 0 функции, соответсвенно $f$ не могла быть голоморфной. 


\textbf{2)}. Аналогично рассмотрим
\begin{equation*}
    f\left(\frac{1}{n}\right) = \frac{1}{n} \cos(\pi n),
\end{equation*}
при чем при $n = 2 k$ $f > 0$ и $n = 2 k +1$ $f < 0$, тогда где-то между случается, что $f = 0$, а значит снова $0$ -- не изолированный 0 функции, соответственно $f$ не может быть голоморфной. 


\textbf{3)}. Просто подберем нужную функцию
\begin{equation*}
    f(z) = \frac{z}{2 + z}, 
    \hspace{10 mm} 
    f\left(\frac{1}{n}\right) = \frac{1}{2n  +1}. 
\end{equation*}

\textbf{4)}. Аналогичная функция подойдёт и здесь:
\begin{equation*}
    f(z) = \frac{z}{2 + z},
    \hspace{10 mm} 
    f\left(\frac{1}{n}\right) = \frac{\cos^2 \pi n}{2 n + 1},
\end{equation*}
где $\cos^2 \pi n = \const$, при $n \in \mathbb{N}$.

\textbf{5)}. Рассмотрим функцию, вида
\begin{equation*}
    f\left(\frac{1}{n}\right) = \frac{1}{2n + \cos \pi n}.
\end{equation*}
Заметим, что при $n_1 = 2k$ и при $n_2 = 2k + 1$:
\begin{equation*}
    f\left(\frac{1}{n_1}\right) = \frac{1}{4k+1}.
    \hspace{5 mm} 
    f\left(\frac{1}{n_2}\right) = \frac{1}{4k+1}.
\end{equation*}
Рассмотрим функцию $f(z) - \frac{z}{2}$, которая будет иметь не изолированный 0 в $z = 0$, т.к. $z/2$ голоморфная функция, можем сделать вывод, что $f(z)$ не могла быть голоморфной.


\textbf{6)}. Просто подберем подходящую функцию:
\begin{equation*}
    f\left(\frac{1}{n}\right) = f\left(- \frac{1}{n}\right) = \frac{1}{n^2},
    \hspace{0.5cm} \Rightarrow \hspace{0.5cm}
    f(z) = z^2.
\end{equation*}

\textbf{7)}. Рассмотрим
\begin{equation*}
    f\left(\frac{1}{n}\right) = f\left(-\frac{1}{n}\right) = \frac{1}{2n + 1}.
\end{equation*}
С одной стороны, из (3), следует, что
\begin{equation*}
    g(z) = \frac{z}{2 + z},
\end{equation*}
<<подходит>>, но $g\left(- \frac{1}{n}\right) \neq g \left(\frac{1}{n}\right)$, так что существование такой голоморфной $f$ противоречило бы теореме о единственности. 


\textbf{8)} Немного по-другому покажем, что
\begin{equation*}
    f\left(\frac{1}{n}\right) = e^{-n},
    \hspace{0.5cm} \overset{?}{\Rightarrow}  \hspace{0.5cm}
    f(z) = e^{-1/|z|},
\end{equation*}
но такая функция не аналитична в нуле, так что такой голоморфной $f$ не существует.



\textbf{9)}. Аналогично (7):
\begin{equation*}
    f\left(\frac{1}{n}\right) = f\left(-\frac{1}{n}\right) = \frac{1}{n^3}.
\end{equation*}
По теореме о единственности, если такая голоморфная $f$ существует, то $f(z) = g(z) = z^3$, но $g(-z) \neq g(z)$, а значит такой голоморфной $f$ не существует. 




\subsection*{\S9, №3}


Пусть функции $f_1$  и $f_2$ регулярны в области $D$ и удовлетворяют уравнения $f'(z) = P(z,\, f(z))$, где $P$ -- многочлен. Покажем, что если в некоторой точке $z_0 \in D$ имеет место равенство $f_1 (z_0) = f_2 (z_0)$, то $f_1 (z) \equiv  f_2 (z)$.

\begin{proof}[$\triangle$]

Для начала заметим, что
\begin{equation*}
    f_1 (z_0) = f_2 (z_0),
    \hspace{0.5cm} \Rightarrow \hspace{0.5cm}
    f_1'(z_0) = P(z_0, f_1(z_0)) = f_2'(z_0) = P(z_0,\, f_2(z_0)).
\end{equation*}
Также заметим, что
\begin{equation*}
    f'(z_0) = \sum_{k,\, m}^{n} c_{mk} z_0^k f(z_0)^m,
    \hspace{0.5cm} \Rightarrow \hspace{0.5cm}
    f''(z_0) = \sum_{k, m}^{n} c_{mk} \left(
        k z_0^{k-1} f^m (z_0) + z_0^k m f^{m-1} (z_0) f'(z_0)
    \right),
    \hspace{0.5cm} \Rightarrow \hspace{0.5cm}
    f_1^{(2)}(z_0) = f_2^{(2)} (z_0),
\end{equation*}
и аналогично для всех остальных производных: $f_1^{(n)}(z_0) = f_2^{(n)} (z_0)$, а значит, по аналитичности $f_1$ и $f_2$ можем сделать вывод, что $f_1 \equiv f_2$. 


\end{proof}




\Tsec{T26}
Фунция $f(z) = \sum_{n=1}^{\infty}  a_n z^n$. Выразим через $f$ сумму ряда $\sum_{n=1}^{\infty} n^3 a_n z^n$. Можем просто несколько раз продифференцировать:
\begin{equation*}
    z f'(z) = \sum_{n = 0}^{\infty} n a_n z^n,
    \hspace{5 mm} 
    z (z f')' = \sum_{n=0}^{\infty} n^2 a_n z^n,
    \hspace{5 mm} 
    z (z (z f')')' = \sum_{n=0}^{\infty} n^3 a_n z^n.
\end{equation*}
Тогда не сложно выразить
\begin{equation*}
    \sum_{n=0}^{\infty} n^3 a_n z^n = z (z (z f')')' = (z f' + f'' z^2)' = 
    f''' z^3 + 3 f'' z^2 + f' z.
\end{equation*}




\Tsec{T27}

Пусть $f(z)$  -- целая функция и для каждого $z_0 \in \mathbb{C}$ в разложении по Тейлору найдётся нулевой коэффициент. Покажем, что $f(z)$ -- полином. 

\begin{proof}[$\triangle$]

Построим множество
\begin{equation*}
    E_n = \left\{
        z \in \mathbb{C} \mid f^{(n)} (z) = 0
    \right\}.
\end{equation*}
При чем множество всех $\{E_n\}$ счётно, $\mathbb{C}$ несчётно, 
и по условию $\forall  z \ \exists n \colon  z \in E_n$, 
 а значит найдётся $N$ такой, что $E_N$ несчётно. Тогда, по теореме о единственности, $f^{(N)} \equiv 0$, а значит $f(z)$ -- полином. 


\end{proof}


\Tsec{T28}

Пусть целая функция $f(z)$ удовлетворяет условию $\Re f' (z) > 0 \ \forall  z \in \mathbb{C}$. Докажем, что $f(z)$ -- полином первой степени с положительным коэффициентом при $z^1$. 

\begin{proof}[$\triangle$]

Воспользуемся теоремой Пикара или теоремой Лиувилля (\href{https://ru.wikipedia.org/wiki/%D0%A2%D0%B5%D0%BE%D1%80%D0%B5%D0%BC%D0%B0_%D0%9B%D0%B8%D1%83%D0%B2%D0%B8%D0%BB%D0%BB%D1%8F_%D0%BE%D0%B1_%D0%BE%D0%B3%D1%80%D0%B0%D0%BD%D0%B8%D1%87%D0%B5%D0%BD%D0%BD%D1%8B%D1%85_%D1%86%D0%B5%D0%BB%D1%8B%D1%85_%D0%B0%D0%BD%D0%B0%D0%BB%D0%B8%D1%82%D0%B8%D1%87%D0%B5%D1%81%D0%BA%D0%B8%D1%85_%D1%84%D1%83%D0%BD%D0%BA%D1%86%D0%B8%D1%8F%D1%85}{доказательство}). 

\begin{to_thr}[теорема Пикара]
    Образ целовй функции $\neq \const$ тождественно равен $\mathbb{C}$, без, быть может, одной точки. 
\end{to_thr}

\begin{to_thr}[теорема Лиувилля]
    Если целая функция $f(z)$ ограничен, то $f(z)$ константа. 
\end{to_thr}

Построим функцию $g(z)$ вида
\begin{equation*}
    g(z) = \frac{1}{1 + f'(z)},
\end{equation*}
которая по условию ($\Re f' > 0$ и $f$ -- целая) является целой. Тогда
\begin{equation*}
    g(z) = \frac{1}{(x+1) + i y} = \underbrace{\frac{x + 1}{(x+1)^2 + y^2}}_{< 1} - i \underbrace{\frac{y}{(x+1)^2 + y^2}}_{< 1},
    \hspace{0.5cm} \Rightarrow \hspace{0.5cm}
    |g(z)| \leq \sqrt{2},
\end{equation*}
а значит, по теореме Лиувилля:
\begin{equation*}
    g(z) = \const,
    \hspace{0.5cm} \Rightarrow \hspace{0.5cm}
    f'(z) = \const,
    \hspace{0.5cm} \Rightarrow \hspace{0.5cm}
    f(z) = a z + b, \ \ a > 0.
\end{equation*}




\end{proof}

