\newpage

\section{B4}

\begin{to_thr}[]
    Если $\forall i$  $|a_{ii}| \geq \sum_{i,j} |a_{ij}| +\varepsilon$, то метод Зейделя сходится.
\end{to_thr}

\begin{proof}[$\triangle$]
    Введем погрешность решения $\delta^{(k+1)} = x^{(k)}-x^{(*)}$, для которой верно, что $\hat{^(k+1)} = R \delta^{(k)}$, а значит
    \begin{equation*}
        \delta_i^{(k+1)} = \frac{1}{a_{ii}}\left(
            \sum_{j=1}^{i-1} a_{ij} \delta_j^{(k+1)} + \sum_{j=i+1}^{n} a_{ij} \delta_j^{k}
        \right),
    \end{equation*}
    откуда получаем оценку
    \begin{equation*}
        |\delta_i^{(k+1)}| \leq \max_{1 \leq j < i} |\delta_j^{k+1}| \alpha_i  + \max_{i < j \leq n} |\delta_j^{(k)}| \beta_i,
        \hspace{0.5cm} \Rightarrow \hspace{0.5cm}
        \|\delta^{(k+1)}\| \leq 
         \alpha_i \|\delta^{(k+1)}\| + \beta_i \|\delta^{(k)}\|,
         \hspace{5 mm} 
         \alpha_i+ \beta_i < 1,
    \end{equation*}
    в силу диагонального преобладания. 
    Выражая норму, находим
    \begin{equation*}
        \|\delta^{(k+1)}\| \leq \frac{\beta_i}{1-\alpha_i} \|\delta^{(k)}\| = \sub{q}{seid}\|\delta^{(k)}\|,
        \hspace{0.5cm} \Rightarrow \hspace{0.5cm}
        \sub{q}{seid} = \frac{\beta_i}{1-\alpha_i}.
    \end{equation*}
\end{proof}

    Осталось вспомнить, что $\sub{q}{gauss} = \alpha_i + \beta_i$, и сравнить $\beta_i/(1-\alpha_i)$  и $\alpha_i + \beta_i$, для которых верно
    \begin{equation*}
        \frac{\beta_i}{1-\alpha_i} < \alpha_i + \beta_i,
        \hspace{0.5cm} \Leftrightarrow \hspace{0.5cm}
        \alpha_1 (\alpha_i + \beta_i) > \alpha_i,
        \hspace{0.5cm} \Rightarrow \hspace{0.5cm}
        \sub{q}{seid} < \sub{q}{gauss},
    \end{equation*}
    а значит метод Зейделя сходится быстрее.