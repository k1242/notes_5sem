% \input{settings/headline_tree.tex}





\newcommand{\progressbar}{% 
	\pgfmathsetmacro{\slidewidth}{\paperwidth}
	\pgfmathsetmacro{\progressstep}{\paperwidth/\inserttotalframenumber}
	\pgfmathsetmacro{\progresspos}{(\insertframenumber - 0.5) * \progressstep}
	\begin{tikzpicture}[scale = 0.035, line width = 1ex]
		\node[inner sep=0pt] (cat) at (\progresspos,0)	{\includegraphics[width=30pt]{settings/cats/cat_red.pdf}};
		\path[red] (0,0) -- (\slidewidth,0);
	\end{tikzpicture}
}

\makeatletter
\setbeamertemplate{footline}
{
\hfill 
% \progressbar %включить котика
    \leavevmode%
    \hbox{
    \begin{beamercolorbox}[wd=.33\paperwidth,ht=2.25ex,dp=1ex,center]{section in head/foot}
        \insertshortauthor~~\beamer@ifempty{\insertshortinstitute}{}{(\insertshortinstitute)} % раскомментить для авторов
        % \insertshortinstitute % раскомментить для вуза
    \end{beamercolorbox}%
    \begin{beamercolorbox}[wd=.33\paperwidth,ht=2.25ex,dp=1ex,center]{section in head/foot}
        % \insertshortdate{}
        \insertshorttitle
    \end{beamercolorbox}%
    \begin{beamercolorbox}[wd=.33\paperwidth,ht=2.25ex,dp=1ex,right]{section in head/foot}%
        \insertframenumber/\inserttotalframenumber \hspace{1mm}
    \end{beamercolorbox}
    }%
}

%
% \usebeamerfont{author in head/foot}\insertshortauthor~~\beamer@ifempty{\insertshortinstitute}{}{(\insertshortinstitute)} % раскомментить для авторов
% \usebeamerfont{author in head/foot}\beamer@ifempty{\insertshortinstitute}{}{\insertshortinstitute}
% \usebeamerfont{title in head/foot}\insertshorttitle
%   % \usebeamerfont{date in head/foot}\insertshortdate{}\hspace*{2em}
%   % \insertframenumber{} / \inserttotalframenumber\hspace*{2ex} 
% \insertshortinstitute \insertframenumber/\inserttotalframenumber


%%% Логотип
% \usepackage{tikz}
% \addtobeamertemplate{headline}{}{%
% \begin{tikzpicture}[remember picture,overlay]
% \node at([shift={(5.5,-0.5)}]current page.north) {\includegraphics[height=.8\headheight]{settings/cat.png}};
% \end{tikzpicture}}



%%% добавить сетку %%%
% \setbeamertemplate{background}{\tikz[overlay, remember picture, help lines]{
%     \foreach \x in {0,...,12} \path (current page.south west) +(\x,0.5) node {\small$\x$};
%     \foreach \y in {0,...,9} \path (current page.south west) +(0.5,\y) node {\small$\y$};
%     \foreach \x in {0,0.5,...,12.5} \draw (current page.south west) ++(\x,0) -- +(0,9.6);
%     \foreach \y in {0,0.5,...,9.5} \draw (current page.south west) ++(0,\y) -- +(12.8,0);
%   }
% }


%%% верхняя полоска с точками %%%
\setbeamertemplate{headline}
{%
  % \begin{beamercolorbox}[colsep=1.5pt]{upper separation line head}
  % \end{beamercolorbox}
  \begin{beamercolorbox}{section in head/foot}
    \vskip0pt\insertnavigation{\paperwidth}\vskip2pt
  \end{beamercolorbox}%
  \begin{beamercolorbox}[colsep=1.5pt]{lower separation line head}
  \end{beamercolorbox}
}
% \setbeamertemplate{headline}{}


%%% добавить маленькие номера слайдов справа снизу %%%
% \addtobeamertemplate{navigation symbols}{}{%
%     \usebeamerfont{footline}%
%     \usebeamercolor[myred]{footline}%
%     \hspace{1em}%
%     \insertframenumber/\inserttotalframenumber
% }




\setbeamertemplate{frametitle}{%
    \nointerlineskip%
    \begin{beamercolorbox}[wd=\paperwidth,ht=2.0ex,dp=0.6ex]{frametitle}
        \hspace*{1ex}\insertframetitle%
    \end{beamercolorbox}%
}