\subsection*{Теория}


% Опыт Франка-Герца подтверждает существование дискретных уровней энергии атомов. 
Рассмотрим столкновение электрона с атомом He. Если энергия электрона недостаточна, чтобы возбудить/ионизировать атом то \textit{упругое столкновение}, электрон не теряет энергию.

При большой разности потенциалов энергия электрона достаточна для возбуждения атомов, а значит происходит \textit{неупругое столкновение}, кинетическая энергия передаётся одному из атомных электронов, в результате чего происходит: \vspace{-2mm}
\begin{itemize*}
    \item \textbf{возбуждение} -- переход одного из атомных электронов на свободный энергетический уровень;
    \item \textbf{ионизация} -- отрыв электрона от атома.
\end{itemize*}.



А значит, измерив $\Delta U_1$ между максимумами и минимумами, отстоящих на равное расстояние, определим  энергию первого возбужденного состояния. 