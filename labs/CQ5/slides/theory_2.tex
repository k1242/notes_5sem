Так как часть электронов в $\ne$:
\begin{equation*}
    \d \alpha \to \frac{\ng-\ne}{\ng + \ne} d \alpha
\end{equation*}

Населенность в двух состояния описывается скоростными уранениями
\begin{align*}
    \dot{\ng} = \phantom{-}\Gamma N_e - \sigma \Phi (\ng - \ne), \\
    \dot{\ne} = - \Gamma \ne + \sigma \Phi (\ng - \ne),
\end{align*}
где $\ng + \ne = N = \const $.

Подставляя $\sigma$ ($v \to -v$), находим:
\begin{equation*}
    \frac{\ne}{N} = \frac{s/2}{1 + s + 4 (\nu - \nu_0(1+  v /c))^2/\Gamma^2}
\end{equation*}
где $s = \Phi/\sub{\Phi}{sat}$, $\sub{\Phi}{sat} = \Gamma/2 \sigma_0$.