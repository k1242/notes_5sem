\subsection*{Результаты}

Итак, была приближена зависимость $V_0 (\omega)$:
\begin{equation*}
    V_0 = a_{\text{fin}} \omega_0 + b_{\text{fin}},
    \hspace{10 mm} 
    \left.\begin{aligned}
        a_{\text{fin}} &= (6.4 \pm 2) 10^{-16} \text{ В/Гц}; \\
        b_{\text{fin}} &= (-1.59 \pm 0.04) 10^{-16} \text{ В}. \\
    \end{aligned}\right.
\end{equation*}
Красная граница может быть оценена, как
\begin{equation*}
    \omega_{\text{border}} = - \frac{b_{\text{fin}}}{a_{\text{fin}}} = (2.5 \pm 0.1) \times 10^{15} \text{Гц}.
\end{equation*}
Также можем найти $\hbar$, по формуле \eqref{einst_eq}:
\begin{align*}
    \hbar_{\text{measured}} = e \frac{d V_0}{d \omega} = e \cdot a_{\text{fin}} &= (1.02 \pm 0.03) \times  10^{-34} \text{Дж}\cdot\text{с}, \\
    \hbar_{\text{table}} &= 1.05 \times  10^{-34} \text{ Дж}\cdot\text{с},
\end{align*}
что подтверждает состоятельность измерений и оценки погрешности. 



\subsection*{Выводы}

Произведена градуировка спектрометра по спектру неона. Для различных длин волн измерена зависимость фототока от катодного потенциала. Показана несостоятельность измерений запирающего потенциала на глаз.

Оценены значения красной границы и постоянной планка.
Значение постоянной планка $\hbar$ совпало в пределах погрешности с табличным:
\begin{equation*}
    \hbar_{\text{measured}} = (1.02 \pm 0.03) \times  10^{-34} \text{Дж}\cdot\text{с}.
\end{equation*}
 Данным методом возможно достаточно точное измерение постоянной планка. 


