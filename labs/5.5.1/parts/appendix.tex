\newpage

\section*{Приложение}




% Создаёем новый разделитель


% Настраиваем значение разделителя для объектов 
\thisfloatsetup{floatrowsep=mysep}

% А вот и сам плавающий объект
\begin{figure}[h]
    \begin{floatrow}
        \ttabbox[0.4\textwidth]{\caption{Измерение прохождения $\gamma$-лучей через железо (Fe)}}{
            \begin{tabular}{rrr}
            \toprule
             $N$, шт. &  $n$, шт. &  $t$, с \\
            \midrule
                 457965 &           1 &       100 \\
                 150958 &           3 &       100 \\
                   5234 &           5 &        10 \\
                   5171 &           5 &        10 \\
                   5018 &           5 &        10 \\
                   1817 &           7 &        10 \\
                   1717 &           7 &        10 \\
                   1824 &           7 &        10 \\
            \bottomrule
            \end{tabular}
        }
        \ttabbox[0.4\textwidth]{\caption{Измерение прохождения $\gamma$-лучей через пробку (Cork)}}{
            \begin{tabular}{rrr}
            \toprule
             $N$, шт. &  $n$, шт. &  $t$, с \\
            \midrule
                  72820 &           8 &        10 \\
                  72144 &           8 &        10 \\
                  74198 &           8 &        10 \\
                  80443 &           2 &        10 \\
                  79852 &           2 &        10 \\
                  77126 &           4 &        10 \\
                  77511 &           4 &        10 \\
                  77865 &           4 &        10 \\
            \bottomrule
            \end{tabular}
        }         
    \end{floatrow}
\end{figure}

\phantom{42}

\thisfloatsetup{floatrowsep=mysep}

\begin{figure}[h]
    \begin{floatrow}
        \ttabbox[0.4\textwidth]{\caption{Измерение прохождения $\gamma$-лучей через железо (Pb)}}{
\begin{tabular}{rrr}
\toprule
 $N$, шт. &  $n$, шт. &  $t$, с \\
\midrule
     483437 &           1 &       100 \\
      16376 &           3 &        10 \\
      16601 &           3 &        10 \\
      16746 &           3 &        10 \\
      28994 &           2 &        10 \\
      28311 &           2 &        10 \\
      28524 &           2 &        10 \\
       9786 &           4 &        10 \\
       5733 &           5 &        10 \\
       5622 &           5 &        10 \\
       5890 &           5 &        10 \\
       3535 &           6 &        10 \\
       3488 &           6 &        10 \\
       3454 &           6 &        10 \\
       2308 &           7 &        10 \\
       2230 &           7 &        10 \\
       2361 &           7 &        10 \\
\bottomrule
\end{tabular}
        }
        \ttabbox[0.4\textwidth]{\caption{Измерение прохождения $\gamma$-лучей через пробку (Al)}}{
\begin{tabular}{rrr}
\toprule
 $N$, шт. &  $n$, шт. &  $t$, с \\
\midrule
      34421 &           8 &       100 \\
       3440 &           8 &        10 \\
       3581 &           8 &        10 \\
       3549 &           8 &        10 \\
     244309 &           3 &       100 \\
      56018 &           1 &        10 \\
      24511 &           3 &        10 \\
      24118 &           3 &        10 \\
      49268 &           7 &       100 \\
      11114 &           5 &        10 \\
      10664 &           5 &        10 \\
      10919 &           5 &        10 \\
\bottomrule
\end{tabular}
        }         
    \end{floatrow}
\end{figure}



% \begin{table}[ht]
%     \centering
%     \caption{Измерение прохождения $\gamma$-лучей через свинец (Pb)}
% \begin{tabular}{rrr}
% \toprule
%  $N$, шт. &  $n$, шт. &  $t$, с \\
% \midrule
%      483437 &           1 &       100 \\
%       16376 &           3 &        10 \\
%       16601 &           3 &        10 \\
%       16746 &           3 &        10 \\
%       28994 &           2 &        10 \\
%       28311 &           2 &        10 \\
%       28524 &           2 &        10 \\
%        9786 &           4 &        10 \\
%        5733 &           5 &        10 \\
%        5622 &           5 &        10 \\
%        5890 &           5 &        10 \\
%        3535 &           6 &        10 \\
%        3488 &           6 &        10 \\
%        3454 &           6 &        10 \\
%        2308 &           7 &        10 \\
%        2230 &           7 &        10 \\
%        2361 &           7 &        10 \\
% \bottomrule
% \end{tabular}
% \end{table}


% \begin{table}[ht]
%     \centering
%     \caption{Измерение прохождения $\gamma$-лучей через железо (Fe)}
% \begin{tabular}{rrr}
% \toprule
%  $N$, шт. &  $n$, шт. &  $t$, с \\
% \midrule
%      457965 &           1 &       100 \\
%      150958 &           3 &       100 \\
%        5234 &           5 &        10 \\
%        5171 &           5 &        10 \\
%        5018 &           5 &        10 \\
%        1817 &           7 &        10 \\
%        1717 &           7 &        10 \\
%        1824 &           7 &        10 \\
% \bottomrule
% \end{tabular}
% \end{table}



% \begin{table}[ht]
%     \centering
%     \caption{Измерение прохождения $\gamma$-лучей через пробку (Cork)}
% \begin{tabular}{rrr}
% \toprule
%  $N$, шт. &  $n$, шт. &  $t$, с \\
% \midrule
%       72820 &           8 &        10 \\
%       72144 &           8 &        10 \\
%       74198 &           8 &        10 \\
%       80443 &           2 &        10 \\
%       79852 &           2 &        10 \\
%       77126 &           4 &        10 \\
%       77511 &           4 &        10 \\
%       77865 &           4 &        10 \\
% \bottomrule
% \end{tabular}
% \end{table}



% \begin{table}[ht]
%     \centering
%     \caption{Измерение прохождения $\gamma$-лучей через алюминий (Al)}
% \begin{tabular}{rrr}
% \toprule
%  $N$, шт. &  $n$, шт. &  $t$, с \\
% \midrule
%       34421 &           8 &       100 \\
%        3440 &           8 &        10 \\
%        3581 &           8 &        10 \\
%        3549 &           8 &        10 \\
%      244309 &           3 &       100 \\
%       56018 &           1 &        10 \\
%       24511 &           3 &        10 \\
%       24118 &           3 &        10 \\
%       49268 &           7 &       100 \\
%       11114 &           5 &        10 \\
%       10664 &           5 &        10 \\
%       10919 &           5 &        10 \\
% \bottomrule
% \end{tabular}
% \end{table}


