\subsection*{Выводы}

Исследована зависимость пропускающей способности от толщины образца для свинца, железа, алюминия и пробки. Можно заметитьЮ что чем плотнее вещество, тем выше коэффциент ослабления. Измерены коэффциенты ослабления для указанных веществ.

 По значениям коэффциента ослабления, получено значение энергии $\gamma$-квантов: $E \approx 0.8$ МэВ. В пределах погрешности значения энергии для трёх веществ совпали. 






 \subsection*{Счётчик Гейгера}


 Также, параллельно с вышеописанной работой было произведено знакомство с прибором для измерения радиационного фона. 

 На рабочем месте уровень радиации составил 15 мкР/час. Вблизи пучка счётчик начинает зашкаливать на значениях более 999 мкР/час. Отдаляясь на 5 см от пучка, наблюдалось значение в районе 60 мкР/час, и на расстояние в районе 10 см, уже 26 мкр/час -- пучок действительно коллимированный.