\subsection*{Измерения}


При установленной заглушке в течение 100 секунд измерялся фон $\sub{N}{bias}$ (в начале и в конце работы):
\begin{equation*}
    \sub{N}{bias} = \mean([2034,\,  1980,\,  2099,\,  2143])/100\, \text{s} = (20.6 \pm 0.6)\, \s,
\end{equation*}
что соответствует погрещности в районе 3\%. 

При открытой заглушке, измерялся свободный поток частиц $N_0$:
\begin{equation*}
    N_0 = \mean([820832,\,  830346,\,  825628])/100\, \text{s} = (8256 \pm 39)\, \s,
\end{equation*}
что соответствует погрешности в 0.5\%.

Далее при различных временах измерялось количество пройденных частиц, при различных толщинах погрощающих веществ, а именно сняты данные по алюминию (Al), железу (Fe), пробке (Cork) и свинцу (Pb). Полные данные приведены в приложении, усредненные значения в таблице \ref{tab:1}, где $n$ -- количество установленных образцов.


\begin{table}[h]
    \centering
    \caption{Усредненные значения прохождения $\gamma$-лучей}
    \begin{tabular}{rrrrr}
    \toprule
     $n$ &  $N_{\text{Fe}}/s$, с$^{-1}$ &  $N_{\text{Al}}/s$, с$^{-1}$ &  $N_{\text{Pb}}/s$, с$^{-1}$ &  $N_{\text{Cork}}/s$, с$^{-1}$ \\
    \midrule
     1 &   4559 &   5581 &   4813 &          \\
     2 &        &        &   2840 &     7994 \\
     3 &   1488 &   2414 &   1636 &          \\
     4 &        &        &    958 &     7729 \\
     5 &    493 &   1069 &    554 &          \\
     6 &        &        &    328 &          \\
     7 &    158 &    472 &    209 &          \\
     8 &        &    329 &        &     7284 \\
    \bottomrule
    \end{tabular}
    \label{tab:1}
\end{table}


С помощью штангенциркуля измерим длину каждого кусочка поглощающего матриала, использованного в работе.
% добавить таблицу