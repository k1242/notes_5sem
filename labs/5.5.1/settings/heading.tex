% document's head

\phantom{42}
\vspace{20mm}

\begin{center}
    \LARGE \textsc{Лабораторная работа №5.5.1} \\
    \vspace{3 mm}
    \large Измерение коэффициента ослабления потока $\gamma$-лучей в веществе и определение их энергии
\end{center}

% \hrule

\phantom{42}

\begin{flushright}
    \begin{tabular}{rr}
    % written by:
        % \textbf{Источник}: 
        % & \href{__ссылка__}{__название__} \\
        % & \\
        % \textbf{Лектор}: 
        % & _ФИО_ \\
        % & \\
        \textbf{Автор работы}: 
        & Хоружий Кирилл \\
        & \\
    % date:
        \textbf{От}: &
        \textit{\today}\\
    \end{tabular}
\end{flushright}

\thispagestyle{empty}

\vspace{10mm}


\subsection*{Цель работы}
\begin{enumerate*}
    \item Ознакомиться с работой сцинтилляционного счетчика.
    \item  С помощью сцинтилляционного счетчика измерить линейные коэффициенты $\mu$ ослабления потока $ \gamma $-лучей в свинце, железе и алюминии; по их величине определить энергию $ \gamma $-квантов.
\end{enumerate*}


\subsection*{Оборудование}
Сцинтилляционный счетчик, штангенциркуль, диски пробки, алюминия, железа и свинца. 



\newpage
