\newpage

\section*{Приложение}




% % Создаёем новый разделитель


% % Настраиваем значение разделителя для объектов 
% \thisfloatsetup{floatrowsep=mysep}

% % А вот и сам плавающий объект
% \begin{figure}[h]
%     \begin{floatrow}
%         \ttabbox[0.4\textwidth]{\caption{Измерение прохождения $\gamma$-лучей через железо (Fe)}}{
%             \begin{tabular}{rrr}
%             \toprule
%              $N$, шт. &  $n$, шт. &  $t$, с \\
%             \midrule
%                  457965 &           1 &       100 \\
%                  150958 &           3 &       100 \\
%                    5234 &           5 &        10 \\
%                    5171 &           5 &        10 \\
%                    5018 &           5 &        10 \\
%                    1817 &           7 &        10 \\
%                    1717 &           7 &        10 \\
%                    1824 &           7 &        10 \\
%             \bottomrule
%             \end{tabular}
%         }
%         \ttabbox[0.4\textwidth]{\caption{Измерение прохождения $\gamma$-лучей через пробку (Cork)}}{
%             \begin{tabular}{rrr}
%             \toprule
%              $N$, шт. &  $n$, шт. &  $t$, с \\
%             \midrule
%                   72820 &           8 &        10 \\
%                   72144 &           8 &        10 \\
%                   74198 &           8 &        10 \\
%                   80443 &           2 &        10 \\
%                   79852 &           2 &        10 \\
%                   77126 &           4 &        10 \\
%                   77511 &           4 &        10 \\
%                   77865 &           4 &        10 \\
%             \bottomrule
%             \end{tabular}
%         }         
%     \end{floatrow}
% \end{figure}

% \phantom{42}

% \thisfloatsetup{floatrowsep=mysep}

% \begin{figure}[h]
%     \begin{floatrow}
%         \ttabbox[0.4\textwidth]{\caption{Измерение прохождения $\gamma$-лучей через железо (Pb)}}{
% \begin{tabular}{rrr}
% \toprule
%  $N$, шт. &  $n$, шт. &  $t$, с \\
% \midrule
%      483437 &           1 &       100 \\
%       16376 &           3 &        10 \\
%       16601 &           3 &        10 \\
%       16746 &           3 &        10 \\
%       28994 &           2 &        10 \\
%       28311 &           2 &        10 \\
%       28524 &           2 &        10 \\
%        9786 &           4 &        10 \\
%        5733 &           5 &        10 \\
%        5622 &           5 &        10 \\
%        5890 &           5 &        10 \\
%        3535 &           6 &        10 \\
%        3488 &           6 &        10 \\
%        3454 &           6 &        10 \\
%        2308 &           7 &        10 \\
%        2230 &           7 &        10 \\
%        2361 &           7 &        10 \\
% \bottomrule
% \end{tabular}
%         }
%         \ttabbox[0.4\textwidth]{\caption{Измерение прохождения $\gamma$-лучей через пробку (Al)}}{
% \begin{tabular}{rrr}
% \toprule
%  $N$, шт. &  $n$, шт. &  $t$, с \\
% \midrule
%       34421 &           8 &       100 \\
%        3440 &           8 &        10 \\
%        3581 &           8 &        10 \\
%        3549 &           8 &        10 \\
%      244309 &           3 &       100 \\
%       56018 &           1 &        10 \\
%       24511 &           3 &        10 \\
%       24118 &           3 &        10 \\
%       49268 &           7 &       100 \\
%       11114 &           5 &        10 \\
%       10664 &           5 &        10 \\
%       10919 &           5 &        10 \\
% \bottomrule
% \end{tabular}
%         }         
%     \end{floatrow}
% \end{figure}



% \begin{table}[ht]
%     \centering
%     \caption{Измерение прохождения $\gamma$-лучей через свинец (Pb)}
% \begin{tabular}{rrr}
% \toprule
%  $N$, шт. &  $n$, шт. &  $t$, с \\
% \midrule
%      483437 &           1 &       100 \\
%       16376 &           3 &        10 \\
%       16601 &           3 &        10 \\
%       16746 &           3 &        10 \\
%       28994 &           2 &        10 \\
%       28311 &           2 &        10 \\
%       28524 &           2 &        10 \\
%        9786 &           4 &        10 \\
%        5733 &           5 &        10 \\
%        5622 &           5 &        10 \\
%        5890 &           5 &        10 \\
%        3535 &           6 &        10 \\
%        3488 &           6 &        10 \\
%        3454 &           6 &        10 \\
%        2308 &           7 &        10 \\
%        2230 &           7 &        10 \\
%        2361 &           7 &        10 \\
% \bottomrule
% \end{tabular}
% \end{table}


\begin{table}[ht]
    \centering
    \caption{Градуировка монохроматора по линиям неона и ртути.}
\begin{tabular}{rr}
\toprule
  $n$, у.е. &   $\lambda$, нм \\
\midrule
1872 & 540.1 \\
2136 & 585.2 \\
2154 & 588.2 \\
2230 & 603.0 \\
2254 & 609.6 \\
2288 & 616.4 \\
2308 & 621.7 \\
2334 & 626.7 \\
2364 & 633.4 \\
2384 & 638.3 \\
2426 & 653.3 \\
2496 & 671.7 \\
2588 & 703.2 \\
2552 & 690.7 \\
2314 & 623.4 \\
2104 & 579.1 \\
2092 & 577.0 \\
1908 & 546.0 \\
1484 & 491.6 \\
 810 & 435.8 \\
 254 & 404.7 \\
\bottomrule
\end{tabular}

\end{table}



% \begin{table}[ht]
%     \centering
%     \caption{Измерение прохождения $\gamma$-лучей через пробку (Cork)}
% \begin{tabular}{rrr}
% \toprule
%  $N$, шт. &  $n$, шт. &  $t$, с \\
% \midrule
%       72820 &           8 &        10 \\
%       72144 &           8 &        10 \\
%       74198 &           8 &        10 \\
%       80443 &           2 &        10 \\
%       79852 &           2 &        10 \\
%       77126 &           4 &        10 \\
%       77511 &           4 &        10 \\
%       77865 &           4 &        10 \\
% \bottomrule
% \end{tabular}
% \end{table}



% \begin{table}[ht]
%     \centering
%     \caption{Измерение прохождения $\gamma$-лучей через алюминий (Al)}
% \begin{tabular}{rrr}
% \toprule
%  $N$, шт. &  $n$, шт. &  $t$, с \\
% \midrule
%       34421 &           8 &       100 \\
%        3440 &           8 &        10 \\
%        3581 &           8 &        10 \\
%        3549 &           8 &        10 \\
%      244309 &           3 &       100 \\
%       56018 &           1 &        10 \\
%       24511 &           3 &        10 \\
%       24118 &           3 &        10 \\
%       49268 &           7 &       100 \\
%       11114 &           5 &        10 \\
%       10664 &           5 &        10 \\
%       10919 &           5 &        10 \\
% \bottomrule
% \end{tabular}
% \end{table}


