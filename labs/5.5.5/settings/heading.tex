% document's head

\phantom{42}
\vspace{20mm}

\begin{center}
    \LARGE \textsc{Лабораторная работа №5.5.5} \\ \phantom{42} \\ 
    Компьютерная сцинтилляционная $\gamma$-спектрометрия
    \vspace{3 mm}
    \large 
\end{center}

% \hrule

\phantom{42}

\begin{flushright}
    \begin{tabular}{rr}
    % written by:
        % \textbf{Источник}: 
        % & \href{__ссылка__}{__название__} \\
        % & \\
        % \textbf{Лектор}: 
        % & _ФИО_ \\
        % & \\
        \textbf{Автор работы}: 
        & Хоружий Кирилл \\
        & \\
    % date:
        \textbf{От}: &
        \textit{\today}\\
    \end{tabular}
\end{flushright}

\thispagestyle{empty}

\vspace{10mm}


\subsection*{Цель работы}
\begin{enumerate*}
    \item Изучить спектр гамма-излучений для образцов 
    $\mathrm{^{22}Na,\, ^{137}Cs,\,  ^{60}Co,\,  ^{241}Am}$ и $\mathrm{^{152}Eu}$. 
    \item Найти для исследумых образцов пики полного поглощения и обратного рассеяния.
    \item Найти для исследумых образцов энергию, соответствующую краю комптоновских электронов.
\end{enumerate*}


% \subsection*{Оборудование}
% Призменный монохроматор-спектрометр УМ-2 ($380$-$1000$ нм), 
% фотоэлемент, покрытый Na$_2$\,K\,Sb(Cs), 
% неоновая лампа,
% усилитель постоянного тока,
% линза, два вольтметра.




\newpage
