\subsection*{Выводы}

Изучен спектр \na, \cs, \co, \am, и \eu. В нем выделен спектр комптоновских электронов, фотопики и пики обратного рассеяния.

 Полученные экспериментально значения фотопиков и края комптоновского спектра сходятся в пределах погрешности с табличными значениями. Наблюдаемые пики обратного рассеяния также ложатся на теоретическую кривую (2). 

Из аппроксимации также получены соответствующие ширины пиков, 
по которым проверена формула (1) для разрешения спектрометра. 