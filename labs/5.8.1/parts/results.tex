\newpage
\subsection*{Результаты}

\textbf{Постоянная Стефана-Больцмана}.
Табличное значение постоянной Стефана-Больцмана $\sub{\sigma}{table} = 5.7 \times 10^{-8} \ \cufrac{Вт}{(м$^2$ T$^4$)}$. Полученное значение составило
\begin{equation*}
    \sub{\sigma}{meas} = (4.8 \pm 0.7) \times 10^{-8} \ \cufrac{Вт}{(м$^2$ T$^4$)},
\end{equation*}
что, с учётом погрешности, чуть ниже табличного значения, что скорее всего объясняется некорректным значением $S$.

\textbf{Постоянная Планка}. Благодаря кубическому корню, чуть лучше обстоят дела с постоянной Планка:
\begin{equation*}
    \sigma = \frac{2 \pi^5 k^4}{15 h^3 c^2},
    \hspace{0.5cm} \Rightarrow \hspace{0.5cm}
    h = \sqrt[3]{\frac{2 \pi^5 k^4}{15 c^2}} \cdot \sqrt[3]{\frac{1}{\sigma}},
    \hspace{0.5cm} \Rightarrow \hspace{0.5cm}
    \sub{h}{meas} = (7.0 \pm 0.4) \times 10^{-34} \text{ Дж} \cdot \text{с},
\end{equation*}
что, кстати, замечательно сходится\footnote{
    Сыграло округление вверх погрешности.
}  с табличным значением $h_{\text{table}} =  6.6 \times 10^{-34} \text{ Дж} \cdot \text{с}$. 


\subsection*{Выводы}


С помощью пирометра исследована яркостная температура накаленных тел с различной испускательной способностью. Определено значение постоянной Стефана-Больцмана
\begin{equation*}
    \sub{\sigma}{meas} = (4.8 \pm 0.7) \times 10^{-8} \ \cufrac{Вт}{(м$^2$ T$^4$)},
\end{equation*}
и постоянной Планка
\begin{equation*}
    \sub{h}{meas} = (7.0 \pm 0.4) \times 10^{-34} \text{ Дж} \cdot \text{с},
\end{equation*}
что частично совпадает с табличными значениями в пределах погрешности. 

Измерение постоянной Стефана-Больцмана проблематично для измерения в силу зависимости от $T^{-4}$, которая сама по себе определяется с большой неточностью. Также в рамках этой работы не представляется возможным определить площадь, с которой происходит излучение, от чего эксперимент носит скорее наблюдательный характер. 

