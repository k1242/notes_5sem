\subsection*{Дополнение}



\thisfloatsetup{floatrowsep=mysep}

% А вот и сам плавающий объект
\begin{figure}[h!]
    \begin{floatrow}
        \ttabbox[0.33\textwidth]{\caption{Измерения $\sub{I}{dark} (U)$}}{
            \begin{tabular}{rr}
\toprule
     $U_1$, В &     $I$, $\mu$А \\
\midrule
   0.1 &    2.0 \\
   3.9 &   10.0 \\
   5.6 &   15.0 \\
   7.1 &   20.0 \\
   8.7 &   25.0 \\
  10.6 &   30.0 \\
  12.7 &   35.0 \\
  14.8 &   40.0 \\
  17.0 &   45.0 \\
  18.6 &   47.5 \\
  20.9 &   48.5 \\
  21.7 &   47.0 \\
  22.8 &   43.0 \\
  23.9 &   37.0 \\
  25.0 &   40.0 \\
  26.4 &   45.0 \\
  28.0 &   50.0 \\
  29.5 &   55.0 \\
  31.0 &   60.0 \\
  32.4 &   65.0 \\
  33.9 &   69.0 \\
  35.3 &   73.0 \\
  36.0 &   74.0 \\
  38.0 &   75.0 \\
  39.5 &   73.0 \\
  41.9 &   71.0 \\
  43.5 &   70.0 \\
  45.5 &   71.0 \\
  49.0 &   75.0 \\
  52.1 &   80.0 \\
  54.9 &   85.0 \\
  58.3 &   90.0 \\
  64.1 &   95.0 \\
  67.2 &   97.0 \\
  68.3 &   97.0 \\
  68.7 &  100.0 \\
\bottomrule
\end{tabular}
        }
        \ttabbox[0.33\textwidth]{\caption{Измерения $\sub{I}{light} (U)$}}{
            \begin{tabular}{rr}
\toprule
   $U_1$, В &     $I$, $\mu$А \\
\midrule
 0.1 &  0.0 \\
 5.3 & 10.0 \\
 8.4 & 20.0 \\
11.8 & 30.0 \\
15.8 & 40.0 \\
17.7 & 45.0 \\
18.7 & 47.0 \\
19.1 & 48.0 \\
20.2 & 48.5 \\
20.5 & 48.0 \\
21.0 & 48.0 \\
21.9 & 47.0 \\
22.9 & 45.0 \\
23.3 & 42.0 \\
23.5 & 38.0 \\
23.6 & 36.0 \\
23.8 & 29.5 \\
23.8 & 29.0 \\
24.0 & 28.0 \\
24.7 & 26.0 \\
25.5 & 27.0 \\
26.2 & 29.0 \\
27.6 & 35.0 \\
29.0 & 40.0 \\
30.3 & 45.0 \\
31.6 & 50.0 \\
33.1 & 55.0 \\
34.6 & 60.0 \\
35.5 & 62.0 \\
36.9 & 64.0 \\
38.1 & 65.0 \\
39.2 & 64.0 \\
39.9 & 63.0 \\
41.7 & 60.0 \\
43.9 & 57.0 \\
44.4 & 56.5 \\
45.3 & 56.0 \\
45.7 & 55.5 \\
46.1 & 56.0 \\
47.3 & 56.5 \\
50.5 & 60.0 \\
53.8 & 65.0 \\
57.0 & 70.0 \\
60.9 & 75.0 \\
66.9 & 77.0 \\
67.5 & 77.0 \\
68.0 & 78.0 \\
74.8 & 81.0 \\
\bottomrule
\end{tabular}

        }         
        \ttabbox[0.33\textwidth]{\caption{Измерения $\sub{I}{photo} (U)$}}{
            \begin{tabular}{rr}
\toprule
     $U_1$, В &     $I$, $\mu$А \\
\midrule
   0.1 &    0.0 \\
   7.0 &   10.0 \\
  10.1 &   20.0 \\
  13.6 &   30.0 \\
  17.5 &   40.0 \\
  19.3 &   44.0 \\
  20.8 &   45.0 \\
  20.9 &   45.5 \\
  22.4 &   44.0 \\
  23.5 &   40.0 \\
  24.0 &   35.0 \\
  24.1 &   24.0 \\
  24.2 &   22.0 \\
  24.3 &   20.5 \\
  24.5 &   19.0 \\
  24.8 &   18.0 \\
  25.6 &   17.0 \\
  26.9 &   18.0 \\
  27.6 &   20.0 \\
  30.0 &   30.0 \\
  32.6 &   40.0 \\
  35.5 &   50.0 \\
  36.5 &   52.0 \\
  37.3 &   53.0 \\
  38.8 &   53.5 \\
  40.0 &   52.0 \\
  41.4 &   50.0 \\
  44.7 &   45.0 \\
  46.8 &   43.0 \\
  49.1 &   42.5 \\
  49.5 &   43.0 \\
  52.3 &   45.0 \\
  55.2 &   50.0 \\
  58.9 &   55.0 \\
  77.4 &   60.0 \\
\bottomrule
\end{tabular}

        }         
    \end{floatrow}
\end{figure}

\newpage

\phantom{42}


\begin{table}[ht]
    \centering
    \caption{Зависимость коэффициента умноженяи от напряжения}
\begin{tabular}{rrr}
\toprule
    $U$ &           $M$ &         $\Delta M$ \\
\midrule
 0.00 &    0.85 &   0.02\\
 5.00 &    1.00 &   0.02 \\
10.00 &    1.00 &   0.02 \\
15.00 &    1.00 &   0.02 \\
20.00 &    1.00 &   0.02 \\
25.00 &    1.03 &   0.02 \\
30.00 &    1.12 &   0.02 \\
35.00 &    1.31 &   0.02 \\
40.00 &    1.65 &   0.03 \\
45.00 &    2.20 &   0.04 \\
50.00 &    2.95 &   0.05 \\
55.00 &    7.05 &   0.14 \\
58.00 &   17.43 &   0.34 \\
59.00 &   24.38 &   0.48 \\
60.00 &   40.30 &   0.80 \\
61.00 &  104.0 &   2.0 \\
62.00 &  458.5 &   9.1 \\
62.50 & 1095 &  21 \\
63.00 & 2362 &  47 \\
63.50 & 4056 &  81 \\
64.00 & 5886 & 117 \\
64.50 & 7786 & 155 \\
64.70 & 8510 & 170 \\
64.80 & 8973 & 179 \\
64.85 & 9117 & 182 \\
\bottomrule
\end{tabular}
\end{table}