\subsection*{Дополнение}



\thisfloatsetup{floatrowsep=mysep}

% А вот и сам плавающий объект
\begin{figure}[h!]
    \begin{floatrow}
        \ttabbox[0.33\textwidth]{\caption{Измерения $\sub{I}{dark} (U)$}}{
            \begin{tabular}{rr}
\toprule
    U, В &        $\sub{I}{dark}$,  нА\\
\midrule
 0.00 &     0.40 \\
 5.00 &     0.45 \\
10.00 &     0.51 \\
15.00 &     0.58 \\
20.00 &     0.70 \\
25.00 &     0.87 \\
30.00 &     1.17 \\
35.00 &     1.65 \\
40.00 &     2.50 \\
45.00 &     4.15 \\
50.00 &     8.16 \\
55.00 &    13.1 \\
58.00 &    15.8 \\
59.00 &    21.8 \\
60.00 &    34.7 \\
61.00 &    80.7 \\
62.00 &   390 \\
62.50 &  1000 \\
63.00 &  2230 \\
63.50 &  4060 \\
64.00 &  6350 \\
64.50 &  9110 \\
64.70 & 10460 \\
64.80 & 11150 \\
64.85 & 11670 \\
\bottomrule
\end{tabular}
        }
        \ttabbox[0.33\textwidth]{\caption{Измерения $\sub{I}{light} (U)$}}{
            \begin{tabular}{rr}
\toprule
    U, В &        $\sub{I}{light}$,  нА\\
\midrule
 0.00 &     2.36 \\
 5.00 &     2.75 \\
10.00 &     2.81 \\
15.00 &     2.88 \\
20.00 &     2.99 \\
25.00 &     3.24 \\
30.00 &     3.74 \\
35.00 &     4.66 \\
40.00 &     6.29 \\
45.00 &     9.20 \\
50.00 &     14.9 \\
55.00 &     29.2 \\
58.00 &     55.7 \\
59.00 &     77.6 \\
60.00 &      127 \\
61.00 &      319 \\
62.00 &     1440 \\
62.50 &     3508 \\
63.00 &     7640 \\
63.50 &    13350 \\
64.00 &    19830 \\
64.50 &    26940 \\
64.70 &    29950 \\
64.80 &    31700 \\
64.85 &    32550 \\
\bottomrule
\end{tabular}
        }         
        \ttabbox[0.33\textwidth]{\caption{Измерения $\sub{I}{photo} (U)$}}{
            \begin{tabular}{rr}
\toprule
     $U_1$, В &     $I$, $\mu$А \\
\midrule
   0.1 &    0.0 \\
   7.0 &   10.0 \\
  10.1 &   20.0 \\
  13.6 &   30.0 \\
  17.5 &   40.0 \\
  19.3 &   44.0 \\
  20.8 &   45.0 \\
  20.9 &   45.5 \\
  22.4 &   44.0 \\
  23.5 &   40.0 \\
  24.0 &   35.0 \\
  24.1 &   24.0 \\
  24.2 &   22.0 \\
  24.3 &   20.5 \\
  24.5 &   19.0 \\
  24.8 &   18.0 \\
  25.6 &   17.0 \\
  26.9 &   18.0 \\
  27.6 &   20.0 \\
  30.0 &   30.0 \\
  32.6 &   40.0 \\
  35.5 &   50.0 \\
  36.5 &   52.0 \\
  37.3 &   53.0 \\
  38.8 &   53.5 \\
  40.0 &   52.0 \\
  41.4 &   50.0 \\
  44.7 &   45.0 \\
  46.8 &   43.0 \\
  49.1 &   42.5 \\
  49.5 &   43.0 \\
  52.3 &   45.0 \\
  55.2 &   50.0 \\
  58.9 &   55.0 \\
  77.4 &   60.0 \\
\bottomrule
\end{tabular}

        }         
    \end{floatrow}
\end{figure}

\newpage

\phantom{42}


\begin{table}[ht]
    \centering
    \caption{Зависимость коэффициента умноженяи от напряжения}
\begin{tabular}{rrr}
\toprule
    $U$ &           $M$ &         $\Delta M$ \\
\midrule
 0.00 &    0.85 &   0.02\\
 5.00 &    1.00 &   0.02 \\
10.00 &    1.00 &   0.02 \\
15.00 &    1.00 &   0.02 \\
20.00 &    1.00 &   0.02 \\
25.00 &    1.03 &   0.02 \\
30.00 &    1.12 &   0.02 \\
35.00 &    1.31 &   0.02 \\
40.00 &    1.65 &   0.03 \\
45.00 &    2.20 &   0.04 \\
50.00 &    2.95 &   0.05 \\
55.00 &    7.05 &   0.14 \\
58.00 &   17.43 &   0.34 \\
59.00 &   24.38 &   0.48 \\
60.00 &   40.30 &   0.80 \\
61.00 &  104.0 &   2.0 \\
62.00 &  458.5 &   9.1 \\
62.50 & 1095 &  21 \\
63.00 & 2362 &  47 \\
63.50 & 4056 &  81 \\
64.00 & 5886 & 117 \\
64.50 & 7786 & 155 \\
64.70 & 8510 & 170 \\
64.80 & 8973 & 179 \\
64.85 & 9117 & 182 \\
\bottomrule
\end{tabular}
\end{table}