\subsection*{Теория}


Смещение длины волны при рассеянии (эффект Комптона):
\begin{equation}
    \Delta \lambda = \lambda_1 - \lambda_0 = \frac{h}{mc} \left(1 - \cos \theta\right).
\end{equation}
Считая, что $\varepsilon (\theta) = \frac{E_\gamma}{m c^2}= A N$ -- энергия рассеянных $\gamma$-квантов линейно зависит от номера соответствующего канала, найдём, что
\begin{equation}
    \frac{1}{N(\theta)} - \frac{1}{N(0)} = A (1 - \cos \theta),
    \hspace{10 mm} 
    m c^2 = E_\gamma(0) \frac{N_{90}}{N_0 - N_{90}},
\end{equation}
где $N_i$ -- номер канала, соотетствующего максимуму при угле в $i^{\circ}$, и $E_\gamma (0) = E_0$ -- энергия испускаемых источникм $\gamma$-квантов: $E_\gamma (0) = E_0 = 662$ кэВ.
