\section{Семинар от 16.10.21}

Рассмотрим снова некоторую граничную задачу:
\begin{equation*}
    \hat{L} G(x, y) = \delta(x-y).
\end{equation*}
Запишем граничные условия:
\begin{equation*}
    \alpha_1 G(a. y) + \beta_1 G'_x(a, y) = 0,
    \hspace{10 mm} 
    \alpha_2 G(b, y) + \beta_2 G'_x (b, y) = 0,
\end{equation*}
где  $|\alpha_1| + |\beta_2| \neq 0$ и $|\alpha_2| + |\beta_2| \neq 0$.
Можем выписать ответ:
\begin{equation*}
    G(x, y) = \frac{1}{W(y)} \left\{\begin{aligned}
        &v(y) u(x), &x < y; \\
        &v(x) u(y), &x > y,
    \end{aligned}\right.
\end{equation*}
где Вронскиан можно запсиать, как
\begin{equation*}
    W(x) = W(x_9) \exp\left(
        - \int_{x_0}^{x}  Q(t) \d t,
    \right)
\end{equation*}
где $Q(t)$ -- из оператора Штурма-Лиувилля. 

Также решали задачу с периодическими гран. условиями, где $G$ наследовала гран. условия. 
Решать это всё умеем двумя способами: разделяя на $x  > y$ и $x < y$, и через метод Фурье:
\begin{equation*}
    \hat{L} e_n  = \lambda_n e_n,
    \hspace{10 mm} 
    \langle e_n | e_m \rangle = \int_{a}^{b} e_n (x) \bar{e}_m(x)  \d x.
\end{equation*}
Тогда можем найти функцию Грина, как
\begin{equation*}
    G(x, y) = \sum_n g_n (y) e_n (x),
    \hspace{5 mm} 
    \delta(x-y) = \sum_n \delta_n ( y) e_n (x).
\end{equation*}
Находим коэффициенты Фурье:
\begin{equation*}
    g_n (y) = 
    \frac{\langle G | e_n\rangle}{\langle e_n | e_n\rangle},
    \hspace{0.5cm} \Rightarrow \hspace{0.5cm}
    \delta_n (y) = \frac{\bar{e}_n (y)}{\langle e_n | e_n\rangle},
    \hspace{0.5cm} \Rightarrow \hspace{0.5cm}
    g_n (y) = \frac{1}{\lambda_n} \frac{\bar{e}_n (y)}{\langle e_n | e_n\rangle},
\end{equation*}
где мы решали уравнение, вида $\hat{L} G = \delta(x-y)$. 
Проблема возникает при $\lambda_n = 0$. 



\textbf{Решение}. Наличие у оператора собственного числа $\lambda_n = 0$ называется нулевой модой. Рассмотрим оператор:
\begin{equation*}
    \hat{L} = \partial_x^2,
\end{equation*}
для которого $e_n (x) = e^{i n x}$, где $\langle e_n | e_n\rangle = 2 \pi$, где $e_0 = 1$ и $\lambda_{0} = 0$. Пусть тогда
\begin{equation*}
    \delta(x) = \sum \frac{\bar{e}_n (0) e_n (x)}{\langle e_n | e_n\rangle} = \sum \frac{e^{i n x}}{2 \pi},
    \hspace{10 mm} 
    G(x) = \sum  g_n e_n (x). 
\end{equation*}
но для $\hat{L} G = \delta(x)$ оказывается нет решений (справа $e_0$ есть, а слева нет). То есть
\begin{equation*}
    \Ker \hat{L} \neq \{0\},
    \hspace{10 mm} 
    \Ker \hat{L} + \Im \hat{L} = \mathcal H,
\end{equation*}
поэтому всегда имеем ввиду, что $\hat{L} \hat{L}^{-1} = \mathbbm{1}$, но только для $\im \hat{L}$. 

В общем, проблему уйдёт, если рассмотрим уравнение, вида
\begin{equation*}
    \hat{L} G(x) = \delta(x) - e_0(x) = \delta(x) - \frac{1}{2 \pi},
\end{equation*}
то есть справа единичный оператор только на образе $\im \hat{L}$. 



Если в источнике есть нулевая мода, то уравнение не имеет решений. 



\textbf{Алгоритм (Фурье)}. Раскладываем 
\begin{equation*}
    G(x) = \sum_{n \neq 0} g_n e_n,
    \hspace{10 mm} 
    \delta(x) = \sum \frac{e_n (x)}{2 \pi},
    \hspace{0.5cm} \Rightarrow \hspace{0.5cm}
    \hat{L} G = \delta(x) - \frac{1}{2\pi}.
\end{equation*}
Знаем, что $\lambda_n g_n = \frac{1}{2\pi}$, а значит
\begin{equation*}
    g_n(x) = \frac{1}{2\pi} \frac{1}{- n^2},
    \hspace{0.5cm} \Rightarrow \hspace{0.5cm}   
    G(x) = \sum_{n \neq 0} \frac{1}{2\pi} \frac{1}{-n62} e^{i n x},
\end{equation*}
и рассмотрим $0 < x < \pi$, суммирая это через вычеты, записываем
\begin{equation*}
    f(z) = \frac{e^{zx}}{2 \pi z^2}, 
    \hspace{0.5cm} \Rightarrow \hspace{0.5cm}   
    G(x) = \sum \oint_{in} \frac{\d z}{2 \pi i} f(z) g(z).
\end{equation*}
Соответственно, выберем
\begin{align*}
    g(z) = \frac{\pi e^{- \pi z}}{\sh (\pi z)}
\end{align*}
тогда
\begin{equation*}
    f(z) g(z) = \frac{\pi}{z^2} \frac{e^{(x-\pi)z}}{\sh \pi z},
\end{equation*}
получаем, что интеграл по душам вправо/влево  равен $0$, и остается только вычет в $z = 0$:
\begin{equation*}
    G(z) = - \res_0 f(z) g(z) = \ldots = - \frac{x^2}{4 \pi} + \frac{x}{2} - \frac{\pi}{6}.
\end{equation*}



\textbf{Алгоритм (сшивка)}. Решим задачу
\begin{equation*}
    \partial_x^2 G(x) = \delta(x) - \frac{1}{2\pi}.
\end{equation*}
Разбиваем $x < 0$ и $x > 0$:
\begin{align*}
    &x < 0, 
    & G = -\tfrac{x^2}{4 \pi} + a x + b, \\
    &x > 0, 
    & G = -\tfrac{x^2}{4 \pi} + c x + \varpi, 
\end{align*}
учитываем граничные условия:
\begin{equation*}
    G(-0) = G(+0),
    \hspace{5 mm} 
    G'(+0) - G'(-0) = 1,
    \hspace{0.5cm} \Rightarrow \hspace{0.5cm}   
    b = \varpi.
\end{equation*}
Также получаем, что $-a = b$.

Учтём, что $e_0$ не входит в $G$:
\begin{equation*}
    \langle G | e_0\rangle = 0 = \int_{-\pi}^{+\pi}G(x) \d x = 0,
    \hspace{0.5cm} \Rightarrow \hspace{0.5cm}
    b = - \frac{\pi}{6},
\end{equation*}
так и получаем все необходиме условия на $G(x, y)$. 



\subsection{Многомерие \texorpdfstring{$\mathbb{R}^3$}{R3}}

Рассмотрим $\mathbb{R}^3$:
\begin{equation*}
    \nabla^2 f = \varphi,
\end{equation*}
где все линейно, всё хорошо. Как обычно будем искать функцию, виде
\begin{equation*}
    f(\vc{r}) = \int_{\mathbb{R}^3} G(\vc{r}  - \vc{r}') \varphi(\vc{r}) \d^3 r. 
\end{equation*}
Функцию Грина найдём в виде
\begin{equation*}
    \nabla^2 G(\vc{r}) = \delta(r^3) = \delta(x) \delta(y) \delta(z),
    \hspace{10 mm}  
    \int f(\vc{r}) \delta(\vc{r}- \vc{r}') \d^3 \vc{r}' = f(\vc{r}').
\end{equation*}
Можем свести уравнение Лапласа, к уравнению Дебая:
\begin{equation*}
    (\nabla^2 - \kappa^2) G(\vc{r}) = \delta(\vc{r}),
\end{equation*} 
которое очень удобно раскладывать по Фурье:
\begin{align*}
    &\text{ПФ}: 
    &G(\vc{k}) &= \int_{\mathbb{R}^3} G(\vc{r}) e^{- i \smallvc{k} \cdot \smallvc{r}} \d \vc{r}, \\
    &\text{ОПФ}: 
    &G(\vc{r}) &= \int_{\mathbb{R}^3} G(\vc{r}) e^{i \smallvc{k} \cdot \smallvc{r}} \frac{\d \vc{k}}{(2 \pi)^3}.
\end{align*}
Также вспомним, что
\begin{equation*}
    \partial_m G(\vc{r}) e^{- i \smallvc{k} \cdot \smallvc{r}} \d \vc{r} = i k_m G(\vc{k}),
\end{equation*}
а значит
\begin{equation*}
    (-k^2 - \kappa^2) G(\vc{k}) = 1,
    \hspace{0.5cm} \Rightarrow \hspace{0.5cm}
    G(\vc{k})=- \frac{1}{k^2 + \kappa^2},
    \hspace{0.5cm} \Rightarrow \hspace{0.5cm}
    G(\vc{r}) = \int_{\mathbb{R}^3} \frac{e^{i \smallvc{k} \cdot \smallvc{r}}}{k^2 + \kappa^2} \frac{\d \vc{k}}{(2 \pi)^3}.
\end{equation*}
Переходим в сферические координаты, получаем, что
\begin{equation*}
    G(\vc{r}) = - \frac{2 \pi}{(2 \pi)^3} \int_{0}^{\infty} \frac{k^2}{k^2 + \kappa^2} \int_{0}^{\pi} 
    \sin \theta  e^{i k r \cos \theta}  \d \theta \d k = 
    - \frac{e^{- \kappa r}}{4 \pi r}
    .
\end{equation*}
Устремляя $\kappa \to 0$, находим
\begin{equation*}
     \nabla^2 G = \delta(\vc{r}),
     \hspace{0.5cm} \Rightarrow \hspace{0.5cm}
     G = - \frac{1}{4 \pi r}.
 \end{equation*} 





% \textbf{Пример}. Пусть 

\subsection{Многомерие \texorpdfstring{$\mathbb{R}^2$}{R2}}

Для Гаусса можно найти, что
\begin{equation*}
    G^{[\dim = n]}(x) = \frac{1}{\sigma_{n-1}} \frac{1}{r^{n-2}},
\end{equation*}
где $\sigma_{n-1}$ -- площадь $n-1$ мерной сферы. 



Вообще часто задача формулируется в виде задачи Дирихле:
\begin{equation*}
    \nabla^2 f = 0, 
    \hspace{5 mm} 
    f'_{\partial D} = f_0 (\vc{r}),
\end{equation*}
то есть функция задана на границе некоторой области. Пусть
\begin{equation*}
    f(z) = u(z) = i v(z),
    \hspace{5 mm} 
    \nabla^2 u = \nabla^2 v = 0.
\end{equation*}

Пусть знаем комплексную функцию $f(z)$ такую, что $\Re f |_{\partial D} = f_0$, тогда $\Re f(z)$ решает задачу Дирихле.
Далее конформным преобразованием переводим любое $D$ в круг, в круге задача Дирихле решается, а дальше отображаем назад. 


Пусть задана функция $u_0 (x) = u(x, 0)$. Вообще можно было бы разложить по Фурье $u$, и записать
\begin{equation*}
    \nabla^2  u = 0,
    \hspace{5 mm}   
    u(x, 0) = u_0 (x).
\end{equation*}
Тогда
\begin{equation*}
    u(q, y)  = \int_{\mathbb{R}} e^{- i q x} u(x, y) \d x,
    \hspace{0.5cm} \Rightarrow \hspace{0.5cm}
    \nabla^2 u = - q^2 u(q, y) = 0.
\end{equation*}
Так приходим к
\begin{equation*}
    u(q, y) = \exp\left(- |q| y\right) u(q, 0),
    \hspace{0.5cm} \Rightarrow \hspace{0.5cm}   
    u(x, y) = \int_{\mathbb{R}} \frac{\d q}{2\pi} e^{i q x} \underbrace{e^{- |q| y}}_{h(q)} u(q, 0).
\end{equation*}
Произведение Фурье образов -- свёртка:
\begin{equation*}
    u(x, y) = \int_{\mathbb{R}} \d \xi h(x - \xi,\,  y) u_0 (\xi).
\end{equation*}
Найдём, что
\begin{equation*}
    \int_{\mathbb{R}} \frac{\d q}{2 \pi} e^{i q x} e^{-|q| y} = \frac{y}{\pi (x^2 + y^2)}.
\end{equation*}
Подставляем, и находим:
\begin{equation*}
    u(x, y) = \int_{\mathbb{R}} d \xi \, \frac{y/\pi}{(x-\xi)^2 + y^2} u_0 (\xi),
\end{equation*}
где 
\begin{equation*}
    \frac{y/\pi}{(x-\xi)^2 + y^2} = \Im \frac{-1}{x + i y - \xi},
    \hspace{0.5cm} \Rightarrow \hspace{0.5cm}   
    f(z) = - \frac{1}{\pi} \int_{\mathbb{R}} \frac{1}{z-\xi} u_0 (\xi),
\end{equation*}
что в некотором смысле привело нас к интегралу Коши, так что и $\nabla^2 f = 0$ и гран. условия удовлетворяются. 



\textbf{Пример}. Рассмотрим
\begin{equation*}
    u_0 (x) = \frac{1}{1 + x^2},
    \hspace{5 mm} 
    u(x, y) = \int_{\mathbb{R}} d \xi \frac{1}{\pi} \frac{y}{y^2 + (x-\xi)^2} \frac{1}{1+\xi^2} = \frac{1 + y}{x^2 + (1+y)^2}.
\end{equation*}

