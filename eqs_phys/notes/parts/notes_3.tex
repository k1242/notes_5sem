% Лаврентьев Шаббат -- более строгий рассказ.

\section{Семинар от 25.09.21 (Фурье и Лаплас)}

% трансляционно инвариантна система
\textbf{Про Фурье}. 
Как раньше нашли
\begin{equation*}
    L(\partial_t) G(t) = \delta(t),
    \hspace{0.5cm} \Rightarrow \hspace{0.5cm}
    \hat{x} (\omega) = \int_{\mathbb{R}} e^{- i \omega t} x(t) \d t,
    \hspace{5 mm} 
    x(t) = \int_{\mathbb{R}} e^{i \omega t} \hat{x}(\omega) \frac{\d \omega}{2\pi}.
\end{equation*}
Для этого должно выполняться
\begin{equation*}
    \int |x(t)| \d d < + \infty.
\end{equation*}
\textit{Например}, для $\partial_t + \gamma$:
\begin{equation*}
    (\partial_t + \gamma) G(t) = \delta(t),
    \hspace{0.5cm} \Rightarrow \hspace{0.5cm}
    \int_{\mathbb{R}} \frac{\d t}{\ldots} e^{- i \omega t} \d t = x(t) e^{- i \omega t} \bigg|_{-\infty}^{+\infty},
    \hspace{0.5cm} \Rightarrow \hspace{0.5cm}
    (i \omega + \gamma) \hat{G} (\omega) = 1,
    \hspace{0.5cm} \Rightarrow \hspace{0.5cm}
    \hat{G}(\omega) = \frac{1}{i \omega + \gamma}.
\end{equation*}
Так приходим к уравнению
\begin{equation*}
    G(t) = \int_{\mathbb{R}} \frac{e^{i \omega t}}{\omega - i \gamma} \frac{\d \omega}{2\pi} = 
    \left\{\begin{aligned}
        &e^{- \gamma t}, &t > 0
        &0, &t<0
    \end{aligned}\right.
    \hspace{0.5cm} \Rightarrow \hspace{0.5cm}
    \hat{G}(\omega) = \theta(t) e^{- |t|}.
\end{equation*}

Однако, при $\hat{L} = \partial_t - \gamma$ мы бы получили
\begin{equation*}
    G_A (t) = \theta(-t) e^{\gamma t}, 
\end{equation*}
хотя вообще должно быть (если посчитать через неопределенные коэффициенты)
\begin{equation*}
    G_R (t) = \theta(t) e^{\gamma t},
\end{equation*}
которая растёт.


В методе с Фурье будут получаться функции Грина затухающие, но, возможно, без причинности. 
В методе неопределенных коэффициентов исходим из причинности, но может быть рост $\sim e^{\gamma t}$. 


Кроме того, в Фурье всегда предполагается $x(t \to - \infty) = 0$ и $x(t \to + \infty) = 0$. Также может случиться
\begin{equation*}
    (\partial_t^2 + \omega_0^2) G(t) = \delta(t),
    \hspace{0.5cm} \Rightarrow \hspace{0.5cm}
    \hat{G}(\omega) = \frac{1}{\omega^2 - \omega_0^2},
\end{equation*}
с особенностями на вещественной оси, что можно решить, сместив полюса в $\mathbb{C}$.





\textbf{Свёртка}. Рассмотрим уравнение
\begin{equation*}
    L(\partial_t) x(t) = f(t),
    \hspace{5 mm} 
    L(\partial_t) G(t) = \delta(t).
\end{equation*}
Фурье переводит 
\begin{equation*}
    \int_{\mathbb{R}} \partial_x^n x(t) e^{- i \omega t} \d t = (i \omega)^n \hat{x} (\omega).
\end{equation*}
Тогда
\begin{equation*}
    L(i \omega) \hat{x} (\omega) = \hat{f}(\omega),
    \hspace{5 mm} 
    L(i \omega)  \hat{G} (\omega) = 1,
    \hspace{0.5cm} \Rightarrow \hspace{0.5cm}
    \hat{G}(\omega) = \frac{1}{L(i \omega)}.
\end{equation*}
Также нашли, что
\begin{equation*}
    \hat{x}(\omega) = \frac{\hat{f}(\omega)}{L(i \omega)} = \hat{f}(\omega) \hat{G}(\omega),
    \hspace{0.5cm} \Rightarrow \hspace{0.5cm}
    x(t) = \int_{-\infty}^{+\infty} G(t-s) f(s) \d s.
\end{equation*}



\textbf{Преобразование Лапласа}. Пусть есть некоторое преобразование
\begin{equation*}
    \tilde{f}(p) = \int_0^{\infty} e^{- p t} f(t) \d t,
\end{equation*}
где подразумевается, что $\Re p \geq 0$ и, вообще, в Фурье можно $p \in \mathbb{C}$. 

Пусть $p = i \omega$, где $\omega \in \mathbb{R}$. Тогда
\begin{equation*}
    \tilde{f} (i \omega) = \int_{\mathbb{R}} e^{- i \omega t} f(t) \d t = \hat{f} (\omega),
    \hspace{0.5cm} \Rightarrow \hspace{0.5cm}
    f(t)  =\int_{\mathbb{R}} \hat{f}(\omega) e^{i \omega t} \frac{\d \omega}{2\pi} = 
    \int_{\mathbb{R}} \tilde{f} (i \omega) e^{i \omega t} \frac{\d \omega}{2 \pi} = 
    \int_{- i \infty}^{i \infty} e^{p t} \tilde{f} (p) \frac{\d p}{2\pi}.
\end{equation*}
В вычислениях выше мы предполагали, что $f(t \to \infty) = 0$. 

Обойдём это, пусть $|f(t)| < M e^{s t}$, при $s > 0$. Возьмём $p_0 > s$, тогда
\begin{equation*}
    \tilde{f} (p) = \int_{\mathbb{R}} e^{- p_0 t} e^{-(p-p_0)t} f(t) \d t = \tilde{g}(p-p_0),
\end{equation*}
где вводе $g(t) = e^{- i p_0 t} f(t)$, которая уже убывает на бесконечности. Обратно:
\begin{equation*}
    g(t) = \int_{-i\infty}^{+i\infty}  \tilde{g}(p) e^{pt} \frac{\d p}{2\pi} = 
    \int_{p_0 - i \omega}^{p_0 + i \omega} \tilde{g}(p-p_0) e^{- p_0 t} e^{pt} \frac{\d p}{2\pi i}.
\end{equation*}
Так пришли к форме обращения
\begin{equation}
    f(t) = \int_{p_0 - i \infty}^{p_0 + i \omega} \tilde{f}(p) \frac{\d p}{2 \pi i},
    \hspace{10 mm} 
    \tilde{f}(p) = \int_{\mathbb{R}} e^{- p_0  t} e^{-(p-p_0)t} f(t) \d t = \tilde{g}(p-p_0),
\end{equation}
где $g(t) = e^{- i p_0 t} f(t)$.



Забавный факт, из леммы Жордана: при $t < 0$ $f(t<0) = 9$, по замыканию дуги по часовой стрелке (вправо). Выбирая $p_0$ так, чтобы все особенности лежали левее $p_0$, можем получать причинные функции. 




\textbf{Производная}. Найдём преобразование Лапласа для $\partial_t f(t)$:
\begin{equation*}
    \int_{0}^{\infty} \frac{d f}{d t} e^{- p t} \d t = f e^{- pt} \bigg|_0^{\infty} + p \int_{0}^{\infty} 
    f(t) e^{-pt} \d t = p \tilde{f}(p) - f(+0).
\end{equation*}
Но, для функции Грина $L(\partial_t) G(t) = \delta(t)$, тогда
\begin{equation*}
    L(\partial_t) G_\varepsilon(t) = \delta(t-\varepsilon),
    \hspace{10 mm} 
    G_\varepsilon (t) = G(t-\varepsilon),
    \hspace{0.5cm} \Rightarrow \hspace{0.5cm}
    G_\varepsilon(0)=  0,
\end{equation*}
где $G_\varepsilon \to G(t)$ при $\varepsilon \to 0$.  

Преобразуем\footnote{
    Здесь и далее $f(t)$ -- функция, $f(\omega) = \hat{f}(\omega)$ -- Фурье образ, $f(p) = \tilde{f}(p)$ -- преобразование Лапласа.
}  по Лапласу уравнения выше
\begin{equation*}
     L(p) G(p) = e^{p \varepsilon} = 1,
     \hspace{0.5cm} \Rightarrow \hspace{0.5cm}
     G_\varepsilon (p) = \frac{1}{L(p)},
     \hspace{0.5cm} \overset{\varepsilon \to 0}{\Rightarrow}  \hspace{0.5cm}
     G(p) = \frac{1}{L(p)}.
 \end{equation*} 
Так получаем
\begin{equation}
    G(t) = \int_{p_0 - i \infty}^{p_0 + i \omega} \frac{e^{pt}}{\tilde{L}(p)} \frac{\d p}{2\pi i},
\end{equation}
где $p_0$ правее всех особенностей. 




\textbf{Пример}. Рассмотрим $L = \partial_t + \gamma$, тогда
\begin{equation*}
    (p+\gamma) G(p) = 1,
    \hspace{0.5cm} \Rightarrow \hspace{0.5cm}
    G(p) = \frac{1}{p + \gamma},
    \hspace{0.5cm} \Rightarrow \hspace{0.5cm}
    G(t) = \int_{-i \infty}^{i \infty} \frac{e^{pt}}{p+\gamma} \frac{\d p}{2 \pi i} = \theta(t) e^{- \gamma t}.
\end{equation*}
Аналогично, пусть $L = \partial_t^2 + \omega^2$, тогда $L G(t) = \delta(t)$, и
\begin{equation*}
    G(p) = \frac{1}{p^2 + \omega^2},
    \hspace{0.5cm} \Rightarrow \hspace{0.5cm}
    G(t) = \int_{p_0 - i \infty}^{p_0 + i \omega} \frac{e^{pt}}{p^2 + \omega^2} \frac{\d p}{2 \pi i} = 
    \left\{\begin{aligned}
        &0, &t < 0 \\
        &\ldots, &t>0
    \end{aligned}\right. 
    = \theta(t) \left(
        \frac{e^{i \omega t}}{2 i \omega} + \frac{e^{- i \omega t}}{- 2 i \omega} 
    \right)
        = \theta(t) \frac{\sin \omega t}{\omega}.
\end{equation*}

В общем виде, пусть $L(\partial_t) G(t) = \delta(t)$, тогда
\begin{equation*}
    L(p) G(p) = 1,
    \hspace{0.5cm} \Rightarrow \hspace{0.5cm}
    G(p) \frac{1}{L(p)},
    \hspace{0.5cm} \Rightarrow \hspace{0.5cm}
    G(t) = \int_{p_0 - i \infty}^{p_0 + i \omega} \frac{e^{p t}}{L(p)} \frac{\d p}{2 \pi i}.
\end{equation*}

Поговорим про свёртку:
\begin{equation*}
    L x  = f,
    \hspace{0.5cm} \Rightarrow \hspace{0.5cm}
    L(p) x(p) = f(p),
    \hspace{5 mm} 
    L(p) G(p)  =1,
    \hspace{0.5cm} \Rightarrow \hspace{0.5cm}
    G(p) = \frac{1}{L(p)}.
\end{equation*}
Тогда получается
\begin{equation*}
    x(p) = \frac{f(p)}{L(p)} = f(p) G(p),
    \hspace{0.5cm} \Rightarrow \hspace{0.5cm}
    x(t) = \int_{0}^{t} G(t-s) f(s) \d s.
\end{equation*}


\textbf{Уравнение Вольтера}. Иногда бывает уравнения на $x(s)$ вида
\begin{equation}
    f(t) = \int_{0}^{t} x(s) K(t-s) \d s.
\end{equation}
Через преобразрвание Лапласа, находим
\begin{equation}
    f(p) = x(p) K(p),
    \hspace{0.5cm} \Rightarrow \hspace{0.5cm}
    x(p) = \frac{f(p)}{K(p)}. 
\end{equation}
В общем виде тогда находим
\begin{equation*}
    x(t) = \int_{p_0 - i \infty}^{p_0 + i \omega} \frac{f(p)}{K(p)} e^{pt} \frac{\d p}{2\pi i}.
\end{equation*}
Кстати, забавный факт:
\begin{equation}
    \int_{p_0 - i \infty}^{p_0 + i \omega} 1 \cdot e^{pt} \frac{\d p}{2 \pi i} = e^{p_0 t} 
    \int_{-i \infty}^{i \infty} e^{p t} \frac{\d p }{2 \pi i} = 
    e^{p_0 t} \int_{-\infty}^{+\infty} e^{i \omega t} \frac{\d \omega}{2 \pi} = \delta(t),
\end{equation}
то есть преобразование Лапласа от константы -- дельта функция. 



Рассмотрим, например
\begin{equation*}
    \int_{-i \infty}^{i \infty} \frac{p + 1 -1}{p+1} e^{pt} \frac{\d p}{2 \pi i} = \int_{- i \infty}^{i \infty}  e^{p t} \frac{\d p}{2 \pi i} - \int_{-i\infty}^{+i\infty}  \frac{e^{pt}}{p+1} \frac{\d p}{2 \pi} = \delta(t) - \theta(t) e^{- t}.
\end{equation*}
Также верно, что
\begin{equation*}
    \int_{- i \infty}^{i \infty} p e^{pt} \frac{\d p}{2 \pi i} = \delta'(t).
\end{equation*}
Действительно,
\begin{equation*}
    \frac{d }{d t} \left(
        \int_{- i \infty}^{i \infty}
         e^{p t} \frac{\d p}{2 \pi i}
    \right) = \frac{\d}{\d t} \delta(t) = \delta'(t).
\end{equation*}

Важно, что можно делать функции маленькими
\begin{equation}
    \int_{p_0 - i \infty}^{p_0 + i \omega} f(p) e^{pt} \frac{\d p}{2 \pi i} = 
    \left(\frac{d }{d t} \right)^n \int_{p_0 - i \infty}^{p_0 + i \omega} \frac{f(p)}{p^n} e^{pt} \frac{\d p}{2 \pi i}.
\end{equation}



\textbf{Неоднородная релаксация}. 
Рассмотрим уравнение
\begin{equation*}
    (\partial_t + \gamma(t)) G(t,s) = \delta(t-s),
    \hspace{10 mm} 
    x(t) = \int_{-\infty}^{+\infty} G(t, s) f(s) \d s,
\end{equation*}
где продолжаем требовать причинность $G(t,s>t) = 0$. Для начала, рассмотрим $t>s$, тогда
\begin{equation*}
    (\partial_t + \gamma(t)) G(t) = 0,
    \hspace{0.5cm} \Rightarrow \hspace{0.5cm}
    \frac{d G}{G}  = - \gamma(t) \d t,
    \hspace{0.5cm} \Rightarrow \hspace{0.5cm}
    G(t, s) = A(s) \exp\left(
        - \int_{t_0}^{t} \gamma(t') \d t'
    \right).
\end{equation*}
Также записываем граничные условия:
\begin{equation*}
    \int_{s-\varepsilon}^{s+\varepsilon} \ldots \d s,
    \hspace{0.5cm} \Rightarrow \hspace{0.5cm}
    G(s+0, s) = 1.
\end{equation*}
Так можем найти
\begin{equation}
    A(s) \exp\left(
        - \int_{t_0}^{s} \gamma(t') \d t'
    \right) = 1,
    \hspace{0.5cm} \Rightarrow \hspace{0.5cm}  
    G(t, s) = \theta(t-s) \exp\left(
        - \int_{s}^{t} \gamma(t') \d t'
    \right),
\end{equation}
где мы разбили
\begin{equation*}
    \int_{t_0}^t = \int_{t_0}^{s}  + \int_{s}^{t},
\end{equation*}
и получили, что хотели.



\phantom{42}

\textit{Комментарий про дельта функцию}. Главное, нужно показать, что
\begin{equation*}
    \int_{-\infty}^{+\infty} \delta_a (x) = 1,
    \hspace{10 mm}  
    \lim_{a \to 0} \delta_a (x) =0, \text{ при } x \neq 0.
\end{equation*}
Вообще можем плодить дельтаобразные последовательности, взяв $f$ с единичным интегралом и 
\begin{equation*}
    \delta_a (x) = \frac{1}{a} f\left(\frac{x}{a}\right).
\end{equation*}



\textit{Комментарий про преобразование Лапласа}. Для функции вида
\begin{equation*}
    \frac{1}{\sqrt{p+\alpha}},
\end{equation*}
необходим аппарат разрезов, так что её можно сделать с шифтом на неделю. 

На следующей недели будет контрольная. Необходим аппарат метода неопределенных коэффициентов, матричные экспоненты, решение диффуров через Фурье (не всегда причинный результат), а также преобразование Лапласа. Вычеты скорее всего в районе второго порядка и меньше.  Ещё полезно вспонить, как записывать начальные условия: осцияллятор, осциллятор с затуханием. 




