% document's head

\begin{center}
    \LARGE \textsc{Задание по курсу \\  <<Уравнения математической физики>>}
\end{center}

\hrule

\phantom{42}

\begin{flushright}
    \begin{tabular}{rr}
    % written by:
        % \textbf{Источник}: 
        % & \href{__ссылка__}{__название__} \\
        % & \\
        % \textbf{Лектор}: 
        % & _ФИО_ \\
        % & \\
        \textbf{Автор}: 
        & Хоружий Кирилл \\
        % & Примак Евгений \\
        & \\
    % date:
        \textbf{От}: &
        \textit{\today}\\
    \end{tabular}
\end{flushright}

\thispagestyle{empty}
% \tableofcontents
% \newpage



% дописать в заголовочный файл перечекнутую const


% \section*{Программа ГОСа по физике}

\subsection*{Механика}
\begin{enumerate*}
\item Законы Ньютона. Движение тел в инерциальных и неинерциальных системах отсчета.
\item Принцип относительности Галилея и принцип относительности Эйнштейна. Преобразования Лоренца. Инвариантность интервала.
\item Законы сохранения энергии и импульса в классической механике. Упругие и неупругие столкновения.
\item Уравнение движения релятивистской частицы под действием внешней силы. Импульс и энергия
релятивистской частицы.
\item Закон всемирного тяготения и законы Кеплера. Движение тел в поле тяготения.
\item Закон сохранения момента импульса. Уравнение моментов. Вращение твердого тела вокруг неподвижной оси. Гироскопы.
\end{enumerate*}


\subsection*{МСС}
\begin{enumerate*}
\setcounter{enumi}{6}
\item Течение идеальной жидкости. Уравнение непрерывности. Уравнение Бернулли.
\item Закон вязкого течения жидкости. Формула Пуазейля. Число Рейнольдса, его физический смысл.
\item Упругие деформации. Модуль Юнга и коэффициент Пуассона. Энергия упругой деформации.

\end{enumerate*}


\subsection*{Термодинамика}
\begin{enumerate*}
\setcounter{enumi}{9}
\item Уравнение состояния идеального газа, его объяснение на основе молекулярно-кинетической теории. Уравнение неидеального газа Ван-дер-Ваальса.
\item Квазистатические процессы. Первое начало термодинамики. Количество теплоты и работа. Внутренняя энергия. Энтальпия.
\item Второе начало термодинамики. Цикл Карно. Энтропия и закон ее возрастания. Энтропия идеального газа. Статистический смысл энтропии.
\item Термодинамические потенциалы. Условия равновесия термодинамических систем.
\item Распределения Максвелла и Больцмана.
\item Теплоемкость. Закон равномерного распределения энергии по степеням свободы. Зависимость теплоемкости газов от температуры.
\item Фазовые переходы. Уравнение Клапейрона-Клаузиуса. Диаграммы состояний.
\item Явления переноса: диффузия, теплопроводность, вязкость. Коэффициенты переноса в газах. Уравнение стационарной теплопроводности.
\item Поверхностное натяжение. Формула Лапласа. Свободная энергия и внутренняя энергия поверхности.
\item Флуктуации в термодинамических системах.
\item Броуновское движение, закон Эйнштейна-Смолуховского. Связь диффузии и подвижности (соотношение Эйнштейна).
\end{enumerate*}


\subsection*{Электричество и теория поля}
\begin{enumerate*}
\setcounter{enumi}{20}
\item Закон Кулона. Теорема Гаусса в дифференциальной и интегральной формах. Теорема о циркуляции для статического электрического поля. Потенциал. Уравнение Пуассона.
\item Электростатическое поле в веществе. Вектор поляризации, электрическая индукция. Граничные
условия для векторов $\vc{E}$ и $\vc{D}$.
\item Магнитное поле постоянных токов в вакууме. Основные уравнения магнитостатики в вакууме. Закон Био-Савара. Сила Ампера. Сила Лоренца.
\item Магнитное поле в веществе. Основные уравнения магнитостатики в веществе. Граничные условия
для векторов $\vc{B}$ и $\vc{H}$.
\item Закон Ома в цепи постоянного тока. Переходные процессы в электрических цепях.
\item Электромагнитная индукция в движущихся и неподвижных проводниках. ЭДС индукции. Само- и
взаимоиндукция. Теорема взаимности.
\item Система уравнений Максвелла в интегральной и дифференциальной формах. Ток смещения. Материальные уравнения.
\item Закон сохранения энергии для электромагнитного поля. Вектор Пойнтинга. Импульс электромагнитного поля.
\item Квазистационарные токи. Свободные и вынужденные колебания в электрических цепях. Явление
резонанса. Добротность колебательного контура, ее энергетический смысл.
\item Спектральное разложение электрических сигналов. Спектры колебаний, модулированных по амплитуде и фазе.
\item Электрические флуктуации. Дробовой и тепловой шумы. Предел чувствительности электроизмерительных приборов.
\item Электромагнитные волны. Волновое уравнение. Уравнение Гельмгольца.
\item Электромагнитные волны в волноводах. Критическая частота. Объемные резонаторы.
\item Плазма. Плазменная частота. Диэлектрическая проницаемость плазмы. Дебаевский радиус.
\end{enumerate*}


\subsection*{Оптика}
\begin{enumerate*}
\setcounter{enumi}{34}
\item Интерференция волн. Временная и пространственная когерентность. Соотношение неопределенностей.
\item Принцип Гюйгенса-Френеля. Зоны Френеля. Дифракция Френеля и Фраунгофера. Границы применимости геометрической оптики.
\item Спектральные приборы (призма, дифракционная решетка, интерферометр Фабри-Перо) и их основные характеристики.
\item Дифракционный предел разрешения оптических и спектральных приборов. Критерий Рэлея.
\item Пространственное фурье-преобразование в оптике. Дифракция на синусоидальных решетках. Теория Аббе формирования изображения.
\item Принципы голографии. Голограмма Габора. Голограмма с наклонным опорным пучком. Объемные
голограммы.
\item Волновой пакет. Фазовая и групповая скорости. Формула Рэлея. Классическая теория дисперсии.
Нормальная и аномальная дисперсии.
\item Поляризация света. Угол Брюстера. Оптические явления в одноосных кристаллах.
\item Дифракция рентгеновских лучей. Формула Брэгга-Вульфа. Показатель преломления вещества для
рентгеновских лучей.
\end{enumerate*}


\subsection*{Квантовая механика}
\begin{enumerate*}
\setcounter{enumi}{43}
\item Квантовая природа света. Внешний фотоэффект. Уравнение Эйнштейна. Эффект Комптона.
\item Спонтанное и вынужденное излучение. Инверсная заселенность уровней. Принцип работы лазера.
\item Излучение абсолютно черного тела. Формула Планка, законы Вина и Стефана-Больцмана.
\item Корпускулярно-волновой дуализм. Волны де Бройля. Опыты Девиссона-Джермера и Томсона по
дифракции электронов.
\item Волновая функция. Операторы координаты и импульса. Средние значения физических величин.
Соотношение неопределенности для координаты и импульса. Уравнение Шредингера.
\item Строение водородоподобного атома. Уровни энергии и кратность их вырождения. Спектр излучения атома водорода.
\item Опыты Штерна и Герлаха. Спин электрона. Орбитальный и спиновый магнитные моменты электрона.
\item Тождественность частиц. Симметрия волновой функции относительно перестановки частиц. Бозоны и фермионы. Принцип Паули. Электронная структура атомов. Таблица Менделеева.
\item Тонкая и сверхтонкая структуры оптических спектров. Правила отбора при поглощении и испускании фотонов атомами.
\item Эффект Зеемана в слабых магнитных полях.
\item Эффект Зеемана в сильных магнитных полях.
\item Ядерный и электронный магнитный резонансы.
\item Виды распадов. Закон радиоактивного распада. Период полураспада и время жизни.
\item Туннелирование частиц сквозь потенциальный барьер. Альфа-распад. Закон Гейгера-Нэттола и его
объяснение.
\item Виды бета-распадов. Объяснение непрерывности энергетического спектра электронов распада.
Нейтрино.
\item Ядерные реакции. Составное ядро. Сечение нерезонансных реакций. Закон Бете.
\item Резонансные ядерные реакции, формула Брейта-Вигнера. Упругий и неупругие каналы реакции.
\item Деление ядер под действием нейтронов. Принцип работы ядерного реактора на тепловых нейтронах.
\item Соотношение неопределенностей для энергии и времени. Оценка времени жизни виртуальных частиц, радиусов сильного и слабого взаимодействий.
\item Фундаментальные взаимодействия и фундаментальные частицы (лептоны, кварки и переносчики
взаимодействий). Кварковая структура адронов
\end{enumerate*}
% \section*{ТеорМин}



\textbf{Вычеты}. Интеграл по дуге может быть найден, как
\begin{align*}
    \int_C f(z) \d z = 2 \pi i \sum_{z_j} \res_{z_j} f(z),
    \hspace{5 mm} 
    \res_{z_j} f(z) &= \lim_{\varepsilon \to 0} \varepsilon \int_0^{2\pi} \frac{\d \varphi}{2\pi} e^{i \varphi} f(z_j + \varepsilon e^{i \varphi}) \\ 
    &= \frac{1}{(m-1)!} \lim_{z \to z_j} \left(
        \frac{d^{m-1} }{d z^{m-1}} (z-z_j)^m f(z)
    \right),
\end{align*}
где $m$ -- степень полюса. 



\begin{to_lem}[лемма Жордана]
    Пусть $f(z)$ непрерывна в замкнутой области $G = \{z \mid \Im z \geq 0,\,  |z| \geq R_0 > 0\}$. Обозначим через $C_R$ полуокружность $|z| = R,\, \Im x \geq 0$ и пусть верно, что $\lim_{R \to \infty} \max |f(z)| =0$. тогда при $a > 0$
    \begin{equation*}
        \lim_{R \to \infty} \int_{C_R} f(z) e^{i a z} \d z = 0,
    \end{equation*}
    аналогичное верно при $C_R$ с $\Im x \leq 0$ и $a < 0$. 
\end{to_lem}



\textbf{Функция Грина}. Всегда и всюду, уравнение вида
\begin{equation*}
    L x(t) = \varphi(t), 
    \hspace{5 mm}   
    x(t) = \int_{-\infty}^{t} G(t-s) \varphi(s) \d s,
    \hspace{5 mm} 
    L G = \delta(t).
\end{equation*}
И, если хочется добавить начальные условия, то например, для $L = \partial_t^2$ будет
\begin{equation*}
    x(t) = \dot{x}(0) G(t) + x(0) \dot{G}(t) + \int_{0}^{t} G(t-s) \varphi(s) \d s.
\end{equation*}




\textbf{Матричное уравнение}. Решение линейного уравнения для векторной величины $\vc{y}$
\begin{equation*}
    \frac{d \vc{y}}{d t} + \hat{\Gamma} \vc{y} = \vc{\chi},
\end{equation*}
может быть найдено, через функцию Грина, вида
\begin{equation*}
    \hat{G} (t) = \theta(t) \exp\left(- \hat{\Gamma} t\right),
    \hspace{10 mm} 
    \vc{y}(t) = \int_{-\infty}^{t}  \hat{G}(t-s) \vc{\chi}(s) \d s.
\end{equation*}


% \textbf{Преобразование Фурье}. 



\textbf{Преобразование Лапласа}. Преобразование Лапласа функциии $\Phi(t)$ определяется, как
\begin{equation*}
    \tilde{\Phi}(p) = \int_{0}^{\infty}  \exp(-pt) \Phi(t) \d t,
    \hspace{10 mm} 
    \Phi(t) = \int_{c-i \infty}^{c+i \infty} \frac{\d p}{2 \pi i} \exp(pt) \tilde{\Phi}(p),
\end{equation*}
где далее $c$ выбираем правее всех особенностей для причинности. 

Решение уравнения $L(\partial_t) G(t) = \delta(t)$ может быть найдено, как
\begin{equation*}
    G(t) = \int_{c-i \infty}^{c+i \infty} \frac{\d p}{2 \pi i} \exp(p t) \tilde{G}(p),
    \hspace{10 mm} \tilde{G} (p) = \frac{1}{L(p)},
    \hspace{0.5cm} \Rightarrow \hspace{0.5cm}
    G(t) = \sum_i \res_i \frac{\exp(pt)}{L(p)},
\end{equation*}
где суммирование идёт по полюсам $1/L(p)$. 

Важно, что можно делать функции маленькими
\begin{equation}
    \int_{p_0 - i \infty}^{p_0 + i \omega} \tilde{f}(p) e^{pt} \frac{\d p}{2 \pi i} = 
    \left(\frac{d }{d t} \right)^n \int_{p_0 - i \infty}^{p_0 + i \omega} \frac{\tilde{f}(p)}{p^n} e^{pt} \frac{\d p}{2 \pi i}.
\end{equation}




\textbf{Уравнение Вольтерра}. Интегральное уравнение Вольтерра первого рода с однородным ядром:
\begin{equation*}
    \int_{0}^{t}  K(t-s) f(s) \d s = \varphi(t).
\end{equation*}
Решение может быть найдено через обратное преобразование Лапласа
\begin{equation*}
    f(t) = \int_{c-i \infty}^{c+i \infty} \frac{d p}{2 \pi i} \exp(pt) \tilde{f}(p),
    \hspace{10 mm} 
    \tilde{f}(p) = \frac{\tilde{\varphi}(p)}{\tilde{K}(p)}.
\end{equation*}
Но есть один нюанс. При $K(t),\, \varphi(t) \overset{p \to \infty}{\to} K_0,\, \varphi_0$ получается, что $\tilde{K}(p),\, \tilde{\varphi}(p) \approx \frac{K_0}{p},\, \frac{\varphi_0}{p}$, тогда
\begin{equation*}
    f(t) = \frac{\varphi_0}{K_0} \delta(t) + \int_{c-i \infty}^{c+i \infty} \frac{d p}{2 \pi i} \exp(p t)
    \left(
        \frac{\tilde{\varphi}}{\tilde{K}} - \frac{\varphi_0}{K_0}
    \right),
\end{equation*}
при этом в отсутствие аналитичности в нуле нет ничего страшного. 


\textbf{Неоднородная релаксация}. Для одномерного случая
\begin{equation*}
    \big(\partial_t + \gamma(t)\big) x(t) = \varphi(t),
    \hspace{0.5cm} \Rightarrow \hspace{0.5cm}
    x(t) = \int_{-\infty}^{+\infty}  G(t,s) \varphi(s) \d s,
    \hspace{5 mm} 
    G(t,\,  s) = \theta(t-s) \exp\left(
        - \int_{s}^{t} \gamma(\tau) \d \tau
    \right),
\end{equation*}
где всё также $G(t, s>t) = 0$ в силу стремления к принципу причинности. 


% многомерная неоднородная релаксация


% % пылинка, что несет баланс

Введем лучистую энергию, раскладывая по частотам или длинам волн:
\begin{equation*}
    u = \int_0^\infty u_\omega \d \omega = \int_{0}^{\infty} u_\lambda \d \lambda,
\end{equation*}
где $u_\lambda$ и $u_\omega$ -- спектральные плотности лучистой энергии. При этом
\begin{equation*}
    \lambda = \frac{2 \pi c}{\omega},
    \hspace{5 mm} 
    \frac{d \lambda}{\lambda} = - \frac{d \omega}{\omega},
    \hspace{10 mm} 
    u_\lambda = \frac{\omega}{\lambda} u_\omega,
    \hspace{5 mm} 
    u_\omega = \frac{\lambda}{\omega}  u_\lambda.
\end{equation*}
В теорфизе обычно $u_\omega$, в эксперименте чаще $u_\lambda$ (как удобно). 

Поток лучистой энергии, проходящий за время $\d t$ через площадку $\d s$ в пределах телесного угла $\d \Omega$, ось которого перпендикулярна к площадке $\d s$, можно представить, как
\begin{equation*}
    \d \Phi = I \d s \d \Omega \d t,
    \hspace{10 mm} 
    I = \int_{0}^{\infty} I_\omega \d \omega,
\end{equation*}
где $I$ -- удельная интенсивность излучения, а $I_\omega$ -- удельная интенсивность излучения частоты $\omega$. 


Для равновесного излучения несложно выписать связь:
\begin{equation*}
    u = \frac{4 \pi}{c} I,
    \hspace{5 mm} 
    u_\omega = \frac{4 \pi}{c} I_\omega. 
\end{equation*}



\textbf{Закон Кирхгофа}. Для непрозрачного и поглощающего тела верно, что поток лучистой энергии, излучаемый площадкой $\d s$ поверхности тела внутрь телесного угла $\d \Omega$:
\begin{equation*}
    \d \Phi = E_\omega \d s \cos \varphi \d \Omega \d \omega \d t,
\end{equation*}
где $\varphi$ -- угол между направлением излучения и нормалью к площадке $\d s$. Велична $E_\omega$ -- \textit{излучаетльная способность} поверхности тела, в направлении угла $\varphi$. 


\textit{Поглощательной способностью} $A_\omega$ поверхности для излучения той же частоты, называется величина, показывающая, какая доля энергии падающего излучения, поглощается рассматриваемой поверхностью. Величины $E_\omega$ и $A_\omega$ -- характеристики тела, определяемые только температурой. 


Рассмотрев тело в ящике, можем получить, что
\begin{equation*}
    \frac{E_\omega}{A_\omega} =  I_\omega,
\end{equation*}
таким обрахом $\frac{E_\omega}{A_\omega}$ -- универсальная функция только частоты и температуры для каждого тела. 



\begin{to_def}
    \textit{Абсолютно черным} называется телос $A_\omega = 1 \ \forall \omega$.
\end{to_def}

Далее излучательную способность АЧТ примем за $e_\omega \equiv I_\omega$. Излучение АЧТ изотропно, а значит подчиняется \textit{закону Ламберта}:
\begin{equation*}
    \frac{\d \Phi}{\d \Omega \d s \cos \theta} = B_\theta = \const(\theta).
\end{equation*}


\textbf{Закон Стефана-Больцмана}. Выведем этот закон, \textit{методом циклов}. Пусть есть некоторая оболочка, при увеличении объема на $\d V$ за счёт давления света совершается работа $\mathcal P \d V$, где $\mathcal P = \frac{1}{3} u$,  а $u$ --интегральная плотность лучистой энергии. Внутренняя энергия излучения в оболочке  $uV$, откуда находим
\begin{equation*}
    \mathcal P \d V = - d(uV),
    \hspace{0.5cm} \Rightarrow \hspace{0.5cm}
    \frac{4}{3} u \d V + V \d u = 0,
    \hspace{0.5cm} \Rightarrow \hspace{0.5cm}
    u V^{4/3} = \const, \hspace{5 mm} \mathcal P V^{4/3} = \const,
\end{equation*}
так получили \textit{уравнения адиабаты} для изотропного изучения, с постоянной адиабаты $\gamma = 4/3$. 



В силу эффекта Дполера, при адиабатическом сжатии должен меняться спектральный состав, пусть $\omega \to \omega'$, при этом:
\begin{equation*}
    u_\omega \d \omega \cdot V^{4/3} = u'_{\omega'} \d \omega \cdot V'{}^{4/3} = \const,
\end{equation*}
где $V'$ и $u'_{\omega'}$ -- объем и спектральная плотность энергии излучения частоты $\omega'$ в конце процесса. 


Произведем теперь над излучением АЧТ \textit{цикл Карно} (см. Сивухин, т. IV, \S 115).  А можно этого и не делать, а подставить $U = V u(T)$ и $\mathcal P = \frac{1}{3} u(T)$ в формулу
\begin{equation*}
    \left(\frac{\partial U}{\partial V} \right)_T = T \left(\frac{\partial \mathcal P}{\partial T} \right)_V - \mathcal P,
    \hspace{0.5cm} \Rightarrow \hspace{0.5cm}
    u/T^4 = \const,
\end{equation*}
что и составляет закон Стефана-Больцмана. 


Пользуясь формулой Планка, можем уточнить, что
\begin{equation*}
    u = \frac{h}{\pi^2 c^3} \int_{0}^{\infty} \frac{\omega^3 \d \omega}{e^{\hbar \omega / k T} - 1} = 
    \frac{k^4 T^4}{\pi^2 c^3 \hbar^3} \int_{0}^{\infty} \frac{x^3 e^{-x}}{1 - e^{-x}} \d x = \frac{8}{15} \frac{\pi^5 k^4}{c^3 h^3} T^4 = \frac{\pi^2 k^4}{15 c^3 \hbar^3} T^4.
\end{equation*}

На практике удобнее говорить про энергетическую светимость $S$ для АЧТ, которая связана с яркостью $B$ излучающей поверхности соотношением $S = \pi B = \pi I = c u/4$, а значит
\begin{equation*}
    S = \sigma T^4,
    \hspace{10 mm} 
    \sigma = \frac{\pi^2 k^4}{60 c^2 /\hbar^3} = \frac{2 \pi^5 k^4}{15 c^2 h^3} = 5.670 \times 10^{-8} 
    \ \text{Вт}\cdot\text{м}^{-2}\cdot\text{К}^{-4},
\end{equation*}
где $\sigma$ -- \textit{постоянная Стефана-Больцмана}.



\textbf{Теорема Вина}. Рассмотрим сферически симметричную систему (вообще вроде можно показать что в общем случае изотропия излучения сохраняется), сожмем от $V_1$ до $V_2$, уравновесим (необратимый процесс), расщирим от $V_2$ до $V_1$, получим адиабатический \textit{обратимый} круговой процесс, что невозможно, а значит верна следующая теорема:

\begin{to_thr}[теорема Вина]
    Равновесное излучение, в оболочке с идеально отражающими стенками, остается равновесным при квазистатическом изменении объема системы.
\end{to_thr}


Рассмотрим сферическую оболочку с идеально зеркальными стенками. Рассмотрим луч, падающий под углом $\theta$, тогда время между думя последовательными отражениями равно $\Delta t = (2r/c) \cos \theta$, за это время радиум оболочки получит приращение $\Delta r = r\delta \Delta t$. При При каждом отражении происходит доплеровское изменение частоты:
\begin{equation*}
    \frac{\Delta \omega}{\omega} = - \frac{2 \dot{r} \cos \theta}{c} = - \frac{2 \Delta r \cos \theta}{c \Delta t} = - \frac{\Delta r}{r},
    \hspace{0.5cm} \Rightarrow \hspace{0.5cm}
    \frac{\d \omega}{\omega} + \frac{\d r}{r} = 0,
    \hspace{0.5cm} \Rightarrow \hspace{0.5cm}
    \omega r = \const.
\end{equation*}
Так как $r \sim V^{1/3}$, то можно записать в чуть более общем виде:
\begin{equation*}
    \omega^3 V = \const,
\end{equation*}
что объединяя с другми адиабатическими инвариантами и законом Стефана-Больцмана, находим \textit{закон смещения Вина} в наиболее общей форме:
\begin{equation*}
    \frac{\omega^4}{u} = \const,
    \hspace{5 mm} 
    \frac{\omega}{T} = \const,
    \hspace{5 mm} 
    \frac{u_\omega \d \omega}{\omega^4} = \const.
\end{equation*}
По теореме Вина излучение остается равновесным, так что можно было бы такж и нагревать/охлаждать стенки, да и вообще: полученные результаты -- свойства только самого равновесного излучения, не связанные с процессами.


\textbf{Максимумы спектральной плотности}. 
Их последней формулы можем получить\footnote{
    Интегрируя $u_\omega (\omega, T) = T^3 \cdot \varphi_1\left(\frac{\omega}{T}\right)$, находим, что
    $u = \int_{0}^{\infty} u_\omega \d \omega = T^4 \int_{0}^{\infty} \varphi(\omega/T) \d (\omega/T) = a T^4$.
}   
\begin{equation*}
    u_\omega (\omega, T) = \frac{\omega^4}{\omega'{}^4} \frac{d \omega'}{d \omega} u'_{\omega'} (\omega', T) = \frac{T^3}{T'{}^3} u'_{\omega'} \left(\frac{T'}{T} \omega,\, T'\right) = \const(T'),
    \hspace{0.5cm} \Rightarrow \hspace{0.5cm}  
    u_\omega (\omega, T) = T^3 \cdot \varphi_1\left(\frac{\omega}{T}\right) = \omega^3 f_1\left(\frac{\omega}{T}\right)
    ,
\end{equation*}
где $\varphi,\,  f$ -- универсальные функции. Аналогично можно переписать, в виде
\begin{equation*}
    u_\lambda = T^5 \varphi_2 (\lambda T),
    \hspace{10 mm}  
    u_\lambda = \frac{1}{\lambda^5} f_2(\lambda T).
\end{equation*}


Найдём теперь максимумы $u_\lambda$ обозначив, за $\sub{\lambda}{max}$:
\begin{equation*}
    \frac{d \varphi_2}{d \lambda} = T \frac{d \varphi_2}{d (\lambda T)} = 0,
    \hspace{0.5cm} \Rightarrow \hspace{0.5cm}
    \frac{d \varphi_2}{d (\lambda T)} = 0.
\end{equation*}
Таким образом, при всех температурах максимум получается при одном и том же значении $\lambda T$, а значит выполняется \textit{закон смещения Вина}:
\begin{align*}
    \sub{\lambda}{max} \cdot T &= b_\lambda = \const ,
    &b_\lambda &= 2.898 \times 10^{6} \ \,  \unm{нм}{К} \\
    \sub{\nu}{max} /T &= b_\nu = \const,
    &b_\nu &= 5.879 \times 10^{10} \, \unm{Гц}{К}.
\end{align*}

 

 Введем $\beta = h c / \lambda kT$, тогда задача сводится к отысканию минимума: 
 \begin{equation*}
     \frac{1}{\beta^5}(e^\beta - 1) \to \min,
     \hspace{0.5cm} \Rightarrow \hspace{0.5cm}
     e^{-\beta} + \frac{\beta}{5} - 1 =0,
     \hspace{0.5cm} \Rightarrow \hspace{0.5cm}
     \beta = 4.9651142,
     \hspace{1cm}
     b_\lambda = \sub{\lambda}{max} T = \frac{h c}{k \beta}.
 \end{equation*}


При поиске $\beta_\omega$ уравнение получится, вида
\begin{equation*}
    (3-\beta_\omega)e^{\beta_\omega} - 3 =0,
    \hspace{0.5cm} 
    \beta_\omega = \frac{\hbar \omega}{kT} = \frac{h c}{\lambda k T},
    \hspace{0.5cm} \Rightarrow \hspace{0.5cm}
    \beta_\omega = 2.821,
    \hspace{5 mm} 
    \sub{\lambda}{max}^{\text{по }\omega} = \frac{h c}{k \beta_\omega} \frac{1}{T}.
\end{equation*}
Стоит заметить, что $\sub{\lambda}{max}^{\text{по }\omega} / \sub{\lambda}{max} = \beta / \beta_\omega \approx 1.76$.





\textbf{Формула Планка}. \red{опускаем кусок вывода про стоячие волны}
\\
 Итак, считая, что на каждую стоячую волну приходится $\bar{\E} = kT$, то записав энергию равновесного излучения в полости в спектральном интервале $\d \omega$ в виде $V u_\omega \d \omega$, получаем:
 \begin{equation}
     u_\omega  = \frac{\omega^2}{\pi^2 c^3} \bar{\E} \overset{*}{=}  \frac{kT}{\pi^2 c^3} \omega^2,
     \label{rj0}
 \end{equation}
где равенство со звёздочкой -- формула Рэлея-Джинса, верная при малых $\omega$. 

Однако, считая, что существует минимальный квант энергии света, по теореме Больцмана, вероятности возбуждения энергетических уровней осциллятора пропорциональны
\begin{equation*}
    1,\, e^{-\E_0/kT}, e^{- 2 \E_0/kT}, \ldots,
    \hspace{0.5cm} \Rightarrow \hspace{0.5cm}
    \bar{\E} = \frac{\sum_{n=0}^{\infty}  n \E_0 e^{- n \E_0 / kT}}{\sum_{n=0}^{\infty} e^{-n \E_0 / kT}} = \E_0 \frac{\sum_{n=0}^{\infty} n e^{-n x}}{\sum_{n=0}^{\infty} e^{-nx}},
\end{equation*}
где введено обозначение $x = \E_0 / kT$. Вспоминая, что
\begin{equation*}
    \sum_{n=0}^{\infty} e^{-n x} = \frac{1}{1 - e^{-x}},
    \hspace{5 mm} 
    \sum_{n=0}^{\infty} n e^{-nx} = \frac{e^{-x}}{(1-e^{-x})^2},
    \hspace{0.5cm} \Rightarrow \hspace{0.5cm}
    \bar{\E} = \frac{\E_0}{e^{\E_0/ k T} - 1}.
\end{equation*}
Подставляя это в формулу \eqref{rj0}, находим
\begin{equation*}
    u_\omega (\omega, T) = \frac{\omega^2}{\pi^2 c^3} \frac{\E_0}{e^{\E_0/ kT}-1}.
\end{equation*}
А теперь внимание, гений Планка предложил подобрать $\E_0$ так, чтобы выполнялся закон смещения Вина:
\begin{equation*}
     u_\omega (\omega, T) = \omega^3 f_1\left(\frac{\omega}{T}\right),
     \hspace{0.5cm} \Rightarrow \hspace{0.5cm}
     \frac{1}{\pi^2 c^3} \frac{\E_0 / \omega}{e^{\E_0 / kT} - 1} = f\left(\frac{\omega}{T}\right),
\end{equation*}
но $\E_0$ -- характеристика только самого осциллятора, а значит $\E_0 = \const (T)$, тогда $\E_0 = \E_0(\omega)$, откуда находим
\begin{equation*}
    \E_0 = \hbar \omega,
\end{equation*}
где $\hbar$ -- постоянная Планка. Подставляя, находим
\begin{equation}
    u_\omega = \frac{\hbar \omega^3}{\pi^2 c^3} = \frac{1}{e^{\hbar \omega /  kT} - 1},
    \hspace{5 mm} 
    u_\nu = \frac{8 \pi h \nu^3}{c^3} \frac{1}{e^{h \nu / kT}-1},
    \hspace{10 mm} 
    u_\lambda = \frac{8 \pi h c}{\lambda^5} \frac{1}{e^{h c / \lambda k T} - 1},
\end{equation}
что и называют \textit{формулой Планка}.

% В пределе $\hbar \omega / k T \gg 1$, получается
% \begin{equation*}
%     u_\omega = \frac{\hbar \omega^3}{\pi^2 c^3} e^{- \hbar \omega / kT}.
% \end{equation*}


% дописать 5. -- вывод из статистики Бозе-Эйнтштейна.
% % напомнить про лишние штрихи в #1 (упс)

\section{Неделя II}

\subsection*{№1 (1.1.4)}

Найдём функию Грина $G(t)$ уравнения
\begin{equation*}
    L(\partial_t) x(t) = \varphi(t),
    \hspace{5 mm} 
    L(\partial_t) = \partial_t^4 + 4 \nu^2 \partial_t^2 + 3 \nu^4.
\end{equation*}
Функция Грина может быть найдена, как решение уравнения
\begin{equation*}
    L(\partial_t) G(t) = \delta(t)l,
    \hspace{0.5cm} \Rightarrow \hspace{0.5cm} 
    G(t) = \theta(t) \cdot \left(b_1 e^{-\nu t} + b_2 e^{i \nu t} + b_3 e^{- i \sqrt{3} \nu t} + b_4 e^{i \sqrt{3} \nu t}\right),
\end{equation*}
где воспользовались разложением
\begin{equation*}
    L(z) = (z + i \nu) (z- i \nu) (z - i \sqrt{3} \nu) (z + i \sqrt{3} \nu).
\end{equation*}
Интегрируя от $-\varepsilon$ до $+\varepsilon$ уравнение на $G(t)$ находим, что
\begin{equation*}
    \partial_t^3 G(+0)  = 1, \hspace{5 mm} \partial_t^2 G(+0) = \partial_t^1 G(+0) = G(+0) = 0,
\end{equation*}
откуда получаем СЛУ на $\{b_1,\, b_2,\, b_3,\, b_4\}$:
\begin{equation*}
    \left.\begin{aligned}
        b_1+b_2+b_3+b_4 &=0, \\ 
        b_1-b_2+\sqrt{3} \left(b_3-b_4\right)  &=0, \\
        b_1+b_2+3 \left(b_3+b_4\right)  &=0, \\
        b_1-b_2+3 \sqrt{3} \left(b_3-b_4\right)  &=-\tfrac{i}{ \nu^3}, \\
    \end{aligned}\right\}
    \hspace{0.5cm} \Rightarrow \hspace{0.5cm}
    b_1 = \frac{i}{4 \nu^3}, \hspace{2.5 mm}, 
    b_2 = -\frac{i}{4 \nu^3}, \hspace{2.5 mm},
    b_3 = -\frac{i}{4 \sqrt{3} \nu^3}, \hspace{2.5 mm},
    b_4 = \frac{i}{4 \sqrt{3} \nu^3}.
\end{equation*}
Так получаем решение, вида
\begin{equation*}
    G(t) = \frac{\theta(t)}{2 \sqrt{3}\,  \nu^3} \left(
        \sqrt{3}  \sin(\nu t) - \sin(\sqrt{3} \nu t)
    \right).
\end{equation*}






\subsection*{№2 (1.1.5)}

Найдём функцию Грина для уравнения, вида
\begin{equation*}
    (\partial_t^2 + \nu^2)^2 x (t) = \varphi(t).
\end{equation*}

Аналогично предыдущему номеру, сначала находим $G(t>0)$:
\begin{equation*}
    G(t>0) = b_1 e^{i \nu t} + b_2 t e^{i \nu} + b_3 e^{- i \nu t} + b_4 t e^{- i \nu t},
\end{equation*}
где секулярные члены возникли из-за кратности корней.

Также, интегрируя уравнение на $G(t)$ от $-\varepsilon$ до $\varepsilon$, получаем аналогичное условие
\begin{equation*}
    \partial_t^3 G(+0)  = 1, \hspace{5 mm} \partial_t^2 G(+0) = \partial_t^1 G(+0) = G(+0) = 0,
\end{equation*}
и приходим к СЛУ на коэффициенты $\{b_1,\, b_2,\, b_3,\, b_4\}$:
\begin{equation*}
    \left.\begin{aligned}
        b_1+b_3&=0, \\
        i \left(b_1-b_3\right) \nu +b_2+b_4&=0, \\ 
        \nu  \left(\left(b_1+b_3\right) \nu -2 i \left(b_2-b_4\right)\right)&=0, \\
        \nu ^2 \left(-3 \left(b_2+b_4\right)-i \left(b_1-b_3\right) \nu \right)&=1, \\
    \end{aligned}\right\}
    \hspace{0.5cm} \Rightarrow \hspace{0.5cm}
    b_1 = -\frac{i}{4 \nu ^3}, \hspace{2.5 mm} 
    b_2 = -\frac{1}{4 \nu ^2}, \hspace{2.5 mm} 
    b_3 = \frac{i}{4 \nu ^3}, \hspace{2.5 mm} 
    b_4 = -\frac{1}{4 \nu ^2}.
\end{equation*}
Получаем решение, вида
\begin{equation*}
    G(t) = \frac{\theta(t)}{2 \nu^3} \left( \vp
        \sin(\nu t) - \nu t \cos(\nu t)
    \right).
\end{equation*}




\subsection*{№3 (1.1.8)}

Для системы уравнений, вида
\begin{equation*}
    (\partial_t + \hat{\Gamma}) \vc{y}(t)  = \vc{\xi}(t),
    \hspace{5 mm} 
    \Gamma = \lambda \delta_{i,j} + \delta_{i,j-1},
\end{equation*}
найдём функцию Грина $G(t)$, как решение уравнения
\begin{equation*}
    (\partial_t +   \hat{\Gamma}) G(t) = \delta(t) \mathbb{E},
    \hspace{0.5cm} \Rightarrow \hspace{0.5cm}
    G(t) = \theta(t) \exp\left(- \hat{\Gamma} t\right).
\end{equation*}
Осталось найти $\exp(-\hat{\Gamma} t)$, как матричную экспоненту, от жордановой клетки. 

Для начала заметим, что
\begin{equation*}
    \delta_{i,j-1}^2 = \delta_{i,j-1} \delta_{j, k} = \delta_{i+1, k-1} = \delta_{i, k-2},
\end{equation*}
и так далее, то есть $\delta_{i, j-1}$ -- нильпотентный оператор, с $\delta_{i, j-1}^4 = 0$.

Посмотрим на степени $\hat{\Gamma}$:
\begin{align*}
    \hat{\Gamma}^2 &= \delta_{i,j} + 2 \delta_{i,j-1} + \delta_{i, j-2} \\
    \hat{\Gamma}^3 &= \delta_{i,j} + 3 \delta_{i,j-1} + 3 \delta_{i, j-2} + \delta_{i, j-3}\\
    \hat{\Gamma}^4 &= \delta_{i,j} + 4 \delta_{i,j-1} + 6 \delta_{i, j-2} + 4\delta_{i, j-3} + 
    \delta_{i,j-4}, 
\end{align*}
но $\delta_{i,j-4} = 0$, так что можем явно выделить на побочных диагоналях соответсвтующие экспоненты:
\begin{equation*}
    G(t) = \theta(t) e^{- \lambda t} 
    \left(
        \begin{array}{cccc}
         1 & -t & \tfrac{t^2}{2} & -\tfrac{t^3}{6} \\
         0 & 1 & -t & \tfrac{t^2}{2} \\
         0 & 0 & 1 & -t \\
         0 & 0 & 0 & 1 \\
        \end{array}
    \right),
\end{equation*}
где появившиеся $t^k$ -- секулярные члены. 




\subsection*{№4}


В частотном представлении для оператора $\partial_t^2 + \omega_0^2$ можем <<найти>> функцию Грина, приводящую к
\begin{equation*}
    G(\omega) = \frac{1}{\omega_0^2 - \omega^2},
    \hspace{0.5cm} \Rightarrow \hspace{0.5cm}
    G(t) = \int_{-\infty}^{+\infty}  \frac{e^{- i \omega t}}{\omega_0^2 - \omega^2} \frac{\d \omega}{2\pi},
\end{equation*}
с особенностями на вещественной оси.

Регуляризуем интеграл, рассмотрением <<затухающего>> осцилятора, тогда
\begin{equation*}
    G(t) = \int_{-\infty}^{+\infty} \underbrace{\frac{e^{- i \omega t}}{(\omega_0 - \omega + i \varepsilon_1)(\omega_0 + \omega + i \varepsilon_2)}}_{F(w)} \frac{\d \omega}{2\pi}.
\end{equation*}
Получилось два полюса:
\begin{equation*}
    \omega_1 = \omega_0 + i \varepsilon_1,
    \hspace{5 mm} 
    \omega_2 = - \omega_0 - i \varepsilon_2.
\end{equation*}
Соответсвенно, по лемме Жордана, наличие/отсутствие вклада от $\varepsilon_{1,2}$ будет зависеть от выбора знаков в $\varepsilon_{1,2} \to \pm 0$. 

Для начала найдём вычеты по каждому полюсу:
\begin{equation*}
    2 \pi i \cdot \res_{\omega_1} F(\omega) = 
    i \varepsilon e^{i \varphi} F(\omega_1) = 
    - i{\varepsilon e^{i \varphi}} \frac{e^{i t \omega_0}}{2 \omega_0 + i (\varepsilon_1 + \varepsilon_2) + \varepsilon e^{i \varphi}} \overset{\varepsilon \to 0}{\approx} 
-\frac{i}{2 \omega_0} e^{- i t \omega_0}.
\end{equation*}
Аналогично, для второго полюса:
\begin{equation*}
    2 \pi i \cdot \res_{\omega_2} F(\omega) =  \ldots = \frac{i}{2 \omega_0} e^{i t \omega_0}.
\end{equation*}

Сразу заметим, что при вхождение только отного вычета невозможно выполнение условия о $G(0) = 0$, тогда рассмотрим $\varepsilon_1 \to + 0$ и $\varepsilon_2 \to - 0$, тогда оба полюча находятся в верхней полуплоскости, по которой и происходит обход \textit{по} часовой стрелке:
\begin{equation*}
    G(t) =  \textcolor{grey}{\theta (-t)} \frac{1}{\omega_0} \sin(- \omega_0 t),
\end{equation*}
что соответствует опережающей функции Грина ($\partial_t G(t=0) = -1$). 

Теперь найдём, что при $\varepsilon_1 \to -0$ и $\varepsilon_2 \to + 0$ оба вычета в нижней полуплоскости, что приведет к смене знака:
\begin{equation*}
    G(t) = \textcolor{grey}{\theta (t)} \frac{1}{\omega_0} \sin(\omega_0 t),
\end{equation*}
что и соответствует запаздывающей функции Грина (см. ур. \eqref{w1osc}, $\partial_t G(t=0) = 1$), что не может не радовать. 



% Лаврентьев Шаббат -- более строгий рассказ.

\section{Семинар от 25.09.21 (Фурье и Лаплас)}

% трансляционно инвариантна система
\textbf{Про Фурье}. 
Как раньше нашли
\begin{equation*}
    L(\partial_t) G(t) = \delta(t),
    \hspace{0.5cm} \Rightarrow \hspace{0.5cm}
    \hat{x} (\omega) = \int_{\mathbb{R}} e^{- i \omega t} x(t) \d t,
    \hspace{5 mm} 
    x(t) = \int_{\mathbb{R}} e^{i \omega t} \hat{x}(\omega) \frac{\d \omega}{2\pi}.
\end{equation*}
Для этого должно выполняться
\begin{equation*}
    \int |x(t)| \d d < + \infty.
\end{equation*}
\textit{Например}, для $\partial_t + \gamma$:
\begin{equation*}
    (\partial_t + \gamma) G(t) = \delta(t),
    \hspace{0.5cm} \Rightarrow \hspace{0.5cm}
    \int_{\mathbb{R}} \frac{\d t}{\ldots} e^{- i \omega t} \d t = x(t) e^{- i \omega t} \bigg|_{-\infty}^{+\infty},
    \hspace{0.5cm} \Rightarrow \hspace{0.5cm}
    (i \omega + \gamma) \hat{G} (\omega) = 1,
    \hspace{0.5cm} \Rightarrow \hspace{0.5cm}
    \hat{G}(\omega) = \frac{1}{i \omega + \gamma}.
\end{equation*}
Так приходим к уравнению
\begin{equation*}
    G(t) = \int_{\mathbb{R}} \frac{e^{i \omega t}}{\omega - i \gamma} \frac{\d \omega}{2\pi} = 
    \left\{\begin{aligned}
        &e^{- \gamma t}, &t > 0
        &0, &t<0
    \end{aligned}\right.
    \hspace{0.5cm} \Rightarrow \hspace{0.5cm}
    \hat{G}(\omega) = \theta(t) e^{- |t|}.
\end{equation*}

Однако, при $\hat{L} = \partial_t - \gamma$ мы бы получили
\begin{equation*}
    G_A (t) = \theta(-t) e^{\gamma t}, 
\end{equation*}
хотя вообще должно быть (если посчитать через неопределенные коэффициенты)
\begin{equation*}
    G_R (t) = \theta(t) e^{\gamma t},
\end{equation*}
которая растёт.


В методе с Фурье будут получаться функции Грина затухающие, но, возможно, без причинности. 
В методе неопределенных коэффициентов исходим из причинности, но может быть рост $\sim e^{\gamma t}$. 


Кроме того, в Фурье всегда предполагается $x(t \to - \infty) = 0$ и $x(t \to + \infty) = 0$. Также может случиться
\begin{equation*}
    (\partial_t^2 + \omega_0^2) G(t) = \delta(t),
    \hspace{0.5cm} \Rightarrow \hspace{0.5cm}
    \hat{G}(\omega) = \frac{1}{\omega^2 - \omega_0^2},
\end{equation*}
с особенностями на вещественной оси, что можно решить, сместив полюса в $\mathbb{C}$.





\textbf{Свёртка}. Рассмотрим уравнение
\begin{equation*}
    L(\partial_t) x(t) = f(t),
    \hspace{5 mm} 
    L(\partial_t) G(t) = \delta(t).
\end{equation*}
Фурье переводит 
\begin{equation*}
    \int_{\mathbb{R}} \partial_x^n x(t) e^{- i \omega t} \d t = (i \omega)^n \hat{x} (\omega).
\end{equation*}
Тогда
\begin{equation*}
    L(i \omega) \hat{x} (\omega) = \hat{f}(\omega),
    \hspace{5 mm} 
    L(i \omega)  \hat{G} (\omega) = 1,
    \hspace{0.5cm} \Rightarrow \hspace{0.5cm}
    \hat{G}(\omega) = \frac{1}{L(i \omega)}.
\end{equation*}
Также нашли, что
\begin{equation*}
    \hat{x}(\omega) = \frac{\hat{f}(\omega)}{L(i \omega)} = \hat{f}(\omega) \hat{G}(\omega),
    \hspace{0.5cm} \Rightarrow \hspace{0.5cm}
    x(t) = \int_{-\infty}^{+\infty} G(t-s) f(s) \d s.
\end{equation*}



\textbf{Преобразование Лапласа}. Пусть есть некоторое преобразование
\begin{equation*}
    \tilde{f}(p) = \int_0^{\infty} e^{- p t} f(t) \d t,
\end{equation*}
где подразумевается, что $\Re p \geq 0$ и, вообще, в Фурье можно $p \in \mathbb{C}$. 

Пусть $p = i \omega$, где $\omega \in \mathbb{R}$. Тогда
\begin{equation*}
    \tilde{f} (i \omega) = \int_{\mathbb{R}} e^{- i \omega t} f(t) \d t = \hat{f} (\omega),
    \hspace{0.5cm} \Rightarrow \hspace{0.5cm}
    f(t)  =\int_{\mathbb{R}} \hat{f}(\omega) e^{i \omega t} \frac{\d \omega}{2\pi} = 
    \int_{\mathbb{R}} \tilde{f} (i \omega) e^{i \omega t} \frac{\d \omega}{2 \pi} = 
    \int_{- i \infty}^{i \infty} e^{p t} \tilde{f} (p) \frac{\d p}{2\pi}.
\end{equation*}
В вычислениях выше мы предполагали, что $f(t \to \infty) = 0$. 

Обойдём это, пусть $|f(t)| < M e^{s t}$, при $s > 0$. Возьмём $p_0 > s$, тогда
\begin{equation*}
    \tilde{f} (p) = \int_{\mathbb{R}} e^{- p_0 t} e^{-(p-p_0)t} f(t) \d t = \tilde{g}(p-p_0),
\end{equation*}
где вводе $g(t) = e^{- i p_0 t} f(t)$, которая уже убывает на бесконечности. Обратно:
\begin{equation*}
    g(t) = \int_{-i\infty}^{+i\infty}  \tilde{g}(p) e^{pt} \frac{\d p}{2\pi} = 
    \int_{p_0 - i \omega}^{p_0 + i \omega} \tilde{g}(p-p_0) e^{- p_0 t} e^{pt} \frac{\d p}{2\pi i}.
\end{equation*}
Так пришли к форме обращения
\begin{equation}
    f(t) = \int_{p_0 - i \infty}^{p_0 + i \omega} \tilde{f}(p) \frac{\d p}{2 \pi i},
    \hspace{10 mm} 
    \tilde{f}(p) = \int_{\mathbb{R}} e^{- p_0  t} e^{-(p-p_0)t} f(t) \d t = \tilde{g}(p-p_0),
\end{equation}
где $g(t) = e^{- i p_0 t} f(t)$.



Забавный факт, из леммы Жордана: при $t < 0$ $f(t<0) = 9$, по замыканию дуги по часовой стрелке (вправо). Выбирая $p_0$ так, чтобы все особенности лежали левее $p_0$, можем получать причинные функции. 




\textbf{Производная}. Найдём преобразование Лапласа для $\partial_t f(t)$:
\begin{equation*}
    \int_{0}^{\infty} \frac{d f}{d t} e^{- p t} \d t = f e^{- pt} \bigg|_0^{\infty} + p \int_{0}^{\infty} 
    f(t) e^{-pt} \d t = p \tilde{f}(p) - f(+0).
\end{equation*}
Но, для функции Грина $L(\partial_t) G(t) = \delta(t)$, тогда
\begin{equation*}
    L(\partial_t) G_\varepsilon(t) = \delta(t-\varepsilon),
    \hspace{10 mm} 
    G_\varepsilon (t) = G(t-\varepsilon),
    \hspace{0.5cm} \Rightarrow \hspace{0.5cm}
    G_\varepsilon(0)=  0,
\end{equation*}
где $G_\varepsilon \to G(t)$ при $\varepsilon \to 0$.  

Преобразуем\footnote{
    Здесь и далее $f(t)$ -- функция, $f(\omega) = \hat{f}(\omega)$ -- Фурье образ, $f(p) = \tilde{f}(p)$ -- преобразование Лапласа.
}  по Лапласу уравнения выше
\begin{equation*}
     L(p) G(p) = e^{p \varepsilon} = 1,
     \hspace{0.5cm} \Rightarrow \hspace{0.5cm}
     G_\varepsilon (p) = \frac{1}{L(p)},
     \hspace{0.5cm} \overset{\varepsilon \to 0}{\Rightarrow}  \hspace{0.5cm}
     G(p) = \frac{1}{L(p)}.
 \end{equation*} 
Так получаем
\begin{equation}
    G(t) = \int_{p_0 - i \infty}^{p_0 + i \omega} \frac{e^{pt}}{\tilde{L}(p)} \frac{\d p}{2\pi i},
\end{equation}
где $p_0$ правее всех особенностей. 




\textbf{Пример}. Рассмотрим $L = \partial_t + \gamma$, тогда
\begin{equation*}
    (p+\gamma) G(p) = 1,
    \hspace{0.5cm} \Rightarrow \hspace{0.5cm}
    G(p) = \frac{1}{p + \gamma},
    \hspace{0.5cm} \Rightarrow \hspace{0.5cm}
    G(t) = \int_{-i \infty}^{i \infty} \frac{e^{pt}}{p+\gamma} \frac{\d p}{2 \pi i} = \theta(t) e^{- \gamma t}.
\end{equation*}
Аналогично, пусть $L = \partial_t^2 + \omega^2$, тогда $L G(t) = \delta(t)$, и
\begin{equation*}
    G(p) = \frac{1}{p^2 + \omega^2},
    \hspace{0.5cm} \Rightarrow \hspace{0.5cm}
    G(t) = \int_{p_0 - i \infty}^{p_0 + i \omega} \frac{e^{pt}}{p^2 + \omega^2} \frac{\d p}{2 \pi i} = 
    \left\{\begin{aligned}
        &0, &t < 0 \\
        &\ldots, &t>0
    \end{aligned}\right. 
    = \theta(t) \left(
        \frac{e^{i \omega t}}{2 i \omega} + \frac{e^{- i \omega t}}{- 2 i \omega} 
    \right)
        = \theta(t) \frac{\sin \omega t}{\omega}.
\end{equation*}

В общем виде, пусть $L(\partial_t) G(t) = \delta(t)$, тогда
\begin{equation*}
    L(p) G(p) = 1,
    \hspace{0.5cm} \Rightarrow \hspace{0.5cm}
    G(p) \frac{1}{L(p)},
    \hspace{0.5cm} \Rightarrow \hspace{0.5cm}
    G(t) = \int_{p_0 - i \infty}^{p_0 + i \omega} \frac{e^{p t}}{L(p)} \frac{\d p}{2 \pi i}.
\end{equation*}

Поговорим про свёртку:
\begin{equation*}
    L x  = f,
    \hspace{0.5cm} \Rightarrow \hspace{0.5cm}
    L(p) x(p) = f(p),
    \hspace{5 mm} 
    L(p) G(p)  =1,
    \hspace{0.5cm} \Rightarrow \hspace{0.5cm}
    G(p) = \frac{1}{L(p)}.
\end{equation*}
Тогда получается
\begin{equation*}
    x(p) = \frac{f(p)}{L(p)} = f(p) G(p),
    \hspace{0.5cm} \Rightarrow \hspace{0.5cm}
    x(t) = \int_{0}^{t} G(t-s) f(s) \d s.
\end{equation*}


\textbf{Уравнение Вольтера}. Иногда бывает уравнения на $x(s)$ вида
\begin{equation}
    f(t) = \int_{0}^{t} x(s) K(t-s) \d s.
\end{equation}
Через преобразрвание Лапласа, находим
\begin{equation}
    f(p) = x(p) K(p),
    \hspace{0.5cm} \Rightarrow \hspace{0.5cm}
    x(p) = \frac{f(p)}{K(p)}. 
\end{equation}
В общем виде тогда находим
\begin{equation*}
    x(t) = \int_{p_0 - i \infty}^{p_0 + i \omega} \frac{f(p)}{K(p)} e^{pt} \frac{\d p}{2\pi i}.
\end{equation*}
Кстати, забавный факт:
\begin{equation}
    \int_{p_0 - i \infty}^{p_0 + i \omega} 1 \cdot e^{pt} \frac{\d p}{2 \pi i} = e^{p_0 t} 
    \int_{-i \infty}^{i \infty} e^{p t} \frac{\d p }{2 \pi i} = 
    e^{p_0 t} \int_{-\infty}^{+\infty} e^{i \omega t} \frac{\d \omega}{2 \pi} = \delta(t),
\end{equation}
то есть преобразование Лапласа от константы -- дельта функция. 



Рассмотрим, например
\begin{equation*}
    \int_{-i \infty}^{i \infty} \frac{p + 1 -1}{p+1} e^{pt} \frac{\d p}{2 \pi i} = \int_{- i \infty}^{i \infty}  e^{p t} \frac{\d p}{2 \pi i} - \int_{-i\infty}^{+i\infty}  \frac{e^{pt}}{p+1} \frac{\d p}{2 \pi} = \delta(t) - \theta(t) e^{- t}.
\end{equation*}
Также верно, что
\begin{equation*}
    \int_{- i \infty}^{i \infty} p e^{pt} \frac{\d p}{2 \pi i} = \delta'(t).
\end{equation*}
Действительно,
\begin{equation*}
    \frac{d }{d t} \left(
        \int_{- i \infty}^{i \infty}
         e^{p t} \frac{\d p}{2 \pi i}
    \right) = \frac{\d}{\d t} \delta(t) = \delta'(t).
\end{equation*}

Важно, что можно делать функции маленькими
\begin{equation}
    \int_{p_0 - i \infty}^{p_0 + i \omega} f(p) e^{pt} \frac{\d p}{2 \pi i} = 
    \left(\frac{d }{d t} \right)^n \int_{p_0 - i \infty}^{p_0 + i \omega} \frac{f(p)}{p^n} e^{pt} \frac{\d p}{2 \pi i}.
\end{equation}



\textbf{Неоднородная релаксация}. 
Рассмотрим уравнение
\begin{equation*}
    (\partial_t + \gamma(t)) G(t,s) = \delta(t-s),
    \hspace{10 mm} 
    x(t) = \int_{-\infty}^{+\infty} G(t, s) f(s) \d s,
\end{equation*}
где продолжаем требовать причинность $G(t,s>t) = 0$. Для начала, рассмотрим $t>s$, тогда
\begin{equation*}
    (\partial_t + \gamma(t)) G(t) = 0,
    \hspace{0.5cm} \Rightarrow \hspace{0.5cm}
    \frac{d G}{G}  = - \gamma(t) \d t,
    \hspace{0.5cm} \Rightarrow \hspace{0.5cm}
    G(t, s) = A(s) \exp\left(
        - \int_{t_0}^{t} \gamma(t') \d t'
    \right).
\end{equation*}
Также записываем граничные условия:
\begin{equation*}
    \int_{s-\varepsilon}^{s+\varepsilon} \ldots \d s,
    \hspace{0.5cm} \Rightarrow \hspace{0.5cm}
    G(s+0, s) = 1.
\end{equation*}
Так можем найти
\begin{equation}
    A(s) \exp\left(
        - \int_{t_0}^{s} \gamma(t') \d t'
    \right) = 1,
    \hspace{0.5cm} \Rightarrow \hspace{0.5cm}  
    G(t, s) = \theta(t-s) \exp\left(
        - \int_{s}^{t} \gamma(t') \d t'
    \right),
\end{equation}
где мы разбили
\begin{equation*}
    \int_{t_0}^t = \int_{t_0}^{s}  + \int_{s}^{t},
\end{equation*}
и получили, что хотели.



\phantom{42}

\textit{Комментарий про дельта функцию}. Главное, нужно показать, что
\begin{equation*}
    \int_{-\infty}^{+\infty} \delta_a (x) = 1,
    \hspace{10 mm}  
    \lim_{a \to 0} \delta_a (x) =0, \text{ при } x \neq 0.
\end{equation*}
Вообще можем плодить дельтаобразные последовательности, взяв $f$ с единичным интегралом и 
\begin{equation*}
    \delta_a (x) = \frac{1}{a} f\left(\frac{x}{a}\right).
\end{equation*}



\textit{Комментарий про преобразование Лапласа}. Для функции вида
\begin{equation*}
    \frac{1}{\sqrt{p+\alpha}},
\end{equation*}
необходим аппарат разрезов, так что её можно сделать с шифтом на неделю. 

На следующей недели будет контрольная. Необходим аппарат метода неопределенных коэффициентов, матричные экспоненты, решение диффуров через Фурье (не всегда причинный результат), а также преобразование Лапласа. Вычеты скорее всего в районе второго порядка и меньше.  Ещё полезно вспонить, как записывать начальные условия: осцияллятор, осциллятор с затуханием. 





\section{Семинар от 12.09.21}

Раннее решалась задача Коши, вида $L(\partial_t) x(t) = \varphi(t)$. Можо рассмотреть другой класс задач:
\begin{equation*}
    f(0) = f(\pi) = 0,
    \hspace{5 mm} 
    (\partial_x^2 + 1) f(x) = 0,
    \hspace{0.5cm} \Rightarrow \hspace{0.5cm}
    f(x) = A \sin x,
\end{equation*}
где $A$ -- любая, то есть решение не единственно. Более того, решение может не существовать. 

Однако, покуда мы рассматриваем диффуры первого порядка, граничное условие всего одно: значение функции в точке: $x(0) = x_0$, что эквивалентно задаче Коши. 


\subsection{Задача Штурма-Лиувилля}


% \textbf{Задача Штурма-Лиувилля}. 
Интереснее на диффурах II порядка, один из наиболее ярких примеров: \textit{задача Штурма-Лиувилля}:
\begin{equation*}
    \hat{L} = \partial_x^2 + Q(x) \partial_x + U(x),
    \hspace{5 mm} 
    \hat{L} f(x) = \varphi(x),
    \hspace{5 mm} 
    \left\{\begin{aligned}
        \alpha_1 f(a) + \beta_1 f'(a) &= 0 \\
        \alpha_2 f(b) + \beta_2 f'(b) &= 0,
    \end{aligned}\right.
\end{equation*}
где\footnote{
    Часто можно встретить нулевые граничные условия: $f(a) = f(b) = 0$. 
}  $|\alpha_1| + |\beta_2| \neq 0$ и $|\alpha_2| + |\beta_2| \neq 0$.

Заметим, что уравнение линейно: если $\varphi = \varphi_1 + \varphi_2$, то $f = f_1 + f_2$, а значит ответ можно найти в виде
\begin{equation*}
    f(x) \int_{a}^{b} G(x,y) \varphi(y) \d y,
\end{equation*}
однако система теперь не является транслционно инвариантной. 


Граничные условия на $G$:
\begin{equation*}
    \alpha_1 f(a) + \beta_1 f'(a) = \int_{a}^{b} 
    \underbrace{\left(\alpha_1 G(a, y) + \beta_1 G'_x (a, y)\right) }_{\text{непрерывен}}\varphi(y) \d y = 0,
\end{equation*}
что верно $\forall  \varphi$. По лемме Дюбуа-Реймона, можем свести уравнение к виду
\begin{equation*}
    \alpha_1 G(a, y) + \beta_1 G'_x (a, y) \equiv 0,
\end{equation*}
то есть функция Грина $G$ наследует граничные условия.  Аналогично,
\begin{equation*}
    \alpha_2 G(b, y) + \beta_2 G'_x (b, y) = 0.
\end{equation*}

Запищем уравнение на $G(x, y)$:
\begin{equation*}
    \varphi(y) = \delta(y-y'),
    \hspace{0.5cm} \Rightarrow \hspace{0.5cm}
    f(x) = \int_{a}^{b} G(x, y) \delta(y-y') \d y = G(x, y'),
    \hspace{0.5cm} \Rightarrow \hspace{0.5cm}
    \hat{L} G(x, y) = \delta(x-y).
\end{equation*}
Решения имеет смысл разбить на $x \neq y$, и, в частности, рассмотрим $x < y$:
\begin{equation*}
    \left\{\begin{aligned}
        \hat{L} G(x, y) &= 0 \\
        \alpha_1 G(a, y) + \beta_1 G'_x (a, y) &= 0
    \end{aligned}\right.
    ,
    \hspace{0.5cm} \Rightarrow \hspace{0.5cm}
    G(x,y) = A(y) \cdot u(x),
    \hspace{0.5cm} \Rightarrow \hspace{0.5cm}
    \left\{\begin{aligned}
        \hat{L} u(x) &= 0 \\
        \alpha_1 u(a) + \beta_1 u'(a) &= 0
    \end{aligned}\right.
\end{equation*}
Более того, почему бы и не доопределить $u(a) = - \beta_1$ и $u'(a) = \alpha_1$, таким образом свели задачу к задаче Коши, решение которой существует и единственно. 


Аналогично для $x > y$:
\begin{equation*}
    \hat{L} G(x, y) = 0,
    \hspace{0.5cm} \Rightarrow \hspace{0.5cm}
    G(x, y) = B(y) v(x),
    \hspace{0.5cm} \Rightarrow \hspace{0.5cm}
    \left\{\begin{aligned}
        \hat{L} v &= 0 \\
        \alpha_2 v(b) + \beta_2 v'(b) &= 0
    \end{aligned}\right.
\end{equation*}
где снова есть задача Коши, решение которой существует и единственно. 

\textbf{Сшивка}. Во-первых заметим, что $G$ непрерывна, а $G'$ испытыывает скачок:
\begin{equation*}
    G(y + 0, y) = G(y-0, y),
    \hspace{0.5cm} \Rightarrow \hspace{0.5cm}
    A(y) u(y) = B(y) v(y).
\end{equation*}
Интегрируя, находим
\begin{equation*}
    G'_x (y + 0, y) - G'_x (y-0, y) = 1,
    \hspace{0.5cm} \Rightarrow \hspace{0.5cm}
    B(y) v'(y) - A(y) u'(y) = 1.
\end{equation*}
Собирая уравнения вместе, находим, что
\begin{equation*}
    B(y) \underbrace{\left(
        \frac{v' (y) u(y) - v(y) u'(y)}{u(y)}
    \right)}_{W[u, v]} = 1,
    \hspace{0.5cm} \Rightarrow \hspace{0.5cm}
    B(y) = \frac{u(y)}{W},\hspace{5 mm} 
    A(y) = \frac{v(y)}{W(y)},
\end{equation*}
где $W[u, v]$ -- вронскиан. Итого, можем выписать ответ:
\begin{equation*}
    G(x, y) = \frac{1}{W(y)} \left\{\begin{aligned}
        &v(y) u(x), &x < y; \\
        &v(x) u(y), &x > y.
    \end{aligned}\right.
\end{equation*}
Можем записать, когда решение $\exists$ и $!$:
\begin{equation*}
    W \neq 0,
    \hspace{0.5cm} \Rightarrow \hspace{0.5cm}
    \text{Sol} \  \exists \& !.
\end{equation*}
Отсюда вытекает теорема Стеклова:

\begin{to_thr}[теорема Стеклова]
    Если $u,\, v$ -- спец. ФСР, то решение существует и единственно:
    \begin{equation*}
        f(x) = \int_{a}^{b} G(x, y) \varphi(y) \d y,
        \hspace{10 mm} \hat{L}^{-1} \varphi = f.
    \end{equation*}
    Если $W = \const$, то $G(x, y) =G(y,x)$ -- симметричное ядро, а значит $L^{-1}$ -- симметричный, самосапряженный оператор $\Rightarrow$ у $\hat{L}$ есть ОНБ из собственных функций. 
\end{to_thr}




\textbf{Про вронскиан}. Можно записать формулу Лиувилля-Остроградского
\begin{equation*}
    W(x) = 
    \det \begin{pmatrix}
        u & v  \\
        u' & v'  \\
    \end{pmatrix}
    =
    W(x_0) \exp\left(
        - \int_{x_0}^{x}  Q(z) \d z
    \right).
\end{equation*}

\begin{to_def}
    \textit{Специальной ФСР} называется решение уравнении $\hat{L} u = 0$ и $\alpha_1 u(a) + \beta_1 u'(a) = 0$, и аналогичного уравнения по $v(x)$ с граничным условием в $b$, 
    если $W[u, v] \neq 0$, то есть $u$ и $v$ линейной независимы. 
\end{to_def}



\textbf{Пример I}. Рассмотрим уравнения
\begin{equation*}
    \left\{\begin{aligned}
        \partial_x^2 f(x) = \varphi(x) \\
        f(a) = f(b) = 0
    \end{aligned}\right.
    \hspace{0.5cm} \Rightarrow \hspace{0.5cm}
    u(x) = x - a,
    \hspace{5 mm} 
    v(x) = x - b,
    \hspace{0.5cm} \Rightarrow \hspace{0.5cm}
    W = \begin{pmatrix}
        u & v  \\
        u' & v'  \\
    \end{pmatrix} = b-a = \const,
\end{equation*}
а значит
\begin{equation*}
    G(x, y) = \frac{1}{b-a} \left\{\begin{aligned}
        &(y-b)(x-a), &x < y \\
        &(x-b)(y-a), &x > y.
    \end{aligned}\right.
\end{equation*}


\textbf{Пример II}. Рассмотрим двумерный цилиндр, радуса $R$, вне которого $\rho(r > R) = 0$, $\rho(\vc{r}) = \rho(r)$. Рассмотрим уравнения Лапласа:
\begin{equation*}
    \nabla^2 \varphi = - 4 \pi \rho,
    \hspace{0.5cm} \Rightarrow \hspace{0.5cm}
    (\partial_r^2 + \tfrac{1}{r} \partial_r) \varphi = - 4 \pi \rho.
\end{equation*}
Добавим граничные условия: потенциал определен с точностью до константы, так что пусть $\varphi(R) = 0$, также хотим конечность $\varphi$ при $r=0$, так что пусть $\varphi(0) = 1$.

Получили задачу, где при $r < r'$
\begin{equation*}
    \left\{\begin{aligned}
        &\left(\partial_r^2 + \tfrac{1}{r} \partial_r\right) u(r) = 0,
        &u(0) = 1
    \end{aligned}\right.
    \hspace{0.5cm} \Rightarrow \hspace{0.5cm}
    u' = \frac{C}{r},
    \hspace{0.5cm} \Rightarrow \hspace{0.5cm}
    u(r) = C \ln r + D = 1.
\end{equation*}
Аналогично, рассмотрим $r > r'$:
\begin{equation*}
    \left\{\begin{aligned}
        &\left(\partial_r^2 + \tfrac{1}{r} \partial_r\right) v(r) = 0,
        &v(R) = 0,
    \end{aligned}\right.
    \hspace{0.5cm} \Rightarrow \hspace{0.5cm}  
    v(R) = C' \ln r + D',
    \hspace{0.5cm} \Rightarrow \hspace{0.5cm}   
    v = \ln \left(\frac{r}{R}\right).
\end{equation*}
Сразу вычислим 
\begin{equation*}
    W[u,\,  v] = \det \begin{pmatrix}
        1 & \ln r/R  \\
        0 & 1/r  \\
    \end{pmatrix} = \frac{1}{r},
    \hspace{0.5cm} \Rightarrow \hspace{0.5cm}
    G(r, r') = r' \left\{\begin{aligned}
        & \ln \tfrac{r'}{R}, & r < r'
        & \ln \frac{r}{R}, & r > r'
    \end{aligned}\right.
    \hspace{0.5cm} \Rightarrow \hspace{0.5cm}
    \varphi(r) = \int_{0}^{R} G(r, r') \left(- 4 \pi \rho(r')\right) \d r'.
\end{equation*}




\subsection{Задача с периодическими условиями}

Рассмотрим такой же $\hat{L}$, и граничные условия в виде
\begin{equation*}
    \left\{\begin{aligned}
        f(a) &= f(b) \\
        f'(a) &= f'(b),
    \end{aligned}\right.
\end{equation*}
то есть решение периодично. 



\textbf{Пример}.  Рассмотрим задачу
\begin{equation*}
    \hat{L} = \partial_x^2 + \kappa^2,
\end{equation*}
с условиями на $[-\pi, \pi]$. 

При $x < y$:
\begin{equation*}
    G(x, y) = A_1 (y) \sin \kappa(x + \pi) + B_1 (y) \cos \kappa( x + \pi),
\end{equation*}
и аналогично для $x > y$:
\begin{equation*}
    G(x, y) = A_2 \sin \kappa (x - \pi) + B_2 (y) \cos \kappa (x - \pi).
\end{equation*}
Запишем граничные условия:
\begin{align*}
    G(- \pi, y) = G(\pi, y), \hspace{0.5cm} \Rightarrow \hspace{0.5cm}
    B_1 (y) = B_2 (y) \overset{\mathrm{def}}{=} B(y) \\
    G'_x (-\pi, y) = G'_x (\pi, y),
    \hspace{0.5cm} \Rightarrow \hspace{0.5cm}
    A_1 (y) = A_2 (y) \overset{\mathrm{def}}{=}  A(y).
\end{align*}
Тогда нашли, что
\begin{equation*}
    G(x, y) = \left\{\begin{aligned}
        &A \sin \kappa (x + \pi) + B \cos \kappa (x + \pi) \\
        &A \sin \kappa (x - \pi) + B \cos \kappa (x - \pi) \\
    \end{aligned}\right.
\end{equation*}
Теперь запишем непрерывность:
\begin{equation*}
     A \sin \kappa (x + \pi) + B \cos \kappa (x + \pi) 
     = 
     A \sin \kappa (x - \pi) + B \cos \kappa (x - \pi).
\end{equation*}
А также скачок производной
\begin{equation*}
    G'_x(y + 0, y) - G'_x (y-0, y) = 1,
    \hspace{0.25cm} \Rightarrow \hspace{0.25cm}
        A \cos \kappa (x - \pi) - B \sin \kappa (x - \pi)  - 
        A \cos \kappa (x + \pi) + B \cos \kappa (x + \pi)
        = \kappa^{-1}.
\end{equation*}
Решая эту систему находим, что
\begin{equation*}
    2 \sin \pi \kappa 
    \begin{pmatrix}
        \cos \kappa y & - \sin \kappa y  \\
        \sin \xi y & \cos \kappa y  \\
    \end{pmatrix} \begin{pmatrix}
        A  \\
        B  \\
    \end{pmatrix}
    = \begin{pmatrix}
        0 \\ 1/\kappa
    \end{pmatrix},
    \hspace{0.25cm} \Rightarrow \hspace{0.25cm}
    \begin{pmatrix}
        A \\ B
    \end{pmatrix} = 
    \frac{1}{2 \sin \pi \kappa} \begin{pmatrix}
        \cos \kappa y & \sin \kappa y  \\
        \sin \kappa y & \cos \kappa y  \\
    \end{pmatrix}
    \begin{pmatrix}
        0 \\ 1/\kappa
    \end{pmatrix} = 
    \frac{1}{2 \kappa \sin \pi \kappa} \begin{pmatrix}
        \sin xy \\ \cos xy
    \end{pmatrix}.
\end{equation*}
Подставляя в $G(x, y)$, находим\footnote{
    К дз будет полезно заметить, что $G(x, y) = G(x-y)$ -- задача трансляционно инвариантна. 
} 
\begin{equation*}
    G(x, y) = \frac{1}{2 \kappa \sin \pi \kappa}
    \left\{\begin{aligned}
        &\cos \left(\kappa(x-y) + \kappa \pi\right), & x < y\\
        &\cos(\kappa (x-y) - \kappa \pi), & x > y.
    \end{aligned}\right.
\end{equation*}
Всё это было, повторимся, для уравнения:
\begin{equation*}
    \left(\partial_x^2 + \kappa^2\right) f(x) = \varphi(x),
    \hspace{0.5cm} \Rightarrow \hspace{0.5cm}   
    f(x) = 
    \int_{-\pi}^{+\pi} G(x, y) \varphi(y) \d y. 
\end{equation*}





\section{Семинар от 16.09.21}

Рассмотрим снова некоторую граничную задачу:
\begin{equation*}
    \hat{L} G(x, y) = \delta(x-y).
\end{equation*}
Запишем граничные условия:
\begin{equation*}
    \alpha_1 G(a. y) + \beta_1 G'_x(a, y) = 0,
    \hspace{10 mm} 
    \alpha_2 G(b, y) + \beta_2 G'_x (b, y) = 0,
\end{equation*}
где  $|\alpha_1| + |\beta_2| \neq 0$ и $|\alpha_2| + |\beta_2| \neq 0$.
Можем выписать ответ:
\begin{equation*}
    G(x, y) = \frac{1}{W(y)} \left\{\begin{aligned}
        &v(y) u(x), &x < y; \\
        &v(x) u(y), &x > y,
    \end{aligned}\right.
\end{equation*}
где Вронскиан можно запсиать, как
\begin{equation*}
    W(x) = W(x_9) \exp\left(
        - \int_{x_0}^{x}  Q(t) \d t,
    \right)
\end{equation*}
где $Q(t)$ -- из оператора Штурма-Лиувилля. 

Также решали задачу с периодическими гран. условиями, где $G$ наследовала гран. условия. 
Решать это всё умеем двумя способами: разделяя на $x  > y$ и $x < y$, и через метод Фурье:
\begin{equation*}
    \hat{L} e_n  = \lambda_n e_n,
    \hspace{10 mm} 
    \langle e_n | e_m \rangle = \int_{a}^{b} e_n (x) \bar{e}_m(x)  \d x.
\end{equation*}
Тогда можем найти функцию Грина, как
\begin{equation*}
    G(x, y) = \sum_n g_n (y) e_n (x),
    \hspace{5 mm} 
    \delta(x-y) = \sum_n \delta_n ( y) e_n (x).
\end{equation*}
Находим коэффициенты Фурье:
\begin{equation*}
    g_n (y) = 
    \frac{\langle G | e_n\rangle}{\langle e_n | e_n\rangle},
    \hspace{0.5cm} \Rightarrow \hspace{0.5cm}
    \delta_n (y) = \frac{\bar{e}_n (y)}{\langle e_n | e_n\rangle},
    \hspace{0.5cm} \Rightarrow \hspace{0.5cm}
    g_n (y) = \frac{1}{\lambda_n} \frac{\bar{e}_n (y)}{\langle e_n | e_n\rangle},
\end{equation*}
где мы решали уравнение, вида $\hat{L} G = \delta(x-y)$. 
Проблема возникает при $\lambda_n = 0$. 



\textbf{Решение}. Наличие у оператора собственного числа $\lambda_n = 0$ называется нулевой модой. Рассмотрим оператор:
\begin{equation*}
    \hat{L} = \partial_x^2,
\end{equation*}
для которого $e_n (x) = e^{i n x}$, где $\langle e_n | e_n\rangle = 2 \pi$, где $e_0 = 1$ и $\lambda_{0} = 0$. Пусть тогда
\begin{equation*}
    \delta(x) = \sum \frac{\bar{e}_n (0) e_n (x)}{\langle e_n | e_n\rangle} = \sum \frac{e^{i n x}}{2 \pi},
    \hspace{10 mm} 
    G(x) = \sum  g_n e_n (x). 
\end{equation*}
но для $\hat{L} G = \delta(x)$ оказывается нет решений (справа $e_0$ есть, а слева нет). То есть
\begin{equation*}
    \Ker \hat{L} \neq \{0\},
    \hspace{10 mm} 
    \Ker \hat{L} + \Im \hat{L} = \mathcal H,
\end{equation*}
поэтому всегда имеем ввиду, что $\hat{L} \hat{L}^{-1} = \mathbbm{1}$, но только для $\im \hat{L}$. 

В общем, проблему уйдёт, если рассмотрим уравнение, вида
\begin{equation*}
    \hat{L} G(x) = \delta(x) - e_0(x) = \delta(x) - \frac{1}{2 \pi},
\end{equation*}
то есть справа единичный оператор только на образе $\im \hat{L}$. 



Если в источнике есть нулевая мода, то уравнение не имеет решений. 



\textbf{Алгоритм (Фурье)}. Раскладываем 
\begin{equation*}
    G(x) = \sum_{n \neq 0} g_n e_n,
    \hspace{10 mm} 
    \delta(x) = \sum \frac{e_n (x)}{2 \pi},
    \hspace{0.5cm} \Rightarrow \hspace{0.5cm}
    \hat{L} G = \delta(x) - \frac{1}{2\pi}.
\end{equation*}
Знаем, что $\lambda_n g_n = \frac{1}{2\pi}$, а значит
\begin{equation*}
    g_n(x) = \frac{1}{2\pi} \frac{1}{- n^2},
    \hspace{0.5cm} \Rightarrow \hspace{0.5cm}   
    G(x) = \sum_{n \neq 0} \frac{1}{2\pi} \frac{1}{-n62} e^{i n x},
\end{equation*}
и рассмотрим $0 < x < \pi$, суммирая это через вычеты, записываем
\begin{equation*}
    f(z) = \frac{e^{zx}}{2 \pi z^2}, 
    \hspace{0.5cm} \Rightarrow \hspace{0.5cm}   
    G(x) = \sum \oint_{in} \frac{\d z}{2 \pi i} f(z) g(z).
\end{equation*}
Соответственно, выберем
\begin{align*}
    g(z) = \frac{\pi e^{- \pi z}}{\sh (\pi z)}
\end{align*}
тогда
\begin{equation*}
    f(z) g(z) = \frac{\pi}{z^2} \frac{e^{(x-\pi)z}}{\sh \pi z},
\end{equation*}
получаем, что интеграл по душам вправо/влево  равен $0$, и остается только вычет в $z = 0$:
\begin{equation*}
    G(z) = - \res_0 f(z) g(z) = \ldots = - \frac{x^2}{4 \pi} + \frac{x}{2} - \frac{\pi}{6}.
\end{equation*}



\textbf{Алгоритм (сшивка)}. Решим задачу
\begin{equation*}
    \partial_x^2 G(x) = \delta(x) - \frac{1}{2\pi}.
\end{equation*}
Разбиваем $x < 0$ и $x > 0$:
\begin{align*}
    &x < 0, 
    & G = -\tfrac{x^2}{4 \pi} + a x + b, \\
    &x > 0, 
    & G = -\tfrac{x^2}{4 \pi} + c x + \varpi, 
\end{align*}
учитываем граничные условия:
\begin{equation*}
    G(-0) = G(+0),
    \hspace{5 mm} 
    G'(+0) - G'(-0) = 1,
    \hspace{0.5cm} \Rightarrow \hspace{0.5cm}   
    b = \varpi.
\end{equation*}
Также получаем, что $-a = b$.

Учтём, что $e_0$ не входит в $G$:
\begin{equation*}
    \langle G | e_0\rangle = 0 = \int_{-\pi}^{+\pi}G(x) \d x = 0,
    \hspace{0.5cm} \Rightarrow \hspace{0.5cm}
    b = - \frac{\pi}{6},
\end{equation*}
так и получаем все необходиме условия на $G(x, y)$. 



\subsection{Многомерие \texorpdfstring{$\mathbb{R}^3$}{R3}}

Рассмотрим $\mathbb{R}^3$:
\begin{equation*}
    \nabla^2 f = \varphi,
\end{equation*}
где все линейно, всё хорошо. Как обычно будем искать функцию, виде
\begin{equation*}
    f(\vc{r}) = \int_{\mathbb{R}^3} G(\vc{r}  - \vc{r}') \varphi(\vc{r}) \d^3 r. 
\end{equation*}
Функцию Грина найдём в виде
\begin{equation*}
    \nabla^2 G(\vc{r}) = \delta(r^3) = \delta(x) \delta(y) \delta(z),
    \hspace{10 mm}  
    \int f(\vc{r}) \delta(\vc{r}- \vc{r}') \d^3 \vc{r}' = f(\vc{r}').
\end{equation*}
Можем свести уравнение Лапласа, к уравнению Дебая:
\begin{equation*}
    (\nabla^2 - \kappa^2) G(\vc{r}) = \delta(\vc{r}),
\end{equation*} 
которое очень удобно раскладывать по Фурье:
\begin{align*}
    &\text{ПФ}: 
    &G(\vc{k}) &= \int_{\mathbb{R}^3} G(\vc{r}) e^{- i \smallvc{k} \cdot \smallvc{r}} \d \vc{r}, \\
    &\text{ОПФ}: 
    &G(\vc{r}) &= \int_{\mathbb{R}^3} G(\vc{r}) e^{i \smallvc{k} \cdot \smallvc{r}} \frac{\d \vc{k}}{(2 \pi)^3}.
\end{align*}
Также вспомним, что
\begin{equation*}
    \partial_m G(\vc{r}) e^{- i \smallvc{k} \cdot \smallvc{r}} \d \vc{r} = i k_m G(\vc{k}),
\end{equation*}
а значит
\begin{equation*}
    (-k^2 - \kappa^2) G(\vc{k}) = 1,
    \hspace{0.5cm} \Rightarrow \hspace{0.5cm}
    G(\vc{k})=- \frac{1}{k^2 + \kappa^2},
    \hspace{0.5cm} \Rightarrow \hspace{0.5cm}
    G(\vc{r}) = \int_{\mathbb{R}^3} \frac{e^{i \smallvc{k} \cdot \smallvc{r}}}{k^2 + \kappa^2} \frac{\d \vc{k}}{(2 \pi)^3}.
\end{equation*}
Переходим в сферические координаты, получаем, что
\begin{equation*}
    G(\vc{r}) = - \frac{2 \pi}{(2 \pi)^3} \int_{0}^{\infty} \frac{k^2}{k^2 + \kappa^2} \int_{0}^{\pi} 
    \sin \theta  e^{i k r \cos \theta}  \d \theta \d k = 
    - \frac{e^{- \kappa r}}{4 \pi r}
    .
\end{equation*}
Устремляя $\kappa \to 0$, находим
\begin{equation*}
     \nabla^2 G = \delta(\vc{r}),
     \hspace{0.5cm} \Rightarrow \hspace{0.5cm}
     G = - \frac{1}{4 \pi r}.
 \end{equation*} 





% \textbf{Пример}. Пусть 

\subsection{Многомерие \texorpdfstring{$\mathbb{R}^2$}{R2}}

Для Гаусса можно найти, что
\begin{equation*}
    G^{[\dim = n]}(x) = \frac{1}{\sigma_{n-1}} \frac{1}{r^{n-2}},
\end{equation*}
где $\sigma_{n-1}$ -- площадь $n-1$ мерной сферы. 



Вообще часто задача формулируется в виде задачи Дирихле:
\begin{equation*}
    \nabla^2 f = 0, 
    \hspace{5 mm} 
    f'_{\partial D} = f_0 (\vc{r}),
\end{equation*}
то есть функция задана на границе некоторой области. Пусть
\begin{equation*}
    f(z) = u(z) = i v(z),
    \hspace{5 mm} 
    \nabla^2 u = \nabla^2 v = 0.
\end{equation*}

Пусть знаем комплексную функцию $f(z)$ такую, что $\Re f |_{\partial D} = f_0$, тогда $\Re f(z)$ решает задачу Дирихле.
Далее конформным преобразованием переводим любое $D$ в круг, в круге задача Дирихле решается, а дальше отображаем назад. 


Пусть задана функция $u_0 (x) = u(x, 0)$. Вообще можно было бы разложить по Фурье $u$, и записать
\begin{equation*}
    \nabla^2  u = 0,
    \hspace{5 mm}   
    u(x, 0) = u_0 (x).
\end{equation*}
Тогда
\begin{equation*}
    u(q, y)  = \int_{\mathbb{R}} e^{- i q x} u(x, y) \d x,
    \hspace{0.5cm} \Rightarrow \hspace{0.5cm}
    \nabla^2 u = - q^2 u(q, y) = 0.
\end{equation*}
Так приходим к
\begin{equation*}
    u(q, y) = \exp\left(- |q| y\right) u(q, 0),
    \hspace{0.5cm} \Rightarrow \hspace{0.5cm}   
    u(x, y) = \int_{\mathbb{R}} \frac{\d q}{2\pi} e^{i q x} \underbrace{e^{- |q| y}}_{h(q)} u(q, 0).
\end{equation*}
Произведение Фурье образов -- свёртка:
\begin{equation*}
    u(x, y) = \int_{\mathbb{R}} \d \xi h(x - \xi,\,  y) u_0 (\xi).
\end{equation*}
Найдём, что
\begin{equation*}
    \int_{\mathbb{R}} \frac{\d q}{2 \pi} e^{i q x} e^{-|q| y} = \frac{y}{\pi (x^2 + y^2)}.
\end{equation*}
Подставляем, и находим:
\begin{equation*}
    u(x, y) = \int_{\mathbb{R}} d \xi \, \frac{y/\pi}{(x-\xi)^2 + y^2} u_0 (\xi),
\end{equation*}
где 
\begin{equation*}
    \frac{y/\pi}{(x-\xi)^2 + y^2} = \Im \frac{-1}{x + i y - \xi},
    \hspace{0.5cm} \Rightarrow \hspace{0.5cm}   
    f(z) = - \frac{1}{\pi} \int_{\mathbb{R}} \frac{1}{z-\xi} u_0 (\xi),
\end{equation*}
что в некотором смысле привело нас к интегралу Коши, так что и $\nabla^2 f = 0$ и гран. условия удовлетворяются. 



\textbf{Пример}. Рассмотрим
\begin{equation*}
    u_0 (x) = \frac{1}{1 + x^2},
    \hspace{5 mm} 
    u(x, y) = \int_{\mathbb{R}} d \xi \frac{1}{\pi} \frac{y}{y^2 + (x-\xi)^2} \frac{1}{1+\xi^2} = \frac{1 + y}{x^2 + (1+y)^2}.
\end{equation*}



\section{Семинар от 23.10.21}


\textbf{$\Gamma$-функция}. Найдем некоторые интересные свойства:
\begin{equation*}
    \Gamma(z) = \int_{0}^{\infty} t^{z-1} e^{-t} \d t \overset{t = \tau x}{=} 
    x^z \int_{0}^{\infty} \tau^{z-1} e^{- \tau x} \d \tau,
    \hspace{10 mm} 
    \frac{1}{x^z} = \frac{1}{\Gamma(z)} \int_{0}^{\infty} \tau^{z-1} e^{- \tau x} \d \tau.
\end{equation*}
Также знаем $\Gamma(n+1) = n!$, $\Gamma(2n+1)$, $\Gamma(1/2) = \sqrt{\pi}$ и т.д.



\textbf{Аналитическое продолжение $\Gamma$-функции}. Пусть есть две функции $\varphi_1$ и $f_2$, равные друг другу на сходящемся множестве точек $z_i \in \mathcal D_1 \cap \mathcal D_2$. Так и строим аналитическое продолжение для $f_1$ функуцией $f_2$.
% обнудение тригонометрии, Карлов.


Можно сказать, что 
\begin{equation*}
    \Gamma(z-1) = \frac{\Gamma(z)}{z-1}, \hspace{10 mm} \Re z > 0.
\end{equation*}
Но давайте сыграем в чудеса. Изначально определяли
\begin{equation*}
    \Gamma(z) = \int_{0}^{\infty} t^{z-1} e^{-t} \d t,
    \hspace{5 mm} 
    t^{z-1} = e^{(z-1) \ln t}.
\end{equation*}
Выберем такую связную область, чтобы точку $t = 0$ нельзя было бы обойти, и получим $\ln t = \ln |t| + i \varphi$. 

Сверху $\ln (|t| + i 0) = \ln |t| + 2 \pi i$, и снизу $\ln (|t| + i 0) = \ln |t| + 2 \pi i$. Тогда верно, что
\begin{equation*}
    \int_{0}^{\infty} e^{(z-1)\ln t} e^{-t} \d t,
    \hspace{10 mm} 
    \text{up}: \ \ e^{(z-1) \ln |t|} e^{-|t|},
    \hspace{10 mm} 
    \text{down}: \ \ e^{(z-1) \ln |t|} e^{-|t|} e^{2 \pi i z}.
\end{equation*}
Сложим интеграл поверху и понизу, получим 
\begin{equation*}
    I = \int e^{(z-1)\ln t} e^{-t} \d t = (1 -e^{2 \pi i z}) \Gamma(z) 
    = \int_C e^{(z-1) \ln t} e^{- t} \d t = \int_C t^{z-1} e^{-t} \d t.
\end{equation*}
Особенность есть только в точке $0$. Таким образом находим аналитическое продолжение:
\begin{equation}
    \Gamma(z) = \frac{1}{1-e^{2 \pi i z}} \int_C t^{z-1} e^{-t} \d t.
\end{equation}
Видим, что у $\Gamma(z)$ есть особенности $z \in \mathbb{Z}$, где $z \in \mathbb{N}$ -- УОТ, и $z \in \mathbb{Z}\ \mathbb{N}$ -- полюса первого порядка.


Рассмотрим $z = -n$, тогда интегрируем
\begin{equation*}
    \int_C t^{-n-1} e^{-t} \d t = 2 \pi i \frac{1}{n!} \left(\frac{d }{d t} \right)^n e^{-t} = 
    - \frac{2\pi i (-1)^n}{n!}.
\end{equation*}
Итого находим, что
\begin{equation*}
    \res_{-n} \Gamma(z) = \frac{(-1)^n}{n!},
\end{equation*}
что позволяет определить преобразование Мелина от $\Gamma(z)$:
\begin{equation*}
    \int_{-i\infty}^{+i\infty}  \Gamma(z) x^{-z} \d x,
    \hspace{10 mm} 
    M(f(z)) = \int_{0}^{\infty} t^{z-1} f(t) \d t,
\end{equation*}
но это к слову. 


Найдём теперь $\Gamma(n)$:
\begin{equation*}
    \Gamma(n) = \lim_{z\to n} \frac{1}{1 - e^{2 \pi i z}} \int_C t^{z-1 + n  -n} e^{-t} \d t = 
    \bigg/
        t^{z-n} \approx 1 + (z-n) \ln t
    \bigg/ \overset{\frac{a}{b}\to \frac{a'}{b'}}{=}  
    \frac{1}{2\pi i} \int_C t^{n-1} \ln t e^{-t} \d t.
\end{equation*}
Теперь делаем обратную интерацию, <<сдувая>> логарифм к $\Re t$. Здесь всё также $\ln (|t| + i0) = \ln |t|$ и $\ln (|t| - i 0) = \ln |t| + 2 \pi i$. Тогда
\begin{equation*}
    \left(\int_C = \int_{\text{up}} + \int_{\text{down}}\right) 
    \frac{1}{1-e^{2 \pi i z}} \int_C t^{z-1} e^{-t} \d t
    = - 2 \pi i \int_0^{\infty} t^{n-1} e^{-t} \d t = (n-1)!.
\end{equation*}


\textbf{$B$-функция}. Рассмотрим функцию, вида
\begin{equation*}
      B(\alpha,\, \beta) = \int_{0}^{1} t^{\alpha-1} (1-t)^{\beta-1} \d t,
      \hspace{5 mm} 
      \Re \alpha, \beta > 0.
\end{equation*}  
Сделаем замену переменных
\begin{equation*}
    B(\alpha,\, \beta) = \int_{0}^{1}  t^{\alpha-1} (1-t)^{\beta-1} \d t \overset{t = y/s}{=} 
    \int_{0}^{s} d y\ y^{\alpha-1} (s-y)^{\beta-1} / s^{\alpha + \beta -1}.
\end{equation*}
Нетрудно получить, что
\begin{equation*}
    B(\alpha, \beta) \Gamma(\alpha + \beta) = \int_{0}^{\infty}  d s\ e^{-s} \int_{0}^{s} dy\ y^{\alpha-1} (s-y)^{\beta-1} = 
    \int_{0}^{\infty} dy\ y^{\alpha-1} \int_y^{\infty} ds\ e^{-s + y - y} (s-y)^{\beta-1} = 
    \int_{0}^{\infty}  dy \ y^{\alpha-1} e^{-y} \int_{0}^{\infty} dx\ e^{-x} x^{\beta-1},
\end{equation*}
а значит
\begin{equation}
    B(\alpha,\, \beta) = \frac{\Gamma(\alpha) \Gamma(\beta)}{\Gamma(\alpha + \beta)},
\end{equation}
что и является аналитическим продолжением $B$-функции. 


Например,
\begin{equation*}
    \int_0^{\pi/2} \sin^\alpha \varphi \cos^\beta \varphi \d \varphi =  \frac{1}{2} B\left(\frac{a+1}{2},\, \frac{b+1}{2}\right).
\end{equation*}
Также верно, что
\begin{equation*}
    \int_{0}^{\infty}  \frac{x^m}{(1+x^n)^k} \d x = \bigg/
        t = \frac{1}{1+x^n}
    \bigg/.
\end{equation*}
Аналогично можем получить, что
\begin{equation}
    \Gamma(z) \Gamma\left(z + \tfrac{1}{2}\right) = \frac{\sqrt{\pi}}{2^{2z-1}} \Gamma(2 z).
\end{equation}
Ну действительно, представим
\begin{equation*}
    \Gamma(z) \Gamma(z + \tfrac{1}{2}) = \frac{\Gamma(\frac{1}{2}) \Gamma(z)^2}{B(z,\, \tfrac{1}{2})} = 
    \frac{\sqrt{\pi} B(z, z)}{B(z,\, \frac{1}{2})} \Gamma(2z).
\end{equation*}
Осталось раскрыть
\begin{equation*}
    B(z, z) = 2 \int_{0}^{\pi/2} d \varphi\ \left(\sin \varphi \cos \varphi\right)^{2z-1} = 
    \frac{2}{2^{2z-1}} \int_{0}^{\pi/2} d \varphi \ \sin^{2 z-1} \varphi.
\end{equation*}
Теперь, уже интегрируя двойной угол, находим
\begin{equation*}
    B(z, z) = \frac{2}{2^{2z-1}} \frac{1}{2} B(z,\, \tfrac{1}{2}) = \frac{1}{2^{2z-1}} B(z,\, \tfrac{1}{2}).
\end{equation*}

Ещё один забавный факт:
\begin{equation*}
    \Gamma(z) \Gamma(1-z) = \frac{\pi}{\sin \pi z},
\end{equation*}
что также совершает аналитическое продолжение. Действительно,
\begin{equation*}
    B(z,\, 1-z) = \int_{0}^{1}  t^{z-1} (1-t)^{-z} \d t = \int_0^1 \left(\frac{t}{1-t}\right)^z \frac{\d t}{t}.
\end{equation*}
Тут логично ввести $x = \frac{t}{1-t} = -1 + \frac{1}{1-t}$, а значит
\begin{equation*}
    t = \frac{x}{x+1}, \hspace{5 mm} 
    \d t = \frac{\d x}{(x+1)^2}.
\end{equation*}
Продолжая жонглировать переменными
\begin{equation*}
    B(z,\, 1-z) = \int_{0}^{\infty} x^z \frac{x+1}{x} \frac{\d x}{(x + 1)^2} = 
    \int_{0}^{\infty} \frac{x^{z-1}}{x+1} \d x.
\end{equation*}
Который снова удобно посчитать через разрезы. 
\begin{equation*}
    B(z,\, 1-z) = \int_{\text{up}} \frac{x^{z-1}}{1+x} \d x =
    \frac{1}{1-e^{2 \pi i z}} \int_C \frac{x^{z-1}}{1+x} \d x,
\end{equation*}
но тут уже можно замкнуть дугу на бесконечности, вклад от котрой нулевой.  Осталось найти вычет в точке $-1$, тогда
\begin{equation*}
    \int_{\text{up}} \frac{x^{z-1}}{1+x} \d x = \frac{1}{1-e^{2\pi i z}}    \res_{-1} = \frac{2 \pi i (-1) e^{\pi i z}}{1 - e^{2 \pi i z}} = \frac{2 \pi i}{e^{\pi i z} - e^{-i \pi z}} = \frac{\pi}{\sin \pi z}.
\end{equation*}

\textbf{Дигамма-функция}. По определнию $\psi(z)$:
\begin{equation*}
    \psi(z) \overset{\mathrm{def}}{=}  \left(\ln \Gamma(z)\right)' = \frac{\Gamma'(z)}{\Gamma(z)}.
\end{equation*}
Заметим, что $\psi(1) = - \gamma$, где $\gamma$ -- постоянная Эйлера-Маскерони. Найдём
\begin{equation*}
    \psi(z+1)= \left(\ln z + \ln \Gamma(z)\right) = \frac{1}{z} + \psi(z)/
\end{equation*}
Забавный факт:
\begin{equation*}
    \psi(N+1) = \frac{1}{N} + \psi(N) = \sum_{n=1}^{N} \frac{1}{n} + \psi(1),
\end{equation*}
где $\sum_{n=1}^{N} \frac{1}{n}$ -- $N$-е гармоническое число. 


Также найдем, что
\begin{equation*}
    \psi(x + N + 1) = \frac{1}{x + N} + \psi(x+ N) = \frac{1}{x+N} + \ldots + \frac{1}{x+1} + \psi(x+1).
\end{equation*}
Вспомним, что $\Gamma(z) \Gamma(1-z) = \frac{\pi}{\sin \pi z}$. Тогда
\begin{equation*}
    \psi(-z) - \psi(z) = \pi \ctg \pi z.
\end{equation*}
Найдём асимптотику 
\begin{equation*}
    \Gamma(z \to \infty) = \sqrt{2 \pi z} e^{z \ln z - z} = \sqrt{2 \pi z} \left(\frac{z}{e}\right)^{z},
\end{equation*}
что и составляет формулу Стирлинга. 

Также для $\psi(z\to \infty)$:
\begin{equation*}
    \psi(z\to \infty) = \left(\ln \Gamma(z)\right)' = \ln z + \frac{1}{2z} + o(1) = \ln z + o(1).
\end{equation*}




% асимптотика интегралов Лапласа
\textbf{Метод перевала}.  Представим семейство интегралов с параметром $\lambda$:
\begin{equation*}
    I_\lambda = \int_{-\infty}^{+\infty} g(x) e^{\lambda f(x)} \d x.
\end{equation*}
При этом предположим, что $f(x)$ такая, что существует единственный максимум в точке $x_0$. Тогда
\begin{equation*}
    I_\lambda \approx g(x_0) \int_{-\infty}^{+\infty} e^{\lambda f(x)} \d x.
\end{equation*}
Теперь воспользуемся аналитичностью функции $f(x)$:
\begin{equation*}
    f(x) = f(x_0) + \frac{f'(x_0)}{2} (x-x_0)^2 + \frac{f''(x_0)}{2} (x-x_0)^2 + o(x-x_0)^2.
\end{equation*}
Подставляя в интеграл, находим
\begin{equation*}
    I_\lambda=  g(x_0) e^{\lambda f(x_0)} \int_{-\infty}^{+\infty} e^{\lambda f'(x_0) (x-x_0)^2/2} \d x = g(x_0) e^{\lambda f(x_0)} \sqrt{\frac{2\pi}{|\lambda f'' (x_0) |}}.
\end{equation*}

Пусть $\lambda$ нет. Тогда достаточно потребовать $|f''(x_0)|$ большой -- максимум резкий. 
Тогда
\begin{equation*}
    |f''(x_0) (x-x_0)^2| \sim 1,
    \hspace{0.5cm} \Rightarrow \hspace{0.5cm}   
    |x-x_0| \frac{1}{\sqrt{f'(x_0)}},
    \hspace{0.5cm} \Rightarrow \hspace{0.5cm}   
    |f'''(x_0) (x-x_0^3)| \ll 1,
    \hspace{0.5cm} \Rightarrow \hspace{0.5cm}
    (f'')^3 \gg (f''')^2.
\end{equation*}

Посмотрим на $\Gamma$-функцию:
\begin{equation*}
    \Gamma(z+1) = \int_{0}^{\infty}  t^{z} e^{-t} \d t  = \int_0^\infty 
    e^{z \ln t - t} \d t.
\end{equation*}
Тогда $f(t) = z \ln t - t$. Подставляем в критерий, видим что макимум у $f$ резкий.

Подставляем, находим
\begin{equation*}
    \Gamma(z+1) \approx e^{z \ln z - z} \sqrt{2 \pi z},
\end{equation*}
что и составляет формулу Стирлинга, верной на всей комплексной плоскости.















% \section{Семинар от 31.10.21 (функция Эйри)}

% Рассмотрим решение уранения
% \begin{equation*}
%     f'' - x f  = 0,
%     \hspace{10 mm} 
%     f(x \to + \infty) \to 0,
% \end{equation*}
% которое называется функцией Эйри. 






% 12:30
% accuracy:

% (False, False): 389, 
% (True, True): 179
% (False, True): 7
% (True, False): 15}

 % acc
 % Sensitivity 

 % (False, False): 369, 
% (True, True): 139
% (False, True): 27
% (True, False): 55}



% Brain tumor localization and segmentation from magnetic resonance imaging (MRI) are hard and important tasks for several applications in the field of medical analysis.


% Automatic segmentation of brain tumors from medical images is important for clinical assessment and treatment planning of brain tumors. Recent years have seen an increasing use of convolutional neural networks (CNNs) for this task


подходящий
% suitable
% 

So, we decided to take advantage of convolutional neural networks, work with brain images:
in particular, we made some successes in solving the problem
Brain tumor localization and segmentation from magnetic resonance imaging (MRI) 

its important tasks for several applications in the field of medical analysis.

The slide shows an example of MRI Image and the area with anomaly, which is necessary to allocate.


The development of the algorithm looks like this: 

A MRI picture is  the neuralnet input and 
expected output: the mask of the brain tumor area.

we minimize the difference on the training set,
 then test out neural net on data, that Neuranet before did not see

 Results are presented on the slide,

For better clarity, results are given in training on different amounts of data, the second column corresponds to two times more volume of the training data.
 As you can see the generalizing ability of the algorithm  very sensitive to data. 

 However, in any case, it was possible to correctly detect 94 percent of cases of the disease, which is a fairly good result in this area.





Moreover, the algorithm with an accuracy of 96 percent correctly segmented the area of the disease, 

The slide is demonstrated for comparison made by manual segmentation, and segmentation obtained automatically using our algorithm.

So, it can potentially easily simplify (automate), speed up and improve solving the problem of Brain Tumor Localization and Segmentation, especially after learning and testing on large amounts of data.


The amazing feature of the deep learning algorithms, which I would like to highlight, that a very wide class of tasks related with segmenting/classification  the image or signals can be resolved (Automated improved) 

only by the presence of a sufficient number of marked examples for learning









