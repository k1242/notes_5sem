% document's head

\phantom{42}
\vspace{20mm}

\begin{center}
    \LARGE \textsc{Лабораторная работа} \\
    \vspace{3 mm}
    \large Измерение частоты темнового счёта и шум-фактора твердотельного фотоумножителя SiPM
\end{center}

% \hrule

\phantom{42}

\begin{flushright}
    \begin{tabular}{rr}
    % written by:
        % \textbf{Источник}: 
        % & \href{__ссылка__}{__название__} \\
        % & \\
        % \textbf{Лектор}: 
        % & _ФИО_ \\
        % & \\
        \textbf{Автор работы}: 
        & Хоружий Кирилл \\ 
        & Кузнецова Арина \\
        & Евгений Дедков \\
        & Александр Двуреченский \\
        & Яушев Михаил  \\ 
        & \\
    % date:
        \textbf{От}: &
        \textit{\today}\\
    \end{tabular}
\end{flushright}

\thispagestyle{empty}

\vspace{10mm}


\subsection*{Цель работы}
Измерение частоты темнового счёта и шум-фактора твердотельного фотоумножителя SiPM.

\subsection*{Оборудование}
Источник-измеритель Keithley 236 (smu), измерительный модуль\,(фильтр низких частот, фотодетектор, фильтр высоких частот, усилитель), длинная линия с волновым сопротивлением 50\,Ом, осциллограф Tektronix TDS7104, пинцет.

\begin{figure}[h]
    \centering
    \includegraphics[width=0.5\textwidth]{figures/exp.png}
    \caption{Блок-схема измерительной установки}
    \label{fig:exp}
\end{figure}

Источник измеритель smu, состоящий из управляемого источника напряжения vs и измерителя тока, через фильтр низких частот измерительного модуля обеспечивает фотодетектор напряжением обратного смещения. 

Сигнал с фотодетектора проходит через фильтр высоких частот и усиливается при помощи amp. Сигнал с выход усилителя попадает в длинную линию и передается на вход осциллографа scope. 


\newpage



% дописать в заголовочный файл перечекнутую const


% \subsection*{Оптическая глубина}


Интенсивность слабого одиночного луча, проходящего через ячейку описывается законом Бэра:
\begin{equation*}
    d I / d x = - \alpha I,
    \hspace{10 mm} 
    \alpha = \alpha (\nu).
\end{equation*}
В хорошем приближение\footnote{
    Для слабого луча [1].
}  $\alpha \neq \alpha(x)$ . 
Введем оптическую длину $\tau(\nu) = l \alpha(\nu)$, тогда
\begin{equation*}
    \sub{I}{out} = \sub{I}{in} e^{- \alpha(\nu) l} = \sub{I}{in} e^{- \tau(\nu)}.
\end{equation*}
Вклад от группы атомов $(v,\, v + d v)$ в $\tau(\nu)$ можем быть записан, как
\begin{equation*}
    d \alpha (\nu, v) = \sigma(\nu, v) \, dn(v),
    \hspace{0.5cm} \Rightarrow \hspace{0.5cm}    
    d \tau(\nu, v) = l \sigma(\nu, v) \, dn(v).
\end{equation*}
Коэффициент поглощения $\sigma(\nu, v)$ имеет Лоренцовский профиль с натуральной шириной $\Gamma$ (?) и смещенной по Допплеру резонансной частотой
\begin{equation}
    \label{1_4}
    \sigma(\nu, v) = \sigma_0 \frac{\Gamma^2/4}{(\nu - \nu_0(1 - v / c))^2 + \Gamma^2/4},
\end{equation}
где $\sigma_0$ -- резонансное сечение поглощения\footnote{
    [1], problem 1: $\sigma_0 \sim n$ атомов в ячейке.
} , зависящее от вида дипольного перехода и поляризации падающего света [1$\Rightarrow$4]. 

Часть атомов $d n (v)$ с определенной скорости можем найти из распределения Больцмана
\begin{equation*}
    d n(v) = n_0 \sqrt{\frac{m}{2 \pi \sub{k}{B} T}} \exp\left(
        - \frac{m v^2}{2 \sub{k}{B} T} 
    \right) \d v,
\end{equation*} 
где $n_0 = N/V$ -- концентрация атомов в ячейке. 

Собирая все вместе (?) приходим к выражению
\begin{equation}
    \label{1_6}
    d \tau (\nu, v) = \frac{2}{\pi} \frac{\tau_0}{\sigma_0 \Gamma} \frac{\nu_0}{c} 
     \sigma(\nu, v) \exp\left(
        - \frac{m v^2}{2 \sub{k}{B} T}
     \right) \d v,
\end{equation}
где $\tau_0$  -- сответствующая нормировка такая, что для резонанса $\tau_0 = \int_v \d \tau (\nu_0,\, v)$.



Для насышенной спектроскопии нужно учесть эффект от дополнительного насыщающего лазерного луча. Из-за него значительная часть атомов в ячейке будут в возбужденном состоянии. Так как атомы могут поглощать свет только когда они в невозможденном состоянии, к \eqref{1_6} добаваить фактор $(\ng-\ne)/N$, описывающей разницу 
между количеством атомов в возбужденном состоянии $\ne$ и невозбужденном $\ng$. 



\subsection*{Скоростные уравнения}


 Населенность в двух состояния описывается скоростными уранениями
\begin{align*}
    \dot{\ng} = \phantom{-}\Gamma N_e - \sigma \Phi (\ng - \ne), \\
    \dot{\ne} = - \Gamma \ne + \sigma \Phi (\ng - \ne),
\end{align*}
где первое слагаемое отвечает спонтанной эмиссии, и второе насыщению лазером. $\Phi = I / h \nu$ -- насыщающий поток фотонов. Учитывая, что $\ng + \ne = N = \const $, можем получить диффур первого порядка на $\ne$:
\begin{equation*}
    \dot{\ne} = - (\Gamma \ne + 2 \sigma \Phi) \ne + \sigma \Phi N.
\end{equation*}
Решение можем быть найдено в виде
\begin{equation*}
    \ne (t) = \left[
        \Gamma \ne(0) - \frac{N \sigma \Phi}{\Gamma + 2 \sigma \Phi} 
    \right] e^{-(\Gamma + 2 \sigma \Phi) t} + \frac{N \sigma \Phi}{\Gamma + 2 \sigma \Phi}.
\end{equation*}

Заметим, что при $\Phi = 0$:
\begin{equation*}
    \ne (t) = \ne (0) e^{- \Gamma t},
\end{equation*}
а в случае слаюого насыщающего луча $\sigma \Phi \ll \Gamma$, и изначальной популяции в невозбужденном состоянии,
\begin{equation*}
    \ne (t) = \frac{N \sigma \Phi}{\Gamma} \left(1 - e^{- \Gamma t}\right),
\end{equation*}
достигающий стационарного состояния после $\Gamma^{-1}$ с $\ne = N \sigma \Phi / \Gamma \ll N$.   Наконец, при $\sigma \Phi \gg \Gamma$, получаем насыщенный переход
\begin{equation*}
    \ne (t) = \left[\ne(0) - N/2\right]e^{-2 \sigma \Phi t} + N / 2 \to N/2.
\end{equation*}
Под насыщением понимаем, что $\ne = N/2$, большие значения по понятным причинам невозможны $\forall  \Phi$, по крайней мере для двухуровневых систем. 

Также наблюдается увеличение <<мощности>> ширины линии перехода, в пределе $(\Gamma + 2 \sigma \Phi) t \gg 1$, получаем
\begin{equation*}
     \frac{\ne (\infty)}{N} = \frac{\sigma \Phi}{\Gamma + 2 \sigma \Gamma}.
\end{equation*} 
Вспоминая уранение \eqref{1_4} с $\Delta \nu = \nu - \nu_0(1+  v /c)$ (минус, т.к. допплеровский сдвиг в другую сторону), можем переписать уравнение в виде
\begin{equation*}
    \frac{\ne(\infty)}{N} = \frac{\sigma_0 \Phi \Gamma / 4}{\Delta \nu^2 + \Gamma^2/4 + \sigma_0 \Phi \Gamma / 2},
    \hspace{0.5cm} \Rightarrow \hspace{0.5cm}
    \boxed{
        \frac{\ne}{N} = \frac{s/2}{1 + s + 4 \Delta \nu^2/\Gamma^2}
    }
\end{equation*}
, где ввели параметр насыщения $s = \Phi/\sub{\Phi}{sat}$, $\sub{\Phi}{sat} = \Gamma/2 \sigma_0$.

Получился лоренцев профиль с уширением, полуширина (FWHM) которого зависит от $\Phi$:
\begin{equation*}
    \text{FWHM} = \frac{\Gamma}{2} \sqrt{1 + \frac{2 \sigma_0 \Phi}{\Gamma}}.
\end{equation*}

% Таким образом можем посчитать простую насыщенную спектроскопию для двухуронего атома. 

Интенсивность насыщения $\sub{I}{sat}$ может быть выражена, как [1 $\Rightarrow$ 4] 
\begin{equation*}
    \sub{I}{sat} = 2 \pi^2 h c \Gamma / 3 \lambda^3.
\end{equation*}
Например, для $^{87} \text{Rb}$ с натуральной шириной $\Gamma = 6$ МГц, $\sub{I}{sat} = 1.65 \text{ мВт}/\text{см}^2$.


\subsection*{Итоговая картина для двухуровнего атома}


Собираем всё вместе, в зависимости от мощности насыщающего лазера некоторое количество атомов будет находиться в возбужденном состоянии:
\begin{equation*}
    \frac{\ne}{N} = \frac{s/2}{1 + s + 4 (\Delta_+ \nu)^2/\Gamma^2},
    \hspace{5 mm} 
    \frac{\sigma(\nu, v) }{\sigma_0} = \frac{1}{4 (\Delta_{-}\nu)^2 / \Gamma^2 + 1},
    \hspace{5 mm} 
    \Delta_{\pm} \nu = \nu - \nu_0(1\pm  v /c).
\end{equation*}
Тогда
\begin{equation}
    \frac{\sub{I}{out}}{\sub{I}{int}} = \exp\left[
        - \kappa
        \int_{-\infty}^{\infty} 
        \left(
            1 - 2 \frac{N_e (\nu, v)}{N}
        \right)
        \frac{\sigma(\nu, v)}{\sigma_0}  \exp\left(
        - \frac{m v^2}{2 \sub{k}{B} T} 
    \right) \d v
    \right]
    , 
\end{equation}
где
\begin{equation*}
    s = \Phi/\sub{\Phi}{sat},
    \hspace{5 mm} 
    \sub{\Phi}{sat} = \Gamma/2 \sigma_0,
    \hspace{5 mm}
    \kappa = \sigma_0 n l \sqrt{\frac{m}{2 \pi \sub{k}{B} T}},
\end{equation*}
можно подставить, но пока не нужно:
\begin{equation*}
    \frac{\sub{I}{out}}{\sub{I}{int}} = \exp\left[
        - \kappa
        \int_{-\infty}^{\infty} 
        \left(
            1 - \frac{s}{1 + s + 4 (\Delta_+ \nu)^2/\Gamma^2}
        \right)
        \frac{1}{4 (\Delta_{-}\nu)^2 / \Gamma^2 + 1}  \exp\left(
        - \frac{m v^2}{2 \sub{k}{B} T} 
    \right) \d v
    \right] = \exp\left[
        - \kappa F(s, \nu)
    \right]
    .
\end{equation*}



\newpage

\subsection*{Оценка контрастности}

В первом приближении, не зная значения $\kappa$, можем оценить его, зная глубину доплеровского провала в резнансе $\nu_0$. Введем для удобства приведенную интенсивность $\beta \overset{\mathrm{def}}{=}  \sub{I}{out}/\sub{I}{in}$, далее в этом разделе всегда полагаем $\nu = \nu_0$, тогда
\begin{equation*}
    \beta(s=0) \overset{\mathrm{def}}{=}  \beta_0 = e^{- \kappa F(0)},
    \hspace{0.5cm} \Rightarrow \hspace{0.5cm}
    \kappa = \frac{\ln 1/\beta_0}{F(0)},
\end{equation*}
где $1-\beta_0$ -- глубина доплеровского провала.

Тогда контрастность спектроскопии $K$, определенную, как отношение высоты лэмбоского пика к глубине доплеровского провала, можем найти, как
\begin{equation*}
    K(s) = \frac{e^{-\kappa F(s)}-e^{-\kappa F(0)}}{1-e^{-\kappa F(0)}} = 
    \frac{\beta_0^{F(s)/F(0)}-\beta_0}{1-\beta_0}.
\end{equation*}
Ниже на рисунке приведены значения контрастности $K(s)$ для различных $\beta$.

\begin{figure}[h]
    \centering
    \includegraphics[width=0.5\textwidth]{"D:\\Kami\\git_folder\\notes_5sem\\rqc\\saturation_spectr_simulation\\K.pdf"}
    %\caption{}
    %\label{fig:}
\end{figure}



% \section*{ТеорМин №1}
\addcontentsline{toc}{section}{ТеорМин №1}

\textbf{Вычеты}. Интеграл по дуге может быть найден, как
\begin{align*}
    \int_C f(z) \d z = 2 \pi i \sum_{z_j} \res_{z_j} f(z),
    \hspace{5 mm} 
    \res_{z_j} f(z) &= \lim_{\varepsilon \to 0} \varepsilon \int_0^{2\pi} \frac{\d \varphi}{2\pi} e^{i \varphi} f(z_j + \varepsilon e^{i \varphi}) \\ 
    &= \frac{1}{(m-1)!} \lim_{z \to z_j} \left(
        \frac{d^{m-1} }{d z^{m-1}} (z-z_j)^m f(z)
    \right),
\end{align*}
где $m$ -- степень полюса. 



\begin{to_lem}[лемма Жордана]
    Пусть $f(z)$ непрерывна в замкнутой области $G = \{z \mid \Im z \geq 0,\,  |z| \geq R_0 > 0\}$. Обозначим через $C_R$ полуокружность $|z| = R,\, \Im x \geq 0$ и пусть верно, что $\lim_{R \to \infty} \max |f(z)| =0$. тогда при $a > 0$
    \begin{equation*}
        \lim_{R \to \infty} \int_{C_R} f(z) e^{i a z} \d z = 0,
    \end{equation*}
    аналогичное верно при $C_R$ с $\Im x \leq 0$ и $a < 0$. 
\end{to_lem}



\textbf{Функция Грина}. Всегда и всюду, уравнение вида
\begin{equation*}
    L x(t) = \varphi(t), 
    \hspace{5 mm}   
    x(t) = \int_{-\infty}^{t} G(t-s) \varphi(s) \d s,
    \hspace{5 mm} 
    L G = \delta(t).
\end{equation*}
И, если хочется добавить начальные условия, то например, для $L = \partial_t^2$ будет
\begin{equation*}
    x(t) = \dot{x}(0) G(t) + x(0) \dot{G}(t) + \int_{0}^{t} G(t-s) \varphi(s) \d s.
\end{equation*}




\textbf{Матричное уравнение}. Решение линейного уравнения для векторной величины $\vc{y}$
\begin{equation*}
    \frac{d \vc{y}}{d t} + \hat{\Gamma} \vc{y} = \vc{\chi},
\end{equation*}
может быть найдено, через функцию Грина, вида
\begin{equation*}
    \hat{G} (t) = \theta(t) \exp\left(- \hat{\Gamma} t\right),
    \hspace{10 mm} 
    \vc{y}(t) = \int_{-\infty}^{t}  \hat{G}(t-s) \vc{\chi}(s) \d s.
\end{equation*}


% \textbf{Преобразование Фурье}. 



\textbf{Преобразование Лапласа}. Преобразование Лапласа функциии $\Phi(t)$ определяется, как
\begin{equation*}
    \tilde{\Phi}(p) = \int_{0}^{\infty}  \exp(-pt) \Phi(t) \d t,
    \hspace{10 mm} 
    \Phi(t) = \int_{c-i \infty}^{c+i \infty} \frac{\d p}{2 \pi i} \exp(pt) \tilde{\Phi}(p),
\end{equation*}
где далее $c$ выбираем правее всех особенностей для причинности. 

Решение уравнения $L(\partial_t) G(t) = \delta(t)$ может быть найдено, как
\begin{equation*}
    G(t) = \int_{c-i \infty}^{c+i \infty} \frac{\d p}{2 \pi i} \exp(p t) \tilde{G}(p),
    \hspace{10 mm} \tilde{G} (p) = \frac{1}{L(p)},
    \hspace{0.5cm} \Rightarrow \hspace{0.5cm}
    G(t) = \sum_i \res_i \frac{\exp(pt)}{L(p)},
\end{equation*}
где суммирование идёт по полюсам $1/L(p)$. 

Важно, что можно делать функции маленькими
\begin{equation}
    \int_{p_0 - i \infty}^{p_0 + i \omega} \tilde{f}(p) e^{pt} \frac{\d p}{2 \pi i} = 
    \left(\frac{d }{d t} \right)^n \int_{p_0 - i \infty}^{p_0 + i \omega} \frac{\tilde{f}(p)}{p^n} e^{pt} \frac{\d p}{2 \pi i}.
\end{equation}




\textbf{Уравнение Вольтерра}. Интегральное уравнение Вольтерра первого рода с однородным ядром:
\begin{equation*}
    \int_{0}^{t}  K(t-s) f(s) \d s = \varphi(t).
\end{equation*}
Решение может быть найдено через обратное преобразование Лапласа
\begin{equation*}
    f(t) = \int_{c-i \infty}^{c+i \infty} \frac{d p}{2 \pi i} \exp(pt) \tilde{f}(p),
    \hspace{10 mm} 
    \tilde{f}(p) = \frac{\tilde{\varphi}(p)}{\tilde{K}(p)}.
\end{equation*}
Но есть один нюанс. При $K(t),\, \varphi(t) \overset{p \to \infty}{\to} K_0,\, \varphi_0$ получается, что $\tilde{K}(p),\, \tilde{\varphi}(p) \approx \frac{K_0}{p},\, \frac{\varphi_0}{p}$, тогда
\begin{equation*}
    f(t) = \frac{\varphi_0}{K_0} \delta(t) + \int_{c-i \infty}^{c+i \infty} \frac{d p}{2 \pi i} \exp(p t)
    \left(
        \frac{\tilde{\varphi}}{\tilde{K}} - \frac{\varphi_0}{K_0}
    \right),
\end{equation*}
при этом в отсутствие аналитичности в нуле нет ничего страшного. 


\textbf{Неоднородная релаксация}. Для одномерного случая
\begin{equation*}
    \big(\partial_t + \gamma(t)\big) x(t) = \varphi(t),
    \hspace{0.5cm} \Rightarrow \hspace{0.5cm}
    x(t) = \int_{-\infty}^{+\infty}  G(t,s) \varphi(s) \d s,
    \hspace{5 mm} 
    G(t,\,  s) = \theta(t-s) \exp\left(
        - \int_{s}^{t} \gamma(\tau) \d \tau
    \right),
\end{equation*}
где всё также $G(t, s>t) = 0$ в силу стремления к принципу причинности. 


% многомерная неоднородная релаксация


% \section{Неделя I}

\subsection*{№1}

Рассмотрим уравнение на $G(t)$
\begin{equation}
    (\partial_t + \gamma) G(t) = \delta(t),
    \label{w1_eq1}
\end{equation}
с учетом принципа причинности $g(t<0) = 0$. 

При $t > 0$ $\delta(t) = 0$,  так что
\begin{equation*}
    \partial_t G(t) = - \gamma G(t),
    \hspace{0.5cm} \Rightarrow \hspace{0.5cm}
    G(t) = A \exp(- \gamma t).
\end{equation*}
Проинтегрируем уравнение \eqref{w1_eq1} от $-\varepsilon$ до $\varepsilon$:
\begin{equation*}
    G(\varepsilon) - \cancel{G(-\varepsilon)} + \cancel{\int_{-\varepsilon}^{\varepsilon} \gamma G(t) \d t } = \int \delta(t) \d t = 1,
    \hspace{0.5cm} \Rightarrow \hspace{0.5cm}
    G(\varepsilon) = 1, 
    \hspace{0.5cm} \Rightarrow \hspace{0.5cm}
    A = 1.
\end{equation*}
Таким образом, искомая функция Грина $G(t)$:
\begin{equation*}
    G(t) = \theta(t) \cdot \exp\left(- \gamma t\right),
\end{equation*}
где $\theta(t)$ обеспечивает $G(t) = 0$ при $t<0$.




\subsection*{№2}

Рассмотрим уравнение, вида
\begin{equation*}
    (\partial_t^2 + \omega^2) \varphi(t) = g(t),
    \hspace{5 mm}   
    g(t) = \left\{\begin{aligned}
        &0, &t\notin [0, \tau]; \\
        &- \tfrac{v}{\tau l}, &t \in [0, \tau], \\
    \end{aligned}\right.
    \label{w1_t2}
\end{equation*}
с нулевым начальным условием $\varphi(t<0)=0$.
Функция Грина $G(t)$ для оператора $(\partial_t^2 + \omega^2)$ равна\footnote{
    Конспект, уравнение (1.11).
} 
\begin{equation*}
    G(t) = \theta(t) \frac{1}{\omega} \sin(\omega t).
\end{equation*}
Далее найдём вид $\varphi(t)$ при $t < \tau$ (красная линия рис. \ref{fig:I2}):
\begin{equation*}
    \varphi(t < \tau) = \frac{1}{\omega} \int_{-\infty}^{t} \sin \omega(t-s) \ g(s) \d t = \frac{1}{\omega} \int_0^t \sin \omega(t-s) \frac{v}{2l} \d (t-s) = 
    \frac{v}{l \tau} \frac{1}{\omega^2} \left(
        \cos(\omega t) - 1
    \right).
\end{equation*}

\begin{figure}[ht]
    \centering
    \includegraphics[width=0.5\textwidth]{figures/T2.pdf}
    \caption{Сшивка решений в I.2}
    \label{fig:I2}
\end{figure}

Теперь решим\footnote{
    Конспект, уравнение (1.12).
}  задачу Коши с начальным условием при $t = \tau$, введя переменную $T = t-\tau$:
\begin{equation*}
    \varphi(T) = \varphi(t-\tau) = \dot{\varphi}(\tau) G(t-\tau) + \varphi(\tau) \dot{G}(t-\tau) + 0 = 
    \frac{v}{lt} \frac{1}{\omega^2} \left(
        \cos \omega t - \cos \omega(t-\tau)
    \right).
\end{equation*}
 получая синюю кривую на рис. \ref{fig:I2}.

 Итого, решение уравнения \eqref{w1_t2} (фиолетовая кривая, рис \ref{fig:I2}):
 \begin{equation*}
     \varphi(t) = \frac{v}{l \tau} \frac{1}{\omega^2}\left\{\begin{aligned}
        &0
        , &t < 0; \\
        & \cos \omega t - 1
        , &t \in [0, \tau]; \\
        & \cos \omega t - \cos \omega(t-\tau)
        , &t > \tau. \\
     \end{aligned}\right.
 \end{equation*}


 \subsection*{№3}


 Найдём значение интеграла, вида
 \begin{equation*}
     I_1= \int_{-\infty}^{+\infty} \frac{1}{(x^2 + a^2)^2} \d x.
 \end{equation*}
 Заметим, что уравнение $z^2 + a^2 = 0$ имеет корни в $z_{1, 2} = a^{\pm i \pi/2}$, тогда
 \begin{equation*}
     I_1 = 2 \pi i \cdot \text{res}_{z_1} = 2 \pi i 
     \lim_{z \to z_1} \cdot \left(
        \frac{1}{(z- z_2)^2}
     \right)' = -4 \pi i \cdot \lim_{z \to z_1} \left(
        \frac{1}{(z-z_2)^3}
     \right) = - 4 \pi i \frac{1}{(2 i a)^3} = \frac{\pi}{2 a^3}.
 \end{equation*}


% \textbf{Фотоэффект}. Максимальная кинетическая энергия, которой будут обладать электроны, вылетевшие при фотоэффекте определяется формулой Эйнштейна:
\begin{equation*}
    \frac{1}{2} \sub{m}{e} \sub{v}{max}^2 = \hbar \omega - A.
\end{equation*}

\textbf{Эффект Комптона}, -- изменение длины волны $\lambda' - \lambda$ в длинноволновую сторону спектра при рассеянии излучения. Смещение не зависит от состава тела и длины падающей волны,  но пропорционально $\sin^2 (\theta/2)$, где $\theta$ -- угол рассеяния. Рассмотрев упругое столкновение фотона и электрона, можем получить:
\begin{equation*}
    \frac{\sub{\E}{ph} \sub{\E}{ph}'}{c^2} + \frac{\sub{\E}{ph}' \sub{\E}{0}}{c^2} - 
    \frac{\sub{\E}{ph} \sub{\E}{0}}{c^2} - \sub{\vc{p}}{ph} \cdot \sub{\vc{p}}{ph}' = 0,
    \hspace{5 mm} \Leftrightarrow \hspace{5 mm} 
    1 - \cos \theta = \sub{m}{e} c \left(\frac{1}{\sub{p}{ph}'}-\frac{1}{\sub{p}{ph}}\right),
\end{equation*}
где $\theta$ -- угол рассеяния, т.е. угол между $\sub{\vc{p}}{pf}$ и $\sub{\vc{p}}{pf}$. Считая, что $\sub{p}{ph}' = h/\lambda'$ и $\sub{p}{ph} = h/\lambda$, находим
\begin{equation*}
    \lambda' - \lambda = \sub{\lambda}{K} (1- \cos \theta)  = 2 \sub{\lambda}{K} \sin^2 \frac{\theta}{2},
    \hspace{0.5cm} \Rightarrow \hspace{0.5cm}
    \sub{\lambda}{K} = \frac{h}{\sub{m}{e} c}  = 2.43 \cdot 10^{-3} \text{ нм},
\end{equation*}
так и находим комптоновскую длину\footnote{
    Формально, $\sub{\lambda}{K}$ можно рассматривать, как длину волны де Бройля, которой соответетствует величина импульса, равная инвариантной длине четырехмерного вектора энергии-импульса в пространстве Минковского. 
}  для электрона. Также можно встретить приведенную комптоновскую длину для электрона
\begin{equation*}
    \sub{\lambdabar}{K}  = \frac{\hbar}{\sub{m}{e} c} = \frac{\sub{\lambda}{K}}{2 \pi} = 3.86 \cdot 10^{-4} \text{ нм},
\end{equation*}
где электрон предполагается неподвижным. Движущийся электрон может передать свою энергию фотону, а сам остановиться -- обратный эффект Комптона.  Несмещенная компонента возникает из рассеяния на связанных электронах. 

Также можем посмотреть на направление вылета электрона отдачи:
\begin{equation*}
    \tg \varphi = \frac{\ctg(\theta /2)}{1 + \hbar \omega / (\sub{m}{e} c^2)}.
\end{equation*}



% Лаврентьев Шаббат -- более строгий рассказ.

\section{Семинар от 25.09.21 (Фурье и Лаплас)}

% трансляционно инвариантна система
\textbf{Про Фурье}. 
Как раньше нашли
\begin{equation*}
    L(\partial_t) G(t) = \delta(t),
    \hspace{0.5cm} \Rightarrow \hspace{0.5cm}
    \hat{x} (\omega) = \int_{\mathbb{R}} e^{- i \omega t} x(t) \d t,
    \hspace{5 mm} 
    x(t) = \int_{\mathbb{R}} e^{i \omega t} \hat{x}(\omega) \frac{\d \omega}{2\pi}.
\end{equation*}
Для этого должно выполняться
\begin{equation*}
    \int |x(t)| \d d < + \infty.
\end{equation*}
\textit{Например}, для $\partial_t + \gamma$:
\begin{equation*}
    (\partial_t + \gamma) G(t) = \delta(t),
    \hspace{0.5cm} \Rightarrow \hspace{0.5cm}
    \int_{\mathbb{R}} \frac{\d t}{\ldots} e^{- i \omega t} \d t = x(t) e^{- i \omega t} \bigg|_{-\infty}^{+\infty},
    \hspace{0.5cm} \Rightarrow \hspace{0.5cm}
    (i \omega + \gamma) \hat{G} (\omega) = 1,
    \hspace{0.5cm} \Rightarrow \hspace{0.5cm}
    \hat{G}(\omega) = \frac{1}{i \omega + \gamma}.
\end{equation*}
Так приходим к уравнению
\begin{equation*}
    G(t) = \int_{\mathbb{R}} \frac{e^{i \omega t}}{\omega - i \gamma} \frac{\d \omega}{2\pi} = 
    \left\{\begin{aligned}
        &e^{- \gamma t}, &t > 0
        &0, &t<0
    \end{aligned}\right.
    \hspace{0.5cm} \Rightarrow \hspace{0.5cm}
    \hat{G}(\omega) = \theta(t) e^{- |t|}.
\end{equation*}

Однако, при $\hat{L} = \partial_t - \gamma$ мы бы получили
\begin{equation*}
    G_A (t) = \theta(-t) e^{\gamma t}, 
\end{equation*}
хотя вообще должно быть (если посчитать через неопределенные коэффициенты)
\begin{equation*}
    G_R (t) = \theta(t) e^{\gamma t},
\end{equation*}
которая растёт.


В методе с Фурье будут получаться функции Грина затухающие, но, возможно, без причинности. 
В методе неопределенных коэффициентов исходим из причинности, но может быть рост $\sim e^{\gamma t}$. 


Кроме того, в Фурье всегда предполагается $x(t \to - \infty) = 0$ и $x(t \to + \infty) = 0$. Также может случиться
\begin{equation*}
    (\partial_t^2 + \omega_0^2) G(t) = \delta(t),
    \hspace{0.5cm} \Rightarrow \hspace{0.5cm}
    \hat{G}(\omega) = \frac{1}{\omega^2 - \omega_0^2},
\end{equation*}
с особенностями на вещественной оси, что можно решить, сместив полюса в $\mathbb{C}$.





\textbf{Свёртка}. Рассмотрим уравнение
\begin{equation*}
    L(\partial_t) x(t) = f(t),
    \hspace{5 mm} 
    L(\partial_t) G(t) = \delta(t).
\end{equation*}
Фурье переводит 
\begin{equation*}
    \int_{\mathbb{R}} \partial_x^n x(t) e^{- i \omega t} \d t = (i \omega)^n \hat{x} (\omega).
\end{equation*}
Тогда
\begin{equation*}
    L(i \omega) \hat{x} (\omega) = \hat{f}(\omega),
    \hspace{5 mm} 
    L(i \omega)  \hat{G} (\omega) = 1,
    \hspace{0.5cm} \Rightarrow \hspace{0.5cm}
    \hat{G}(\omega) = \frac{1}{L(i \omega)}.
\end{equation*}
Также нашли, что
\begin{equation*}
    \hat{x}(\omega) = \frac{\hat{f}(\omega)}{L(i \omega)} = \hat{f}(\omega) \hat{G}(\omega),
    \hspace{0.5cm} \Rightarrow \hspace{0.5cm}
    x(t) = \int_{-\infty}^{+\infty} G(t-s) f(s) \d s.
\end{equation*}



\textbf{Преобразование Лапласа}. Пусть есть некоторое преобразование
\begin{equation*}
    \tilde{f}(p) = \int_0^{\infty} e^{- p t} f(t) \d t,
\end{equation*}
где подразумевается, что $\Re p \geq 0$ и, вообще, в Фурье можно $p \in \mathbb{C}$. 

Пусть $p = i \omega$, где $\omega \in \mathbb{R}$. Тогда
\begin{equation*}
    \tilde{f} (i \omega) = \int_{\mathbb{R}} e^{- i \omega t} f(t) \d t = \hat{f} (\omega),
    \hspace{0.5cm} \Rightarrow \hspace{0.5cm}
    f(t)  =\int_{\mathbb{R}} \hat{f}(\omega) e^{i \omega t} \frac{\d \omega}{2\pi} = 
    \int_{\mathbb{R}} \tilde{f} (i \omega) e^{i \omega t} \frac{\d \omega}{2 \pi} = 
    \int_{- i \infty}^{i \infty} e^{p t} \tilde{f} (p) \frac{\d p}{2\pi}.
\end{equation*}
В вычислениях выше мы предполагали, что $f(t \to \infty) = 0$. 

Обойдём это, пусть $|f(t)| < M e^{s t}$, при $s > 0$. Возьмём $p_0 > s$, тогда
\begin{equation*}
    \tilde{f} (p) = \int_{\mathbb{R}} e^{- p_0 t} e^{-(p-p_0)t} f(t) \d t = \tilde{g}(p-p_0),
\end{equation*}
где вводе $g(t) = e^{- i p_0 t} f(t)$, которая уже убывает на бесконечности. Обратно:
\begin{equation*}
    g(t) = \int_{-i\infty}^{+i\infty}  \tilde{g}(p) e^{pt} \frac{\d p}{2\pi} = 
    \int_{p_0 - i \omega}^{p_0 + i \omega} \tilde{g}(p-p_0) e^{- p_0 t} e^{pt} \frac{\d p}{2\pi i}.
\end{equation*}
Так пришли к форме обращения
\begin{equation}
    f(t) = \int_{p_0 - i \infty}^{p_0 + i \omega} \tilde{f}(p) \frac{\d p}{2 \pi i},
    \hspace{10 mm} 
    \tilde{f}(p) = \int_{\mathbb{R}} e^{- p_0  t} e^{-(p-p_0)t} f(t) \d t = \tilde{g}(p-p_0),
\end{equation}
где $g(t) = e^{- i p_0 t} f(t)$.



Забавный факт, из леммы Жордана: при $t < 0$ $f(t<0) = 9$, по замыканию дуги по часовой стрелке (вправо). Выбирая $p_0$ так, чтобы все особенности лежали левее $p_0$, можем получать причинные функции. 




\textbf{Производная}. Найдём преобразование Лапласа для $\partial_t f(t)$:
\begin{equation*}
    \int_{0}^{\infty} \frac{d f}{d t} e^{- p t} \d t = f e^{- pt} \bigg|_0^{\infty} + p \int_{0}^{\infty} 
    f(t) e^{-pt} \d t = p \tilde{f}(p) - f(+0).
\end{equation*}
Но, для функции Грина $L(\partial_t) G(t) = \delta(t)$, тогда
\begin{equation*}
    L(\partial_t) G_\varepsilon(t) = \delta(t-\varepsilon),
    \hspace{10 mm} 
    G_\varepsilon (t) = G(t-\varepsilon),
    \hspace{0.5cm} \Rightarrow \hspace{0.5cm}
    G_\varepsilon(0)=  0,
\end{equation*}
где $G_\varepsilon \to G(t)$ при $\varepsilon \to 0$.  

Преобразуем\footnote{
    Здесь и далее $f(t)$ -- функция, $f(\omega) = \hat{f}(\omega)$ -- Фурье образ, $f(p) = \tilde{f}(p)$ -- преобразование Лапласа.
}  по Лапласу уравнения выше
\begin{equation*}
     L(p) G(p) = e^{p \varepsilon} = 1,
     \hspace{0.5cm} \Rightarrow \hspace{0.5cm}
     G_\varepsilon (p) = \frac{1}{L(p)},
     \hspace{0.5cm} \overset{\varepsilon \to 0}{\Rightarrow}  \hspace{0.5cm}
     G(p) = \frac{1}{L(p)}.
 \end{equation*} 
Так получаем
\begin{equation}
    G(t) = \int_{p_0 - i \infty}^{p_0 + i \omega} \frac{e^{pt}}{\tilde{L}(p)} \frac{\d p}{2\pi i},
\end{equation}
где $p_0$ правее всех особенностей. 




\textbf{Пример}. Рассмотрим $L = \partial_t + \gamma$, тогда
\begin{equation*}
    (p+\gamma) G(p) = 1,
    \hspace{0.5cm} \Rightarrow \hspace{0.5cm}
    G(p) = \frac{1}{p + \gamma},
    \hspace{0.5cm} \Rightarrow \hspace{0.5cm}
    G(t) = \int_{-i \infty}^{i \infty} \frac{e^{pt}}{p+\gamma} \frac{\d p}{2 \pi i} = \theta(t) e^{- \gamma t}.
\end{equation*}
Аналогично, пусть $L = \partial_t^2 + \omega^2$, тогда $L G(t) = \delta(t)$, и
\begin{equation*}
    G(p) = \frac{1}{p^2 + \omega^2},
    \hspace{0.5cm} \Rightarrow \hspace{0.5cm}
    G(t) = \int_{p_0 - i \infty}^{p_0 + i \omega} \frac{e^{pt}}{p^2 + \omega^2} \frac{\d p}{2 \pi i} = 
    \left\{\begin{aligned}
        &0, &t < 0 \\
        &\ldots, &t>0
    \end{aligned}\right. 
    = \theta(t) \left(
        \frac{e^{i \omega t}}{2 i \omega} + \frac{e^{- i \omega t}}{- 2 i \omega t} = \theta(t) \frac{\sin \omega t}{\omega}.
    \right)
\end{equation*}

В общем виде, пусть $L(\partial_t) G(t) = \delta(t)$, тогда
\begin{equation*}
    L(p) G(p) = 1,
    \hspace{0.5cm} \Rightarrow \hspace{0.5cm}
    G(p) \frac{1}{L(p)},
    \hspace{0.5cm} \Rightarrow \hspace{0.5cm}
    G(t) = \int_{p_0 - i \infty}^{p_0 + i \omega} \frac{e^{p t}}{L(p)} \frac{\d p}{2 \pi i}.
\end{equation*}

Поговорим про свёртку:
\begin{equation*}
    L x  = f,
    \hspace{0.5cm} \Rightarrow \hspace{0.5cm}
    L(p) x(p) = f(p),
    \hspace{5 mm} 
    L(p) G(p)  =1,
    \hspace{0.5cm} \Rightarrow \hspace{0.5cm}
    G(p) = \frac{1}{L(p)}.
\end{equation*}
Тогда получается
\begin{equation*}
    x(p) = \frac{f(p)}{L(p)} = f(p) G(p),
    \hspace{0.5cm} \Rightarrow \hspace{0.5cm}
    x(t) = \int_{0}^{t} G(t-s) f(s) \d s.
\end{equation*}


\textbf{Уравнение Вольтера}. Иногда бывает уравнения на $x(s)$ вида
\begin{equation}
    f(t) = \int_{0}^{t} x(s) K(t-s) \d s.
\end{equation}
Через преобразрвание Лапласа, находим
\begin{equation}
    f(p) = x(p) K(p),
    \hspace{0.5cm} \Rightarrow \hspace{0.5cm}
    x(p) = \frac{f(p)}{K(p)}. 
\end{equation}
В общем виде тогда находим
\begin{equation*}
    x(t) = \int_{p_0 - i \infty}^{p_0 + i \omega} \frac{f(p)}{K(p)} e^{pt} \frac{\d p}{2\pi i}.
\end{equation*}
Кстати, забавный факт:
\begin{equation}
    \int_{p_0 - i \infty}^{p_0 + i \omega} 1 \cdot e^{pt} \frac{\d p}{2 \pi i} = e^{p_0 t} 
    \int_{-i \infty}^{i \infty} e^{p t} \frac{\d p }{2 \pi i} = 
    e^{p_0 t} \int_{-\infty}^{+\infty} e^{i \omega t} \frac{\d \omega}{2 \pi} = \delta(t),
\end{equation}
то есть преобразование Лапласа от константы -- дельта функция. 



Рассмотрим, например
\begin{equation*}
    \int_{-i \infty}^{i \infty} \frac{p + 1 -1}{p+1} e^{pt} \frac{\d p}{2 \pi i} = \int_{- i \infty}^{i \infty}  e^{p t} \frac{\d p}{2 \pi i} - \int_{-i\infty}^{+i\infty}  \frac{e^{pt}}{p+1} \frac{\d p}{2 \pi} = \delta(t) - \theta(t) e^{- t}.
\end{equation*}
Также верно, что
\begin{equation*}
    \int_{- i \infty}^{i \infty} p e^{pt} \frac{\d p}{2 \pi i} = \delta'(t).
\end{equation*}
Действительно,
\begin{equation*}
    \frac{d }{d t} \left(
        \int_{- i \infty}^{i \infty}
         e^{p t} \frac{\d p}{2 \pi i}
    \right) = \frac{\d}{\d t} \delta(t) = \delta'(t).
\end{equation*}

Важно, что можно делать функции маленькими
\begin{equation}
    \int_{p_0 - i \infty}^{p_0 + i \omega} f(p) e^{pt} \frac{\d p}{2 \pi i} = 
    \left(\frac{d }{d t} \right)^n \int_{p_0 - i \infty}^{p_0 + i \omega} \frac{f(p)}{p^n} e^{pt} \frac{\d p}{2 \pi i}.
\end{equation}



\textbf{Неоднородная релаксация}. 
Рассмотрим уравнение
\begin{equation*}
    (\partial_t + \gamma(t)) G(t,s) = \delta(t-s),
    \hspace{10 mm} 
    x(t) = \int_{-\infty}^{+\infty} G(t, s) f(s) \d s,
\end{equation*}
где продолжаем требовать причинность $G(t,s>t) = 0$. Для начала, рассмотрим $t>s$, тогда
\begin{equation*}
    (\partial_t + \gamma(t)) G(t) = 0,
    \hspace{0.5cm} \Rightarrow \hspace{0.5cm}
    \frac{d G}{G}  = - \gamma(t) \d t,
    \hspace{0.5cm} \Rightarrow \hspace{0.5cm}
    G(t, s) = A(s) \exp\left(
        - \int_{t_0}^{t} \gamma(t') \d t'
    \right).
\end{equation*}
Также записываем граничные условия:
\begin{equation*}
    \int_{s-\varepsilon}^{s+\varepsilon} \ldots \d s,
    \hspace{0.5cm} \Rightarrow \hspace{0.5cm}
    G(s+0, s) = 1.
\end{equation*}
Так можем найти
\begin{equation}
    A(s) \exp\left(
        - \int_{t_0}^{s} \gamma(t') \d t'
    \right) = 1,
    \hspace{0.5cm} \Rightarrow \hspace{0.5cm}  
    G(t, s) = \theta(t-s) \exp\left(
        - \int_{s}^{t} \gamma(t') \d t'
    \right),
\end{equation}
где мы разбили
\begin{equation*}
    \int_{t_0}^t = \int_{t_0}^{s}  + \int_{s}^{t},
\end{equation*}
и получили, что хотели.



\phantom{42}

\textit{Комментарий про дельта функцию}. Главное, нужно показать, что
\begin{equation*}
    \int_{-\infty}^{+\infty} \delta_a (x) = 1,
    \hspace{10 mm}  
    \lim_{a \to 0} \delta_a (x) =0, \text{ при } x \neq 0.
\end{equation*}
Вообще можем плодить дельтаобразные последовательности, взяв $f$ с единичным интегралом и 
\begin{equation*}
    \delta_a (x) = \frac{1}{a} f\left(\frac{x}{a}\right).
\end{equation*}



\textit{Комментарий про преобразование Лапласа}. Для функции вида
\begin{equation*}
    \frac{1}{\sqrt{p+\alpha}},
\end{equation*}
необходим аппарат разрезов, так что её можно сделать с шифтом на неделю. 

На следующей недели будет контрольная. Необходим аппарат метода неопределенных коэффициентов, матричные экспоненты, решение диффуров через Фурье (не всегда причинный результат), а также преобразование Лапласа. Вычеты скорее всего в районе второго порядка и меньше.  Ещё полезно вспонить, как записывать начальные условия: осцияллятор, осциллятор с затуханием. 





\section{Семинар от 12.09.21}

Раннее решалась задача Коши, вида $L(\partial_t) x(t) = \varphi(t)$. Можо рассмотреть другой класс задач:
\begin{equation*}
    f(0) = f(\pi) = 0,
    \hspace{5 mm} 
    (\partial_x^2 + 1) f(x) = 0,
    \hspace{0.5cm} \Rightarrow \hspace{0.5cm}
    f(x) = A \sin x,
\end{equation*}
где $A$ -- любая, то есть решение не единственно. Более того, решение может не существовать. 

Однако, покуда мы рассматриваем диффуры первого порядка, граничное условие всего одно: значение функции в точке: $x(0) = x_0$, что эквивалентно задаче Коши. 


\subsection{Задача Штурма-Лиувилля}


% \textbf{Задача Штурма-Лиувилля}. 
Интереснее на диффурах II порядка, один из наиболее ярких примеров: \textit{задача Штурма-Лиувилля}:
\begin{equation*}
    \hat{L} = \partial_x^2 + Q(x) \partial_x + U(x),
    \hspace{5 mm} 
    \hat{L} f(x) = \varphi(x),
    \hspace{5 mm} 
    \left\{\begin{aligned}
        \alpha_1 f(a) + \beta_1 f'(a) &= 0 \\
        \alpha_2 f(b) + \beta_2 f'(b) &= 0,
    \end{aligned}\right.
\end{equation*}
где\footnote{
    Часто можно встретить нулевые граничные условия: $f(a) = f(b) = 0$. 
}  $|\alpha_1| + |\beta_2| \neq 0$ и $|\alpha_2| + |\beta_2| \neq 0$.

Заметим, что уравнение линейно: если $\varphi = \varphi_1 + \varphi_2$, то $f = f_1 + f_2$, а значит ответ можно найти в виде
\begin{equation*}
    f(x) \int_{a}^{b} G(x,y) \varphi(y) \d y,
\end{equation*}
однако система теперь не является транслционно инвариантной. 


Граничные условия на $G$:
\begin{equation*}
    \alpha_1 f(a) + \beta_1 f'(a) = \int_{a}^{b} 
    \underbrace{\left(\alpha_1 G(a, y) + \beta_1 G'_x (a, y)\right) }_{\text{непрерывен}}\varphi(y) \d y = 0,
\end{equation*}
что верно $\forall  \varphi$. По лемме Дюбуа-Реймона, можем свести уравнение к виду
\begin{equation*}
    \alpha_1 G(a, y) + \beta_1 G'_x (a, y) \equiv 0,
\end{equation*}
то есть функция Грина $G$ наследует граничные условия.  Аналогично,
\begin{equation*}
    \alpha_2 G(b, y) + \beta_2 G'_x (b, y) = 0.
\end{equation*}

Запищем уравнение на $G(x, y)$:
\begin{equation*}
    \varphi(y) = \delta(y-y'),
    \hspace{0.5cm} \Rightarrow \hspace{0.5cm}
    f(x) = \int_{a}^{b} G(x, y) \delta(y-y') \d y = G(x, y'),
    \hspace{0.5cm} \Rightarrow \hspace{0.5cm}
    \hat{L} G(x, y) = \delta(x-y).
\end{equation*}
Решения имеет смысл разбить на $x \neq y$, и, в частности, рассмотрим $x < y$:
\begin{equation*}
    \left\{\begin{aligned}
        \hat{L} G(x, y) &= 0 \\
        \alpha_1 G(a, y) + \beta_1 G'_x (a, y) &= 0
    \end{aligned}\right.
    ,
    \hspace{0.5cm} \Rightarrow \hspace{0.5cm}
    G(x,y) = A(y) \cdot u(x),
    \hspace{0.5cm} \Rightarrow \hspace{0.5cm}
    \left\{\begin{aligned}
        \hat{L} u(x) &= 0 \\
        \alpha_1 u(a) + \beta_1 u'(a) &= 0
    \end{aligned}\right.
\end{equation*}
Более того, почему бы и не доопределить $u(a) = - \beta_1$ и $u'(a) = \alpha_1$, таким образом свели задачу к задаче Коши, решение которой существует и единственно. 


Аналогично для $x > y$:
\begin{equation*}
    \hat{L} G(x, y) = 0,
    \hspace{0.5cm} \Rightarrow \hspace{0.5cm}
    G(x, y) = B(y) v(x),
    \hspace{0.5cm} \Rightarrow \hspace{0.5cm}
    \left\{\begin{aligned}
        \hat{L} v &= 0 \\
        \alpha_2 v(b) + \beta_2 v'(b) &= 0
    \end{aligned}\right.
\end{equation*}
где снова есть задача Коши, решение которой существует и единственно. 

\textbf{Сшивка}. Во-первых заметим, что $G$ непрерывна, а $G'$ испытыывает скачок:
\begin{equation*}
    G(y + 0, y) = G(y-0, y),
    \hspace{0.5cm} \Rightarrow \hspace{0.5cm}
    A(y) u(y) = B(y) v(y).
\end{equation*}
Интегрируя, находим
\begin{equation*}
    G'_x (y + 0, y) - G'_x (y-0, y) = 1,
    \hspace{0.5cm} \Rightarrow \hspace{0.5cm}
    B(y) v'(y) - A(y) u'(y) = 1.
\end{equation*}
Собирая уравнения вместе, находим, что
\begin{equation*}
    B(y) \underbrace{\left(
        \frac{v' (y) u(y) - v(y) u'(y)}{u(y)}
    \right)}_{W[u, v]} = 1,
    \hspace{0.5cm} \Rightarrow \hspace{0.5cm}
    B(y) = \frac{u(y)}{W},\hspace{5 mm} 
    A(y) = \frac{v(y)}{W(y)},
\end{equation*}
где $W[u, v]$ -- вронскиан. Итого, можем выписать ответ:
\begin{equation*}
    G(x, y) = \frac{1}{W(y)} \left\{\begin{aligned}
        &v(y) u(x), &x < y; \\
        &v(x) u(y), &x > y.
    \end{aligned}\right.
\end{equation*}
Можем записать, когда решение $\exists$ и $!$:
\begin{equation*}
    W \neq 0,
    \hspace{0.5cm} \Rightarrow \hspace{0.5cm}
    \text{Sol} \  \exists \& !.
\end{equation*}
Отсюда вытекает теорема Стеклова:

\begin{to_thr}[теорема Стеклова]
    Если $u,\, v$ -- спец. ФСР, то решение существует и единственно:
    \begin{equation*}
        f(x) = \int_{a}^{b} G(x, y) \varphi(y) \d y,
        \hspace{10 mm} \hat{L}^{-1} \varphi = f.
    \end{equation*}
    Если $W = \const$, то $G(x, y) =G(y,x)$ -- симметричное ядро, а значит $L^{-1}$ -- симметричный, самосапряженный оператор $\Rightarrow$ у $\hat{L}$ есть ОНБ из собственных функций. 
\end{to_thr}




\textbf{Про вронскиан}. Можно записать формулу Лиувилля-Остроградского
\begin{equation*}
    W(x) = 
    \det \begin{pmatrix}
        u & v  \\
        u' & v'  \\
    \end{pmatrix}
    =
    W(x_0) \exp\left(
        - \int_{x_0}^{x}  Q(z) \d z
    \right).
\end{equation*}

\begin{to_def}
    \textit{Специальной ФСР} называется решение уравнении $\hat{L} u = 0$ и $\alpha_1 u(a) + \beta_1 u'(a) = 0$, и аналогичного уравнения по $v(x)$ с граничным условием в $b$, 
    если $W[u, v] \neq 0$, то есть $u$ и $v$ линейной независимы. 
\end{to_def}



\textbf{Пример I}. Рассмотрим уравнения
\begin{equation*}
    \left\{\begin{aligned}
        \partial_x^2 f(x) = \varphi(x) \\
        f(a) = f(b) = 0
    \end{aligned}\right.
    \hspace{0.5cm} \Rightarrow \hspace{0.5cm}
    u(x) = x - a,
    \hspace{5 mm} 
    v(x) = x - b,
    \hspace{0.5cm} \Rightarrow \hspace{0.5cm}
    W = \begin{pmatrix}
        u & v  \\
        u' & v'  \\
    \end{pmatrix} = b-a = \const,
\end{equation*}
а значит
\begin{equation*}
    G(x, y) = \frac{1}{b-a} \left\{\begin{aligned}
        &(y-b)(x-a), &x < y \\
        &(x-b)(y-a), &x > y.
    \end{aligned}\right.
\end{equation*}


\textbf{Пример II}. Рассмотрим двумерный цилиндр, радуса $R$, вне которого $\rho(r > R) = 0$, $\rho(\vc{r}) = \rho(r)$. Рассмотрим уравнения Лапласа:
\begin{equation*}
    \nabla^2 \varphi = - 4 \pi \rho,
    \hspace{0.5cm} \Rightarrow \hspace{0.5cm}
    (\partial_r^2 + \tfrac{1}{r} \partial_r) \varphi = - 4 \pi \rho.
\end{equation*}
Добавим граничные условия: потенциал определен с точностью до константы, так что пусть $\varphi(R) = 0$, также хотим конечность $\varphi$ при $r=0$, так что пусть $\varphi(0) = 1$.

Получили задачу, где при $r < r'$
\begin{equation*}
    \left\{\begin{aligned}
        &\left(\partial_r^2 + \tfrac{1}{r} \partial_r\right) u(r) = 0,
        &u(0) = 1
    \end{aligned}\right.
    \hspace{0.5cm} \Rightarrow \hspace{0.5cm}
    u' = \frac{C}{r},
    \hspace{0.5cm} \Rightarrow \hspace{0.5cm}
    u(r) = C \ln r + D = 1.
\end{equation*}
Аналогично, рассмотрим $r > r'$:
\begin{equation*}
    \left\{\begin{aligned}
        &\left(\partial_r^2 + \tfrac{1}{r} \partial_r\right) v(r) = 0,
        &v(R) = 0,
    \end{aligned}\right.
    \hspace{0.5cm} \Rightarrow \hspace{0.5cm}  
    v(R) = C' \ln r + D',
    \hspace{0.5cm} \Rightarrow \hspace{0.5cm}   
    v = \ln \left(\frac{r}{R}\right).
\end{equation*}
Сразу вычислим 
\begin{equation*}
    W[u,\,  v] = \det \begin{pmatrix}
        1 & \ln r/R  \\
        0 & 1/r  \\
    \end{pmatrix} = \frac{1}{r},
    \hspace{0.5cm} \Rightarrow \hspace{0.5cm}
    G(r, r') = r' \left\{\begin{aligned}
        & \ln \tfrac{r'}{R}, & r < r'
        & \ln \frac{r}{R}, & r > r'
    \end{aligned}\right.
    \hspace{0.5cm} \Rightarrow \hspace{0.5cm}
    \varphi(r) = \int_{0}^{R} G(r, r') \left(- 4 \pi \rho(r')\right) \d r'.
\end{equation*}




\subsection{Задача с периодическими условиями}

Рассмотрим такой же $\hat{L}$, и граничные условия в виде
\begin{equation*}
    \left\{\begin{aligned}
        f(a) &= f(b) \\
        f'(a) &= f'(b),
    \end{aligned}\right.
\end{equation*}
то есть решение периодично. 



\textbf{Пример}.  Рассмотрим задачу
\begin{equation*}
    \hat{L} = \partial_x^2 + \kappa^2,
\end{equation*}
с условиями на $[-\pi, \pi]$. 

При $x < y$:
\begin{equation*}
    G(x, y) = A_1 (y) \sin \kappa(x + \pi) + B_1 (y) \cos \kappa( x + \pi),
\end{equation*}
и аналогично для $x > y$:
\begin{equation*}
    G(x, y) = A_2 \sin \kappa (x - \pi) + B_2 (y) \cos \kappa (x - \pi).
\end{equation*}
Запишем граничные условия:
\begin{align*}
    G(- \pi, y) = G(\pi, y), \hspace{0.5cm} \Rightarrow \hspace{0.5cm}
    B_1 (y) = B_2 (y) \overset{\mathrm{def}}{=} B(y) \\
    G'_x (-\pi, y) = G'_x (\pi, y),
    \hspace{0.5cm} \Rightarrow \hspace{0.5cm}
    A_1 (y) = A_2 (y) \overset{\mathrm{def}}{=}  A(y).
\end{align*}
Тогда нашли, что
\begin{equation*}
    G(x, y) = \left\{\begin{aligned}
        &A \sin \kappa (x + \pi) + B \cos \kappa (x + \pi) \\
        &A \sin \kappa (x - \pi) + B \cos \kappa (x - \pi) \\
    \end{aligned}\right.
\end{equation*}
Теперь запишем непрерывность:
\begin{equation*}
     A \sin \kappa (x + \pi) + B \cos \kappa (x + \pi) 
     = 
     A \sin \kappa (x - \pi) + B \cos \kappa (x - \pi).
\end{equation*}
А также скачок производной
\begin{equation*}
    G'_x(y + 0, y) - G'_x (y-0, y) = 1,
    \hspace{0.25cm} \Rightarrow \hspace{0.25cm}
        A \cos \kappa (x - \pi) - B \sin \kappa (x - \pi)  - 
        A \cos \kappa (x + \pi) + B \cos \kappa (x + \pi)
        = \kappa^{-1}.
\end{equation*}
Решая эту систему находим, что
\begin{equation*}
    2 \sin \pi \kappa 
    \begin{pmatrix}
        \cos \kappa y & - \sin \kappa y  \\
        \sin \xi y & \cos \kappa y  \\
    \end{pmatrix} \begin{pmatrix}
        A  \\
        B  \\
    \end{pmatrix}
    = \begin{pmatrix}
        0 \\ 1/\kappa
    \end{pmatrix},
    \hspace{0.25cm} \Rightarrow \hspace{0.25cm}
    \begin{pmatrix}
        A \\ B
    \end{pmatrix} = 
    \frac{1}{2 \sin \pi \kappa} \begin{pmatrix}
        \cos \kappa y & \sin \kappa y  \\
        \sin \kappa y & \cos \kappa y  \\
    \end{pmatrix}
    \begin{pmatrix}
        0 \\ 1/\kappa
    \end{pmatrix} = 
    \frac{1}{2 \kappa \sin \pi \kappa} \begin{pmatrix}
        \sin xy \\ \cos xy
    \end{pmatrix}.
\end{equation*}
Подставляя в $G(x, y)$, находим\footnote{
    К дз будет полезно заметить, что $G(x, y) = G(x-y)$ -- задача трансляционно инвариантна. 
} 
\begin{equation*}
    G(x, y) = \frac{1}{2 \kappa \sin \pi \kappa}
    \left\{\begin{aligned}
        &\cos \left(\kappa(x-y) + \kappa \pi\right), & x < y\\
        &\cos(\kappa (x-y) - \kappa \pi), & x > y.
    \end{aligned}\right.
\end{equation*}
Всё это было, повторимся, для уравнения:
\begin{equation*}
    \left(\partial_x^2 + \kappa^2\right) f(x) = \varphi(x),
    \hspace{0.5cm} \Rightarrow \hspace{0.5cm}   
    f(x) = 
    \int_{-\pi}^{+\pi} G(x, y) \varphi(y) \d y. 
\end{equation*}



\subsection{Другой способ}

Дорустим мы $\mathcal H$, соответственно есть $\langle x | y\rangle$ -- эрмитово. Допустим есть некоторый сопряженный оператор $\langle A x | y\rangle = \langle x | A^* y \rangle$. При чём требуем, что $y \colon $ обе части непрерывны по $x$:
\begin{equation*}
    \mathcal D (A^*) = \left\{
        y \colon  \langle A x | y\rangle \text{ непр. по } x,
        \ \ x \in \mathcal D A.
    \right\}
\end{equation*}
Получается, что было бы здорово, если бы выполнялось $\mathcal D (A) = \mathcal D(A^*)$ и $A = A^*$, тогда $\langle A x | y\rangle = \langle x | A y\rangle$, и $A$ -- ССО.


\begin{to_thr}[thr Гильберта-Шмидта]
    Если $A$ -- компактный\footnote{
        $\mathcal D (A)$ -- компакт в гильбертовом пространстве.
    }  ССО, то у $A$ есть ОНБ из собственных векторов. 
\end{to_thr}


Далее верим, что базис счётный, есть условие ортогональности и $A e_n = \lambda_n e_n$. Вернемся к оператору Штурма-Лиувилля:
\begin{equation*}
    \hat{L} = A(x) \partial_x^2 + B(x) \partial_x + C(x),
\end{equation*}
и живем мы в $\mathcal H = L_2[a, b]$, со скалярным произведением, вида
\begin{equation*}
    \langle f|g\rangle = \int_{a}^{b} f \bar{g} \d x.
\end{equation*}

Покажем, что для задачи\footnote{
    с периодическими гран. усовиями?
}  Штурма-Лиувилля $\hat{L}$  симметричен, при $B(x) = A'(x)$. 

Действительно, 
\begin{equation*}
    \langle \hat{L} f | g\rangle = \int_{a}^{b} 
    \left(
        \partial_x (A \partial_x) + C 
    \right) f \bar{g} \d x,
\end{equation*}
где  $\langle C f | g\rangle = \langle  f  | C g\rangle$, так что дальше опустим.
\begin{equation*}
    \langle \hat{L} f | g\rangle \sim
    \int_{a}^{b} \partial_x (A \partial_x f) g \d x = 
    \ldots + A \partial_x f \bar{g} \bigg|_a^b - 
    \int_{a}^{b}  \frac{d \bar{g}}{d x}  \d f = 
    A \frac{d f}{d x} g \bigg|_a^b - A f \frac{d \bar{g}}{d x}  \bigg|_a^b + \langle f | \hat{L} g\rangle.
\end{equation*}
% Рассмотрим теперь периодические граничные условия:
% \begin{equation*}
%     f
% \end{equation*}


\textbf{Задача Штурма-Лиувилля}. Для $f$ и $g$ верно, что
\begin{equation*}
    \left.\begin{aligned}
        \alpha_1 f(a) + \beta_1 f'(a) = 0 \\
        \alpha_2 f(b) + \beta_2 f'(b) = 0
    \end{aligned}\right.
    \hspace{10 mm} 
    A(b) \left[
        f'(b) \bar{g}(b) - f(b) \bar{g}'(b)
    \right] - A(a) \left[\ldots\right] \sim
    \det \begin{pmatrix}
        \bar{g} & f  \\
        \bar{g}' & f'  \\
    \end{pmatrix} = 0,
\end{equation*}
в силу граничных условий, а значит концов не будет:
\begin{equation*}
    \langle \hat{L} f | g\rangle = \langle f | \hat{L} g\rangle,
\end{equation*}
ура!)

Аналогично для периодических граничных условий:
\begin{equation*}
    \langle \hat{L} f | g\rangle = A f' \bar{g} \bigg|_a^b - A f \bar{g}' \bigg|_a^b + \langle f | \hat{L} g\rangle,
\end{equation*}
но из периодических граничных условий сразу получаем, что $\hat{L}$ симметричный, а значт, скорее всего, $\hat{L}$ -- ССО. 

А теперь внимание:
\begin{equation*}
    \hat{L} f = \varphi,
    \hspace{0.5cm} \Rightarrow \hspace{0.5cm}
    f = \underbrace{\int_{a}^{b}  dt\, G(x, y)}_{\text{комп } \hat{L}^{-1}} \varphi(y) = \hat{L}^{-1} \varphi,
\end{equation*}
где 
\begin{equation*}
    \hat{L} = A \partial_x^2 + A' \partial_x + C,
    \hspace{0.5cm} \Rightarrow \hspace{0.5cm}
    L^{-1} \text{ симметричный}
    \hspace{0.5cm} \Rightarrow \hspace{0.5cm}
    G(x, y) = G(y, x).
\end{equation*}

Итак, для операторов Штурма-Лиувилля ищем собственные функции:
\begin{equation*}
    \left\{\begin{aligned}
        &\hat{L} e_n (x) = \lambda_n e_n (x) \\
        &\text{гран. усл.}
    \end{aligned}\right.
    \hspace{0.5cm} \Rightarrow \hspace{0.5cm}
    f(x) = \sum_n f_n e_n(x),
\end{equation*}
домножая на $e_m$, находим, что
\begin{equation*}
    \langle f | e_m\rangle = f_m \langle e_m | e_n\rangle,
    \hspace{0.5cm} \Rightarrow \hspace{0.5cm}
    f_m = \frac{\langle f | e_m\rangle}{\langle e_m | e_n\rangle},
    \hspace{10 mm} 
    \langle f | e_m\rangle = \int_{a}^{b}  f(x) \bar{e}_m (x) \d x.
\end{equation*}

Метод Фурье решения краевых задач:
\begin{equation*}
    L(\partial_t) u(x, t) = \hat{A}_x u(x, t) + f(x, t),
\end{equation*}
плюс граничные условия.  Раскладывая,
\begin{equation*}
    \hat{A}_x e_n = \lambda_n e_n,
    \hspace{10 mm} 
    u = \sum_n u_n(t) e_n (x),
    \hspace{0.5cm} \Rightarrow \hspace{0.5cm}
    f(x, t) = \sum f_n (t) e_n (x),
\end{equation*}
а значит
\begin{equation*}
    \sum_n \left(
        L(\partial_t) u_n (t) - \lambda_n u_n (t) - f_n (t)
    \right) e_n (x) = 0,
\end{equation*}
откуда можем находить $u_n (t)$:
\begin{equation*}
    u(x, t) = \sum_n u_n (t) e_n (x). 
\end{equation*}


\textbf{Пример}. Рассмотрим снова задачу, вида
\begin{equation*}
    (\partial_x^2 + \kappa^2) G(x, y) = \delta(x-y)
\end{equation*}
и решим методом Фурье. Получим систему, вида
\begin{equation*}
    \left\{\begin{aligned}
        \partial_x^2 e_n = \lambda_n e_n \\
        e_n (-\pi) = e_n (\pi) \\ 
        e_n'(-\pi) = e_n'(\pi) \\
    \end{aligned}\right.
    \hspace{0.5cm} \Rightarrow \hspace{0.5cm}
    e = \alpha e^{i q x} + \beta e^{- i q x}, 
    \hspace{5 mm} 
    \alpha e^{i q \pi} + \beta e^{- i q \pi} = \alpha e^{- i q \pi} + \beta e^{i q \pi}.
    \hspace{0.25cm} \Rightarrow \hspace{0.25cm}
    \alpha \sin \pi q = \beta \sin \pi q,
\end{equation*}
а значит $q = n$, $n \in \mathbb{Z}$, вот и дискретность:
\begin{equation*}
    \lambda^2 = - n^2,
    \hspace{0.5cm} \Rightarrow \hspace{0.5cm}
    e_n = e^{i n x}, \hspace{5 mm} 
    e_{-n} = e^{- i nx}.
\end{equation*} 

% ...


% \subsection{Суммирование рядов в ТФКП}









% \subsection{Асимптотика решений}





\section{Семинар от 16.09.21}

Рассмотрим снова некоторую граничную задачу:
\begin{equation*}
    \hat{L} G(x, y) = \delta(x-y).
\end{equation*}
Запишем граничные условия:
\begin{equation*}
    \alpha_1 G(a. y) + \beta_1 G'_x(a, y) = 0,
    \hspace{10 mm} 
    \alpha_2 G(b, y) + \beta_2 G'_x (b, y) = 0,
\end{equation*}
где  $|\alpha_1| + |\beta_2| \neq 0$ и $|\alpha_2| + |\beta_2| \neq 0$.
Можем выписать ответ:
\begin{equation*}
    G(x, y) = \frac{1}{W(y)} \left\{\begin{aligned}
        &v(y) u(x), &x < y; \\
        &v(x) u(y), &x > y,
    \end{aligned}\right.
\end{equation*}
где Вронскиан можно запсиать, как
\begin{equation*}
    W(x) = W(x_9) \exp\left(
        - \int_{x_0}^{x}  Q(t) \d t,
    \right)
\end{equation*}
где $Q(t)$ -- из оператора Штурма-Лиувилля. 

Также решали задачу с периодическими гран. условиями, где $G$ наследовала гран. условия. 
Решать это всё умеем двумя способами: разделяя на $x  > y$ и $x < y$, и через метод Фурье:
\begin{equation*}
    \hat{L} e_n  = \lambda_n e_n,
    \hspace{10 mm} 
    \langle e_n | e_m \rangle = \int_{a}^{b} e_n (x) \bar{e}_m(x)  \d x.
\end{equation*}
Тогда можем найти функцию Грина, как
\begin{equation*}
    G(x, y) = \sum_n g_n (y) e_n (x),
    \hspace{5 mm} 
    \delta(x-y) = \sum_n \delta_n ( y) e_n (x).
\end{equation*}
Находим коэффициенты Фурье:
\begin{equation*}
    g_n (y) = 
    \frac{\langle G | e_n\rangle}{\langle e_n | e_n\rangle},
    \hspace{0.5cm} \Rightarrow \hspace{0.5cm}
    \delta_n (y) = \frac{\bar{e}_n (y)}{\langle e_n | e_n\rangle},
    \hspace{0.5cm} \Rightarrow \hspace{0.5cm}
    g_n (y) = \frac{1}{\lambda_n} \frac{\bar{e}_n (y)}{\langle e_n | e_n\rangle},
\end{equation*}
где мы решали уравнение, вида $\hat{L} G = \delta(x-y)$. 
Проблема возникает при $\lambda_n = 0$. 



\textbf{Решение}. Наличие у оператора собственного числа $\lambda_n = 0$ называется нулевой модой. Рассмотрим оператор:
\begin{equation*}
    \hat{L} = \partial_x^2,
\end{equation*}
для которого $e_n (x) = e^{i n x}$, где $\langle e_n | e_n\rangle = 2 \pi$, где $e_0 = 1$ и $\lambda_{0} = 0$. Пусть тогда
\begin{equation*}
    \delta(x) = \sum \frac{\bar{e}_n (0) e_n (x)}{\langle e_n | e_n\rangle} = \sum \frac{e^{i n x}}{2 \pi},
    \hspace{10 mm} 
    G(x) = \sum  g_n e_n (x). 
\end{equation*}
но для $\hat{L} G = \delta(x)$ оказывается нет решений (справа $e_0$ есть, а слева нет). То есть
\begin{equation*}
    \Ker \hat{L} \neq \{0\},
    \hspace{10 mm} 
    \Ker \hat{L} + \Im \hat{L} = \mathcal H,
\end{equation*}
поэтому всегда имеем ввиду, что $\hat{L} \hat{L}^{-1} = \mathbbm{1}$, но только для $\im \hat{L}$. 

В общем, проблему уйдёт, если рассмотрим уравнение, вида
\begin{equation*}
    \hat{L} G(x) = \delta(x) - e_0(x) = \delta(x) - \frac{1}{2 \pi},
\end{equation*}
то есть справа единичный оператор только на образе $\im \hat{L}$. 



Если в источнике есть нулевая мода, то уравнение не имеет решений. 



\textbf{Алгоритм (Фурье)}. Раскладываем 
\begin{equation*}
    G(x) = \sum_{n \neq 0} g_n e_n,
    \hspace{10 mm} 
    \delta(x) = \sum \frac{e_n (x)}{2 \pi},
    \hspace{0.5cm} \Rightarrow \hspace{0.5cm}
    \hat{L} G = \delta(x) - \frac{1}{2\pi}.
\end{equation*}
Знаем, что $\lambda_n g_n = \frac{1}{2\pi}$, а значит
\begin{equation*}
    g_n(x) = \frac{1}{2\pi} \frac{1}{- n^2},
    \hspace{0.5cm} \Rightarrow \hspace{0.5cm}   
    G(x) = \sum_{n \neq 0} \frac{1}{2\pi} \frac{1}{-n62} e^{i n x},
\end{equation*}
и рассмотрим $0 < x < \pi$, суммирая это через вычеты, записываем
\begin{equation*}
    f(z) = \frac{e^{zx}}{2 \pi z^2}, 
    \hspace{0.5cm} \Rightarrow \hspace{0.5cm}   
    G(x) = \sum \oint_{in} \frac{\d z}{2 \pi i} f(z) g(z).
\end{equation*}
Соответственно, выберем
\begin{align*}
    g(z) = \frac{\pi e^{- \pi z}}{\sh (\pi z)}
\end{align*}
тогда
\begin{equation*}
    f(z) g(z) = \frac{\pi}{z^2} \frac{e^{(x-\pi)z}}{\sh \pi z},
\end{equation*}
получаем, что интеграл по душам вправо/влево  равен $0$, и остается только вычет в $z = 0$:
\begin{equation*}
    G(z) = - \res_0 f(z) g(z) = \ldots = - \frac{x^2}{4 \pi} + \frac{x}{2} - \frac{\pi}{6}.
\end{equation*}



\textbf{Алгоритм (сшивка)}. Решим задачу
\begin{equation*}
    \partial_x^2 G(x) = \delta(x) - \frac{1}{2\pi}.
\end{equation*}
Разбиваем $x < 0$ и $x > 0$:
\begin{align*}
    &x < 0, 
    & G = -\tfrac{x^2}{4 \pi} + a x + b, \\
    &x > 0, 
    & G = -\tfrac{x^2}{4 \pi} + c x + \varpi, 
\end{align*}
учитываем граничные условия:
\begin{equation*}
    G(-0) = G(+0),
    \hspace{5 mm} 
    G'(+0) - G'(-0) = 1,
    \hspace{0.5cm} \Rightarrow \hspace{0.5cm}   
    b = \varpi.
\end{equation*}
Также получаем, что $-a = b$.

Учтём, что $e_0$ не входит в $G$:
\begin{equation*}
    \langle G | e_0\rangle = 0 = \int_{-\pi}^{+\pi}G(x) \d x = 0,
    \hspace{0.5cm} \Rightarrow \hspace{0.5cm}
    b = - \frac{\pi}{6},
\end{equation*}
так и получаем все необходиме условия на $G(x, y)$. 



\subsection{Многомерие \texorpdfstring{$\mathbb{R}^3$}{R3}}

Рассмотрим $\mathbb{R}^3$:
\begin{equation*}
    \nabla^2 f = \varphi,
\end{equation*}
где все линейно, всё хорошо. Как обычно будем искать функцию, виде
\begin{equation*}
    f(\vc{r}) = \int_{\mathbb{R}^3} G(\vc{r}  - \vc{r}') \varphi(\vc{r}) \d^3 r. 
\end{equation*}
Функцию Грина найдём в виде
\begin{equation*}
    \nabla^2 G(\vc{r}) = \delta(r^3) = \delta(x) \delta(y) \delta(z),
    \hspace{10 mm}  
    \int f(\vc{r}) \delta(\vc{r}- \vc{r}') \d^3 \vc{r}' = f(\vc{r}').
\end{equation*}
Можем свести уравнение Лапласа, к уравнению Дебая:
\begin{equation*}
    (\nabla^2 - \kappa^2) G(\vc{r}) = \delta(\vc{r}),
\end{equation*} 
которое очень удобно раскладывать по Фурье:
\begin{align*}
    &\text{ПФ}: 
    &G(\vc{k}) &= \int_{\mathbb{R}^3} G(\vc{r}) e^{- i \smallvc{k} \cdot \smallvc{r}} \d \vc{r}, \\
    &\text{ОПФ}: 
    &G(\vc{r}) &= \int_{\mathbb{R}^3} G(\vc{r}) e^{i \smallvc{k} \cdot \smallvc{r}} \frac{\d \vc{k}}{(2 \pi)^3}.
\end{align*}
Также вспомним, что
\begin{equation*}
    \partial_m G(\vc{r}) e^{- i \smallvc{k} \cdot \smallvc{r}} \d \vc{r} = i k_m G(\vc{k}),
\end{equation*}
а значит
\begin{equation*}
    (-k^2 - \kappa^2) G(\vc{k}) = 1,
    \hspace{0.5cm} \Rightarrow \hspace{0.5cm}
    G(\vc{k})=- \frac{1}{k^2 + \kappa^2},
    \hspace{0.5cm} \Rightarrow \hspace{0.5cm}
    G(\vc{r}) = \int_{\mathbb{R}^3} \frac{e^{i \smallvc{k} \cdot \smallvc{r}}}{k^2 + \kappa^2} \frac{\d \vc{k}}{(2 \pi)^3}.
\end{equation*}
Переходим в сферические координаты, получаем, что
\begin{equation*}
    G(\vc{r}) = - \frac{2 \pi}{(2 \pi)^3} \int_{0}^{\infty} \frac{k^2}{k^2 + \kappa^2} \int_{0}^{\pi} 
    \sin \theta  e^{i k r \cos \theta}  \d \theta \d k = 
    - \frac{e^{- \kappa r}}{4 \pi r}
    .
\end{equation*}
Устремляя $\kappa \to 0$, находим
\begin{equation*}
     \nabla^2 G = \delta(\vc{r}),
     \hspace{0.5cm} \Rightarrow \hspace{0.5cm}
     G = - \frac{1}{4 \pi r}.
 \end{equation*} 





% \textbf{Пример}. Пусть 

\subsection{Многомерие \texorpdfstring{$\mathbb{R}^2$}{R2}}

Для Гаусса можно найти, что
\begin{equation*}
    G^{[\dim = n]}(x) = \frac{1}{\sigma_{n-1}} \frac{1}{r^{n-2}},
\end{equation*}
где $\sigma_{n-1}$ -- площадь $n-1$ мерной сферы. 



Вообще часто задача формулируется в виде задачи Дирихле:
\begin{equation*}
    \nabla^2 f = 0, 
    \hspace{5 mm} 
    f'_{\partial D} = f_0 (\vc{r}),
\end{equation*}
то есть функция задана на границе некоторой области. Пусть
\begin{equation*}
    f(z) = u(z) = i v(z),
    \hspace{5 mm} 
    \nabla^2 u = \nabla^2 v = 0.
\end{equation*}

Пусть знаем комплексную функцию $f(z)$ такую, что $\Re f |_{\partial D} = f_0$, тогда $\Re f(z)$ решает задачу Дирихле.
Далее конформным преобразованием переводим любое $D$ в круг, в круге задача Дирихле решается, а дальше отображаем назад. 


Пусть задана функция $u_0 (x) = u(x, 0)$. Вообще можно было бы разложить по Фурье $u$, и записать
\begin{equation*}
    \nabla^2  u = 0,
    \hspace{5 mm}   
    u(x, 0) = u_0 (x).
\end{equation*}
Тогда
\begin{equation*}
    u(q, y)  = \int_{\mathbb{R}} e^{- i q x} u(x, y) \d x,
    \hspace{0.5cm} \Rightarrow \hspace{0.5cm}
    \nabla^2 u = - q^2 u(q, y) = 0.
\end{equation*}
Так приходим к
\begin{equation*}
    u(q, y) = \exp\left(- |q| y\right) u(q, 0),
    \hspace{0.5cm} \Rightarrow \hspace{0.5cm}   
    u(x, y) = \int_{\mathbb{R}} \frac{\d q}{2\pi} e^{i q x} \underbrace{e^{- |q| y}}_{h(q)} u(q, 0).
\end{equation*}
Произведение Фурье образов -- свёртка:
\begin{equation*}
    u(x, y) = \int_{\mathbb{R}} \d \xi h(x - \xi,\,  y) u_0 (\xi).
\end{equation*}
Найдём, что
\begin{equation*}
    \int_{\mathbb{R}} \frac{\d q}{2 \pi} e^{i q x} e^{-|q| y} = \frac{y}{\pi (x^2 + y^2)}.
\end{equation*}
Подставляем, и находим:
\begin{equation*}
    u(x, y) = \int_{\mathbb{R}} d \xi \, \frac{y/\pi}{(x-\xi)^2 + y^2} u_0 (\xi),
\end{equation*}
где 
\begin{equation*}
    \frac{y/\pi}{(x-\xi)^2 + y^2} = \Im \frac{-1}{x + i y - \xi},
    \hspace{0.5cm} \Rightarrow \hspace{0.5cm}   
    f(z) = - \frac{1}{\pi} \int_{\mathbb{R}} \frac{1}{z-\xi} u_0 (\xi),
\end{equation*}
что в некотором смысле привело нас к интегралу Коши, так что и $\nabla^2 f = 0$ и гран. условия удовлетворяются. 



\textbf{Пример}. Рассмотрим
\begin{equation*}
    u_0 (x) = \frac{1}{1 + x^2},
    \hspace{5 mm} 
    u(x, y) = \int_{\mathbb{R}} d \xi \frac{1}{\pi} \frac{y}{y^2 + (x-\xi)^2} \frac{1}{1+\xi^2} = \frac{1 + y}{x^2 + (1+y)^2}.
\end{equation*}

\section{Семинар от 23.10.21}


\textbf{$\Gamma$-функция}. Найдем некоторые интересные свойства:
\begin{equation*}
    \Gamma(z) = \int_{0}^{\infty} t^{z-1} e^{-t} \d t \overset{t = \tau x}{=} 
    x^z \int_{0}^{\infty} \tau^{z-1} e^{- \tau x} \d \tau,
    \hspace{10 mm} 
    \frac{1}{x^z} = \frac{1}{\Gamma(z)} \int_{0}^{\infty} \tau^{z-1} e^{- \tau x} \d \tau.
\end{equation*}
Также знаем $\Gamma(n+1) = n!$, $\Gamma(2n+1)$, $\Gamma(1/2) = \sqrt{\pi}$ и т.д.



\textbf{Аналитическое продолжение $\Gamma$-функции}. Пусть есть две функции $\varphi_1$ и $f_2$, равные друг другу на сходящемся множестве точек $z_i \in \mathcal D_1 \cap \mathcal D_2$. Так и строим аналитическое продолжение для $f_1$ функуцией $f_2$.
% обнудение тригонометрии, Карлов.


Можно сказать, что 
\begin{equation*}
    \Gamma(z-1) = \frac{\Gamma(z)}{z-1}, \hspace{10 mm} \Re z > 0.
\end{equation*}
Но давайте сыграем в чудеса. Изначально определяли
\begin{equation*}
    \Gamma(z) = \int_{0}^{\infty} t^{z-1} e^{-t} \d t,
    \hspace{5 mm} 
    t^{z-1} = e^{(z-1) \ln t}.
\end{equation*}
Выберем такую связную область, чтобы точку $t = 0$ нельзя было бы обойти, и получим $\ln t = \ln |t| + i \varphi$. 

Сверху $\ln (|t| + i 0) = \ln |t| + 2 \pi i$, и снизу $\ln (|t| + i 0) = \ln |t| + 2 \pi i$. Тогда верно, что
\begin{equation*}
    \int_{0}^{\infty} e^{(z-1)\ln t} e^{-t} \d t,
    \hspace{10 mm} 
    \text{up}: \ \ e^{(z-1) \ln |t|} e^{-|t|},
    \hspace{10 mm} 
    \text{down}: \ \ e^{(z-1) \ln |t|} e^{-|t|} e^{2 \pi i z}.
\end{equation*}
Сложим интеграл поверху и понизу, получим 
\begin{equation*}
    I = \int e^{(z-1)\ln t} e^{-t} \d t = (1 -e^{2 \pi i z}) \Gamma(z) 
    = \int_C e^{(z-1) \ln t} e^{- t} \d t = \int_C t^{z-1} e^{-t} \d t.
\end{equation*}
Особенность есть только в точке $0$. Таким образом находим аналитическое продолжение:
\begin{equation}
    \Gamma(z) = \frac{1}{1-e^{2 \pi i z}} \int_C t^{z-1} e^{-t} \d t.
\end{equation}
Видим, что у $\Gamma(z)$ есть особенности $z \in \mathbb{Z}$, где $z \in \mathbb{N}$ -- УОТ, и $z \in \mathbb{Z}\ \mathbb{N}$ -- полюса первого порядка.


Рассмотрим $z = -n$, тогда интегрируем
\begin{equation*}
    \int_C t^{-n-1} e^{-t} \d t = 2 \pi i \frac{1}{n!} \left(\frac{d }{d t} \right)^n e^{-t} = 
    - \frac{2\pi i (-1)^n}{n!}.
\end{equation*}
Итого находим, что
\begin{equation*}
    \res_{-n} \Gamma(z) = \frac{(-1)^n}{n!},
\end{equation*}
что позволяет определить преобразование Мелина от $\Gamma(z)$:
\begin{equation*}
    \int_{-i\infty}^{+i\infty}  \Gamma(z) x^{-z} \d x,
    \hspace{10 mm} 
    M(f(z)) = \int_{0}^{\infty} t^{z-1} f(t) \d t,
\end{equation*}
но это к слову. 


Найдём теперь $\Gamma(n)$:
\begin{equation*}
    \Gamma(n) = \lim_{z\to n} \frac{1}{1 - e^{2 \pi i z}} \int_C t^{z-1 + n  -n} e^{-t} \d t = 
    \bigg/
        t^{z-n} \approx 1 + (z-n) \ln t
    \bigg/ \overset{\frac{a}{b}\to \frac{a'}{b'}}{=}  
    \frac{1}{2\pi i} \int_C t^{n-1} \ln t e^{-t} \d t.
\end{equation*}
Теперь делаем обратную интерацию, <<сдувая>> логарифм к $\Re t$. Здесь всё также $\ln (|t| + i0) = \ln |t|$ и $\ln (|t| - i 0) = \ln |t| + 2 \pi i$. Тогда
\begin{equation*}
    \left(\int_C = \int_{\text{up}} + \int_{\text{down}}\right) 
    \frac{1}{1-e^{2 \pi i z}} \int_C t^{z-1} e^{-t} \d t
    = - 2 \pi i \int_0^{\infty} t^{n-1} e^{-t} \d t = (n-1)!.
\end{equation*}


\textbf{$B$-функция}. Рассмотрим функцию, вида
\begin{equation*}
      B(\alpha,\, \beta) = \int_{0}^{1} t^{\alpha-1} (1-t)^{\beta-1} \d t,
      \hspace{5 mm} 
      \Re \alpha, \beta > 0.
\end{equation*}  
Сделаем замену переменных
\begin{equation*}
    B(\alpha,\, \beta) = \int_{0}^{1}  t^{\alpha-1} (1-t)^{\beta-1} \d t \overset{t = y/s}{=} 
    \int_{0}^{s} d y\ y^{\alpha-1} (s-y)^{\beta-1} / s^{\alpha + \beta -1}.
\end{equation*}
Нетрудно получить, что
\begin{equation*}
    B(\alpha, \beta) \Gamma(\alpha + \beta) = \int_{0}^{\infty}  d s\ e^{-s} \int_{0}^{s} dy\ y^{\alpha-1} (s-y)^{\beta-1} = 
    \int_{0}^{\infty} dy\ y^{\alpha-1} \int_y^{\infty} ds\ e^{-s + y - y} (s-y)^{\beta-1} = 
    \int_{0}^{\infty}  dy \ y^{\alpha-1} e^{-y} \int_{0}^{\infty} dx\ e^{-x} x^{\beta-1},
\end{equation*}
а значит
\begin{equation}
    B(\alpha,\, \beta) = \frac{\Gamma(\alpha) \Gamma(\beta)}{\Gamma(\alpha + \beta)},
\end{equation}
что и является аналитическим продолжением $B$-функции. 


Например,
\begin{equation*}
    \int_0^{\pi/2} \sin^\alpha \varphi \cos^\beta \varphi \d \varphi =  \frac{1}{2} B\left(\frac{a+1}{2},\, \frac{b+1}{2}\right).
\end{equation*}
Также верно, что
\begin{equation*}
    \int_{0}^{\infty}  \frac{x^m}{(1+x^n)^k} \d x = \bigg/
        t = \frac{1}{1+x^n}
    \bigg/.
\end{equation*}
Аналогично можем получить, что
\begin{equation}
    \Gamma(z) \Gamma\left(z + \tfrac{1}{2}\right) = \frac{\sqrt{\pi}}{2^{2z-1}} \Gamma(2 z).
\end{equation}
Ну действительно, представим
\begin{equation*}
    \Gamma(z) \Gamma(z + \tfrac{1}{2}) = \frac{\Gamma(\frac{1}{2}) \Gamma(z)^2}{B(z,\, \tfrac{1}{2})} = 
    \frac{\sqrt{\pi} B(z, z)}{B(z,\, \frac{1}{2})} \Gamma(2z).
\end{equation*}
Осталось раскрыть
\begin{equation*}
    B(z, z) = 2 \int_{0}^{\pi/2} d \varphi\ \left(\sin \varphi \cos \varphi\right)^{2z-1} = 
    \frac{2}{2^{2z-1}} \int_{0}^{\pi/2} d \varphi \ \sin^{2 z-1} \varphi.
\end{equation*}
Теперь, уже интегрируя двойной угол, находим
\begin{equation*}
    B(z, z) = \frac{2}{2^{2z-1}} \frac{1}{2} B(z,\, \tfrac{1}{2}) = \frac{1}{2^{2z-1}} B(z,\, \tfrac{1}{2}).
\end{equation*}

Ещё один забавный факт:
\begin{equation*}
    \Gamma(z) \Gamma(1-z) = \frac{\pi}{\sin \pi z},
\end{equation*}
что также совершает аналитическое продолжение. Действительно,
\begin{equation*}
    B(z,\, 1-z) = \int_{0}^{1}  t^{z-1} (1-t)^{-z} \d t = \int_0^1 \left(\frac{t}{1-t}\right)^z \frac{\d t}{t}.
\end{equation*}
Тут логично ввести $x = \frac{t}{1-t} = -1 + \frac{1}{1-t}$, а значит
\begin{equation*}
    t = \frac{x}{x+1}, \hspace{5 mm} 
    \d t = \frac{\d x}{(x+1)^2}.
\end{equation*}
Продолжая жонглировать переменными
\begin{equation*}
    B(z,\, 1-z) = \int_{0}^{\infty} x^z \frac{x+1}{x} \frac{\d x}{(x + 1)^2} = 
    \int_{0}^{\infty} \frac{x^{z-1}}{x+1} \d x.
\end{equation*}
Который снова удобно посчитать через разрезы. 
\begin{equation*}
    B(z,\, 1-z) = \int_{\text{up}} \frac{x^{z-1}}{1+x} \d x =
    \frac{1}{1-e^{2 \pi i z}} \int_C \frac{x^{z-1}}{1+x} \d x,
\end{equation*}
но тут уже можно замкнуть дугу на бесконечности, вклад от котрой нулевой.  Осталось найти вычет в точке $-1$, тогда
\begin{equation*}
    \int_{\text{up}} \frac{x^{z-1}}{1+x} \d x = \frac{1}{1-e^{2\pi i z}}    \res_{-1} = \frac{2 \pi i (-1) e^{\pi i z}}{1 - e^{2 \pi i z}} = \frac{2 \pi i}{e^{\pi i z} - e^{-i \pi z}} = \frac{\pi}{\sin \pi z}.
\end{equation*}

\textbf{Дигамма-функция}. По определнию $\psi(z)$:
\begin{equation*}
    \psi(z) \overset{\mathrm{def}}{=}  \left(\ln \Gamma(z)\right)' = \frac{\Gamma'(z)}{\Gamma(z)}.
\end{equation*}
Заметим, что $\psi(1) = - \gamma$, где $\gamma$ -- постоянная Эйлера-Маскерони. Найдём
\begin{equation*}
    \psi(z+1)= \left(\ln z + \ln \Gamma(z)\right) = \frac{1}{z} + \psi(z)/
\end{equation*}
Забавынй факт:
\begin{equation*}
    \psi(N+1) = \frac{1}{N} + \psi(N) = \sum_{n=1}^{N} \frac{1}{n} + \psi(1),
\end{equation*}
где $\sum_{n=1}^{N} \frac{1}{n}$ -- $N$-е гармоническое число. 


Также найдем, что
\begin{equation*}
    \psi(x + N + 1) = \frac{1}{x + N} + \psi(x+ N) = \frac{1}{x+N} + \ldots + \frac{1}{x+1} + \psi(x+1).
\end{equation*}
Вспомним, что $\Gamma(z) \Gamma(1-z) = \frac{\pi}{\sin \pi z}$. Тогда
\begin{equation*}
    \psi(-z) - \psi(z) = \pi \ctg \pi z.
\end{equation*}
Найдём асимптотику 
\begin{equation*}
    \Gamma(z \to \infty) = \sqrt{2 \pi z} e^{z \ln z - z} = \sqrt{2 \pi z} \left(\frac{z}{e}\right)^{z},
\end{equation*}
что и составляет формулу Стирлинга. 

Также для $\psi(z\to \infty)$:
\begin{equation*}
    \psi(z\to \infty) = \left(\ln \Gamma(z)\right)' = \ln z + \frac{1}{2z} + o(1) = \ln z + o(1).
\end{equation*}




% асимптотика интегралов Лапласа
\textbf{Метод перевала}.  Представим семейство интегралов с параметром $\lambda$:
\begin{equation*}
    I_\lambda = \int_{-\infty}^{+\infty} g(x) e^{\lambda f(x)} \d x.
\end{equation*}
При этом предположим, что $f(x)$ такая, что существует единственный максимум в точке $x_0$. Тогда
\begin{equation*}
    I_\lambda \approx g(x_0) \int_{-\infty}^{+\infty} e^{\lambda f(x)} \d x.
\end{equation*}
Теперь воспользуемся аналитичностью функции $f(x)$:
\begin{equation*}
    f(x) = f(x_0) + \frac{f'(x_0)}{2} (x-x_0)^2 + \frac{f''(x_0)}{2} (x-x_0)^2 + o(x-x_0)^2.
\end{equation*}
Подставляя в интеграл, находим
\begin{equation*}
    I_\lambda= = g(x_0) e^{\lambda f(x_0)} \int_{-\infty}^{+\infty} e^{\lambda f'(x_0) (x-x_0)^2/2} \d x = g(x_0) e^{\lambda f(x_0)} \sqrt{\frac{2\pi}{|\lambda f'' (x_0) |}}.
\end{equation*}

Пусть $\lambda$ нет. Тогда достаточно потребовать $|f''(x_0)|$ большой -- максимум резкий. 
Тогда
\begin{equation*}
    |f''(x_0) (x-x_0)^2| \sim 1,
    \hspace{0.5cm} \Rightarrow \hspace{0.5cm}   
    |x-x_0| \frac{1}{\sqrt{f'(x_0)}},
    \hspace{0.5cm} \Rightarrow \hspace{0.5cm}   
    |f'''(x_0) (x-x_0^3)| \ll 1,
    \hspace{0.5cm} \Rightarrow \hspace{0.5cm}
    (f'')^3 \gg (f''')^2.
\end{equation*}

Посмотрим на $\Gamma$-функцию:
\begin{equation*}
    \Gamma(z+1) = \int_{0}^{\infty}  t^{z} e^{-t} \d t  = \int_0^\infty 
    e^{z \ln t - t} \d t.
\end{equation*}
Тогда $f(t) = z \ln t - t$. Подставляем в критерий, видим что макимум у $f$ резкий.

Подставляем, находим
\begin{equation*}
    \Gamma(z+1) \approx e^{z \ln z - z} \sqrt{2 \pi z},
\end{equation*}
что и составляет формулу Стирлинга, верной на всей комплексной плоскости.















\section{Семинар от 31.10.21 (функция Эйри)}

Рассмотрим решение уранения
\begin{equation*}
    f'' - x f  = 0,
    \hspace{10 mm} 
    f(x \to + \infty) \to 0.
\end{equation*}




% 12:30
% accuracy:

% (False, False): 389, 
% (True, True): 179
% (False, True): 7
% (True, False): 15}

 % acc
 % Sensitivity 

 % (False, False): 369, 
% (True, True): 139
% (False, True): 27
% (True, False): 55}



% Brain tumor localization and segmentation from magnetic resonance imaging (MRI) are hard and important tasks for several applications in the field of medical analysis.


% Automatic segmentation of brain tumors from medical images is important for clinical assessment and treatment planning of brain tumors. Recent years have seen an increasing use of convolutional neural networks (CNNs) for this task


подходящий
% suitable
% 

So, we decided to take advantage of convolutional neural networks, work with brain images:
in particular, we made some successes in solving the problem
Brain tumor localization and segmentation from magnetic resonance imaging (MRI) 

its important tasks for several applications in the field of medical analysis.

The slide shows an example of MRI Image and the area with anomaly, which is necessary to allocate.


The development of the algorithm looks like this: 

A MRI picture is  the neuralnet input and 
expected output: the mask of the brain tumor area.

we minimize the difference on the training set,
 then test out neural net on data, that Neuranet before did not see

 Results are presented on the slide,

For better clarity, results are given in training on different amounts of data, the second column corresponds to two times more volume of the training data.
 As you can see the generalizing ability of the algorithm  very sensitive to data. 

 However, in any case, it was possible to correctly detect 94 percent of cases of the disease, which is a fairly good result in this area.





Moreover, the algorithm with an accuracy of 96 percent correctly segmented the area of the disease, 

The slide is demonstrated for comparison made by manual segmentation, and segmentation obtained automatically using our algorithm.

So, it can potentially easily simplify (automate), speed up and improve solving the problem of Brain Tumor Localization and Segmentation, especially after learning and testing on large amounts of data.


The amazing feature of the deep learning algorithms, which I would like to highlight, that a very wide class of tasks related with segmenting/classification  the image or signals can be resolved (Automated improved) 

only by the presence of a sufficient number of marked examples for learning









