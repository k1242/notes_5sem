% базовая подстройка
\renewcommand{\d}{\, d}
\renewcommand{\leq}{\leqslant}
\renewcommand{\geq}{\geqslant}
\renewcommand{\kappa}{\varkappa}


% авторские команды
\newcommand{\vc}[1]{\mbox{\boldmath $#1$}}
\newcommand{\smallvc}[1]{\scalebox{0.65}{\mbox{\boldmath $#1$}}}
\newcommand{\T}{^{\text{T}}}
\newcommand{\con}{^{\dag}}
\newcommand{\sub}[2]{#1_{\textnormal{#2}}}
\newcommand{\vp}{\vphantom{\dfrac{1}{2}}}

% операторы (просто прямой текст)
\renewcommand{\Im}{\mathop{\mathrm{Im}}\nolimits}
\renewcommand{\Re}{\mathop{\mathrm{Re}}\nolimits}
\newcommand{\diag}{\mathop{\mathrm{diag}}\nolimits}
\newcommand{\card}{\mathop{\mathrm{card}}\nolimits}
\newcommand{\grad}{\mathop{\mathrm{grad}}\nolimits}
\renewcommand{\div}{\mathop{\mathrm{div}}\nolimits}
\newcommand{\rot}{\mathop{\mathrm{rot}}\nolimits}
\newcommand{\Ker}{\mathop{\mathrm{ker}}\nolimits}
\newcommand{\spec}{\mathop{\mathrm{spec}}\nolimits}
\newcommand{\sign}{\mathop{\mathrm{sign}}\nolimits}
\newcommand{\tr}{\mathop{\mathrm{tr}}\nolimits}
\newcommand{\rg}{\mathop{\mathrm{rg}}\nolimits}
\newcommand{\const}{\textnormal{const}}
\newcommand{\res}{\mathop{\mathrm{res}}\limits}

% теорвер
% \renewcommand{\P}{\mathop{\mathrm{P}}\nolimits}
% \newcommand{\E}{\mathop{\mathrm{E}}\nolimits}
% \newcommand{\D}{\mathop{\mathrm{D}}\nolimits}
\newcommand{\cov}{\mathop{\mathrm{cov}}\nolimits}
\newcommand{\var}{\mathop{\mathrm{var}}\nolimits}
\newcommand{\cor}[2]{\langle\hspace{-1.2pt}\langle #1,\, #2 \rangle\hspace{-1.2pt}\rangle}

% цветной текст
\newcommand{\red}[1]{\textcolor{red}{#1}}
\newcommand{\green}[1]{\textcolor{urlcolor}{#1}}
\newcommand{\blue}[1]{\textcolor{ublue}{#1}}


% символы
\newcommand{\cmark}{\text{\ding{51}}}
\newcommand{\xmark}{\text{\ding{55}}}


% подгрузка pdf_tex картинок
% \newcommand{\incfig}[1]{%
%     \def\svgwidth{\columnwidth}
%     \import{./figures/}{#1.pdf_tex}
% }


% специфично к квантам
\newcommand{\ket}[1]{\left| #1 \right\rangle}
\newcommand{\bra}[1]{\left\langle #1 \right|}

\newcommand{\dppp}{\frac{d^3 p}{(2 \pi \hbar)^3}}


\newcommand*\xbar[1]{%
  \hbox{%
    \vbox{%
      \hrule height 0.4pt % The actual bar
      \kern0.2ex%         % Distance between bar and symbol
      \hbox{%
        \kern-0.1em%      % Shortening on the left side
        \ensuremath{#1}%
        \kern-0.1em%      % Shortening on the right side
      }%
    }%
  }%
} 