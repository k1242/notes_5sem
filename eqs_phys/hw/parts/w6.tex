\section{Неделя VI}





\subsection*{№1 (3.1.1)}

Найдём объём $d$-мерного единичного шара $B_n$. Для этого рассмотрим интеграл от $f(\vc{x}) = e^{-|\smallvc{x}|^2}$:
\begin{equation*}
    I^n = \int_{\mathbb{R}^n} e^{-|\smallvc{x}|^2} \d \vc{x} = \int_0^1 \mu\left[e^{-|\smallvc{x}|^2} \leq y\right] \d y = \int_{0}^{1} \mu\left[|\vc{x}|^2 \leq - \ln y\right] \d y.
\end{equation*}
Заметим, что справа стоит объем шара, радиуса $r = \sqrt{-\ln y}$, равный $r^{n} B_n$. Тогда
\begin{equation*}
    I^n = B_n \int_{0}^{1} (- \ln y)^{n/2} \d y.  
\end{equation*}
Заменяя $- \ln y = t$, находим
\begin{equation*}
    I^n = B_n \int_{0}^{\infty} t^{n/2} e^{-t} \d t = B_n \Gamma\left(\tfrac{n}{2}+1\right),
    \hspace{0.5cm} \Rightarrow \hspace{0.5cm}
    B_n = \frac{\pi^{n/2}}{\Gamma\left(\tfrac{n}{1}+1\right)},
\end{equation*}
где $I$ -- гауссов интеграл: $I = \sqrt{\pi}$. Нетрудно от объема шара перейти к площади сферы, дифференцируя $B_n r^n$ по $r$:
\begin{equation*}
    S_n = \frac{n \pi^{n/2}}{ \Gamma\left(\tfrac{n}{2}+1\right)},
\end{equation*}
только здесь размерность сферы $n-1$ (поверхность $n$-мерного шара). 




\subsection*{№2 (3.1.3)}

Найдём значение
\begin{equation*}
    I_c = \int_{0}^{\infty}  u^{z-1} \cos u \d u = \frac{1}{2} \int_{0}^{\infty} u^{z-1} \left(e^{i u} + e^{-iu}\right) \d u.
\end{equation*}
Для этого вычислим интегралы, вида
\begin{equation*}
    I_1 = \int_{0}^{\infty} u^{z-1} e^{i u} \d u = \bigg/ u = i \alpha \bigg/ = 
    i^z \int_{0}^{\infty} \alpha^{z-1} e^{-\alpha} \d \alpha = i^{z} \Gamma(z).
\end{equation*}
Аналогично находим
\begin{equation*}
    I_2 = \int_{0}^{\infty} u^{z-1} e^{-i u} \d u = \bigg/ u = -i \alpha \bigg/ = 
    (-i)^z \int_{0}^{\infty} \alpha^{z-1} e^{-\alpha} \d \alpha = (-i)^{z} \Gamma(z).
\end{equation*}
Вспоминая, что
\begin{equation*}
    i^{z} = e^{z \ln i} = e^{i \pi z /2}, \hspace{10 mm} 
    (-i)^z = e^{z \ln -i} = e^{- i \pi z  /2}.
\end{equation*}
Тогда
\begin{equation*}
    I_c = \frac{1}{2}\left(I_1 + I_2\right) = \frac{1}{2} \Gamma(z) \left(e^{i \pi z / 2} + e^{- i \pi z/2}\right) = \Gamma(z) \cos \left(\tfrac{\pi z}{2}\right).
\end{equation*}
Аналогично находим, что
\begin{equation*}
    I_s = \int_{0}^{\infty} u^{z-1} \sin u \d u = \frac{1}{2i}\left(I_1-I_2\right) = \Gamma(z) \sin\left(\tfrac{\pi z}{2}\right).
\end{equation*}






\subsection*{№4 (3.1.7)}

Найдём значение интеграла,  вида
\begin{equation*}
    I = \int_{0}^{1}  \ln \Gamma(z) \d z.
\end{equation*}
Делая замену $z = 1 - z$, находим, что
\begin{equation*}
    I = - \int_{1}^{0} \ln \Gamma(1-z) \d z = \int_{0}^{1} \ln \Gamma(1-z) \d z.
\end{equation*}
Вспоминая, что
\begin{equation*}
    \Gamma(z) \Gamma(1-z) = \frac{\pi}{\sin \pi z},
\end{equation*}
и складывая два представления, находим:
\begin{equation*}
    2 I = \int_{0}^{1} \d z \ln \left[
        \Gamma(z) \Gamma(1-z)
    \right] \d z = \ln \pi - \int_{0}^{1} \ln \sin \pi z \d z
     = \ln \pi - \frac{2}{\pi} \int_{0}^{\pi/2} \ln \sin z \d z
    .
\end{equation*}
Введем $I_1 = \int_{0}^{\pi/2} \ln \sin  z \d z$, тогда
\begin{equation*}
    I_1 = \int_{0}^{\pi/2} \ln \sin  z \d z = \int_{0}^{\pi/2} \ln \cos z \d z,
    \hspace{0.5cm} \Rightarrow \hspace{0.5cm}
    2 I_1 = \int_{0}^{\pi/2} \ln\left[\tfrac{\sin 2z}{2}\right] \d z = 
    \int_{0}^{\pi/2} \ln \sin 2 x - \frac{\pi}{2} \ln 2.
\end{equation*}
Заметим, что
\begin{equation*}
    \int_{0}^{\pi/2} \ln \sin 2 z = \frac{1}{2}\int_{0}^{\pi} \ln \sin z \d z = I_1,
\end{equation*}
а значит мы нашли значения для $I_1$:
\begin{equation*}
    I_1 = \int_{0}^{\pi/2} \ln \sin  z \d z = - \frac{\pi}{2} \ln 2.
\end{equation*}
Подставляя $I_1$ в выражение для $I$, находим
\begin{equation*}
    2 I = \ln \pi - \frac{2}{\pi} \left(-\frac{\pi}{2} \ln 2\right) = \ln \pi + \ln 2 = \ln 2 \pi,
    \hspace{0.5cm} \Rightarrow \hspace{0.5cm}
    I = \int_{0}^{1} \ln \Gamma(z) \d z = \frac{1}{2}\ln 2 \pi.
\end{equation*}




\subsection*{№5}
Найдём значение функции $S(z)$, вида
\begin{equation*}
    S(z) = \sum_{n=0}^{\infty} \left(\frac{1}{n+1}-\frac{1}{n+z}\right).
\end{equation*}
Для этого рассотрим только сумму до некоторого $N$ следующих рядов:
\begin{align*}
    \sum_{n=0}^{N} \frac{1}{n+1} &= \sum_{n=1}^{N+1} \frac{1}{n} = - \psi(1) + \psi(N+2), \\
    \sum_{n=0}^{N} \frac{1}{n+z} &\overset{*}{=}  - \psi(z) + \psi(N+1+z),
\end{align*}
где $\psi(z)$ -- дигамма функция: $\psi(z) = \Gamma'(z)/\Gamma(z)$. Равенство со звёздочкой можно получить из следующих соображений:
\begin{equation*}
    \psi(z+1) = \frac{1}{z} + \psi(z),
    \hspace{0.5cm} \Rightarrow \hspace{0.5cm}
    \psi(N + z + 1) = \frac{1}{N+z} + \psi(N+z) = \ldots = \psi(z) + \sum_{n=0}^{N} \frac{1}{n+z}.
\end{equation*}
Также заметим, что
\begin{equation*}
    \lim_{N \to \infty} \left(
        \psi(N+2) - \psi\left(N+1 +z\right)
    \right) =
     \lim_{N \to \infty} \left(
     \frac{1}{N+1} + \frac{1}{N} + 
            \psi(N) - \frac{1}{N+z} -\psi\left(N +z\right)
        \right) = \lim_{N \to \infty} \left(
            \psi(N) - \psi(N+z)
        \right),
\end{equation*}
и так далее вплоть до поиска предела разницы $\psi(N) - \psi(N+\alpha)$, где $\alpha \in (0, 1)$. Асимптотикой $\psi (z)$ является $\ln z$, так что 
\begin{equation*}
    \lim_{N \to \infty} \left(
        \psi(N) - \psi\left(N+\alpha\right)
    \right) = 0,
\end{equation*}
а значит искомое значение $S(z)$:
\begin{equation*}
    S(z) = \psi(z) - \psi(1).
\end{equation*}





