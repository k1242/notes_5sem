\section{Неделя VII}

\subsection*{№1 (3.2.1)}

Известно, что 
\begin{equation*}
    \Ai (x) = \frac{1}{\pi} \int_{0}^{\infty} \cos\left(x u + \frac{u^3}{3}\right) \d u,
    \hspace{10 mm} 
    \Bi(x) = \frac{1}{\pi} \int_{0}^{\infty} \left[
        \exp\left(x u - \frac{u^3}{3}\right) + \sin\left(x u + \frac{u^3}{3}\right)
    \right] \d u.
\end{equation*}
Найдём значения в $x = 0$, переходя к $u = t^{1/3}$:
\begin{equation*}
    \Ai(0) = \frac{1}{\pi} \int_{0}^{\infty} \cos( u^3/3) \d u = 
    \frac{1}{\pi} \int_{0}^{\infty} \frac{\cos(t/3) }{3 t^{2/3}}\d t = \frac{1}{3^{2/3} \Gamma\left(\frac{2}{3}\right)}.
\end{equation*}
Аналогично для $\Bi$:
\begin{equation*}
    \Bi(0) = \frac{1}{\pi} \int_{0}^{\infty} \left[
        \exp\left( - u^3/3\right) + \sin( u^3/3)
    \right] \d u = 
    \frac{1}{\pi} \int_{0}^{\infty} \frac{e^{-\frac{t}{3}}+\sin \left(\frac{t}{3}\right)}{3 t^{2/3}} \d t =  
    \frac{\Gamma \left(\frac{1}{3}\right)}{3^{2/3} \pi } -\frac{\Gamma \left(-\frac{2}{3}\right)}{3^{5/3} \pi } = \frac{1}{3^{1/6} \Gamma\left(\frac{2}{3}\right)}.
\end{equation*}

Теперь, дифференцируя под знаком интеграла, находим
\begin{equation*}
    \Ai'(0) = 
    \frac{1}{\pi} \int_{0}^{\infty} 
    -\frac{\sin \left(\frac{t}{3}\right)}{3 \sqrt[3]{t}}
    \d t = -\frac{1}{\sqrt[3]{3} \Gamma \left(\frac{1}{3}\right)}.
\end{equation*}
Аналогично для $\Bi$:
\begin{equation*}
    \Bi'(0) = 
    \frac{1}{\pi} \int_{0}^{\infty} 
    \left(
        \frac{e^{-\frac{t}{3}}}{3 \sqrt[3]{t}}+\frac{\cos \left(\frac{t}{3}\right)}{3 \sqrt[3]{t}}
    \right)
    \d t = \frac{\Gamma \left(\frac{2}{3}\right)}{\sqrt[3]{3} \pi } + \frac{\Gamma \left(\frac{2}{3}\right)}{2 \sqrt[3]{3} \pi } = \frac{3^{2/3} \Gamma \left(\frac{2}{3}\right)}{2 \pi } = \frac{3^{1/6}}{\Gamma\left(\frac{1}{3}\right)}.
\end{equation*}


\subsection*{№2 (3.2.2)}

Найдём асимптотику $\Bi(x \to +\infty)$, где $\Bi$:
\begin{equation*}
    \Bi (x) = \frac{1}{2\pi} \int_{C_3-C_2} \exp\left(- \frac{p^3}{3} + xp\right) \d p, \hspace{10 mm} 
    F(p) = - \frac{p^3}{3} + x p.
\end{equation*}
Аналогично $\Ai$ для $\Bi$, методом перевала, при $x \to + \infty$, заметим, что контура $C_2$ и $C_3$ задевают только $+\sqrt{x}$:
\begin{equation*}
    F(\sqrt{x}) = \frac{2}{3} x^{3/2}, \ \ 
    F''(\sqrt{x}) = - 2 \sqrt{x}, \ \ 
    \varphi = \pi,
    \hspace{0.5cm} \Rightarrow \hspace{0.5cm}
    \underset{x\to \infty}{\Bi (x)} = 2 \times \frac{1}{2\pi} \sqrt{\frac{2\pi}{2 \sqrt{x}}} \exp \left(\frac{2}{3} x^{3/2}\right) = \frac{1}{\sqrt{\pi} x^{1/4}} \exp\left(\frac{2}{3} x^{3/2}\right).
\end{equation*}

Теперь найдём асимптотику $\Bi(x \to -\infty)$. Для начала, как и для $\Ai$:
\begin{equation*}
    F\left(\pm i \sqrt{|x|}\right) = \pm \left(
        i \frac{|x|^{3/2}}{3} - i |x|^{3/2}
    \right) = \mp \frac{2}{3} i |x|^{3/2}.
\end{equation*}
Отличие от $\Ai$ будет только в фазе, из-за другого напраления контура в точке $-i\sqrt{|x|}$:
\begin{align*}
    \varphi|_{+i \sqrt{\pi}} &= - \frac{\pi}{2},
    &&\Rightarrow \hspace{0.5cm}   
    \phi = - \frac{\varphi}{2} + \frac{\pi}{2} = +\frac{3}{4} i \pi, \\
    \varphi|_{-i \sqrt{\pi}} &=  \frac{\pi}{2},
    &&\Rightarrow \hspace{0.5cm}
    \phi = - \frac{\varphi}{2} - \frac{\pi}{2} = -\frac{3}{4} i \pi.
\end{align*}
А значит, с учётом раскрытя $\sqrt{2} \cos (a + \pi/4) = \cos a - \sin a$, получаем выражение для $\Bi$ при  $x \to - \infty$:
\begin{equation*}
    \underset{x\to -\infty}{\Bi (x)} = \frac{1}{\sqrt{\pi} |x|^1/4} \cos\left(
        \frac{2}{3} |x|^{3/2} + \frac{\pi}{4}
    \right).
\end{equation*}



% Eu
% 649
% 67
% 119
% 133
% 183
% 243
% 347
% 479
% 1071
% 1323
% 1506
% 1930

% 