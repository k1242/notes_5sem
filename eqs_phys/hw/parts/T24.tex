\subsection*{Т24}


Свели задачу к поиску нулей функции, вида
\begin{equation*}
    f(x, y) = 2 y \cos(y) - 2 y \ch(x) + (x^2 + y^2) \sin(y).
\end{equation*}
Вспомним, что
\begin{equation*}
    \sin y = \sum_{n=0}^{\infty} \frac{(-1)^n y^{2n+1}}{(2n+1)!},
    \hspace{5 mm} 
    \cos y = \sum_{n=0}^{\infty} \frac{(-1)^n y^{2n}}{(2n)!},
    \hspace{5 mm} 
    \ch x = \sum_{n=0}^{\infty} \frac{x^{2n}}{(2n)!}.
\end{equation*}
Посмотрим на раложение в ряд $f(x, y)$ вблизи точки $x = y = 0$:
\begin{align*}
    f(x, y) &= \sum_{n=0}^{\infty} \left(
        2 \frac{(-1)^n y^{2n+1}}{(2n)!} - 2 \frac{y x^{2n}}{(2n)!} + 
        \frac{(-1)^n x^2 y^{2n+1}}{(2n +1)!} + \frac{(-1)^n y^{2n+3}}{(2n+1)!}
    \right) 
    = \\ &= 
    y \sum_{n=0}^{\infty} \left(
        2 \frac{(-1)^n y^{2n}}{(2n)!} - 2 \frac{x^{2n}}{(2n)!} + 
        \frac{(-1)^n x^2 y^{2n}}{(2n +1)!} + \frac{(-1)^n y^{2n+1}}{(2n+1)!}
    \right).
\end{align*}
Понятно, что $y=0$ -- корень уравнения $f(x, y) = 0$. Покажем, что других не существует на $x, y \in[-1, 1]$. Пусть $y \neq 0$, тогда рассмотрим функцию $g(x, y) = f(x, y)/y$:
\begin{equation*}
    g(x, y) = - \frac{(x^2 + y^2)^2}{12} + \frac{(x^2+y^2)y^4}{120} + \sum_{n=3}^{\infty} \left(
        2 \frac{(-1)^n y^{2n}}{(2n)!} - 2 \frac{x^{2n}}{(2n)!} + 
        \frac{(-1)^n x^2 y^{2n}}{(2n +1)!} + \frac{(-1)^n y^{2n+1}}{(2n+1)!}
    \right).
\end{equation*}

