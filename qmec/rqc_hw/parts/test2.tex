\setcounter{section}{1}

\subsection{Спин электрона в переменном \texorpdfstring{$B$}{B}}

Поместили атомы в переменное магнитное поле вида
\begin{equation*}
    \vc{B}(t) = B_0 \vc{z}_0 + B_\bot \vc{x}_0 \cos (\omega t) + B_\bot \vc{y}_0 \sin \omega t,
\end{equation*}
Тогда гамильтониан получился бы
\begin{equation*}
    \hat{V} = - \hat{\vc{\mu}} \cdot \vc{B} = \bigg/
    \hat{\vc{\mu}} = \frac{e}{m_e c} \hat{\vc{S}} \bigg/
    = -  \left(\frac{e \hbar B_\bot}{2 m_e c}\right) \left[ \vp
        \cos (\omega t) (\kb{+}{-}  + \kb{-}{+})
        - i \sin (\omega t) (\kb{+}{-} - \kb{-}{+}
    \right],
    % \hbar \omega_L  \hat{S}_z \cos (\omega T)
     % \hbar \omega \hat{S}_z \cos (\nu t)
\end{equation*}
откуда и находим, что
\begin{equation*}
    \gamma = - \frac{e \hbar B_\bot}{2 m_e c}.
\end{equation*}







\subsection{Электронный парамагнитный резонанс}

Теперь уже внешнее поле имеет вид
\begin{equation*}
    \vc{B}(t) = B_0 \vc{z}_0 \cos(\nu t),
\end{equation*}
с намильтонианом, вида
\begin{equation*}
    \hat{H} = \hat{H}_0  + \hat{V}(t),
    \hspace{5 mm} 
    \hat{V}(t) = \hbar \omega_L \hat{S}_z \cos (\nu t).
\end{equation*}
Выразим $\hat{S}_z$ и $\hat{S}_x$ (для второй части задания) через базисные состояния $J_z, J^2, S^2, I^2$:
\begin{align*}
    \hat{S}_z &= \frac{1}{2} \left(\vp 
        \kb{0}{0}{10} - \kb{1,-1}{1,-1} + \kb{10}{00} + \kb{11}{11}
    \right),
\end{align*}




\setcounter{subsection}{2}
\subsection{Электродипольный переход \texorpdfstring{$1\textnormal{s}\to 2\textnormal{p}$}{1s -> 2p}}



Поместим атом водорода в поле $\vc{E}$ вида
\begin{equation*}
    \vc{E} (z,\, t) = E_0 \vc{\sigma}_+ e^{-i \omega t + i k z} + \cc, 
    \hspace{5 mm} 
    \vc{\sigma}_+ = - \frac{\vc{x}_0 + i \vc{y}_0}{\sqrt{2}},
    \hspace{0.5cm} \Rightarrow \hspace{0.5cm}   
    \vc{E} (z, t) = - \frac{{E}_0}{\sqrt{2}} \left(
        \vp
        \vc{x}_0 \cos(\omega t) + \vc{y}_0 \sin(\omega t - k z)
    \right).
\end{equation*}
Гамильтониан, описывающий динамику тогда
\begin{equation*}
    \hat{H} = \frac{\hat{\vc{p}}^2}{2 m} - \frac{e^2}{\hat{r}} - \hat{\vc{d}} \cdot \vc{E} (\hat{z},\, t) = H_0 + V(t),
\end{equation*}
где возмущение, зависящее от времени,
\begin{equation*}
    \hat{V}(t) = - \frac{e E_0}{\sqrt{2}} \left( \vp
        \hat{\vc{x}} \cdot\vc{x}_0 \cos(\omega t - kz) +\hat{\vc{x}} \cdot \vc{y}_0 \sin(\omega t - k z)
    \right).
\end{equation*}

\textbf{Электродипольное приближение}. Сразу перейдём к рассмотрению электродипольного приближения и перейдём к сферическим координатам:
\begin{equation*}
    \hat{V}(t) = - \frac{e E_0}{\sqrt{2}} \left( \vp
        \hat{\vc{x}} \cdot\vc{x}_0 \cos(\omega t) +\hat{\vc{x}} \cdot \vc{y}_0 \sin(\omega t)
    \right)
    =
    - \frac{e E_0}{\sqrt{2}} \left( \vp
        r \sin \theta \cos \varphi \cos \omega t +
        r \sin \theta \sin \varphi \sin \omega t
    \right)
    .
\end{equation*}

Систему полагаем при $t=0$ в состоянии $\ket{i} = \ket{n=1,l=0,m=0}$. Из нестационарной теории возмущений можем найти оценку (в первом приближении) для вероятности перехода в состояние $\ket{n}$:
\begin{equation*}
    c_n (t) = \frac{1}{i \hbar} \int_0^t e^{i (E_2-E_1)t/\hbar} \bk{n}[V]{i} \d t.
\end{equation*}



Для состояний водорода знаем волновые функции, разложим их на сферический гармоники, и найдём матричные элементы $\hat{V}$:
\begin{equation*}
    \psi_{00} = \frac{1}{2\sqrt{\pi}},
    \hspace{5 mm} 
    \left.\begin{aligned}
        \psi_{1,1} &= - \tfrac{1}{2} e^{i \varphi} \sqrt{\tfrac{3}{2\pi}} \sin \theta, \\
        \psi_{1,1} &= \tfrac{1}{2} \sqrt{\tfrac{3}{\pi}} \cos \theta, \\ 
        \psi_{1,-1} &=  \tfrac{1}{2} e^{-i \varphi} \sqrt{\tfrac{3}{2\pi}} \sin \theta, \\
    \end{aligned}\right.
    \hspace{0.5cm} \Rightarrow \hspace{0.5cm}
    \left.\begin{aligned}
        \psi_{1,1} \hat{V} \psi_{00}\con &= -\tfrac{1}{4\pi} \sqrt{\tfrac{3}{2}} e^{i \varphi} \cos(\varphi - \omega t) \sin^2(\theta). \\
        \psi_{1,0} \hat{V} \psi_{00}\con &= \tfrac{1}{4\pi} \sqrt{\tfrac{3}{4}} \cos(\varphi - \omega t) \sin (2\theta), \\
        \psi_{1,-1} \hat{V} \psi_{00}\con &= \tfrac{1}{4\pi} \sqrt{\tfrac{3}{2}} e^{-i \varphi} \cos(\varphi - \omega t) \sin^2(\theta). \\
    \end{aligned}\right.
\end{equation*}
Осталось проинтегрировать, и получить
\begin{align*}
    \bk{\psi_{1,1}}[\hat{V}]{\psi_{00}} &= -\tfrac{1}{8} \sqrt{\tfrac{3}{2}} \pi  e^{i \omega t}, \\
    \bk{\psi_{1,0}}[\hat{V}]{\psi_{00}} &= 0 \\
    \bk{\psi_{1,-1}}[\hat{V}]{\psi_{00}} &= \tfrac{1}{8} \sqrt{\tfrac{3}{2}} \pi  e^{-i \omega t}.
\end{align*}
Подставим это в выражение для $c_n (t)$, и получим
\begin{align*}
    c_{\ket{2,1,1}} (t) &= -\frac{\pi}{8} \sqrt{\frac{3}{4}} \frac{e E f_r}{\Delta E + \omega \hbar}
    \left(1 - \exp\left(\frac{i t (\Delta E + \omega \hbar)}{\hbar}\right)\right)
    , \\ 
    c_{\ket{2,1,-1}} (t) &= \frac{\pi}{8} \sqrt{\frac{3}{4}} \frac{e E f_r}{\Delta E - \omega \hbar}
    \left(1 - \exp\left(\frac{i t (\Delta E - \omega \hbar)}{\hbar}\right)\right)
\end{align*}
где $f_r$ возникает из интегрирования по $r$, -- некоторая размерная констанста для этого перехода.
Вероятности же перехода получаются равными
\begin{align*}
    |c_{\ket{2,1,1}} (t)|^2 &= 
        \frac{3\pi^2}{64} \left(\frac{E e f_r}{\Delta E + \omega \hbar}\right)^2  \sin^2\left(
            \frac{1}{2} \frac{\Delta E + \omega \hbar}{ \hbar} t
        \right);
    \\
    |c_{\ket{2,1,-1}} (t)|^2 &=  
        \frac{3\pi^2}{64} \left(\frac{E e f_r}{\Delta E - \omega \hbar}\right)^2  \sin^2\left(
            \frac{1}{2} \frac{\Delta E - \omega \hbar}{ \hbar} t
        \right).
    \\
\end{align*}
Получается, что в резонансе при $\omega = \Delta E / \hbar$ будет происходить переход в $\ket{2,1,-1}$, а при $\omega = -\Delta E / \hbar$ будет происходить переход в $\ket{2,1,1}$.
Аналогично можно показать, что при $\vc{E} \parallel \vc{z}_0$ будет происходить переход в $\ket{2,1,0}$.
% , где в начале $|c_{\ket{2,1,-1}} (t)|^2 \sim \frac{3\pi}{256}\left( \frac{e f_r E}{\hbar} \right)^2 \, t^2$.


\textbf{Условие резонанса}. Для резонанса необходимо
\begin{equation*}
    \omega = \frac{E_2 - E_1}{\hbar} \approx 1.5 \times 10^{16} \text{ Гц} \sim \lambda = 19 \text{ нм},
\end{equation*}
что соответствует сильно ультрафиолетовому свету. 

Найдём частоту Раби, так как задача в резонансе сводится в системе с двумя состояниями, можем найти
\begin{equation*}
    \gamma = \frac{e E_0 f_r}{\sqrt{2}} \approx e E_0 a_0,
\end{equation*}
где $a_0$ -- радиус Бора. 