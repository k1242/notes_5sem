researchers utilized an electrostatic lens that magnifies the outgoing electron wave without disrupting its quantum coherence

measurements of outgoing electrons from multiphoton ionization of molecules showed evidence of nodal planes in molecular orbitals

researchers can select and/or manipulate the molecules using static fields or laser techniques to control or induce a dipole

applying a dc electric field that defines a quantization axis in hydrogen and aligns the orbitals before measuring them

direct observation of the transverse orbital state, which is the projection of the orbital onto the plane perpendicular to the electric field
Прямое наблюдение поперечного орбитального состояния, которое является проекцией орбитального на плоскость, перпендикулярно электрическому полю

observe the orbital density of the hydrogen atom by measuring a single interference pattern on a 2D detector (see Fig. 1). This avoids the complex reconstructions of indirect methods
Соблюдайте орбитальную плотность атома водорода путем измерения одной структуры помех на 2D-детекторе (см. Рис. 1). Это позволяет избежать сложных реконструкций косвенных методов



The team starts with a beam of hydrogen atoms that they expose to a transverse laser pulse, which moves the population of atoms from the ground state to the 2s and 2p orbitals via two-photon excitation. A second tunable pulse moves the electron into a highly excited Rydberg state, in which the orbital is typically far from the central nucleus. By tuning the wavelength of the exciting pulse, the authors control the exact quantum numbers of the state they populate, thereby manipulating the number of nodes in the wave function. The laser pulses are tuned to excite those states with principal quantum number n equal to 30.

Команда начинается с пучка атомов водорода, которые они подвергают поперечному лазерному импульсу, который перемещает популяцию атомов из основного состояния на 2S и 2P-орбитали через двухфотонные возбуждения. Второй перестраиваемый импульс перемещает электрон в высоко возбужденное состояние Rydberg, в котором орбиталь обычно далеко от центрального ядра. Настраивая длину волны возбуждающего пульса, авторы контролируют точные квантовые числа состояния, которые они заполняют, тем самым манипулируя количеством узлов в волновой функции. Лазерные импульсы настроены на то, чтобы возбудить эти состояния с главным квантовым числом N, равным 30.



The presence of the dc field places the Rydberg electron above the classical ionization threshold but below the field-free ionization energy. The electron cannot exit against the dc field, but it is a free particle in many other directions. The outgoing electron wave accumulates a different phase, depending on the direction of its initial velocity. The portion of the electron wave initially directed toward the 2D detector (direct trajectories) interferes with the portion initially directed away from the detector (indirect trajectories). This produces an interference pattern on the detector. Stodolna et al. show convincing evidence that the number of nodes in the detected interference pattern exactly reproduces the nodal structure of the orbital populated by their excitation pulse. Thus the photoionization microscope provides the ability to directly visualize quantum orbital features using a macroscopic imaging device.

Наличие полей постоянного тока помещает электрон Rydberg выше порогового значения классического ионизации, но ниже энергии ионизации без полей. Электрон не может выйти против поля постоянного тока, но это свободная частица во многих других направлениях. Исходящая электронная волна накапливает различную фазу в зависимости от направления его начальной скорости. Часть электронной волны, первоначально направленной на 2D-детектор (прямые траектории), мешает участию, изначально направленной от детектора (косвенные траектории). Это производит узор помехи на детекторе. Stodolna et al. Показать убедительные доказательства того, что количество узлов в обнаруженной структуре помех точно воспроизводит узловую структуру орбитальной, населенной, населенной их импульсом возбуждения. Таким образом, микроскоп фотоионизации обеспечивает возможность непосредственно визуализации квантовых орбитальных элементов с использованием устройства макроскопического визуализации.