\Tsec{Задача №8}

\begin{leftrules}
Найти уровни энергии\footnote{
    Формально <<уровень энергии>>, $\delta$-ямая -- всегда мелкая яма, то есть $\exists !$ связное состояние.
} и волновые функции стационарных состояний частицы в потенциале $U(x) = - \frac{\hbar^2}{m}\kappa_0 \delta(x)$.
\end{leftrules}


\textbf{Координатное представление}. 
Cделаем замечание, что $E < 0$, тогда получим
\begin{equation*}
    \hat{H} \psi = - |E| \psi,
    \hspace{1 cm}
    \kappa^2 \overset{\mathrm{def}}{=}  \frac{2 m |E|}{\hbar}.
\end{equation*}
С такой заменой получим:
\begin{equation*}
    -\frac{\hbar^2}{2m} \psi'' - \frac{\hbar^2}{m}\kappa_0 \delta(x) \psi + |E|\psi = 0
    \hspace{1 cm}
    \Rightarrow
    \hspace{1 cm}
    \psi'' - (\kappa - 2 \kappa_0 \delta(x))\psi=0.
\end{equation*}
Мы ожидаем непрерывности от волной функции на границах областей, а именно в точке дельта-ямы, то есть одним из граничных условий будет $\psi(-0) = \psi(+0)$.

Потребовав непрерывности $\psi$, из-за дельта функции,  мы получаем разрыв для первой производной
\begin{equation*}
    \psi'' - (\kappa - 2 \kappa_0 \delta(x))\psi=0
    \hspace{1 cm}
    \overset{\int_{-\varepsilon}^{+\varepsilon} }{\Longrightarrow}
    \hspace{1 cm}
    \psi'(+0) - \psi'(-0) = - 2 \kappa_0 \psi(0).
\end{equation*}

Вне ямы будем наблюдать спад по экспоненте, сама же яма -- по сути точечна, значит такое же поведение будем ожидать и в связном состоянии, таким образом ищем волновую функцию как
\begin{equation*}
    \psi = \left\{\begin{aligned}
        C_1 e^{-\kappa x} \ , \ & x>0\\
        C_2 e^{\kappa x} \ , \ & x<0
    \end{aligned}\right.
\end{equation*}
Из непрерывности получим автоматически, что
\begin{equation*}
    \psi(-0) = \psi(+0)
    \hspace{1 cm}
    \Rightarrow
    \hspace{1 cm}
    C_2 = C_1 = C.
\end{equation*}
Разрыв же первой производной позволит нам найти
\begin{equation*}
    \psi'(+0) - \psi'(-0) = - 2 \kappa_0 \psi(0)
    \hspace{0.5 cm}
    \Rightarrow
    \hspace{0.5 cm}
    - 2 \kappa_0 C = C (-\kappa - \kappa)
    \hspace{0.5 cm}
    \Rightarrow
    \hspace{0.5 cm}
    \kappa = \kappa_0.
\end{equation*}
Таким образом  энергия связного состояния:
\begin{equation*}
    E = - \frac{\hbar^2 \kappa_0^2}{2 m}.
\end{equation*}
Теперь, осталось проверить нормировку нашей волновой функции
\begin{equation*}
    \int_{\mathbb{R}} \psi \psi^* d x = 1
    \hspace{0.5 cm}
    \Rightarrow
    \hspace{0.5 cm}
    C^2 \int_{-\infty}^{+\infty} e^{- 2 \kappa_0 |x|} d x = \frac{C^2}{\kappa_0} \int_{0}^{+\infty} e^{- 2 \kappa_0 x} d 2 \kappa_0 x
    = 
    \frac{C^2}{\kappa_0} = 1
    \hspace{0.5 cm}
    \Rightarrow
    \hspace{0.5 cm}
    \kappa_0 = C^2.
\end{equation*}
Таким образом собирая всё вместе получаем волновую функцию вида:
\begin{equation*}
    \bk{x}{0} = \psi(x) =  \sqrt{\kappa_0} e^{- \kappa_0|x|}.
\end{equation*}

\textbf{Импульсное представление}. Вставляя разбиение единицы, вида $\mathbbm{1} = \int_{\mathbb{R}} \kb{x}{x} \d x$, находим
\begin{equation*}
    \psi(p) = \bk{p}{0} = \int_\mathbb{R} d x \bk{p}{x} \bk{x}{0}
    =
    \bigg/
    \bk{p}{x} = \frac{1}{\sqrt{2 \pi \hbar}} e^{- \frac{i}{\hbar} p x
    }
    \bigg/  
    =
    \frac{\sqrt{\kappa_0}}{\sqrt{2 \pi \hbar}} \int_\mathbb{R} e^{- \kappa_0 x - \frac{i}{h}p x} d x
    =
    \frac{\sqrt{\kappa_0}}{\sqrt{2 \pi \hbar}} \cdot \frac{2 \kappa_0}{\kappa_0^2 + (p/\hbar)^2}
    = \sqrt{\frac{2}{\pi}} \frac{(\kappa_0 \hbar)^{3/2}}{(\kappa_0 \hbar)^2 + p^2}.
\end{equation*}
