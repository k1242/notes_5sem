\Tsec{Задача №15}
\begin{leftrules}
	Квазиклассическое рассмотрение $\alpha$-распада, закон Гейгера–Неттола.
\end{leftrules}
Стоит всё-таки вспомнить потенциал из задания, ведь именно такой потенциал впервые рассматривался в теории $\alpha$-распада.
И так
\begin{equation*}
	U(x) = \left\{
	\begin{aligned}
		& - U_0, &\ &0<x<a\\
		& \tfrac{2 Z e^2}{x}, &\ &x > a
	\end{aligned}
	\right.
\end{equation*}
 Коэффициент прохождения через такой барьер, как отношение входящего потока к выходящему
 \begin{equation*}
 	\mathcal{T} \sim \exp \left( - \frac{2}{\hbar} \int_a^b |p(x)| d x \right),
 	\hspace{1 cm}
 	\int_{a}^{b} |p(x)| d x  = \int_a^b \sqrt{2m |E - U(x)|} d x = \sqrt{2 m} \int_{a}^{2 Ze^2/E} \sqrt{\frac{2 Z e^2}{x} - E} d x.
 \end{equation*}
 Остается только вычислить интеграл, для удобства введём $\beta = \frac{a E}{2 Z e^2}$, тогда
 \begin{equation*}
 	\mathcal{T} \sim \exp\left[- \frac{4 Z e^2}{\hbar} \sqrt{\frac{2 m}{E}} (\arccos \sqrt{\beta} - \sqrt{\beta(1-\beta)})\right]
 \end{equation*}
 И если мы уже далеко отошли от пика $x = a$, то $\beta \ll 1$ и соответственно коэффициент пропускания
 \begin{equation*}
 	\mathcal{T} \sim \exp\left(- \frac{2 \pi Z e^2}{\hbar} \sqrt{\frac{2m}{E}}\right).
 \end{equation*}

 Соответственно альфа частицы из потенциала такого вида могут туннелировать, с известным теперь нам коэффициентом, то есть излучаться. 
 Таким образом вероятность излучения в единицу времени пропорциональна пропусканию: $w = n \mathcal{T}$, где $n$ -- частота столкновений частиц с барьером.
 Если ввести характерную скорость частиц в потенциальной яме ($x<a$), то 
 \begin{equation*}
 	n \sim v / a,
 	\hspace{1 cm}
 	v \sim p/m_\alpha \sim \frac{\hbar}{m_\alpha a}
 	\hspace{0.5 cm}
 	\Rightarrow
 	\hspace{0.5 cm}
 	n \sim \frac{\hbar}{m_\alpha a^2}.
 \end{equation*}
 Вероятность распада в единицу времени $\lambda$ связан с периодом полураспада $T$
 \begin{equation*}
 	T_{1/2} = \frac{\ln 2}{w} \approx \frac{m_\alpha a^2 \ln 2}{\hbar \mathcal{T}} = C_1 \exp\left(\frac{C_2}{\sqrt{E}}\right)
 	\hspace{0.5 cm}
 	\Rightarrow
 	\hspace{0.5 cm}
 	\boxed{\ln T_{1/2} = A + \frac{B}{\sqrt{E}}}
 \end{equation*}
Конечное равенство представляет собой уравнение \textit{Гейгера-Неттола}. Константы выписывать конечно не интересно, ведь и так наши рассуждения носят оценочный характер, но на всякий случай
\begin{equation*}
	A = \ln C_1 \approx \ln \left(\frac{m_\alpha a^2 \ln 2}{\hbar}\right),
	\hspace{1.5 cm}
	B = C_2 \approx \frac{2 \pi Z e^2 \sqrt{2 m}}{\hbar}.
\end{equation*}