\Tsec{Задача №6}
\begin{leftrules}
	Доказать равенство $(\vc{\sigma} \cdot \vc{a})(\vc{\sigma} \cdot \vc{b}) = (\vc{a} \cdot \vc{b}) + i \vc{\sigma}\cdot [\vc{a} \times \vc{b}]$, где $\vc{a}, \vc{b}$ -- векторы.
\end{leftrules}
Напомним, как выглядят матрицы Паули сами по себе
\begin{equation*}
	\vc{\hat{\sigma}}_x = \begin{pmatrix}
	    0 & 1 \\
	    1 & 0 \\
	\end{pmatrix},
	\hspace{1 cm}
	\vc{\hat{\sigma}}_y = \begin{pmatrix}
	    0 & -i \\
	    i & 0 \\
	\end{pmatrix},
	\hspace{1 cm}
	\vc{\hat{\sigma}}_z = \begin{pmatrix}
	    1 & 0 \\
	    0 & -1 \\
	\end{pmatrix},
\end{equation*}
а также как выглядят коммутатор и антикоммутатор
\begin{equation*}
	[\vc{\hat{\sigma}}_\alpha, \vc{\hat{\sigma}}_\beta] 
	= 2 i \varepsilon_{\alpha \beta \gamma} \vc{\hat{\sigma}}_\gamma,
	\hspace{0.5 cm}
	\{\vc{\hat{\sigma}}_\alpha, \vc{\hat{\sigma}}_\beta\}
	= \vc{\hat{\sigma}}_\alpha \vc{\hat{\sigma}}_\beta + \vc{\hat{\sigma}}_\beta \vc{\hat{\sigma}}_\alpha
	= 2 \delta_{\alpha \beta}
\end{equation*}
Таким образом сложив коммутатор и антикоммутатор получим
\begin{equation*}
	\vc{\hat{\sigma}}_\alpha \vc{\hat{\sigma}}_\beta  = \frac{1}{2} [\vc{\hat{\sigma}}_\alpha, \vc{\hat{\sigma}}_\beta] + \frac{1}{2} \{\vc{\hat{\sigma}}_\alpha, \vc{\hat{\sigma}}_\beta\}
	=
	i \varepsilon_{\alpha \beta \gamma} \vc{\hat{\sigma}}_\gamma + \delta_{\alpha\beta},
\end{equation*}
А значит, мы знаем и как найти требуемое скалярное произведение
\begin{equation*}
	(\vc{\sigma} \cdot \vc{a})(\vc{\sigma} \cdot \vc{b}) = 
	\vc{\hat{\sigma}}_\alpha \vc{a}^\alpha + \vc{\hat{\sigma}}_\beta \vc{b}^\beta
	=
	i \vc{\hat{\sigma}}_\gamma \varepsilon_{\alpha \beta \gamma} \vc{a}^\alpha \vc{b}^\beta  + \delta_{\alpha\beta} \vc{a}^\alpha \vc{b}^\beta
	=
	i \vc{\hat{\sigma}}_\gamma [\vc{a} \times \vc{b}]^\gamma + \vc{a}^\alpha \vc{b}_\alpha
	=
	i [\vc{a} \times \vc{b}]\cdot \vc{\hat{\sigma}} + (\vc{a} \cdot \vc{b}).
\end{equation*}
