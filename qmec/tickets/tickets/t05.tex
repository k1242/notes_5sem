\Tsec{Задача №5}
\begin{leftrules}
	Вычислить коммутаторы $\hspace{1 cm} [l_\alpha, x_\beta],
		\hspace{1 cm}
		[l_\alpha, p_\beta]$.
\end{leftrules}
Напомним, что мы работаем с
\begin{equation*}
	l_\alpha = \varepsilon_{\alpha}^{\phantom{0} \beta \gamma} r_\beta p_\gamma
	\hspace{1 cm}
	[l_\alpha, a_\beta] = i \varepsilon_{\alpha \beta }^{\phantom{0}\phantom{0}\gamma} a_\gamma,
	\hspace{1 cm}
	\hat{p}_\gamma = - i \partial_\gamma.
\end{equation*}
Тогда подействуем на функцию состояния
\begin{equation*}
	[l_\alpha, x_\beta] \psi(x)= [ - i \varepsilon_{\alpha \beta' \gamma} x^{\beta'} \partial^{\gamma}, x_\beta] \psi(x)
	=
	- i \varepsilon_{\alpha \beta' \gamma} x^{\beta'} \partial^\gamma (x_\beta \psi(x)) + i \varepsilon_{\alpha \beta' \gamma} x^{\beta'} x_\beta \partial^\gamma \psi(x)
	=
	- i \varepsilon_{\alpha \beta' \beta} x^{\beta'} \psi(x)
\end{equation*}
Таким образом мы свели действие заданного оператора к
\begin{equation*}
	[l_\alpha, x_\beta] = - i \varepsilon_{\alpha \beta \gamma} x^{\beta}s
\end{equation*}
Для другого случая действуем абсолютно аналогично
\begin{equation*}
	[l_\alpha, p_\beta] \psi(x)
	=
	[ - i \varepsilon_{\alpha \beta' \gamma} x^{\beta'} \partial^{\gamma}, -i \partial_\beta] \psi(x)
	=
	-\varepsilon_{\alpha \beta' \gamma} x^{\beta'} \partial^{\gamma} \partial_\beta \psi(x)
	+ \varepsilon_{\alpha \beta' \gamma} \partial_\beta (x^{\beta'} \partial^{\gamma} \psi(x))
	=
	\varepsilon_{\alpha \beta \gamma}  \partial^{\gamma} \psi(x)
\end{equation*}
Таким образом мы свели действие заданного оператора к
\begin{equation*}
	[l_\alpha, p_\beta] = \varepsilon_{\alpha \beta \gamma}  \partial^{\gamma} = - i \varepsilon_{\alpha \beta \gamma} p^\gamma
\end{equation*}