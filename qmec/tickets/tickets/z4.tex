\Tsec{Задача №4}

\begin{leftrules}
Найти операторы рождения и уничтожения для гармонического осцилляора в представлении Гейзенберга. 
\end{leftrules}


\textbf{I}. Запишем уравнение Гейзенберга
\begin{equation*}
    i \hbar \frac{\hat{d f}}{d t}  = i \hbar \frac{\partial \hat{f}}{\partial t} + \left[\hat{f},\, \hat{H}\right].
\end{equation*}
Запищем гамильтониан системы
\begin{equation*}
    \hat{H} = \hbar \omega \left(\hat{a}\con \hat{a} + \frac{1}{2}\right),
\end{equation*}
тогда можем найти
\begin{equation*}
    i \hbar \frac{\hat{d a}}{d t} = \hbar \omega \left[\hat{a},\, \hat{a}\con \hat{a}\right] = \hbar \omega \left(
        \hat{a} \hat{a}\con \hat{a} - \hat{a}\con \hat{a} \hat{a}
    \right) = \hbar \omega \left(
        \left[\hat{a},\, \hat{a}\con\right] \hat{a}
    \right) = \hbar \omega \hat{a},
\end{equation*}
и, решая диффур, находим
\begin{equation*}
    i \hbar \frac{\hat{d a}}{d t} = \hbar \omega \hat{a},
    \hspace{0.5cm} \Rightarrow \hspace{0.5cm}
    \left\{\begin{aligned}
        \hat{a}(t) &= e^{- i \omega t} \hat{a}, \\
        \hat{a}\con(t) &= e^{i \omega t} \hat{a}\con.
    \end{aligned}\right.
\end{equation*}

\textbf{II}. Можно было напрямую, воспользоваться
\begin{equation*}
    \hat{U}(t) = \exp\left(
        - \frac{i}{\hbar} \hat{H} t
    \right),
    \hspace{0.5cm} \Rightarrow \hspace{0.5cm}
    \hat{a}(t) = \hat{a} + (i \omega t) [
        \hat{a}\con \hat{a},\, \hat{a}
    ] + (i \omega t)^2 [\hat{a}\con \hat{a},\, -\hat{a}] + \ldots = \exp(- i \omega t) \hat{a}, 
\end{equation*}
где мы воспользовались равенством, доказанным в У6:
\begin{equation*}
    e^{\xi A} B e^{- \xi A} = B + \xi [A, B] + \frac{1}{2!} \xi^2 \left[A, [A, B]\right] + \ldots
\end{equation*}
