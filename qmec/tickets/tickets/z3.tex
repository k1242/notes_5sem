\Tsec{Задача №3}

Докажем соотношение Фейнмана-Гелмана:
\begin{equation*}
    \partial_\lambda f_n (\lambda) = \bk{n}[\partial_\lambda \hat{f}(\lambda)]{n},
\end{equation*}
где $f_n$ -- собственное значение $\hat{f} \ket{n} = f_n \ket{n}$, то есть $f_n = \bk{n}[\hat{f}]{n}$. 

По формуле Лейбница:
\begin{align*}
    \partial_\lambda f_n 
    &= 
    \bk{n}[\partial_\lambda \hat{f}]{n} + 
    \bk{\partial_\lambda n}[\hat{f}]{n} + \bk{n}[\hat{f}]{\partial_\lambda n} 
    =
    \bk{n}[\partial_\lambda \hat{f}]{n} + \bk{\partial_\lambda n}{n} f_n + 
    \bk{n}{\partial_\lambda n} f_n 
    = \\ &=
    \bk{n}[\partial_\lambda \hat{f}]{n} + f_n \partial_\lambda \bk{n}{n} 
    = 
    \bk{n}[\partial_\lambda \hat{f}]{n},
\end{align*}
что и требовалось доказать.