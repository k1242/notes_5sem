% document's head

\begin{center}
    \LARGE \textsc{Задание по квантовой механике}
\end{center}

\hrule

\phantom{42}

\begin{flushright}
    \begin{tabular}{rr}
    % written by:
        % \textbf{Источник}: 
        % & \href{__ссылка__}{__название__} \\
        % & \\
        % \textbf{Лектор}: 
        % & _ФИО_ \\
        % & \\
        % \textbf{Автор заметок}: 
        \textbf{Автор заметок}: 
        & Хоружий Кирилл \\
        % & Примак Евгений \\
        % & Гурьева Соня \\
        & \\
    % date:
        \textbf{От}: &
        \textit{\today}\\
    \end{tabular}
\end{flushright}
\thispagestyle{empty}


% \phantom{42}

% \noindent
% \begin{minipage}{0.7\textwidth}
%      \begin{quotation}
%         То, что остаётся после всех этих абстракций, не следует ли... считать тем реальным и неизменным содержанием, которое навязывается существам всех видов с одинаковой необходимостью, потому что оно не зависит ни от индивида, ни от момента времени, ни от точки зрения? \\
%         \phantom{42} \hfill \textit{В. И. Ленин}
%     \end{quotation}
% \end{minipage}


% \vfill


% \vspace*{\fill}
% \begin{figure}[h]
%     \centering
%     \includegraphics[width=0.95\textwidth]{figures/pic1.jpeg}
%     %\caption{}
%     %\label{fig:}
% \end{figure}

% \vfill

% \begin{flushright}
%     \textcolor{gray}{
%     \small{Также выражаем благодарность Мещрякову Павлу за консультации по отдельным задачам.}}
% \end{flushright}


% \newpage

\tableofcontents
\newpage