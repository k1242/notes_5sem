Найдём уровни энергии трёхмерного изотропного гармонического осциллятора в ПДСК. Запишем уравнение Шрёдингера примет вид
\begin{equation*}
    \hat{H} \psi = E \psi, 
\end{equation*}
и перейдём к безразмерным величинам
\begin{equation*}
    \hat{\vc{Q}} = \frac{\hat{\vc{q}}}{q_0}, \hspace{5 mm} 
    \hat{\vc{P}} = \frac{\hat{\vc{p}}}{p_0} = - i \partial_{\smallvc{Q}},
    \hspace{10 mm} 
    p_0 = \sqrt{m \omega \hbar}, \hspace{5 mm} 
    q_0 = \sqrt{\frac{\hbar}{m \omega}},
    \hspace{0.5cm} \Rightarrow \hspace{0.5cm}
    \hat{H}_Q = \frac{1}{2}\left(Q^2 + \hat{P}^2\right) = \hat{H}/(\hbar \omega).
\end{equation*}
Для благоприятного разделения переменных\footnote{
    Здесь и далее $\vc{Q} = (x,\, y,\, z)\T$ -- обезразмеренные для удобства перменные.
}  представим $\psi(x, y, z) = \psi_x(x) \psi_y (y) \psi_z (z)$, и $E = E_x + E_y + E_z$:
\begin{equation*}
    \frac{1}{2} \left({x^2 + y^2 + z^2} - \left(\partial_x^2 + \partial_y^2 + \partial_z^2\right)\right) \psi_x \psi_y \psi_z = \frac{1}{\hbar \omega}(E_x + E_y + E_z) \psi_x \psi_y \psi_z.
\end{equation*}
Нетрудно получить
\begin{equation*}
    \left(x^2 - \frac{\psi_x''(x)}{\psi_x(x)} - \frac{2 E_x}{\hbar \omega}\right) \psi_x \psi_y \psi_z + \ldots \left(z^2 - \frac{\psi_z''(z)}{\psi_z(z)} - \frac{2 E_z}{\hbar \omega}\right) \psi_x \psi_y \psi_z = 0,
\end{equation*}
таким образом переменные разделились и мы получили три независимых уравнения одномерных осцилляторов:
\begin{equation*}
    \left\{\begin{aligned}
        \psi_x''(x) + \left(\tfrac{2 E_x}{\hbar \omega} - x^2\right) \psi_x(x) &= 0, \\
        \psi_y''(y) + \left(\tfrac{2 E_y}{\hbar \omega} - y^2\right) \psi_y(y) &= 0, \\
        \psi_z''(z) + \left(\tfrac{2 E_z}{\hbar \omega} - z^2\right) \psi_z(z) &= 0. \\
    \end{aligned}\right.
    \hspace{0.5cm} \Rightarrow \hspace{0.5cm}
    E_i = \hbar \omega \left(\frac{1}{2} + n_i\right).
\end{equation*}
Так приходим к выражению для энергии изотропного гармонического осциллятора через число квантов по каждой из осей:
\begin{equation*}
    E = E_x + E_y + E_z = \hbar \omega \left(\frac{3}{2}+ n_x + n_y + n_z\right),
\end{equation*}
где явно видно вырождение уровней энергии, при $n_x + n_y + n_z = n$. Нетруно посчитать\footnote{
    $n \geq 0$.
} , что
\begin{equation*}
    \#(n) = \card \left\{
        (n_x,\, n_y,\, n_z) \mid 
        n_x + n_y + n_z = n
    \right\}  = \frac{(n+1)(n+2)}{2},
\end{equation*}
что и является кратностью вырождения.

Заметим, что $\#(0) = 1$, $\#(1) = 3$, $\#(2) = 6$, тогда
\begin{align*}
    2 \cdot \big(\#(0)\big) &= 2, \\
    2 \cdot \big(\#(0)+\#(1)\big) &= 8, \\
    2 \cdot \big(\#(0)+\#(1)+\#(2)\big) &= 20,
\end{align*}
что намекает на некоторую связь с магическими числами (ЛЛ3, \S 118: модель оболочек).