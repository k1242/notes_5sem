Выразим оператор координаты и импульса через операторы рождения и уничтожения:
\begin{equation*}
    \left\{\begin{aligned}
        \hat{q} &= \tfrac{q_0}{\sqrt{2}} (\hat{a}\con + \hat{a}), \\ 
        \hat{p} &= \tfrac{i p_0}{\sqrt{2}} (\hat{a}\con - \hat{a})
    \end{aligned}\right.
    \hspace{5 mm} 
    q_0 =\sqrt{ \frac{\hbar}{m \omega}},
    \hspace{5 mm} 
    p_0 = \sqrt{m \omega \hbar}.
\end{equation*}
Для операторов $\hat{a},\, \hat{a}\con$ верно, что
\begin{equation*}
    \hat{a} \ket{n} = \sqrt{n} \ket{n-1}m
    \hspace{5 mm} 
    \hat{a}\con \ket{n} = \sqrt{n+1} \ket{n+1}.
\end{equation*}
Для начала квадрат координаты:
\begin{equation*}
    \hat{q}^2 = \frac{q_0^2}{2} \left(
        \hat{a}\con{}^2 + \hat{a}^2 + \hat{a}\con \hat{a} + \hat{a} \hat{a}\con
    \right) = \frac{\hbar}{m \omega} \left(n + \frac{1}{2}\right) = \frac{E_n}{m \omega^2}.
\end{equation*}
Стоит заметить, что все <<несбалансированные>> $\hat{a}$ и $\hat{a}\con$ не дадут вклада, так как $\bk{n}{m} = \delta_{nm}$.  Поэтому нечетные степени $\langle q^{2k+1}\rangle = \langle p^{2k+1}\rangle = 0$, так как все оператору будут <<несбалансированы>>.

Четвертая степень координаты:
\begin{equation*}
    \langle q^4\rangle = \frac{q_0^4}{4}\left(
        \hat{a}\con{}^2 + \hat{a}^2 + \hat{a}\con \hat{a} + \hat{a} \hat{a}\con
    \right)^2 = \ldots = \frac{1}{4} \left(\frac{\hbar}{m \omega}\right)^2 \left(6 n^2 + 6n + 3\right).
\end{equation*}
Аналогично, находим квадрат импульса
\begin{equation*}
    \langle \hat{p}^2\rangle = \frac{p_0^2}{2}(2n + 2) 
    =
    p_0^2 \left(n + \frac{1}{2}\right)
    = m E_n
\end{equation*}
И его четвертую степень:
\begin{equation*}
    \langle \hat{p}^4\rangle = \frac{p_0^2}{4} (6 n^2 + 6n + 3) = 
    \frac{(m \omega \hbar)^2}{4} (6 n^2 + 6n + 3)
    ,
\end{equation*}
что объясняется ненулевым вкладом только слагаемых с $(-a\con{})^{4/2}$.

\textbf{Полиномы}. 
Если вдруг будет интересно $F_k (n) = \bk{n}[(\hat{a} \pm \hat{a}\con)^k]{n}$, то ниже приведены посчитанные значения для первых нескольких $k$:
\begin{align*}
    F_4(n) &= 6 n^2+6 n+3; \\
    F_6(n) &= 20 n^3+30 n^2+40 n+15; \\
    F_8(n) &= 70 n^4+140 n^3+350 n^2+280 n+105; \\
\end{align*}
Можно, конечно, продолжить..
\begin{align*}
    F_{10}(n) &= 252 n^5+630 n^4+2520 n^3+3150 n^2+2898 n+945; \\
    F_{12}(n) &= 924 n^6+2772 n^5+16170 n^4+27720 n^3+45276 n^2+31878 n+10395; \\ 
    F_{14}(n) &= 3432 n^7+12012 n^6+96096 n^5+210210 n^4+528528 n^3+588588 n^2+453024 n+135135,
\end{align*}
в общем, да. 

\textbf{Соотношение неопределенностей}. Обсудим величину
\begin{equation*}
    \sqrt{\langle x^2\rangle \langle p^2\rangle} = q_0 p_0 \left(n + \frac{1}{2}\right)
     =
     \hbar \left(n + \frac{1}{2}\right) \geq \frac{\hbar}{2},
\end{equation*}
что полностью соответсвует принципу  неопределенности Гейзенберга. 