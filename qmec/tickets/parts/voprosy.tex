\begin{center}
	\huge Вопросы к экзамену по квантовой механике I.
\end{center}
\begin{flushright}
	\large поток лектора В.В.Киселева, ФОПФ — 5 семестр.
\end{flushright}
\hline
\begin{enumerate}
\item Аксиоматические принципы квантовой механики в картине Шредингера
при каноническом квантовании, консервативные системы и спектральная задача, стационарное уравнение Шредингера.

\item Теоремы Эренфеста. Отличие квантовых уравнений для средних от
классических. Гамильтонианы, для которых квантовые уравнения для
средних совпадают с классическими.

\item Совместная точная измеримость наблюдаемых, вывод необходимого и
достаточного условия.

\item Канонический формализм квантования: скобки Пуассона, флуктуации
в классике и в квантовой механике, каноническое квантование, производная оператора по времени.

\item Формализм Дирака. Гильбертово пространство, обозначения Дирака,
формализм бра- и кет-векторов, проекторы и полнота базиса состояний.
Операторы: эрмитово сопряжение, унитарность, наблюдаемые.

\item Собственные векторы и дисперсия. Соотношение неопределенностей
(вывод методом Вейля).

\item Плотность потока вероятности, уравнение непрерывности, случай свободной нерелятивистской частицы.

\item Волновой пакет, фазовая и групповая скорости, неопределенность координаты и импульса.

\item Импульсное представление: базис, уравнение Шредингера, оператор координаты.

\item Интегралы движения, условия вырождения состояний.

\item Неопределенность энергия-время.

\item Представление Гейзенберга, уравнение Гейзенберга, матричная динамика в дискретном спектре.

\item Одномерное движение. Вид потенциала, вронскиан, невырожденность
в случае ограниченного движения. Условие двукратного вырождение в
непрерывном спектре.

\item Одномерное движение. Вронскиан, дискретный спектр, осцилляторная
теорема.

\item Одномерное движение. Вронскиан, задача рассеяния, соотношения взаимности для коэффициентов прохождения и отражения, сохранение потока вероятности.

\item Гармонический осциллятор. Канонический формализм квантования, характерные значения координаты, импульса и энергии, стационарные
уровни, повышающий и понижающий операторы, их коммутатор и действие на состояния с заданным числом квантов энергии, вывод спектра,
полиномы Эрмита.

\item Когерентные состояния гармонического осциллятора, распределение
Пуассона по числу квантов, голоморфное представление, «полнота базиса» когерентных состояний.

\item Трансляции с непрерывным параметром сдвига, генераторы инфинитезимальных преобразований, общий вид преобразования операторов,
связь с классическими преобразованиями физических величин, генератор трансляций для волновых функций.

\item Генераторы унитарных преобразований в квантовой механике и генераторы канонических преобразований в классической механике, каноническое квантование и связь преобразования квантовых наблюдаемых с
преобразованием этих наблюдаемых в классической механике.

\item Повороты декартовых координат, генераторы вращений как обобщенные
импульсы сдвига по углу вращения, момент импульса.

\item Коммутационные соотношения генераторов поворотов с векторными и
скалярными наблюдаемыми, преобразование волновых функций скалярных частиц, орбитальный момент количества движения.

\item Преобразование волновых функций векторных частиц при поворотах,
полный момент количества движения, спин векторной частицы.

\item Квантование момента импульса из коммутационных соотношений. По-
вышающий и понижающий операторы, старший и младший векторы
представления, вырождение квадрата момента по значениям проекции
на ось, полуцелые значения момента импульса.

\item Допустимые значения орбитального момента, базис в сферических координатах, компоненты оператора орбитального момента в сферических координатах, сферические гармоники, полиномы Лежандра и их свойства, $P$-четность сферических гармоник, число узлов.

\item Спин 1/2. Спинор и матрицы Паули, свойства матриц Паули, матрица
преобразований спинора при вращениях на конечный угол, поворот на
угол $2 \pi$.

\item Сопряженный спинор, эквивалентные представления группы вращения
спиноров SU(2), спинорная метрика.

\item Тождественные частицы и принцип запрета Паули, оператор перестановок тождественных частиц, его эрмитовость и собственные значения, антисиметричные и симметричные волновые функции, фермионы и бозоны, связь спина со статистикой и правило суперотбора.

\item Принцип запрета Паули и фермионный осциллятор, антикоммутационные соотношения операторов рождения и уничтожения, когерентные состояния фермионного осциллятора, голоморфное представление и переменные Грассмана.

\item Задача двух тел. Выделение движения центра масс, волновая функция относительного движения.

\item Центрально симметричный потенциал, квантовые числа, связь оператора квадрата орбитального момента с квадратом импульса и генератором масштабных преобразований $(r \cdot p)$, радиальная волновая функция и ее
уравнение Шредингера, центробежный потенциал.

\item Оператор радиального импульса и его эрмитовость, действие оператора радиального импульса на волновую функцию $u(r) = r R(r)$, вывод асимптотического поведения радиальной волновой функции вблизи нуля.

\item Атом водорода. Атомные единицы, асимптотическое поведение радиальной волновой функции в нуле и на бесконечности для связанных состояний.

\item Атом водорода. Главное и радиальное квантовые числа, спектр связанных состояний, вырождение уровней энергии по орбитальному моменту и P-четности.

\item Квазиклассика (1D): вывод правила квантования Бора–Зоммерфельда
из условия на разность фаз между падающей на потенциальный барьер волны и отраженной от него волны, энергетическая плотность состоя-
ний.

\item Квазиклассика (1D): туннельный эффект, виртуальные частицы, критерий применимости квазиклассики в туннельном эффекте; поправки к энергии в задаче с возмущением.

\item Квазиклассика: (3D) метод JWKB (разложение по ~ 2), классический предел и стационарные состояния; (1D) квазиклассическое приближение в областях потенциала, доступных и недоступных для классической частицы; точки поворота и критерий применимости квазиклассики.
\end{enumerate}