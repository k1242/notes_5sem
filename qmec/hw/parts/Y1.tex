В общем и целом нужно найти $A^*$ и $A^{-1}$ для заданного $A$. 
\subsection*{а) Оператор инверсии}
И так, что же такое оператор инверсии, а это $I \psi(x) = \psi(-x)$.
Обрантый оператор должен по определению
\begin{equation*}
	I^{-1} I \psi(x) = \psi(x)
	\hspace{1 cm}
	\overset{x \mapsto -x}{\Longrightarrow}
	\hspace{1 cm}
	I^{-1} \psi(x) = \psi(-x)
	\hspace{1 cm}
	\Rightarrow
	\hspace{1 cm}
	I^{-1} = I.
\end{equation*}
По опредлению сопряженного оператора $(\bk{\Phi}{I \Psi})^* = \bk{\Psi}{I^* \Phi}$\footnote{тут можно стать свидетелем замены строчной пси на заглавную}.
Напомним
$[I\Psi](x) = \Psi(-x)$, что означает уже для состояний $\bk{x}{I \Psi} = \bk{- x}{\Psi}$, c этим знанием
\begin{equation*}
	\bk{\Phi}{I \Psi} = \int_{\mathbb{R}} \bk{\Phi}{x} \bk{x}{I \Psi} d x
	=
	\int_{\mathbb{R}} \bk{\Phi}{x}\bk{-x}{\Psi} d x =\big/ x \mapsto - x \big/ =  \int_{\mathbb{R}} \bk{\Phi}{- x}\bk{x}{\Psi} d x
	=
	\bk{I \Phi}{\Psi} = \bk{\Phi}{I^* \Psi}
	\hspace{0.5 cm}
	\Rightarrow
	\hspace{0.5 cm}
	I^{*} = I.
\end{equation*}
То есть получили, что оператор инверсии унитарен $I I^* = \mathbb{E}$ (единичный оператор).

\subsection*{б) Оператор трансляции}
Оператор трансляции работает $\hat{T}_a \ket{x} = \ket{x+a}$ или так $\bk{x}{T_a\Psi} = \Psi(x + a)$.

Вполне тривиально, что обратный к оператору трансляции это просто $T_{-a}$. Сопряженный же пойдём искать по той же схеме
\begin{equation*}
	\bk{\Phi}{T_a \Psi} = \int_{\mathbb{R}} \bk{\Phi}{x} \bk{x+a}{\Psi} d x 
	= 
	\big/ x \mapsto x - a\big/
	=
	\int_{\mathbb{R}} \bk{\Phi}{x-a} \bk{x}{\Psi} d x 
	=
	\bk{T_a^{*}\Phi}{\Psi}
	=
	\int_{\mathbb{R}} \bk{T_a^{*}\Phi}{x} \bk{x}{\Psi} d x. 
\end{equation*}
Где предпоследние равенство взято просто по определению сопряженного оператора, а последнее неравенство просто по представлению средней величины, тогда видим, что получается следующее
\begin{align*}
	\bk{x}{T_a^* \Phi} = \Phi(x-a)
	\hspace{1 cm}
	\Rightarrow
	\hspace{1 cm}
	T_a^* = T_{-a} = T_a^{-1}.
\end{align*}
Мы вновь получили, что $T_a^* T_a = \mathbb{E}$ -- унитарнвй оператор.