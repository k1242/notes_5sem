\subsection*{Полиномы Лежандра в атоме водорода}


\begin{flushright}
    \textit{Хоружий Кирилл}
\end{flushright}

\textbf{Intro}.
Вспоминаем вид лестничных операторов
\begin{equation*}
    \hat{l}_{\pm} = e^{\pm i \varphi} \left(
        \pm \partial_\theta + i \ctg \theta \partial_\varphi
    \right).
\end{equation*}
А также $Y_{l,l}$:
\begin{equation*}
    Y_{l,l} (\theta, \varphi) = (-1)^l \sqrt{\frac{(2l+1)!}{4 \pi}} \frac{1}{2^l l!} \sin^l \theta e^{i l \varphi}.
\end{equation*}
Применяем несколько раз $\hat{l}_-$:
\begin{equation*}
    \hat{l}_-^l Y_{l,l} (\theta, \varphi) = \partial_{\cos \theta}^l \sin^l \theta Y_{l,l} (\theta),
    \hspace{0.5cm} \Rightarrow \hspace{0.5cm}   
    Y_{l,0} (\theta, \varphi) = \sqrt{\frac{2l + 1}{4 \pi}} \frac{(-1)^l}{2^l l!} \partial_{\cos \theta}^l \sin^{2l} \theta,
\end{equation*}
что удобно переписать в терминах полиномов Лежандра:
\begin{equation*}
    P_n (x) = \frac{1}{2^n n!} \partial_x^n (x^2-1)^n,
    \hspace{10 mm} 
    Y_{l,0} (\theta, \varphi) = \sqrt{\frac{2l+1}{4 \pi}} P_l (\cos \theta).
\end{equation*}


\textbf{Решение}.
Для начала 
\begin{equation*}
    \hat{l}_- \ket{l, m} = \sqrt{(l+m)(l-m+1)} \ket{l,m-1},
    \hspace{5 mm} 
    \hat{l}_- \ket{l, l} = \sqrt{2l} \ket{l, l-1}.
\end{equation*}
Подсталяя дифференциальное представление $\hat{l}_-$, находим
\begin{equation*}
    Y_{1,0} = \frac{\hat{l}_- Y_{1, 1}}{\sqrt{2}},
    \hspace{0.5cm} \Rightarrow \hspace{0.5cm}
    Y_{1, 0} = \frac{1}{2} \sqrt{\frac{3}{\pi}} \cos \theta.
\end{equation*}
Теперь
\begin{equation*}
    Y_{2,0} (\theta, \varphi) = \sqrt{\frac{5}{4 \pi}} P_2 (\cos \theta),
\end{equation*}
где $P_2$ можем найти, как
\begin{equation*}
    P_2 (x) = \frac{3 x P_1 (x) - P_0(x)}{2} = \frac{3 x^2 -1}{2},
\end{equation*}
а значит
\begin{equation*}
    Y_{2, 0} = 
    \frac{1}{4} \sqrt{\frac{5}{\pi}} (3 \cos \theta^2 -1) 
    = \frac{1}{8} \sqrt{\frac{5}{\pi}} \left(3 \cos (2 \theta) + 1\right).
\end{equation*}
Теперь с помощью $\hat{l}_+$, находим
\begin{align*}
    \hat{l}_+ Y_{l,m} (\theta, \varphi) = \sqrt{l (l+1) - m (m+1)} Y_{l, m+1} (\theta, \varphi),
    \hspace{0.5cm} \Rightarrow \hspace{0.5cm}
    Y_{2, 1} &= \frac{\hat{l}_+ Y_{2,0}}{\sqrt{6}} = - \frac{1}{4} \sqrt{\frac{15}{2 \pi}} e^{i \varphi} \sin(2 \theta), \\
    Y_{2, 2} &= \frac{\hat{l}_+ Y_{2,1}}{2} =  \frac{1}{4} \sqrt{\frac{15}{\pi}} e^{2 i \varphi} \sin^2 (\theta).
\end{align*}