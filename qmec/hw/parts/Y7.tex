Давайте посчитаем коммутаторы, в координатном представлении: $\hat{x} = x$ и $\hat{p}_x = - i \hbar \partial_x$, тогда $\hat{p}_x^2 = - \hbar^2 \partial_x^{2}$. 
\\
\textbf{0)} Начнём с нулевого примера, чтобы убедиться, что правильно смотрим на мир:
\begin{align*}
    [\hat{x},\, \hat{p}] &\psi(x) = 
    x (-i \hbar) \partial_x \psi - (-i \hbar) \partial_x (x \psi) = i \hbar \psi + i \hbar x \partial_x \psi - i \hbar x \partial_x \psi = i \hbar \psi,  \\ 
    \Rightarrow \ \ [\hat{x},\, \hat{p}] &= i \hbar.
\end{align*}
\textbf{a)} Аналогично, в смысле операторного равенства,
\begin{align*}
    [\hat{x},\, \hat{p}^2] &\psi(x) 
    = 
    x (-i \hbar)^2 \partial_x^2 \psi - (-i \hbar)^2 \partial_x^2 (x \psi) 
    = 
    - \hbar^2 x \partial_x^2 \psi + \hbar^2 \partial_x (\psi + x \partial_x \psi) 
    = \\ &\phantom{\psi(x)}= 
    - \hbar^2 x \partial_x^2 \psi + \hbar^2 \partial_x \psi + \hbar^2 \partial_x \psi + \hbar^2 x \partial_x^2 \psi = 2 i \hbar \hat{p} \psi,
    \\ \Rightarrow [\hat{x},\, \hat{p}^2] &= 2 i \hbar \hat{p}.
\end{align*}
\textbf{б)} Теперь найдём коммутатор с некоторой функцией $U(x)$:
\begin{align*}
    [U(\hat{x}),\, \hat{p}]&\psi(x) 
    = 
    U(x) (- i \hbar \partial_x \psi) + i \hbar \partial_x (U \psi) 
    = 
    U (- i \hbar \partial_x \psi) + i \hbar  (\psi \partial_x U + U \partial_x \psi)
    = 
    i \hbar (\partial_x U) \psi,
    \\ \Rightarrow [U(\hat{x}),\, \hat{p}] &= 2 i \hbar \hat{p}.
\end{align*}
\textbf{в)} Наконец, 
\begin{align*}
    [U(\hat{x}),\, \hat{p}^2]&\psi(x) 
    = 
    U (-\hbar^2) \partial_x^2 \psi + \hbar^2 \partial_x^2 U \psi
    =
    U (- \hbar^2) \psi'' + \hbar^2 (\psi U'' + 2 U' \psi' + \psi'' U) 
    = \\ &\phantom{\psi(x)}= 
    \hbar^2 (\psi U'' + 2 U' \psi') 
    =
    (\hbar^2 U'' + \hbar 2i U' \hat{p} )\psi,
    \\ \Rightarrow [U(\hat{x}),\, \hat{p}^2] &= \hbar^2 U'' + 2 i \hbar U' \hat{p}.
\end{align*}