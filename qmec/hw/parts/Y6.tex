В этом упражнении казалось бы раскладываем экспоненты в ряд и радуемся жизни. При чем ну ладно даже до третьего члена, всё перемножаем, находим коммутаторы и радуемся жизни. Ведь никогда дальше второго члена всё равно расскладывать не будем\footnote{При чем не только в рамках этого курса, но и весьма вероятно по жизни в принципе.} 

Конечено всегда можно устроить комбинаторные игрища, но кому оно надо?\footnote{Есть ещё вариант посмотреть как это сделал Валерий Валерьевич, что я сделаю завтра утром, то есть через три часа. Клянусь.}
\begin{equation*}
	e^{\xi A} B e^{- i \xi} = \left(1 + \xi A + \frac{\xi^2 A^2}{2} + \frac{\xi^3 A^3}{6} + \ldots\right) \cdot B \cdot \left(1 - \xi A + \frac{\xi^2 A^2}{2} - \frac{\xi^3 A^3}{6} + \ldots\right)
	=
\end{equation*} 
\begin{equation*}
	= B + \underbrace{\xi A B - \xi B A}_{\xi[A,B]} + \underbrace{\frac{\xi^2}{2}  A^2 B + \frac{\xi^2}{2} B A^2 - \xi^2 A B A}_{\frac{\xi^2}{2} (A [A,B] - [A,B] A) = \frac{\xi^2}{2}[A, [A,B]]} + \ldots
	=
	B + \xi[A,B] + \frac{\xi^2}{2!}[A, [A,B]] + \ldots
\end{equation*}
Что и хотелось показать.