\subsection*{б) потенциальная яма}
И так, зададим потенциальную яму и эволюцию нашей системы
\begin{equation*}
	U(x) = \left\{\begin{aligned}
		- U_0 \ , \ &|x| < a/2\\
		0\ , \ &|x|>a/2
	\end{aligned}\right.
	\hspace{1 cm}
	\Rightarrow
	\hspace{1 cm}
	\begin{aligned}
		&\frac{\hbar^2}{2m} \psi''(x) + (U_0 + E) \psi(x) = 0 \ , \ &|x| < a/2\\
		&\frac{\hbar^2}{2m} \psi''(x) + E \psi(x) = 0 \ , \ &|x|>a/2
	\end{aligned}
\end{equation*}
Мы смотрим на энергию в несвязном состоянии, то есть $E>0$, получаем волновую функцию
\begin{equation*}
	\psi(x) = \left\{\begin{aligned}
		1 \cdot e^{i k_1 x} + A e^{- i k_1 x}\ , \ & x<a/2 \\
		B e^{i k_2 x} + C e^{- i k_2 x} \ , \ & |x|<a/2 \\
		D e^{i k_1 x} + 0 \cdot e^{- i k_1 x}\ , \ & x > a/2
	\end{aligned}\right.
	\hspace{1 cm}
	k_1^2 = \frac{2 m E}{\hbar^2},
	\hspace{0.5 cm}
	k_2^2 = \frac{2 m (E + U_0)^2}{\hbar^2}.
\end{equation*}
Где аналогично Т3, мы выбираем волну падающую из $-\infty$ с единичной амплитудой, и соответсвенно из $+\infty$ к нам ничего не приходит.

Теперь на каждой границе нам нужно взять граничные условия
\begin{equation*}
	\left\{\begin{aligned}
		&\psi(\frac{a}{2} - \varepsilon) = \psi(\frac{a}{2} + \varepsilon)\\
		&\psi(-\frac{a}{2} - \varepsilon) = \psi(-\frac{a}{2} + \varepsilon)\\
		&\psi'(\frac{a}{2} - \varepsilon) = \psi'(\frac{a}{2} + \varepsilon)\\
		&\psi'(-\frac{a}{2} - \varepsilon) = \psi'(-\frac{a}{2} + \varepsilon)
	\end{aligned}\right.
	\hspace{1 cm}
	\Rightarrow
	\hspace{1 cm}
	\left\{\begin{aligned}
		&B e^{i k_2 a/2} + C e^{- i k_2 a/2} = D e^{i k_1 a/2}\\
		&e^{- i k_1 a/2} + A e^{i k_1 a/2} = B e^{i k_2 a/2} + C e^{i k_2 a/2}\\
		&k_1 e^{-i k_1 a/2} - k_1 A e^{i k_1 a/2} = k_2 B e^{- i k_2 a/2} - k_2 C e^{i k_2 a/2}\\
		&k_1 D e^{i k_1 a/2} = k_2 B e^{i k_2 a/2} - k_2 C e^{- i a k_2 a/2}
	\end{aligned}\right.
\end{equation*}
В этот раз мы покажаем какие-то алгебраические выкладки, которые ведут к свету, но на самом деле \textit{Wolfram Mathematica} нам в помощь. Удобно заменить экспоненты в сетпенях $k_1$ и $k_2$ на соответсвующие $\alpha_i$, тогда каким-нибудь Гауссом, система решиться. Здесь приведем просто, что досчитать это реально
\begin{equation*}
	\left\{\begin{aligned}
		&B \alpha_2 + \frac{C}{\alpha_2} = D \alpha_1\\
		&\frac{1}{\alpha_1} + A \alpha_1 = \frac{B}{\alpha_2} + C \alpha_2 \\
		&\frac{k_1}{\alpha_1} - A k_1 \alpha_1 = \frac{k_2 B}{\alpha_2} - k_2 C \alpha_2\\
		&D \alpha_1 k_1 = B \alpha_2 k_2 - \frac{C k_2}{\alpha_2}
	\end{aligned}\right.
	\hspace{1 cm}
	\Rightarrow
	\hspace{1 cm}
	\ldots (\text{мы в вас верим})
\end{equation*}

Куда полезней, сейчас понять, что если помучиться и решить данную систему, то мы по сути и найдём ответ на задачу, ведь
\begin{equation*}
		j_\text{in} [e^{i k_1 x}] = - \frac{i \hbar}{2 m}(i k_1 + i k_1) = \frac{\hbar k_1}{m},
		\hspace{1 cm}
		j_\text{out} [D e^{i k_1 x}] = |D|^2\frac{\hbar k_1}{m},
		\hspace{1 cm}
		j_\text{back} [A e^{-i k_1 x}] = |A|^2\frac{-\hbar k_1}{m},
\end{equation*}
То есть \text{in} -- падающая волна, амплитуду которой мы выбрали единицей, \text{out} -- ушедшая в $+\infty$ с амплитудой $D$ и \textit{back} отразившаяся обратно в $-\infty$.

Поверим, что коэффициенты из той системы получаются и соответственно 
\begin{equation*}
	R = \left| \frac{j_\text{back}}{j_\text{in}}\right| 
	= 
	|A|^2 
	= 
	\frac{(k_1^2 - k_2^2)^2 \sin k_1 a}{4 k_2^2 k_1^2 + (k_1^2 + k_2^2)^2 \sin^2 k_1 a},
\end{equation*}
\begin{equation*}
	T = \left| \frac{j_\text{out}}{j_\text{in}}\right|
	=
	|D|^2
	=
	\frac{4 k_1^2 k_2^2}{4 k_1^2 k_2^2 + (k_1^2 - k_2^2)^2 \sin^2 k_1 a}.
\end{equation*}

\subsection*{a) потенциальный барьер}
Всё остаётся почти таким же, только сейчас проследим за сменой знаков кое-где
\begin{equation*}
	U(x) = \left\{\begin{aligned}
		U_0 \ , \ &|x| < a/2\\
		0\ , \ &|x|>a/2
	\end{aligned}\right.
	\hspace{1 cm}
	\Rightarrow
	\hspace{1 cm}
	\begin{aligned}
		&\frac{\hbar^2}{2m} \psi''(x) + ( E - U_0) \psi(x) = 0 \ , \ &|x| < a/2\\
		&\frac{\hbar^2}{2m} \psi''(x) + E \psi(x) = 0 \ , \ &|x|>a/2
	\end{aligned}
\end{equation*}
Теперь если аналогично предыдущему пункту начать решать задачу, то заметим, что нужно лишь заменить одну! переменную $k_1 \mapsto \kappa_1 = \frac{2 m}{\hbar^2}(U_0 - E)$. При чем, $0 < E < U_0$. И тогда по аналогии $k_1 = i \kappa_1$ получаем:
\begin{equation*}
	R = \frac{(\kappa_1^2 + k_2^2)^2 sh^2 \kappa_1 a}{4 \kappa_1^2 k_2^2 + (\kappa_1 + k_2^2)^2 sh^2 \kappa_1 a},
	\hspace{1 cm}
	T = \frac{4 \kappa_1^2 k_2^2}{4 \kappa_1^2 k_2^2 + (\kappa_1^2 + k_2^2)^2 sh^2 \kappa_1 a}.
\end{equation*}
И главное что в прошлом пункте, что сейчас мы получаем сумму $R + T = 1$ что и ожидается.