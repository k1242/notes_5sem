% лекция АВ Турлапова

\section{Квантовая электродинамика I}

Рассмотрим эволюцию по Гейзенбергу: $\hat{A}^H (t) = \hat{U}\con \hat{A} \hat{U}$, тогда
\begin{align*}
    \frac{d }{d t} \hat{A}^H &= 
    \bigg/
        - i \hbar \frac{d \hat{U}\con}{d t}  = \hat{U}\con \hat{H}
    \bigg/ = 
    \frac{d \hat{U}\con}{d t}  \hat{A} \hat{U} + \hat{U}\con \hat{A} \frac{d \hat{U}}{d t} 
    = \\ &=
    - \frac{\hat{U}\con \hat{H} \hat{U}}{i \hbar} \hat{U}\con \hat{A} \hat{U} + \ldots = 
    \frac{1}{i\hbar} \left[\hat{A}^H, \hat{U}\con \hat{H} \hat{U}\right].
\end{align*}
Так, например, при $\hat{H} = \frac{1}{2m} \hat{p}^2 + V(\hat{x})$, тогда
\begin{equation*}
    \frac{d }{d t} \hat{p}^H = - \frac{d }{d \hat{x}^H}  V(\hat{x}^H),
\end{equation*}
где 
\begin{equation*}
    \hat{x}^H = \exp\left(
        \frac{i \hat{H} t}{\hbar}
    \right) \hat{x} \exp\left(
        - \frac{i \hat{H} t}{\hbar}
    \right),
    \hspace{0.5cm} \overset{?}{\Rightarrow}  \hspace{0.5cm}
    \hat{x}^H = \hat{x} + \frac{\hat{p} t}{m}, 
    \text{ при $\hat{V} = 0$.}
\end{equation*}
Движемся к квантовой электродинамике, хотим говорить про моды переменного 
электромагнитного поля. Квантуем только энергию одной выбранной моды. Форма мод есть из классической теории поля. 
Далее $\hat{H},\, \hat{\vc{E}},\, \hat{\vc{B}}$ -- операторы на состояния заданной моды, а $\vc{x}$ и $t$ -- параметры. 
\begin{equation*}
    \hat{H} = \int \frac{d^3 x}{8\pi} \ 
    \left( \vp
        \vc{E}^2 (\vc{x}, t) + \vc{B}^2 (\vc{x}, t)
    \right), \hspace{5 mm} 
    \text{ --- оператор энергии моды.}
\end{equation*}
\textit{Пастулируем}, что $\hat{\vc{E}}$ и $\hat{\vc{B}}$ удовлетворяют уравнениям Максвелла
\begin{equation*}
    \nabla^2 \hat{\vc{E}} -  \frac{1}{c^2} \frac{\partial^2 \hat{\vc{E}}}{\partial t^2} = 0.
\end{equation*}
В наиболее общем виде, будем считать
\begin{equation*}
    \hat{\vc{E}} (\vc{x}, t) = \hat{\alpha}(t) \vc{E}_0 (\vc{x}) + \hat{\alpha}\con (t)  \vc{E}_0^* (\vc{x}), 
    \hspace{5 mm} 
    \hat{\alpha}(t) = \hat{\alpha} (0) e^{- i \omega t}.
\end{equation*}
Пользуясь уравнениями Максвелла
\begin{equation*}
    \rot \hat{\vc{E}} = - \frac{1}{c} \frac{\partial \hat{\vc{B}}}{\partial t},
    \hspace{5 mm} 
    \partial_t = \pm i \omega.
\end{equation*}
Тогда Гамильтониан получается равным
\begin{equation*}
    \hat{H} = \hbar \omega \left(
        \hat{a}\con (t) \hat{\alpha} (t) + 
        \frac{1}{2}\left[
            \hat{\alpha} (t),\, \hat{\alpha}\con (t)
        \right]
    \right) \underbrace{\frac{2}{\hbar \omega} \int \frac{d^3 x}{4\pi} |\vc{E}_0 (\vc{x})|^2}_{\equiv 1, \ \text{-- нормировка}.}
\end{equation*}
Здесь получается $E_0 \sim V^{-1/2}$, что немного контринтуитивно, но позже с эти поработаем. 
Теперь
\begin{equation*}
    \text{\textit{Пастулируем}, что } [\hat{\alpha}(t),\, \hat{\alpha}\con(t)] = 1.
\end{equation*}
тогда
\begin{equation*}
    \hat{H} = \hbar \omega \left(
        \hat{\alpha}\con (t) \hat{\alpha}(t) + \tfrac{1}{2}
    \right).
\end{equation*}
Теперь и получаем, что
\begin{equation*}
    \hat{\alpha}^+ \hat{\alpha} \ket{n} = n \ket{n}, \hspace{5 mm} 
    \hat{H} \ket{n} = \hbar \omega \left(
        n + \frac{1}{2}
    \right), 
    \hspace{5 mm} 
    \hat{\alpha}(t=0) \ket{n} = \sqrt{n} \ket{n-1},
    \hspace{5 mm} 
    \alpha\hbar^+ (t=0) \ket{n} = \sqrt{n+1} \ket{n+1}.
\end{equation*}
Посмотрим, что происходит с полем:
\begin{equation*}
    \bk{n}[\hat{\vc{E}}(\vc{x}, t)]{n} = 0, 
    \hspace{5 mm} 
    \bk{n}[\hat{\vc{B}} (\vc{x}, t)]{n} = 0.
\end{equation*}
Перейдём к рассмотрению когерентных состояний (как у осциллятора)
\begin{equation*}
    \bk{\lambda}[\hat{\vc{E}}(\vc{x}, t)]{\lambda} = \hat{E}_0 \lambda e^{- i \omega t} + \cc
\end{equation*}
где $\bra{\lambda} \hat{\alpha}^+ (t=0) = \bra{\lambda} \lambda^*$.




% мнимая часть показателя преломления
% посмотреть на действительную


% \section{Реализация квнтовых вентилей}


% https://quantumalgorithmzoo.org/
% отрисовка квантовых цепей вычислений
% вычисления для bra ket нотации


