% урматы

\subsection*{Дифференцируемость}


Проверим дифференцируемость
\begin{equation*}
    \alpha + i \beta = \partial-x  + i \partial_x v,
\end{equation*}
откуда находим
\begin{equation*}
    \left\{\begin{aligned}
        \alpha &= \partial_x u \\
        \beta &=1 \partial_x v
    \end{aligned}\right.
\end{equation*}
При этом
\begin{equation*}
    - \beta + i \alpha  =\partial_y u + i \partial_y v.
\end{equation*}
Так получаем условия Коши-Римана (критерий):
\begin{equation*}
    \left\{\begin{aligned}
        \partial_y u &= - \partial_x v \\
        \partial_y v &= \partial_x u
    \end{aligned}\right.
\end{equation*}


Какие функии дифференцируемы? Например полиномы вида $z^n$, иногда бесконченые
\begin{equation*}
    \sum_{n=0}^{\infty} a_n (z-z_0)^n,
    \hspace{5 mm} 
    R = \lim_{n \to \infty} \frac{1}{\sqrt[n]{|a_n|}},
\end{equation*}
который сходится внутри некоторого ряда. Так, например, экспонента
\begin{equation*}
    e^z = \sum_{n=0}^{\infty} \frac{z^n}{n!}. \hspace{5 mm} 
    R = \lim_{n \to \infty} \sqrt[n]{n!} \to \infty.
\end{equation*}
Можно показать, что
\begin{equation*}
    e^{z_1 + z_2} = e^{z_1} e^{z_2}, 
    \hspace{5 mm} 
    \sin z = \frac{e^{iz}-e^{-iz}}{2i}.
\end{equation*}




\subsection*{Отсутствие дифференцируемости в точке}


\textbf{Устранимая особая точка}. Устранимый выколотый разрыв с конечным значением, например $\frac{\sin z}{z},$.
 % при $z=0$. 


\textbf{Полюса}. $\lim_{z \to a} f(z) \to \infty$, например $\frac{1}{z}$. 


\textbf{Существенная особая точка (гадость)}. Случай, когда предела не существует: $\sin(1/z)$, или $e^{1/z}$.


\subsection*{Интегрируемость}


Давайте, как и в Римане, интегрировать по некоторой кривой, параметризованной отрезком $z = z(t)$, $t \in [a, b]$. Определим интеграл первого рода
\begin{equation*}
    \sum f(z_i) |\Delta z_i | \to \int f(z) |d z|,
\end{equation*}
который встречается достаточно редко. Аналогично определим интеграл второго рода
\begin{equation*}
    \sum_i f(z_i) \Delta z_i \to \int_\gamma f(x) \d z,
\end{equation*}
где $f(x) = u + i v$ и $\d z = \d x + i \d y$. 



\subsection*{Интегрируем по областям}

Пусть есть некоторая область $D$, против часовой опоясываем кривой $\gamma$, введем некоторые области внутри и окружим их по часовой стрелке, введем $\Gamma$, как
\begin{equation*}
    \Gamma = \gamma + \gamma_1 + \gamma_2 + \ldots
\end{equation*}

Докажем теорему Коши:
\begin{equation*}
    \int_\Gamma f(z) \d z = 0.
\end{equation*}
если особенностей нет. 

\begin{proof}[$\triangle$]
Рассмотрим интеграл
\begin{equation*}
    \int_\Gamma (u + i v) (\d x + i \d y) = \int_\Gamma (u \d x - v \d y) + 
    i \int_\Gamma (v \d x + u \d y).
\end{equation*}
Воспользуемся условием Коши-Римана и свойством
\begin{equation*}
    \int_\Gamma P \d x + Q \d y = \iint_D \left(\frac{\partial Q}{\partial x} - \frac{\partial P}{\partial y} \right) \d x \d y,
\end{equation*}
тогда
\begin{equation*}
    \int \ldots = 0
\end{equation*}
\end{proof}

В частности можно показать, что для дифференцируемой функции интегрирование от $z_1$ до $z_2$ не зависит от пути интегрирования -- контур можно гнуть.

Если есть особенности:
\begin{equation*}
    \int_\Gamma f(z) \d z = 2 \pi i \sum_n \textnormal{res}_{z=a_n} f(z),
    \hspace{5 mm} 
    \textnormal{res}_{z=a}\, f(z) = \lim_{r \to 0} \frac{1}{2 \pi i} \oint f(z) \d z.
\end{equation*}

\subsection*{Частности}

Первая мысль. Если $a$ -- УОТ, то $\res_{z=a} f(z)= 0$. Вторая мысль, рассмотрим интеграл от $f(z)/z$:
\begin{equation*}
    \oint_\gamma \frac{f(z)}{z} \d z = \oint \frac{f(0) + \ldots}{z} \d z = 
    \oint f(0) \frac{\d z}{z} = f(0) \int_{0}^{\pi} \frac{r i e^{i \varphi} \d \varphi}{r e^{i \varphi}} = 2 \pi i f(0),
\end{equation*}
где воспользовались $z = r e^{i \varphi}$, откуда
\begin{equation*}
    \res_{z=0} \frac{f(z)}{z} = f(0).
\end{equation*}
Посмотрим теперь на
\begin{equation*}
    \oint \frac{f(z)}{z^2} \d z = \oint \frac{f(0) + f'(0) z + \ldots}{z^2} \d z = 
    \oint \left(
        \frac{f(0)}{z^2} + \frac{f'(0)}{z} \d z
    \right),
\end{equation*}
но, так как наличие первообразной гарантирует равенство нулю интеграла, $1/z^2$ выпадает, тогда
\begin{equation*}
    \oint \frac{f(z)}{z^2} \d z = \oint \frac{f'(0)}{z} \d z = 2 \pi i f'(0),
    \hspace{5 mm} 
    \res_{z = 0} \frac{f(z)}{z^2} = f'(0).
\end{equation*}



\subsection*{Ряд Лорана}

Определим ряд вида
\begin{equation*}
    f(z) = \sum_{n=-\infty}^{\infty}  c_n (z-z_0)^n.
\end{equation*}
Если найти разложение до $-1$ члена, то можно получить, что
\begin{equation*}
    \int f(z) \d z = \oint \frac{c_{-1}}{z-z_0} \d z  = 2 \pi i c_{-1}.
\end{equation*}


% Lorem ipsum dolor sit amet, consectetur adipisicing elit, sed do eiusmod
% tempor incididunt ut labore et dolore magna aliqua. Ut enim ad minim veniam,
% quis nostrud exercitation ullamco laboris nisi ut aliquip ex ea commodo
% consequat. Duis aute irure dolor in reprehenderit in voluptate velit esse
% cillum dolore eu fugiat nulla pariatur. Excepteur sint occaecat cupidatat non
% proident, sunt in culpa qui officia deserunt mollit anim id est laborum.

