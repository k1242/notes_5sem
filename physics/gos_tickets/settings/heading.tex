% document's head

\begin{center}
    \LARGE \textsc{Билеты к государственному экзамену по курсу \\ <<общая физика>>}
\end{center}

\hrule

\phantom{42}

\begin{flushright}
    \begin{tabular}{rr}
    % written by:
        % \textbf{Источник}: 
        % & \href{__ссылка__}{__название__} \\
        % & \\
        % \textbf{Лектор}: 
        % & _ФИО_ \\
        % & \\
        \textbf{Авторы заметок}: 
        & Хоружий Кирилл \\
        & Примак Евгений \\
        & \\
    % date:
        \textbf{От}: &
        \textit{\today}\\
    \end{tabular}
\end{flushright}


\thispagestyle{empty}
% \vfill

\vspace{10mm}

\begin{minipage}{0.7\textwidth}
    \ \ \ \ Кельвин так характеризовал роль математики и теоретической работы в физике: он сравнивал 
    экспериментальные данные с зерном, которое получается на полях экспериментальной физики, а 
    математику — с жёрновами мельницы,
    которые перемалывают эти зёрна.  ... 
    Только если вкладывать вполне определённые экспериментальные данные, экспериментальный результат — математический аппарат, теоретическое мышление может дать вам конкретные, полезные, физические результаты.
    \\ 
    \phantom{42} \hfill \textit{П. Л. Капица}
\end{minipage}
\hfill
\begin{minipage}{0.2\textwidth}
    \centering
    \includegraphics[width=1\textwidth]{figures/preview.jpg}
\end{minipage}

% \newpage
\tableofcontents
\newpage

