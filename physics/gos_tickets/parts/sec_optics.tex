

% \ticket{35}{Интерференция волн}
% \toc{Интерференция волн. Временная и пространственная когерентность. Соотношение неопределенностей.}
% \input{tickets/35}

% \ticket{36}{Принцип Гюйгенса-Френеля}
% \toc{Принцип Гюйгенса-Френеля. Зоны Френеля. Дифракция Френеля и Фраунгофера. Границы применимости геометрической оптики}
% \input{tickets/36}

% \ticket{37}{Спектральные приборы}
% \toc{Спектральные приборы (призма, дифракционная решетка, интерферометр Фабри-Перо) и их основные характеристики.}
% \input{tickets/37}

% \ticket{38}{Дифракционный предел разрешения оптических и спектральных приборов}
% \toc{Дифракционный предел разрешения оптических и спектральных приборов. Критерий Рэлея.}
% \input{tickets/38}

% \ticket{39}{Фурье-оптика}
% \toc{Пространственное фурье-преобразование в оптике. Дифракция на синусоидальных решетках. Теория Аббе формирования изображения.}
% \input{tickets/39}

% \ticket{40}{Голография}
% \toc{Принципы голографии. Голограмма Габора. Голограмма с наклонным опорным пучком. Объемные голограммы.}
% \input{tickets/40}

% \ticket{41}{Фазовая и групповая скорости}
% \toc{Волновой пакет. Фазовая и групповая скорости. Формула Рэлея. Классическая теория дисперсии. Нормальная и аномальная дисперсии.}
% \input{tickets/41}

% \ticket{42}{Поляризация света и кристаллоптика}
% \toc{Поляризация света. Угол Брюстера. Оптические явления в одноосных кристаллах.}
% \input{tickets/42}

% \ticket{43}{Дифракция рентгеновских лучей}
% \toc{Дифракция рентгеновских лучей. Формула Брэгга-Вульфа. Показатель преломления вещества для рентгеновских лучей.}
% \input{tickets/43}