% \ticket{44}{Квантовая природа света}
% \toc{Квантовая природа света. Внешний фотоэффект. Уравнение Эйнштейна. Эффект Комптона.}
% \input{tickets/44}


% \ticket{45}{Спонтанное и вынужденное излучение.}
% \toc{Спонтанное и вынужденное излучение. Инверсная заселенность уровней. Принцип работы лазера.}
% \input{tickets/45}


\ticket{46}{Излучение абсолютно черного тела}
\toc{Излучение абсолютно черного тела. Формула Планка, законы Вина и Стефана-Больцмана.}
\input{tickets/46}


% \ticket{47}{Корпускулярно-волновой дуализм}
% \toc{Корпускулярно-волновой дуализм. Волны де Бройля. Опыты Девиссона-Джермера и Томсона по
% дифракции электронов.}
% \input{tickets/47}


% \ticket{48}{Теоретическая основа квантовой механики}
% \toc{Волновая функция. Операторы координаты и импульса. Средние значения физических величин.
% Соотношение неопределенности для координаты и импульса. Уравнение Шредингера.}
% \input{tickets/48}


% \ticket{49}{Водородоподобный атом}
% \toc{Строение водородоподобного атома. Уровни энергии и кратность их вырождения. Спектр излучения атома водорода.}
% \input{tickets/49}


% \ticket{50}{Опыты Штерна-Герлаха}
% \toc{Опыты Штерна и Герлаха. Спин электрона. Орбитальный и спиновый магнитные моменты электрона.}
% \input{tickets/50}


% \ticket{51}{Тождественность частиц и электронная структура атома}
% \toc{Тождественность частиц. Симметрия волновой функции относительно перестановки частиц. Бозоны и фермионы. Принцип Паули. Электронная структура атомов. Таблица Менделеева.}
% \input{tickets/51}


% \ticket{52}{Тонкая и сверхтонкая структуры}
% \toc{Тонкая и сверхтонкая структуры оптических спектров. Правила отбора при поглощении и испускании фотонов атомами.}
% \input{tickets/52}