\subsection*{Оптические нелинейности}

1. Что нелинейно в нелинейной оптике? Принцип суперпозиции для поляризации среды. Материальные уравнения и их связь с уравнениями Максвелла. Механизмы нелинейного взаимодействия излучения со средами: классификация, особенности.

2. Электронные нерезонансные нелинейности. Общий вид материального уравнения. Квадратичные нелинейные явления. Простейший осциллятор как модель нелинейности: два сопутствующих процесса.

\subsection*{Квадратичная нелинейность}


3. Генерация второй оптической гармоники. Понятие фазового синхронизма. Способ выполнения условия фазового синхронизма. Расчет угла фазового синхронизма при ГВГ.

4. Переход от волнового уравнения к уравнениям для медленных амплитуд; разделение уравнений для волн. Решение уравнений генерации второй оптической гармоники в случае точного синхронизма.

4. Генерация второй гармоники в случае слабого преобразования; роль расстройки. Факторы, ограничивающие эффективность преобразования, связь с расстройкой. Роль длины среды.

5. Генерация суммарной и разностной частот. Типы синхронизмов. Вторая гармоника как генерация суммарной частоты. Типы синхронизмов. Периодически поляризованные кристаллы.

6. Оптическое детектирование. Генерация терагерцового излучения; Терагерцовое излучение как процесс генерации разностных частот.

7. Параметрическая генерация света. Основные свойства спонтанного параметрического излучения. Уравнения генерации параметрического излучения. Особенности бигармонического поля.

8. Нерезонансные электронные нелинейности: явления третьего порядка. Простейший осциллятор как модель нелинейности: два сопутствующих процесса.

\subsection*{Кубическая нелинейность}

9. Генерация третьей оптической гармоники. Условие фазового синхронизма. Способ выполнения условия фазового синхронизма. Расчет угла фазового синхронизма при генерации третьей гармоники.

10. Решение уравнений генерации третьей оптической гармоники в случае точного синхронизма. Факторы, ограничивающие эффективность преобразования.

11. Нелинейный показатель преломления среды; связь с кубичной нелинейностью среды. Роль стрикционного и ориентационного механизмов нелинейности.

12. Самофокусировка излучения. Самофокусировка простейшего гауссова пучка света. Критическая мощность при самофокусировке излучения. Фокусировка импульсного излучения.


\subsection*{Другое}

13. Фазовая самомодуляция излучения. Основной результат взаимодействия. Практические применения: сокращение длительности световых импульсов, генерация гребенки частот.

14. Поляризационные эффекты нелинейного показателя преломления. Нелинейность показателя преломления для линейно поляризованного и кругополяризованного света. Слабая волна в среде под действием сильного излучения.

14. Группы кубичных нелинейных явлений. Двухпучковые нелинейные явления. Самодифракция излучения. Самодифракция излучения при различных поляризациях падающих пучков света.

15. Четырех-волновые смешения в нелинейной оптике: Обращение волны и обращение волнового фронта.

16. Электронные нелинейности, резонансное взаимодействие. Полуклассическая модель. Балансные уравнения. Понятие об интенсивности просветления среды. Задача о просветлении среды и изменении показателя преломления.

17. Методы измерения констант нелинейного взаимодействия: метод z-сканирования.

18. «Ядерные» нелинейности. Роль стрикционного и ориентационного механизмов нелинейности.

19. Вынужденное комбинационное рассеяние (ВКР). Роль спонтанного рассеяния. Основные характеристики излучения ВКР. Особенности энергообмена между волнами при ВКР.

20. Вынужденное рассеяние Мандельштама-Бриллюена (ВРМБ). Роль спонтанного рассеяния. Основные характеристики излучения ВРМБ. Особенности энергообмена между волнами при ВРМБ.