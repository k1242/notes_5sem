\Tsec{Билет №9}

\begin{leftrules}
Генерация третьей оптической гармоники. Условие фазового синхронизма. Способ выполнения условия фазового синхронизма. Расчет угла фазового синхронизма при генерации третьей гармоники.
\end{leftrules}





\textit{Условие фазового синхронизма:} вдали от среды видно $E_{2\omega} \propto \delta(\mathbf{k}_{3\omega}-3\mathbf{k}_{\omega})$, что дает условие ''существенности'' третьей гармоники: $\mathbf{k}_{3\omega}=3\mathbf{k}_{\omega}$, что влечет $n_{3\omega}=n_{\omega}$, что едва ли выполняется в кубически нелинейных средах, однако этого можно достигнуть, например, в отрицательном одноосном криталле, в котором пок-тель преломл. для накачки можно настроить равным пок-телю преломл. для третьей гармоники, если последний лежит меж обыкновенным и необыкновенными пок-телями для накачки.


\textit{Выбор угла для выполнения условия синхронизма:} условие на угол в отрицательном одноосном кристалле: $\frac{cos(\theta)^{2}}{n_{0}^{2}(3\omega)} + \frac{sin^{2}(\theta)}{n_{e}^{2}(3\omega)} = \frac{1}{n_{0}^{2}(\omega)}$.

