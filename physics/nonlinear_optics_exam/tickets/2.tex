\Tsec{Билет №2}

\begin{leftrules}
Электронные нерезонансные нелинейности. Общий вид материального уравнения. Квадратичные нелинейные явления. Простейший осциллятор как модель нелинейности: два сопутствующих процесса.
\end{leftrules}



При электронной нерезонансной нелинейности (прозрачные среды) метериальное уравнение $\mathbf{P}=\mathbf{P}(\mathbf{E})$ представимо в виде: $P=\sum\limits_{i=1}^{i \to \infty} \chi_{i} E^{i}$.

\begin{to_def}
    \textit{Квадратичная//кубическая} электронная нерезонансная нелинейность - нелинейность, обусловленная слагаемым $\chi_{2} E^{2}$ \textit{//} $\chi_{3} E^{3}$ в разложении $P=\sum\limits_{i=1}^{i \to \infty} \chi_{i} E^{i}$.
\end{to_def}

Среди явлений квадратичной нелинейности: \\
    \par \textbf{1)} генерация второй гармоники (ГВГ); \\
    \par \textbf{2)} генерация суммарной частоты (ГСЧ); \\
    \par \textbf{3)} генерация разностной частоты (ГРЧ); \\
    \par \textbf{4)} оптическое выпрямление; \\
    \par \textbf{5)} генерация параметрических волн (ГПВ).


\textit{2 сопутствующих процесса:} электрон в атоме рассматривается как осциллятор с пот. эн. $U(x)=\alpha x^{2} + \beta x^{3}$, что в уравнении $\ddot{x} + 2\gamma \dot{x} + \omega_{0}^{2} x + \frac{3\beta e}{m} x^{2} = Acos(\omega t)$ дает нелинейное смещение $x_{LN} = \frac{3\beta e}{2m} \Big( \frac{eA}{m(\omega^{2}-\omega_{0}^{2})} \Big)^{2} \Big( \frac{-1}{\omega_{0}^{2}} + \frac{cos(2\omega t)}{4\omega^{2}-\omega_{0}^{2}} \Big)$, где слагаемое $-1/\omega^{2}$ дает постоянную поляризацию, пропорциональную интенсивности света (оптическое выпрямление), а $\frac{cos(2\omega t)}{4\omega^{2}-\omega_{0}^{2}}$ дает излучение второй гармоники: \\ $P \propto (a+bcos(2\omega t))$.
