\Tsec{Билет №1}

\begin{leftrules}
Что нелинейно в нелинейной оптике? Принцип суперпозиции для поляризации среды. Материальные уравнения и их связь с уравнениями Максвелла. Механизмы нелинейного взаимодействия излучения со средами: классификация, особенности.
\end{leftrules}

Вектор поляризованности $\vc{P}$ в линейной оптике
\begin{equation*}
    P_i = \sum_{k=1}^{3} \alpha_{ik} E_k, 
    \hspace{5 mm} 
    \text{при } \alpha_{ik} = \alpha \mathbbm{1}
    \hspace{5 mm} 
    \vc{P} = \alpha \vc{E},
\end{equation*}
при том же векторе для эдектрической индукции $\vc{D}$:
\begin{equation*}
    \vc{D} = \vc{E} + 4 \pi \vc{P}.
\end{equation*}
В случае квадратичной нелинейности переходим к зависимости, вида
\begin{equation*}
    P_i = \sum_{k=1}^{3} \alpha_{ik}[E] E_k,
    \hspace{5 mm} 
    \alpha_{ik}[E] = \alpha_{ik} + \sum_{j=1}^{3} \chi_{ikj} E_j + \sum_{j=1}^{3} \sum_{m=1}^{3} \theta_{i k j m} E_j E_m + \ldots,
\end{equation*} 
где $\alpha_{ik}$ -- линейная восприимчивость,  $\chi_{ikj}$ -- квадратичная нелинейная восприимчивость, $\theta_{ikjm}$ -- кубическая нелинейная восприимчивость. 


Итого, получаем материальное уравнение, вида
\begin{equation*}
    P_i = \underbrace{\alpha_{ik} E_k}_{P_i^{\text{лин}}} + \underbrace{\chi_{ikj} E_k E_j}_{P_i^{\text{кв}}}+ \underbrace{\theta_{ikjm} E_k E_j E_m}_{P_i^{\text{куб}}} + \ldots
\end{equation*}

