\Tsec{Билет №5}

\begin{leftrules}
Генерация суммарной и разностной частот. Типы синхронизмов. Вторая гармоника как генерация суммарной частоты. Типы синхронизмов. \red{Периодически поляризованные кристаллы.}
\end{leftrules}





\begin{to_def}
    \textit{Генерация суммарной//разностной частоты} - явление, при котором при просвете среды двумя частотами на выходе появляются частота, равная сумме \textit{//} разности исходных.
\end{to_def}

Появляются частоты как слагаемые, получаемые при возведении в квадрат суммарной амплитуды
\begin{equation*}
    A = \frac{1}{2} (A_{1}e^{-i\omega_{1} t +i k_{1}x} + A^*_{1}e^{i\omega-1 t -i k_1 x}) + \\ \frac{1}{2} (A_{2}e^{-i\omega_{2} t +i k_{2}x} + A^*_{2}e^{2i\omega_{2} t -i k_{2}x}).
\end{equation*}


\textit{Условие фазового синхронизма:} вдали от среды видно $E_{\omega_{1}+\omega_{2}} \propto \delta(\mathbf{k}_{1}+\mathbf{k}_{2}-\mathbf{k}_{3})$, что дает условие ''существенности'' суммарной гармоники: $\mathbf{k}_{1}+\mathbf{k}_{2}=\mathbf{k}_{3}$, что уже допусакает неколлинеарность этих векторов, хоть выгоднее и коллинеарность (большая область пересечения). В последнем случае условие влечет $\frac{n_{1}}{\lambda_{1}}+\frac{n_{2}}{\lambda_{2}}=\frac{n_{3}}{\lambda_{3}}$.


\textit{Среди типов синзронизмов:} \\
    \par \textbf{1)} 1-ый: исходные волны обыкновенные, суммарная - необыкновенная;\\
    \par \textbf{2)} 2-ой: исходные волны обыкновенная и необыкновенная, суммарная - необыкновенная


\textit{Генерация второй гармоники как суммарной:} при неколлинеарной ГСЧ от равных частот получился вторая гармоника, но только лишь в области пересечения импульсов двух неколлинеарных волн, которая геометрически пропорциональна длительности импульса. Так с помощью неколлинеарной ГВГ можно судить о длительности импульсов лазера.