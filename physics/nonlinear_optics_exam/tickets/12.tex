\Tsec{Билет №12}

\begin{leftrules}
Самофокусировка излучения. Самофокусировка простейшего гауссова пучка света. Критическая мощность при самофокусировке излучения. Фокусировка импульсного излучения.
\end{leftrules}


\textit{Нелинейный пок-тель преломл.:} кубическая нелинейность порождает нелинейность показателя преломления: $P=\chi_{3} |E|^{2}E \to n=n_{0}+n_{2}I$, где $n_{2}=\frac{12\pi^{2} \chi_{3}}{cn_{0}^{2}}$ ($I$ - интенсивность света).


\textit{Гауссов пучок:} описывется \\ $ E(x,y,z,t) = e^{-i\omega t + i kz} \underbrace{\sqrt{\frac{P}{\pi r_{0}^{2}(1+(z/z_{r})^{2})}}}_{\text{нормировка}}  \underbrace{exp\bigg(-\frac{x^{2}+y^{2}}{2r_{0}^{2}(1+(z/z_{r})^{2})}\bigg)}_{\text{Гаусс}} \underbrace{exp\bigg(-i\frac{x^{2}+y^{2}}{2r_{0}^{2}(z/z_{r}+z_{r}/z)}\bigg)}_{\text{сферич. волн. фронты}} \underbrace{exp\bigg(-i arctg(z/z_{r})\bigg)}_{\text{фаза Гуи}}$, где $r_{0}$ - радиус перетяжки, $z_{r}$ - релеевская длина перетяжки. Четвертый множитель отвечает за кривизну фазовых фронтов. Фаза Гуи отвечает за периодичность фронтов: ближе к перетяжке они реже.


\textit{Критическая мощность:} при расписывании интенсивности и поля гауссова пучка получится в некоторых приближениях получится множитель $exp(i \frac{r^{2}z}{2r_{0}^{2}} (\frac{1}{z_{r}}-\frac{4\pi n_{2}I_{0}}{\lambda}))$, отвечающий за кривизну, который влечет самофокусировку лишь при $I_{0}=\frac{\lambda}{4\pi z_{r}n_{2}}$ $\implies$ критическая можность для самофокусировки $P_{cr}=\frac{\lambda}{8\pi n_{0}n_{2}}$, в которой нет характеристик гауссова пучка.


\textit{Фокус:} пучок самофокусируется на расстоянии $z_{f}=\frac{2,5 r_{0}^{2}}{\lambda} \frac{\sqrt{P_{cr}}}{\sqrt{P}-0,85\sqrt{P_{cr}}}$.


\textit{Импульсное излучение:} в виду ''гауссовости'' импульсов, от больших интенсивностей фокус ближе, от крыльев дальше, что порождает ''бегущий'' фокус.
