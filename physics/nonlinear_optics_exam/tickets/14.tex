\Tsec{Билет №14.1}


Рассматривая поляризацияю излучения в кубично нелинейной среде, придется рассматривать тензерную взаимосвязь: $P_{i}=\chi^{(3)}_{ijkl}E_{j}E_{k}E_{l}^{*}$.


Ввиду распростанения света вдоль одной координаты, то поля ей нормальны и эта координата может не рассматриваться (только иксы и игрики). Ввиду изотропности среды (таковую рассматриваем) элементы тензора должны быть инвариантны к изменению знаков осей ($\implies$ рассматриваем только элементы с четным числом иксов и игриков). Тогда взаиосвязь упрощается: $
\begin{cases}
    P_{x}=\chi_{xxxx}^{(3)}E_{x}E_{x}E_{x}^{(*)} + \chi_{xxyy}^{(3)}E_{x}E_{y}E_{y}^{(*)} + \chi_{xyxy}^{(3)}E_{y}E_{x}E_{y}^{(*)} + \chi_{xyyx}^{(3)}E_{y}E_{y}E_{x}^{(*)} \\
    P_{y}=\chi_{yyyy}^{(3)}E_{y}E_{y}E_{y}^{(*)} + \chi_{yyxx}^{(3)}E_{y}E_{x}E_{x}^{(*)} + \chi_{yxyx}^{(3)}E_{x}E_{y}E_{x}^{(*)} + \chi_{yxxy}^{(3)}E_{x}E_{x}E_{y}^{(*)}
\end{cases}$, где соответствующие элементы для тензора $\chi^{(3)}$ в выражениях для $P_{x}$ и $P_{y}$ равны ввиду изотропии, а также элементы вторых и третьих слагаемых. Итого можно записать: $
\begin{cases}
P_{x}=\kappa_{1}E_{x}E_{x}E_{x}^{(*)} + 2\kappa_{2}E_{x}E_{y}E_{y}^{(*)} + \kappa_{3}E_{y}E_{y}E_{x}^{(*)} \\
P_{y}=\kappa_{1}E_{y}E_{y}E_{y}^{(*)} + 2\kappa_{2}E_{y}E_{x}E_{x}^{(*)} + \kappa_{3}E_{x}E_{x}E_{y}^{(*)}
\end{cases}$, где ввиду инвариантности поворота $\hookrightarrow$ $\kappa_{1}=2\kappa_{2}+\kappa_{3}$.


Нелинейный пок-тель преломл. $n_{2}=\frac{12\pi^{2}}{cn_{0}^{2}} 
\begin{cases}
    \kappa_{1}=2\kappa_{2}+\kappa_{3}, \quad \text{линейная поляризация} \\
    2\kappa_{2}=, \quad \text{круговая поляризация}
\end{cases}$, где ввиду положительности всех $\kappa$ пок-тель для линейной поляризации больше, чем для круговой.


\textit{Слабая волна в поле сильной:} в процессе взаимодействия слабой и сильной волн близких частот $\omega$ и $\omega_{0}$ в среде появляется волна частоты $2\omega_{0}-\omega$. Слабая волна в поле сильной чувствует преломление не как от $\kappa_{1}$ (лин. поляр.), а как от $2\kappa_{1}$ в случае параллельной поляризации сильной и слабой волн, и $2\kappa_{2}$ в случае противоположной поляризации. Характер среды - наведенное двулучепреломление.

