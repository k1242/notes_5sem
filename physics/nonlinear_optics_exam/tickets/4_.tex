\Tsec{Билет №4.2}

\begin{leftrules}
Генерация второй гармоники в случае слабого преобразования; роль расстройки. Факторы, ограничивающие эффективность преобразования, связь с расстройкой. Роль длины среды. 
\\ \phantom{42} \hfill \textit{лекция 2, слайды 13-23.} 
\end{leftrules}


\begin{to_def}
    \textit{''Заданное поле накачки''} - случай, когда, хоть и расстройка ненулевая (вообще говоря, любая), но амплитуда второй гармоники мала и не особо ослабляет амплитуду накачки (аппроксимируемо постоянной).
\end{to_def}

\textit{При заданном поле накачки} решение ур-я для медленных амплитуд: \\$
    \begin{cases}
        A_{2\omega}(z)/A_{\omega}=iGA_{\omega}z sinc(\Delta k z/2) e^{-i\Delta k z/2} \\
        A_{\omega}(z)=const
    \end{cases}$, где $G=\frac{2\pi \omega}{cn}\chi_{2}$, где наблюдается периодические изменение интенсивности второй гармоники по длине среды (из-за sink) между нулем и тем большим значением, чем меньше расстройка. При фиксированной длине среды интереснее подобрать меньшую расстройку. Половина интенсивности теряется уже при $\Delta k L \simeq 2,8$ ($L$ - длина среды).


\textit{Среди причин ограничения эффективности:}
    \par \textbf{1)} пространственная неоднорожность пучка:
    пучок накачки в лучем случае гауссовый, так что на его крыльях $A_{\omega}(0)$ падает, из-за чего падает и $th$, что влечет на выходе наложении $\simeq 100\%$ эфф-ти от центра пучка и плохой эфф-ти от крыльев, что понижает общую эфф-ть; \\
    \par \textbf{2)} временная неоднородность пучка: \\
    проблема аналогична п. 1; \\
    \par \textbf{3)} нарушение частотного синхронизма: \\
    нулевая расстройка для частоты накачки влечет ненулевую расстройку для спектральных крыльев накачки, что понижает эфф-ть: $\Delta k \simeq [k_{2\omega}(2\omega_{0})+2\frac{dk_{2\omega}}{d\omega}(\omega-\omega_{0})]+[2k_{\omega}(\omega_{0})+2\frac{dk_{\omega}}{d\omega}(\omega-\omega_{0})] = 2(\omega-\omega_{0}) \big( v_{gh}^{-1}(2\omega) - v_{gr}^{-1}(\omega) \big)$, так что решение - ограничение спектра накачки: $\Delta \omega < \frac{2,8}{\tau (2\omega) - \tau (\omega)}$, где $\tau$ - групповое время запаздывания $\implies$ время импульса лазера $\tau > \tau(2\omega)-\tau(\omega)$; \\ 
    \par \textbf{4)} нарушение углового синхронизма: \\
    например, в одноосном кристалле дифракционное уширение пучка влечет уширение пучка второй гармоники, которая является в одноосном кристалле необыкновенной волной, а значит ее пок-тель преломл. зависит от угла, и это все приводит к нарушению фазового синхронизма: $\Delta k = \frac{2\omega n_{2\omega}}{c}\alpha (\theta - \theta_{0})$, где $\alpha = \frac{1}{n}n'_{\theta}$ - угол сноса, в стандартных лазерах на стандартных кристаллах равный единицам градусов (не мало). Короче, условие на ширину пучка: $D \leq 2L \alpha$ ($l$ - длина среды).


\textit{Решение проблем ограничения эфф-ти:} \\
    \par \textbf{1)} 90-тиградусный синхронизм: угол сноса в таком случае нулевой; \\
    \par \textbf{2)} периодически поляризованный кристалл: кристалл состоит чередующихся доменов с двумя разными направлениями нелинейной полярной оси, что компенсирует расстройку фазового синхронизма; \\
    \quad \textbf{2)} резонатор: накачку вносят в резонатор Фабри-Перо, который на порядка 2-3 увеличивает ее, как и вторую гармонику.

