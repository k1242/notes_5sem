\Tsec{Билет №20}

\begin{leftrules}
Вынужденное рассеяние Мандельштама-Бриллюена (ВРМБ). Роль спонтанного рассеяния. Основные характеристики излучения ВРМБ. Особенности энергообмена между волнами при ВРМБ.
\\ \phantom{42} \hfill \textit{лекция 12} 
\end{leftrules}

Звуковые волны в среде создают периодические неоднородности плотности, которые можно считать
объёмными дифракционными решетками.

ВРМБ вызвано дифракцией световых волн на таких неоднородностях среды.

Условия дифракции Брегга:
\begin{enumerate}
    \item фронты звуковых волн служат отражающими поверхностями для световых волн
    \item $\Lambda \sin \frac{\Theta}{2} = \lambda$, где $\Lambda$ -- длина звуковой волны
\end{enumerate}

ВРМБ -- кубичная нелинейность.

Смешение по частоте рассеянных волн зависит от угла и максимально для обратного рассеяния $\Delta \nu = \frac{2 v_{\text{зв}}}{\lambda}$.

Основная и рассеянная волны складываются, итоговая интенсивность осциллирует на частоте звуковой волны, что поддерживает возбуждение новых колебаний среды.

Способ описания -- система с двумя уровнями энергии, для которой решается уравнение Шредингера.

Волна на стоксовой частоте усиливается, на антистоксовой ослабевает.
