\Tsec{Билет №18}

\begin{leftrules}
«Ядерные» нелинейности. Роль стрикционного и ориентационного механизмов нелинейности.
\\ \phantom{42} \hfill \textit{лекция 6, слайды 17-19} 
\end{leftrules}

\begin{itemize}
    \item Стрикционная нелинейность

    Втягивание вещества в область повышенного поля (втягивание диэлектрической жидкости в конденсатор). Наблюдается во всех средах

    \item Ориентационная нелинейность. Поворот анизотропных молекул вдоль электрического поля волны, наблюдается в жидкостях и газах

    \item тепловая нелинейность

    \item плазменная нелинейность

\end{itemize}

Стрикционная и ориентационная нелинейности возникают с запаздыванием, у них есть характерные времена релаксации, значительно превышающие период световой волны.

Ядерные нелинейности вызывают кубичные нелинейные явления $\chi^{(3)}$.