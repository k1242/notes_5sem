\Tsec{Билет №10}

\begin{leftrules}
Решение уравнений генерации третьей оптической гармоники в случае точного синхронизма. Факторы, ограничивающие эффективность преобразования.
\end{leftrules}




\textit{Разделение уравнений для волн:} удобно в уравнении для медленных амплитуд представить поле в видел суммы полей двух гармоник: $A = \frac{1}{2} (A_{\omega}e^{-2i\omega t +i k_{\omega}x} + A^*_{\omega}e^{i\omega t -i k_{\omega}x}) + \\ \frac{1}{2} (A_{3\omega}e^{-i\omega t +i k_{3\omega}x} + A*_{3\omega}e^{3i\omega t -i k_{3\omega}x})$, после чего разделить уравнение на 2 для каждой из гармоник (можно в виду тонкости спектров гармоник, которые друг друга не перекрывают, если импульсы достаточно не коротки (нано-/пикосекндные лазеры)) (слагаемые с групповой скоростью можно тоже аккуратно вычеркнуть, так как для, например, наносекундныъ лазеров их расстройка << длительности импульсов): $
    \begin{cases}
        (A_{\omega})'_{z}=\frac{3\pi i \omega}{2cn_{\omega}} \chi_{3} A_{3\omega}A*_{\omega}^{2} e^{i \Delta k z} \\
        (A_{2\omega})'_{z}=\frac{3\pi i \omega}{2cn_{3\omega}} \chi_{3} A_{\omega}^{3} e^{- i \Delta k z} 
    \end{cases}$.


\textit{При точном синхронизме ($\Delta k = 0$)} решение ур-я для медленных амплитуд: \\$
    \begin{cases}
        A_{3\omega}(z)=iA_{\omega}(0)\frac{GA_{\omega}^{2}(0)z}{\sqrt{1+(GA_{\omega}^{2}(0)z)^{2}}} \\
        A_{\omega}(z)=\frac{A(0)}{\sqrt{1+(GA_{\omega}^{2}(0)z)^{2}}}
    \end{cases}$, где $G=\frac{3\pi \omega}{2cn}\chi_{3}$, где интересно, отношение амплитуд (и, ввиду $n_{\omega}=n_{3\omega}$, интенсивностей) третьей гармоники к накачке имеет скорую единичную ассимптотику (из-за корня).


\textit{Среди причин ограничения эффективности:}
    \par \textbf{1)} пространственная неоднорожность пучка:
    пучок накачки в лучем случае гауссовый, так что на его крыльях $A_{\omega}(0)$ падает, из-за чего падает и $th$, что влечет на выходе наложении $\simeq 100\%$ эфф-ти от центра пучка и плохой эфф-ти от крыльев, что понижает общую эфф-ть; \\
    \par \textbf{2)} временная неоднородность пучка: \\
    проблема аналогична п. 1; \\
    \par \textbf{3)} нарушение частотного синхронизма: \\
    нулевая расстройка для частоты накачки влечет ненулевую расстройку для спектральных крыльев накачки, что понижает эфф-ть: $\Delta k \simeq [k_{3\omega}(3\omega_{0})+3\frac{dk_{3\omega}}{d\omega}(\omega-\omega_{0})]+[3k_{\omega}(\omega_{0})+3\frac{dk_{\omega}}{d\omega}(\omega-\omega_{0})] = 3(\omega-\omega_{0}) \big( v_{gh}^{-1}(3\omega) - v_{gr}^{-1}(\omega) \big)$, так что решение - ограничение спектра накачки: $\Delta \omega < \frac{2,8}{\tau (3\omega) - \tau (\omega)}$, где $\tau$ - групповое время запаздывания $\implies$ время импульса лазера $\tau > \tau(3\omega)-\tau(\omega)$; \\ 
    \par \textbf{4)} нарушение углового синхронизма: \\
    например, в одноосном кристалле дифракционное уширение пучка влечет уширение пучка второй гармоники, которая является в одноосном кристалле необыкновенной волной, а значит ее пок-тель преломл. зависит от угла, и это все приводит к нарушению фазового синхронизма: $\Delta k = \frac{3\omega n_{3\omega}}{c}\alpha (\theta - \theta_{0})$, где $\alpha = \frac{1}{n}n'_{\theta}$ - угол сноса, в стандартных лазерах на стандартных кристаллах равный единицам градусов (не мало). Короче, условие на ширину пучка: $D \leq 3L \alpha$ ($l$ - длина среды).
