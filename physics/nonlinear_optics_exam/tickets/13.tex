\Tsec{Билет №13}

\begin{leftrules}
Фазовая самомодуляция излучения. Основной результат взаимодействия. Практические применения: сокращение длительности световых импульсов, генерация гребенки частот.
\end{leftrules}


\textit{Фазовая самомодуляция:} уравнение для медленных амплитуд $\tilde{A}'_{z}=i\frac{\omega}{c} n_{2} |\tilde{A}|^{2}\tilde{A}$ влечет изменение лишь фазы (из-за $i$ в правой части): $\tilde{A}(z)=\tilde{A}(0)exp \big(i\frac{2\pi z}{\lambda} n_{2}I \big)$.

\begin{to_def}
    \textit{Чирп импульса} - производная фазы импульса по времени (фактически, мгновенная частота).
\end{to_def}

При фазовой самомодуляции наблюдается чирп импульса: со временем импульса частота увеличивается, а также уширяется спектр излучения. При втором воздействии на импульс средой с отрицательной дисперсией после самомодуляции крайние частоты уширенного спектро съезжаются - метод укорочения лазерных импульсов. В качетстве сред с отр. дисперсией используются искусственные, например, пара параллельных дифф. решеток, на которых разные частоты отражаются под разными углами, что в частности задерживает красный по отношению к синему - отр. дисп..


Среди всех шумовых импульсов титан-сапфирового лазера можно выделить сильнейший, подавив остальные, путем самофокусировки и расположения дифф. решетки на расстоянии фокуса. На выходе редкие импульсы - гребенка. В случае, когда у нас есть эталонный лазер (про него все знаем) и исследуемый, можно свести излучения каждого по отельности с титан-сапфировым и по разностным частотам определить и разность частот эталонного и исследуемого.
