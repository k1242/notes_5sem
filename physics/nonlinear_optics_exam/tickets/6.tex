\Tsec{Билет №6}

\begin{leftrules}
Оптическое детектирование. Генерация терагерцового излучения; Терагерцовое излучение как процесс генерации разностных частот.
\end{leftrules}




\textit{Оптическое детектирование:} \\
    \textbf{1)} свет детектируется при прохождении через просветленную (то есть без поглощения) квадратично нелинейную среду, по ''бокам'' которой приложены электроды, которые фиксируют напряжение при постоянной поляризации из-за оптического выпрямления; \\
    \textbf{2)} фиксируется положение зеркала (например, с помощью пружинки за ним), на которое падает свет, оказывающий на него давление; \\
    все это очень вкусно, но сигналы с электродов или деформация пружины столь слабы, что тонут в тепловых шумах электроники, потому не так распространено.


\textit{Терагерцовое излучение:} в качестве источника терагерцового излучения можно использовать квадратично нелинейную среду, на которую светят лазером, генерирующим постоянную поляризацию, пропорциональную интенсивности импульса, которая тем скорее меняется, чем короче импульс. Излучение диполей среды (поляризация) $\propto$ второй производной интенсивности, что и дает терагерцовый диапазон от фемптосекундных импульсов. Это востребовано для исследования молекулярного состава различных веществ, так как их линии переходов часто лежат в терагерцовой области. Также исходный импульс лазера можно рассматривать как спектр, для каждой пары частот из которого будет генерироваться своя разностная частота, что дает широкий спектр.


\textit{Терагерцовое излучение как генерация разностных частот:} также исходный импульс лазера можно рассматривать как спектр, для каждой пары частот из которого будет генерироваться своя разностная частота, что дает широкий спектр.
