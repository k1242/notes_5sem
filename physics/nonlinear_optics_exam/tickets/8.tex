\Tsec{Билет №8}

\begin{leftrules}
Нерезонансные электронные нелинейности: явления третьего порядка. Простейший осциллятор как модель нелинейности: два сопутствующих процесса.
\end{leftrules}



\textit{Среди кубических:}\\
    \par \textbf{1)} генерация третьей гармоники; \\
    \par \textbf{2)} нелинейность показателя преломления: \\
     \textbf{2.1)} самофокусировка; \\
     \textbf{2.2)} фазовая самомодуляция; \\
     \textbf{2.3)} самовращение эллипса поляризации; \\
     \textbf{2.4)} образование солитонов; \\
     \textbf{2.5)} другие; \\
    \par \textbf{3)} четырехволновое смещение: \\
     \textbf{3.1)} cамодифракция излучения; \\
     \textbf{3.2)} обращение волнового фронта; \\
     \textbf{3.3)} другие; \\
    \par \textbf{4)} генерация параметрических волн; \\
    \par \textbf{5)} другие.


\textit{2 сопутствующих процесса:} электрон в атоме рассматривается как осциллятор с пот. эн. $U(x)=\alpha x^{2} + \gamma x^{4}$, что в уравнении $\ddot{x} + \omega_{0}^{2} x + \frac{4\gamma e}{m} x^{3} = Acos(\omega t)$ дает нелинейное смещение $x_{LN} = \frac{3\gamma e}{2m} \Big( \frac{eA}{m(\omega^{2}-\omega_{0}^{2})} \Big)^{2} \Big( \frac{cos(\omega t)}{\omega^{2}-omega_{0}^{2}} + \frac{cos(3\omega t)}{9\omega^{2}-\omega_{0}^{2}} \Big)$, где слагаемое $-\frac{cos(\omega t)}{\omega^{2}-omega_{0}^{2}}$ дает первую гармонику (изменение пок-теля преломл. среды = самовоздействие), а $\frac{cos(2\omega t)}{4\omega^{2}-\omega_{0}^{2}}$ дает излучение третьей гармоники: $P \propto (acos(\omega t)+bcos(3\omega t))$.
