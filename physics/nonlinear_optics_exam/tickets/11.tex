\Tsec{Билет №11}

\begin{leftrules}
Нелинейный показатель преломления среды; связь с кубичной нелинейностью среды. Роль стрикционного и ориентационного механизмов нелинейности.
\end{leftrules}




\textit{Среди явлений изменения пок-теля преломл.:} \\
    \par \textbf{1)} стрикционная нелинейность: втягивание вещества в область повышенной интенсивности излучения (что влечет изменение пок-теля преломл.); свойственно любому веществу; \\
    \par \textbf{2)} ориентационная нелинейность: свойственна анизотропным средам (в основном, жидкостям); следстие вытянусти молекул: один из их электронов становится свободным, так что световое поле легко возбудит его движение вдоль остова молекул, но с трудом поперек, что приведет к моменту силы, и остовы повернутся по полу излучения - среда станет двулучепреломляющей; \\
    \par \textbf{3)} тепловая нелинейность: не комметируется; \\
    \par \textbf{4)} плазменная нелинейность: не комментируется.



