\documentclass[a4paper,12pt]{article}

%%% Работа с русским языком
\usepackage{cmap}          % поиск в PDF
\usepackage{mathtext}         % русские буквы в формулах
\usepackage[T2A]{fontenc}      % кодировка
\usepackage[utf8]{inputenc}      % кодировка исходного текста
\usepackage[english,russian]{babel}  % локализация и переносы
\usepackage{indentfirst}
\frenchspacing


%%% Дополнительная работа с математикой
\usepackage{amsmath,amsfonts,amssymb,amsthm,mathtools} % AMS
\usepackage{icomma} % "Умная" запятая: $0,2$ --- число, $0, 2$ --- перечисление

%% Номера формул
%\mathtoolsset{showonlyrefs=true} % Показывать номера только у тех формул, на которые есть \eqref{} в тексте.
%\usepackage{leqno} % Нумерация формул слева

%% Свои команды
\DeclareMathOperator{\sgn}{\mathop{sgn}}

%% Перенос знаков в формулах (по Львовскому)
\newcommand*{\hm}[1]{#1\nobreak\discretionary{}
	{\hbox{$\mathsurround=0pt #1$}}{}}

%%% Работа с картинками
\usepackage{graphicx}  % Для вставки рисунков
\graphicspath{{images/}}  % папки с картинками
\setlength\fboxsep{3pt} % Отступ рамки \fbox{} от рисунка
\setlength\fboxrule{1pt} % Толщина линий рамки \fbox{}
\usepackage{wrapfig} % Обтекание рисунков текстом

%%% Работа с таблицами
\usepackage{array,tabularx,tabulary,booktabs} % Дополнительная работа с таблицами
\usepackage{longtable}  % Длинные таблицы
\usepackage{multirow} % Слияние строк в таблице

%%% Теоремы
\theoremstyle{definition} % Это стиль по умолчанию, его можно не переопределять.
\newtheorem{Th}{Th}[section]
\newtheorem{Lem}{Lem}[section]
\newtheorem{Post}{Post}[section]
\newtheorem{Comm}{Comm}[section]
\newtheorem{Stat}{Stat}[section]
\newtheorem{Proof}{Proof}[section]
\newtheorem{proposition}[Th]{Утверждение}

\theoremstyle{definition} % "Определение"
\newtheorem{Def}{Def}[section]
\newtheorem{Cor}{Cor}[section]
\newtheorem{problem}{Задача}[section]

\theoremstyle{remark} % "Примечание"
\newtheorem*{nonum}{Решение}

%%% Программирование
\usepackage{etoolbox} % логические операторы

%%% Страница
\usepackage{extsizes} % Возможность сделать 14-й шрифт
\usepackage{geometry} % Простой способ задавать поля
\geometry{top=25mm}
\geometry{bottom=35mm}
\geometry{left=10mm}
\geometry{right=10mm}
%
%\usepackage{fancyhdr} % Колонтитулы
%   \pagestyle{fancy}
%\renewcommand{\headrulewidth}{0pt}  % Толщина линейки, отчеркивающей верхний колонтитул
%   \lfoot{Нижний левый}
%   \rfoot{Нижний правый}
%   \rhead{Верхний правый}
%   \chead{Верхний в центре}
%   \lhead{Верхний левый}
%  \cfoot{Нижний в центре} % По умолчанию здесь номер страницы

\usepackage{setspace} % Интерлиньяж
%\onehalfspacing % Интерлиньяж 1.5
%\doublespacing % Интерлиньяж 2
%\singlespacing % Интерлиньяж 1

\usepackage{lastpage} % Узнать, сколько всего страниц в документе.

\usepackage{soul} % Модификаторы начертания

\usepackage{hyperref}
\usepackage[usenames,dvipsnames,svgnames,table,rgb]{xcolor}
\hypersetup{        % Гиперссылки
	unicode=true,           % русские буквы в раздела PDF
	pdftitle={Заголовок},   % Заголовок
	pdfauthor={Автор},      % Автор
	pdfsubject={Тема},      % Тема
	pdfcreator={Создатель}, % Создатель
	pdfproducer={Производитель}, % Производитель
	pdfkeywords={keyword1} {key2} {key3}, % Ключевые слова
	colorlinks=true,         % false: ссылки в рамках; true: цветные ссылки
	linkcolor=red,          % внутренние ссылки
	citecolor=black,        % на библиографию
	filecolor=magenta,      % на файлы
	urlcolor=cyan           % на URL
}

\usepackage{xcolor}
\usepackage{hyperref} 

\definecolor{urlcolor}{HTML}{0000FF} % цвет гиперссылок

\usepackage{csquotes} % Еще инструменты для ссылок

%\usepackage[style=authoryear,maxcitenames=2,backend=biber,sorting=nty]{biblatex}

\usepackage{multicol} % Несколько колонок

\usepackage{tikz} % Работа с графикой
\usepackage{pgfplots}
\usepackage{pgfplotstable}

\begin{document}
\begin{center}
	\LARGE \textsc{Нелинопт (билеты).}
\end{center}

\hrule

\phantom{42}

\begin{flushright}
	\begin{tabular}{rr}
		% written by:
		\textbf{Студент}: 
		& Егоров Д.М. \\
		&\\
		% date:
		\textbf{Дата}: &
		\textit{\today}\\
	\end{tabular}
\end{flushright}

\thispagestyle{empty}
\tableofcontents
\newpage


\section{Оптические нелинейности.}
\subsection{Билет 1.}

\begin{Def}
	\textit{Уравнения Максвелла} - система уравнений, связывающих векторы $\mathbf{E}, \mathbf{D}, \mathbf{B}$ и $\mathbf{H}$  во всех средах: $\scriptsize
	\begin{cases}
		div \mathbf{D} = 4 \pi \rho \\
		div \mathbf{B} = 0 \\
		rot \mathbf{E} = -\dot{\mathbf{B}}/c \\
		rot \mathbf{H} = 4\pi \mathbf{j}/c + \dot{\mathbf{D}}/c
	\end{cases}$, где $\mathbf{j}$ и $\rho$ - плостность тока и заряда (соотв.).
\end{Def}
\begin{Def}
	\textit{Материальные дополнения} - уравнения, связывающие векторы $\mathbf{E}$ и $\mathbf{D}$, $\mathbf{B}$ и $\mathbf{H}$ в конкретной среде.
\end{Def}

	В оптике в качестве материального уравнения часто берется $\mathbf{H}=\mathbf{B}$, после чего можно исключить эти векторы из рассмотрения.

\begin{Def}
	\textit{Вектор поляризация среды = поляризованность среды $\mathbf{P}$} - дипольный момент единицы объема среды.
\end{Def}

	Ур-я Максвелла влекут волновое уравнение: $\Delta \mathbf{E}-\ddot{\mathbf{E}}/c=4\pi \mathbf{P}/c^{2}$, которое требует материального уравнения $\mathbf{P} = \mathbf{P}(\mathbf{E})$.

\begin{Def}
	\textit{Линейная//нелинейная оптика} - оптика сред, в которых зависимость $\mathbf{P}(\mathbf{E})$ линейна \textit{//} нелинейна.
\end{Def}

	\textit{Среди причин оптических нелинейностей:} \\
	\par \textbf{1)} негармонический отклик электронов среды (электронные нелинейности: резонансные и нерезонансные); \\
	\par \textbf{2)} ''ядерные'' нелинейности (движение атомных и молекулярных остовов).


	Виды зависимостей $\mathbf{P}(\mathbf{E})$: \\
     \par \textbf{1)} $P=\chi (\omega) E$ (монохроматические волны); \\
     \par \textbf{2)} $P=|\chi|E_{0}cos(\omega t + \phi)$ (комплексные числа); \\
     \par \textbf{3)} $P_{i}=\chi_{ij}E_{j}$ (тензорная связь - анизотропные среды); \\
	 \par \textbf{4)} $P(\mathbf{r}) = \int H(\mathbf{r}-\mathbf{r}') E(\mathbf{r}') d^{3}r'$ (локальная связь); \\
	 \par \textbf{5)} $P(\omega) = \int \chi (\omega, \omega') E(\omega') d\omega'$ (две частоты); \\
	 \par \textbf{6)} другие.


\subsection{Билет 2.}


	При электронной нерезонансной нелинейности (прозрачные среды) метериальное уравнение $\mathbf{P}=\mathbf{P}(\mathbf{E})$ представимо в виде: $P=\sum\limits_{i=1}^{i \to \infty} \chi_{i} E^{i}$.

\begin{Def}
	\textit{Квадратичная//кубическая} электронная нерезонансная нелинейность - нелинейность, обусловленная слагаемым $\chi_{2} E^{2}$ \textit{//} $\chi_{3} E^{3}$ в разложении $P=\sum\limits_{i=1}^{i \to \infty} \chi_{i} E^{i}$.
\end{Def}

	Среди явлений квадратичной нелинейности: \\
	\par \textbf{1)} генерация второй гармоники (ГВГ); \\
	\par \textbf{2)} генерация суммарной частоты (ГСЧ); \\
	\par \textbf{3)} генерация разностной частоты (ГРЧ); \\
	\par \textbf{4)} оптическое выпрямление; \\
	\par \textbf{5)} генерация параметрических волн (ГПВ).


	\textit{2 сопутствующих процесса:} электрон в атоме рассматривается как осциллятор с пот. эн. $U(x)=\alpha x^{2} + \beta x^{3}$, что в уравнении $\ddot{x} + 2\gamma \dot{x} + \omega_{0}^{2} x + \frac{3\beta e}{m} x^{2} = Acos(\omega t)$ дает нелинейное смещение $x_{LN} = \frac{3\beta e}{2m} \Big( \frac{eA}{m(\omega^{2}-\omega_{0}^{2})} \Big)^{2} \Big( \frac{-1}{\omega_{0}^{2}} + \frac{cos(2\omega t)}{4\omega^{2}-\omega_{0}^{2}} \Big)$, где слагаемое $-1/\omega^{2}$ дает постоянную поляризацию, пропорциональную интенсивности света (оптическое выпрямление), а $\frac{cos(2\omega t)}{4\omega^{2}-\omega_{0}^{2}}$ дает излучение второй гармоники: \\ $P \propto (a+bcos(2\omega t))$.


\section{Квадратичные нелинейности.}
\subsection{Билет 3.}


	\textit{Условие фазового синхронизма:} вдали от среды видно $E_{2\omega} \propto \delta(\mathbf{k}_{2\omega}-2\mathbf{k}_{\omega})$, что дает условие ''существенности'' второй гармоники: $\mathbf{k}_{2\omega}=2\mathbf{k}_{\omega}$, что влечет $n_{2\omega}=n_{\omega}$, что едва ли выполняется в квадратично нелинейных средах, однако этого можно достигнуть, например, в отрицательном одноосном криталле, в котором пок-тель преломл. для накачки можно настроить равным пок-телю преломл. для второй гармоники, если последний лежит меж обыкновенным и необыкновенными пок-телями для накачки.


	\textit{Выбор угла для выполнения условия синхронизма:} условие на угол в отрицательном одноосном кристалле: $\frac{cos(\theta)^{2}}{n_{0}^{2}(2\omega)} + \frac{sin^{2}(\theta)}{n_{e}^{2}(2\omega)} = \frac{1}{n_{0}^{2}(\omega)}$.

	
\subsection{Билет 4.1.}

\begin{Def}
	\textit{Бездифракционное приближение} - приближение, в котором принебрегается поперечным (направлению распространения) расплыванием светового пучка, что влечет обнуление вторых производных по двум координатам (нормальным к координате распространения) в лапласиане волнового уравнения (обусловленно мили-/сантиметровочными длинами рабочих нелинейных сред, а на таких длинах дифр. эффекты мизерны).
\end{Def}
\begin{Def}
	\textit{''Медленная амплитуда''} - случай, когда волна представима произведением множителей, один из которых отвечает за быстро осциллирующую фазу, а второй за медленно изменяющуюся амплитуду.
\end{Def}
\begin{Def}
	\textit{Уравнение для медленных амплитуд} - уравнение $2ik \big( \mathbf{A}'_{z} + \frac{1}{v_{gr}} \dot{\mathbf{A}} \big) e^{-i\omega z + ikt} = 4\pi \ddot{\mathbf{P}}_{NL}/c^{2}$.
\end{Def}

	Использование бездифракционного приближения для медленных амплитуд позволяется убрать в волновом уравнении вторые производных по двум координатам, а оставшиеся вторых производные по координате распространения и времени свести к первым, что переводит волновое уравнение в уравнение для медленных амплитуд.


	\textit{Разделение уравнений для волн:} удобно в уравнении для медленных амплитуд представить поле в видел суммы полей двух гармоник: $A = \frac{1}{2} (A_{\omega}e^{-2i\omega t +i k_{\omega}x} + A*_{\omega}e^{i\omega t -i k_{\omega}x}) + \\ \frac{1}{2} (A_{2\omega}e^{-i\omega t +i k_{2\omega}x} + A*_{2\omega}e^{2i\omega t -i k_{2\omega}x})$, после чего разделить уравнение на 2 для каждой из гармоник (можно в виду тонкости спектров гармоник, которые друг друга не перекрывают, если импульсы достаточно не коротки (нано-/пикосекндные лазеры)) (слагаемые с групповой скоростью можно тоже аккуратно вычеркнуть, так как для, например, наносекундныъ лазеров их расстройка << длительности импульсов): $\scriptsize
	\begin{cases}
		(A_{\omega})'_{z}=\frac{2\pi i \omega}{cn_{\omega}} \chi_{2} A_{2\omega}A*_{\omega} e^{i \Delta k z} \\
		(A_{2\omega})'_{z}=\frac{2\pi i \omega}{cn_{2\omega}} \chi_{2} A_{\omega}^{2} e^{- i \Delta k z} 
	\end{cases}$.

\begin{Def}
	\textit{Расстройка} - $\Delta k = k_{2\omega}-2k_{\omega}$.
\end{Def}

	\textit{При точном синхронизме ($\Delta k = 0$)} решение ур-я для медленных амплитуд: \\$\scriptsize
	\begin{cases}
		A_{2\omega}(z)=iA_{\omega}(0)th(GA_{\omega}(0)z) \\
		A_{\omega}(z)=\frac{A(0)}{ch(GA_{\omega}(0)z)}
	\end{cases}$, где $G=\frac{2\pi \omega}{cn}\chi_{2}$, где интересно, отношение амплитуд (и, ввиду $n_{\omega}=n_{2\omega}$, интенсивностей) второй гармоники к накачке имеет скорую единичную ассимптотику (из-за th).

	
\subsection{Билет 4.2.}

\begin{Def}
	\textit{''Заданное поле накачки''} - случай, когда, хоть и расстройка ненулевая (вообще говоря, любая), но амплитуда второй гармоники мала и не особо ослабляет амплитуду накачки (аппроксимируемо постоянной).
\end{Def}

	\textit{При заданном поле накачки} решение ур-я для медленных амплитуд: \\$\scriptsize
	\begin{cases}
		A_{2\omega}(z)/A_{\omega}=iGA_{\omega}z sinc(\Delta k z/2) e^{-i\Delta k z/2} \\
		A_{\omega}(z)=const
	\end{cases}$, где $G=\frac{2\pi \omega}{cn}\chi_{2}$, где наблюдается периодические изменение интенсивности второй гармоники по длине среды (из-за sink) между нулем и тем большим значением, чем меньше расстройка. При фиксированной длине среды интереснее подобрать меньшую расстройку. Половина интенсивности теряется уже при $\Delta k L \simeq 2,8$ ($L$ - длина среды).


	\textit{Среди причин ограничения эффективности:}
	\par \textbf{1)} пространственная неоднорожность пучка:
	пучок накачки в лучем случае гауссовый, так что на его крыльях $A_{\omega}(0)$ падает, из-за чего падает и $th$, что влечет на выходе наложении $\simeq 100\%$ эфф-ти от центра пучка и плохой эфф-ти от крыльев, что понижает общую эфф-ть; \\
	\par \textbf{2)} временная неоднородность пучка: \\
	проблема аналогична п. 1; \\
	\par \textbf{3)} нарушение частотного синхронизма: \\
	нулевая расстройка для частоты накачки влечет ненулевую расстройку для спектральных крыльев накачки, что понижает эфф-ть: $\Delta k \simeq [k_{2\omega}(2\omega_{0})+2\frac{dk_{2\omega}}{d\omega}(\omega-\omega_{0})]+[2k_{\omega}(\omega_{0})+2\frac{dk_{\omega}}{d\omega}(\omega-\omega_{0})] = 2(\omega-\omega_{0}) \big( v_{gh}^{-1}(2\omega) - v_{gr}^{-1}(\omega) \big)$, так что решение - ограничение спектра накачки: $\Delta \omega < \frac{2,8}{\tau (2\omega) - \tau (\omega)}$, где $\tau$ - групповое время запаздывания $\implies$ время импульса лазера $\tau > \tau(2\omega)-\tau(\omega)$; \\ 
	\par \textbf{4)} нарушение углового синхронизма: \\
	например, в одноосном кристалле дифракционное уширение пучка влечет уширение пучка второй гармоники, которая является в одноосном кристалле необыкновенной волной, а значит ее пок-тель преломл. зависит от угла, и это все приводит к нарушению фазового синхронизма: $\Delta k = \frac{2\omega n_{2\omega}}{c}\alpha (\theta - \theta_{0})$, где $\alpha = \frac{1}{n}n'_{\theta}$ - угол сноса, в стандартных лазерах на стандартных кристаллах равный единицам градусов (не мало). Короче, условие на ширину пучка: $D \leq 2L \alpha$ ($l$ - длина среды).


	\textit{Решение проблем ограничения эфф-ти:} \\
	\par \textbf{1)} 90-тиградусный синхронизм: угол сноса в таком случае нулевой; \\
	\par \textbf{2)} периодически поляризованный кристалл: кристалл состоит чередующихся доменов с двумя разными направлениями нелинейной полярной оси, что компенсирует расстройку фазового синхронизма; \\
	\quad \textbf{2)} резонатор: накачку вносят в резонатор Фабри-Перо, который на порядка 2-3 увеличивает ее, как и вторую гармонику.


\subsection{Билет 5.}

\begin{Def}
	\textit{Генерация суммарной//разностной частоты} - явление, при котором при просвете среды двумя частотами на выходе появляются частота, равная сумме \textit{//} разности исходных.
\end{Def}

	Появляются частоты как слагаемые, получаемые при возведении в квадрат суммарной амплитуды $A = \frac{1}{2} (A_{1}e^{-i\omega_{1} t +i k_{1}x} + A*_{1}e^{i\omega t -i k_{\omega}x}) + \\ \frac{1}{2} (A_{2}e^{-i\omega_{2} t +i k_{2}x} + A*_{2}e^{2i\omega_{2} t -i k_{2}x})$.


	\textit{Условие фазового синхронизма:} вдали от среды видно $E_{\omega_{1}+\omega_{2}} \propto \delta(\mathbf{k}_{1}+\mathbf{k}_{2}-\mathbf{k}_{3})$, что дает условие ''существенности'' суммарной гармоники: $\mathbf{k}_{1}+\mathbf{k}_{2}=\mathbf{k}_{3}$, что уже допусакает неколлинеарность этих векторов, хоть выгоднее и коллинеарность (большая область пересечения). В последнем случае условие влечет $\frac{n_{1}}{\lambda_{1}}+\frac{n_{2}}{\lambda_{2}}=\frac{n_{3}}{\lambda_{3}}$.


	\textit{Среди типов синзронизмов:} \\
	\par \textbf{1)} 1-ый: исходные волны обыкновенные, суммарная - необыкновенная;\\
	\par \textbf{2)} 2-ой: исходные волны обыкновенная и необыкновенная, суммарная - необыкновенная


	\textit{Генерация второй гармоники как суммарной:} при неколлинеарной ГСЧ от равных частот получился вторая гармоника, но только лишь в области пересечения импульсов двух неколлинеарных волн, которая геометрически пропорциональна длительности импульса. Так с помощью неколлинеарной ГВГ можно судить о длительности импульсов лазера.


\subsection{Билет 6.}


	\textit{Оптическое детектирование:} \\
	\textbf{1)} свет детектируется при прохождении через просветленную (то есть без поглощения) квадратично нелинейную среду, по ''бокам'' которой приложены электроды, которые фиксируют напряжение при постоянной поляризации из-за оптического выпрямления; \\
	\textbf{2)} фиксируется положение зеркала (например, с помощью пружинки за ним), на которое падает свет, оказывающий на него давление; \\
	все это очень вкусно, но сигналы с электродов или деформация пружины столь слабы, что тонут в тепловых шумах электроники, потому не так распространено.


	\textit{Терагерцовое излучение:} в качестве источника терагерцового излучения можно использовать квадратично нелинейную среду, на которую светят лазером, генерирующим постоянную поляризацию, пропорциональную интенсивности импульса, которая тем скорее меняется, чем короче импульс. Излучение диполей среды (поляризация) $\propto$ второй производной интенсивности, что и дает терагерцовый диапазон от фемптосекундных импульсов. Это востребовано для исследования молекулярного состава различных веществ, так как их линии переходов часто лежат в терагерцовой области. Также исходный импульс лазера можно рассматривать как спектр, для каждой пары частот из которого будет генерироваться своя разностная частота, что дает широкий спектр.


	\textit{Терагерцовое излучение как генерация разностных частот:} также исходный импульс лазера можно рассматривать как спектр, для каждой пары частот из которого будет генерироваться своя разностная частота, что дает широкий спектр.


\subsection{Билет 7.}

Отсутствует запись, но есть презентация (30.09.2021).

\section{Кубичные нелинейности.}
\subsection{Билет 8.}


	\textit{Среди кубических:}\\
	\par \textbf{1)} генерация третьей гармоники; \\
	\par \textbf{2)} нелинейность показателя преломления: \\
	 \textbf{2.1)} самофокусировка; \\
	 \textbf{2.2)} фазовая амомодуляция; \\
 	 \textbf{2.3)} самовращение эллипса поляризации; \\
	 \textbf{2.4)} образование солитонов; \\
	 \textbf{2.5)} другие; \\
	\par \textbf{3)} четырехволновое смещение: \\
	 \textbf{3.1)} cсамодифракция излучения; \\
	 \textbf{3.2)} обращение волнового фронта; \\
	 \textbf{3.3)} другие; \\
	\par \textbf{4)} генерация параметрических волн; \\
	\par \textbf{5)} другие.


	\textit{2 сопутствующих процесса:} электрон в атоме рассматривается как осциллятор с пот. эн. $U(x)=\alpha x^{2} + \gamma x^{4}$, что в уравнении $\ddot{x} + \omega_{0}^{2} x + \frac{4\gamma e}{m} x^{3} = Acos(\omega t)$ дает нелинейное смещение $x_{LN} = \frac{3\gamma e}{2m} \Big( \frac{eA}{m(\omega^{2}-\omega_{0}^{2})} \Big)^{2} \Big( \frac{cos(\omega t)}{\omega^{2}-omega_{0}^{2}} + \frac{cos(3\omega t)}{9\omega^{2}-\omega_{0}^{2}} \Big)$, где слагаемое $-\frac{cos(\omega t)}{\omega^{2}-omega_{0}^{2}}$ дает первую гармонику (изменение пок-теля преломл. среды = самовоздействие), а $\frac{cos(2\omega t)}{4\omega^{2}-\omega_{0}^{2}}$ дает излучение третьей гармоники: $P \propto (acos(\omega t)+bcos(3\omega t))$.


\subsection{Билет 9.}


	\textit{Условие фазового синхронизма:} вдали от среды видно $E_{2\omega} \propto \delta(\mathbf{k}_{3\omega}-3\mathbf{k}_{\omega})$, что дает условие ''существенности'' третьей гармоники: $\mathbf{k}_{3\omega}=3\mathbf{k}_{\omega}$, что влечет $n_{3\omega}=n_{\omega}$, что едва ли выполняется в кубически нелинейных средах, однако этого можно достигнуть, например, в отрицательном одноосном криталле, в котором пок-тель преломл. для накачки можно настроить равным пок-телю преломл. для третьей гармоники, если последний лежит меж обыкновенным и необыкновенными пок-телями для накачки.


	\textit{Выбор угла для выполнения условия синхронизма:} условие на угол в отрицательном одноосном кристалле: $\frac{cos(\theta)^{2}}{n_{0}^{2}(3\omega)} + \frac{sin^{2}(\theta)}{n_{e}^{2}(3\omega)} = \frac{1}{n_{0}^{2}(\omega)}$.


\subsection{Билет 10.}


	\textit{Разделение уравнений для волн:} удобно в уравнении для медленных амплитуд представить поле в видел суммы полей двух гармоник: $A = \frac{1}{2} (A_{\omega}e^{-2i\omega t +i k_{\omega}x} + A*_{\omega}e^{i\omega t -i k_{\omega}x}) + \\ \frac{1}{2} (A_{3\omega}e^{-i\omega t +i k_{3\omega}x} + A*_{3\omega}e^{3i\omega t -i k_{3\omega}x})$, после чего разделить уравнение на 2 для каждой из гармоник (можно в виду тонкости спектров гармоник, которые друг друга не перекрывают, если импульсы достаточно не коротки (нано-/пикосекндные лазеры)) (слагаемые с групповой скоростью можно тоже аккуратно вычеркнуть, так как для, например, наносекундныъ лазеров их расстройка << длительности импульсов): $\scriptsize
	\begin{cases}
		(A_{\omega})'_{z}=\frac{3\pi i \omega}{2cn_{\omega}} \chi_{3} A_{3\omega}A*_{\omega}^{2} e^{i \Delta k z} \\
		(A_{2\omega})'_{z}=\frac{3\pi i \omega}{2cn_{3\omega}} \chi_{3} A_{\omega}^{3} e^{- i \Delta k z} 
	\end{cases}$.


	\textit{При точном синхронизме ($\Delta k = 0$)} решение ур-я для медленных амплитуд: \\$\scriptsize
	\begin{cases}
		A_{3\omega}(z)=iA_{\omega}(0)\frac{GA_{\omega}^{2}(0)z}{\sqrt{1+(GA_{\omega}^{2}(0)z)^{2}}} \\
		A_{\omega}(z)=\frac{A(0)}{\sqrt{1+(GA_{\omega}^{2}(0)z)^{2}}}
	\end{cases}$, где $G=\frac{3\pi \omega}{2cn}\chi_{3}$, где интересно, отношение амплитуд (и, ввиду $n_{\omega}=n_{3\omega}$, интенсивностей) третьей гармоники к накачке имеет скорую единичную ассимптотику (из-за корня).


	\textit{Среди причин ограничения эффективности:}
	\par \textbf{1)} пространственная неоднорожность пучка:
	пучок накачки в лучем случае гауссовый, так что на его крыльях $A_{\omega}(0)$ падает, из-за чего падает и $th$, что влечет на выходе наложении $\simeq 100\%$ эфф-ти от центра пучка и плохой эфф-ти от крыльев, что понижает общую эфф-ть; \\
	\par \textbf{2)} временная неоднородность пучка: \\
	проблема аналогична п. 1; \\
	\par \textbf{3)} нарушение частотного синхронизма: \\
	нулевая расстройка для частоты накачки влечет ненулевую расстройку для спектральных крыльев накачки, что понижает эфф-ть: $\Delta k \simeq [k_{3\omega}(3\omega_{0})+3\frac{dk_{3\omega}}{d\omega}(\omega-\omega_{0})]+[3k_{\omega}(\omega_{0})+3\frac{dk_{\omega}}{d\omega}(\omega-\omega_{0})] = 3(\omega-\omega_{0}) \big( v_{gh}^{-1}(3\omega) - v_{gr}^{-1}(\omega) \big)$, так что решение - ограничение спектра накачки: $\Delta \omega < \frac{2,8}{\tau (3\omega) - \tau (\omega)}$, где $\tau$ - групповое время запаздывания $\implies$ время импульса лазера $\tau > \tau(3\omega)-\tau(\omega)$; \\ 
	\par \textbf{4)} нарушение углового синхронизма: \\
	например, в одноосном кристалле дифракционное уширение пучка влечет уширение пучка второй гармоники, которая является в одноосном кристалле необыкновенной волной, а значит ее пок-тель преломл. зависит от угла, и это все приводит к нарушению фазового синхронизма: $\Delta k = \frac{3\omega n_{3\omega}}{c}\alpha (\theta - \theta_{0})$, где $\alpha = \frac{1}{n}n'_{\theta}$ - угол сноса, в стандартных лазерах на стандартных кристаллах равный единицам градусов (не мало). Короче, условие на ширину пучка: $D \leq 3L \alpha$ ($l$ - длина среды).


\subsection{Билет 11.}


	\textit{Среди явлений изменения пок-теля преломл.:} \\
	\par \textbf{1)} стрикционная нелинейность: втягивание вещества в область повышенной интенсивности излучения (что влечет изменение пок-теля преломл.); свойственно любому веществу; \\
	\par \textbf{2)} ориентационная нелинейность: свойственна анизотропным средам (в основном, жидкостям); следстие вытянусти молекул: один из их электронов становится свободным, так что световое поле легко возбудит его движение вдоль остова молекул, но с трудом поперек, что приведет к моменту силы, и остовы повернутся по полу излучения - среда станет двулучепреломляющей; \\
	\par \textbf{3)} тепловая нелинейность: не комметируется; \\
	\par \textbf{4)} плазменная нелинейность: не комментируется.


\subsection{Билет 12.}


	\textit{Нелинейный пок-тель преломл.:} кубическая нелинейность порождает нелинейность показателя преломления: $P=\chi_{3} |E|^{2}E \to n=n_{0}+n_{2}I$, где $n_{2}=\frac{12\pi^{2} \chi_{3}}{cn_{0}^{2}}$ ($I$ - интенсивность света).


	\textit{Гауссов пучок:} описывется \\ $\scriptsize E(x,y,z,t) = e^{-i\omega t + i kz} \underbrace{\sqrt{\frac{P}{\pi r_{0}^{2}(1+(z/z_{r})^{2})}}}_{\text{нормировка}}  \underbrace{exp\bigg(-\frac{x^{2}+y^{2}}{2r_{0}^{2}(1+(z/z_{r})^{2})}\bigg)}_{\text{Гаусс}} \underbrace{exp\bigg(-i\frac{x^{2}+y^{2}}{2r_{0}^{2}(z/z_{r}+z_{r}/z)}\bigg)}_{\text{сферич. волн. фронты}} \underbrace{exp\bigg(-i arctg(z/z_{r})\bigg)}_{\text{фаза Гуи}}$, где $r_{0}$ - радиус перетяжки, $z_{r}$ - релеевская длина перетяжки. Четвертый множитель отвечает за кривизну фазовых фронтов. Фаза Гуи отвечает за периодичность фронтов: ближе к перетяжке они реже.


	\textit{Критическая мощность:} при расписывании интенсивности и поля гауссова пучка получится в некоторых приближениях получится множитель $exp(i \frac{r^{2}z}{2r_{0}^{2}} (\frac{1}{z_{r}}-\frac{4\pi n_{2}I_{0}}{\lambda}))$, отвечающий за кривизну, который влечет самофокусировку лишь при $I_{0}=\frac{\lambda}{4\pi z_{r}n_{2}}$ $\implies$ критическая можность для самофокусировки $P_{cr}=\frac{\lambda}{8\pi n_{0}n_{2}}$, в которой нет характеристик гауссова пучка.


	\textit{Фокус:} пучок самофокусируется на расстоянии $z_{f}=\frac{2,5 r_{0}^{2}}{\lambda} \frac{\sqrt{P_{cr}}}{\sqrt{P}-0,85\sqrt{P_{cr}}}$.


	\textit{Импульсное излучение:} в виду ''гауссости'' импульсов, от больших интенсивностей фокус ближе, от крыльев дальше, что порождает ''бегущий'' фокус.


\subsection{Билет 13.}


	\textit{Фазовая самомодуляция:} уравнение для медленных амплитуд $\tilde{A}'_{z}=i\frac{\omega}{c} n_{2} |\tilde{A}|^{2}\tilde{A}$ влечет изменение лишь фазы (из-за $i$ в правой части): $\tilde{A}(z)=\tilde{A}(0)exp \big(i\frac{2\pi z}{\lambda} n_{2}I \big)$.

\begin{Def}
	\textit{Чирп импульса} - производная фазы импульса по времени (фактически, мгновенная частота).
\end{Def}

	При фазовой самомодуляции наблюдается чирп импульса: со временем импульса частота увеличивается, а также уширяется спектр излучения. При втором воздействии на импульс средой с отрицательной дисперсией после самомодуляции крайние частоты уширенного спектро съезжаются - метод укорочения лазерных импульсов. В качетстве сред с отр. дисперсией используются искусственные, например, пара параллельных дифф. решеток, на которых разные частоты отражаются под разными углами, что в частности задерживает красный по отношению к синему - отр. дисп..


	Среди всех шумовых импульсов титан-сапфирового лазера можно выделить сильнейший, подавив остальные, путем самофокусировки и расположения дифф. решетки на расстоянии фокуса. На выходе редкие импульсы - гребенка. В случае, когда у нас есть эталонный лазер (про него все знаем) и исследуемый, можно свести излучения каждого по отельности с титан-сапфировым и по разностным частотам определить и разность частот эталонного и исследуемого.


\subsection{Билет 14.1.}


	Рассматривая поляризацияю излучения в кубично нелинейной среде, придется рассматривать тензерную взаимосвязь: $P_{i}=\chi^{(3)}_{ijkl}E_{j}E_{k}E_{l}^{*}$.


	Ввиду распростанения света вдоль одной координаты, то поля ей нормальны и эта координата может не рассматриваться (только иксы и игрики). Ввиду изотропности среды (таковую рассматриваем) элементы тензора должны быть инвариантны к изменению знаков осей ($\implies$ рассматриваем только элементы с четным числом иксов и игриков). Тогда взаиосвязь упрощается: $\scriptsize
	\begin{cases}
		P_{x}=\chi_{xxxx}^{(3)}E_{x}E_{x}E_{x}^{(*)} + \chi_{xxyy}^{(3)}E_{x}E_{y}E_{y}^{(*)} + \chi_{xyxy}^{(3)}E_{y}E_{x}E_{y}^{(*)} + \chi_{xyyx}^{(3)}E_{y}E_{y}E_{x}^{(*)} \\
		P_{y}=\chi_{yyyy}^{(3)}E_{y}E_{y}E_{y}^{(*)} + \chi_{yyxx}^{(3)}E_{y}E_{x}E_{x}^{(*)} + \chi_{yxyx}^{(3)}E_{x}E_{y}E_{x}^{(*)} + \chi_{yxxy}^{(3)}E_{x}E_{x}E_{y}^{(*)}
	\end{cases}$, где соответствующие элементы для тензора $\chi^{(3)}$ в выражениях для $P_{x}$ и $P_{y}$ равны ввиду изотропии, а также элементы вторых и третьих слагаемых. Итого можно записать: $\scriptsize
\begin{cases}
	P_{x}=\kappa_{1}E_{x}E_{x}E_{x}^{(*)} + 2\kappa_{2}E_{x}E_{y}E_{y}^{(*)} + \kappa_{3}E_{y}E_{y}E_{x}^{(*)} \\
	P_{y}=\kappa_{1}E_{y}E_{y}E_{y}^{(*)} + 2\kappa_{2}E_{y}E_{x}E_{x}^{(*)} + \kappa_{3}E_{x}E_{x}E_{y}^{(*)}
\end{cases}$, где ввиду инвариантности поворота $\hookrightarrow$ $\kappa_{1}=2\kappa_{2}+\kappa_{3}$.


	Нелинейный пок-тель преломл. $n_{2}=\frac{12\pi^{2}}{cn_{0}^{2}} \scriptsize
	\begin{cases}
		\kappa_{1}=2\kappa_{2}+\kappa_{3}, \quad \text{линейная поляризация} \\
		2\kappa_{2}=, \quad \text{круговая поляризация}
	\end{cases}$, где ввиду положительности всех $\kappa$ пок-тель для линейной поляризации больше, чем для круговой.


	\textit{Слабая волна в поле сильной:} в процессе взаимодействия слабой и сильной волн близких частот $\omega$ и $\omega_{0}$ в среде появляется волна частоты $2\omega_{0}-\omega$. Слабая волна в поле сильной чувствует преломление не как от $\kappa_{1}$ (лин. поляр.), а как от $2\kappa_{1}$ в случае параллельной поляризации сильной и слабой волн, и $2\kappa_{2}$ в случае противоположной поляризации. Характер среды - наведенное двулучепреломление.


\subsection{Билет 14.2.}

Отсутствует материал.

\subsection{Билет 15.}

Отсутствует материал.



	
	
	
\end{document}