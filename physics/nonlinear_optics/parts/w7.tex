\subsection*{Нелинейный показатель преломления}

Проялвяется в фазовой самомодуляции, самовращениие эллипса поляризации, самофокусировке, самодефокусировке и т.д. 


Методом медленных амплитуд, можем получить
\begin{equation*}
    \left(\partial_x^2 + \partial_y^2\right) \tilde{A} + 2 i k \, \partial_z \tilde{A} = 
    - 2 \frac{k^2}{n_0} n_2 |\tilde{A}|^2 \tilde{A},
    \hspace{10 mm} 
    |\tilde{A}|^2 = I.
\end{equation*}
И прийти к тому, что пороговая мощность не зависит от параметров пучка. 
% I =  8pi/c E^2


Также можем рассчитать расстояние фокусировки:
\begin{equation*}
    z = 2.5 \frac{r_0^2}{\lambda} \cdot \left(\sqrt{\frac{P}{\sub{P}{крит}}}-0.85\right)^{-1}.
\end{equation*}




Стоит сказать про особенности самофокусировки импульсного излучения: <<бегущие>> фокусы.

\begin{equation*}
    \Sigma \Theta \varpi \varphi \phi \beta \Gamma \Pi \Lambda \Xi
\end{equation*}