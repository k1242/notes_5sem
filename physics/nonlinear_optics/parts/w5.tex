% 30.09.21

% зал великолепен тем, что нет часов, как в игорных домах, поэтому можно заниматься до ночи.

% I. преломление и поглощение, что объединяется в $\varepsilon$.

% II. ...
    % рассеяние Ми,
    % Мандельштам-Релеевское рассеяние,
    % комбинационное рассеяние,

% III. люминесценсия
% IV. Параметрическое излучение. (очень важно для современной квантовой оптики)


% нецентросимметричные среды могут быть квадратично нелинейными. 


\section{Квадратичные нелинейные явления}

\noindent
К этим явлениям относится:  \vspace{-2mm}
\begin{itemize*}
    \item генерация второй оптической гармоники;
    \item оптическое выпрямление;
    \item генерация суммарной частоты, ГСЧ;
    \item генерация разностной частоты, ГРЧ;
    \item генерация параметрических волн.
\end{itemize*}

% Ландсберг и Мандельштам открыл комбинационное рассеяние, но Раман сделал чуть более сомнительную работу, но отправил чуть раньше.
% лаборотория в фиане большой вклад от Ландсберга, почти везде стоит 


\subsection{Генерация параметрических волн}

Стоит заметить, что генерация параметрических волн относится в том числе и к линейной оптике, точнее возникает из спонтанного ($\sim$ линейного) параметрического излучения.

Так сложилось, что только нелинейные среды проявляют в том числе и линейный эффект. Падающая волны с частотой $\omega_P$, генерирует $\omega_1,\, \omega_2 < \omega_P$ (примерно в два раза), однако $\omega_1 + \omega_2 = \omega_P$. 

Можем записать уравнение нелинейной оптики
\begin{align*}
    \frac{d A_1}{d z} &= i \frac{2 \pi \omega_1}{c n_1} \xi^{(2)} A_P A_2^* e^{i k \Delta z}; \\ 
    \frac{d A_2}{d z} &= i \frac{2 \pi \omega_2}{c n_2} \xi^{(2)} A_P A_1^* e^{i k \Delta z}; \\
    \frac{d A_P}{d z} &= i \frac{2 \pi \omega_P}{c n_P} \xi^{(2)} A_1 A_2^* e^{-i k \Delta z}. \\
\end{align*}
Забавно, что эти уравнения не имеют решений, если $A_1(0) = A_2(0) = 0$, так что необходимое условие параметрической генерации:
\begin{equation*}
    A_1 (0) \neq 0,
    \hspace{10 mm} 
    A_2 (0) \neq 0.
\end{equation*}
Стоит заметить, что генерация суммарной и разностной частоты, а также генерация второй гармоники не требовали специфических начальных условий (затравок). 


% вставить картинки из презентации/google
% синхронизм соответсвует накачке второй гармоники. Можно посылать круговую поляризацию, можно линейную.

% \subsection{Спонтанное параметрическое излучения}

% Выделяют 2 типа синхронизма. Всегда выполняется $\vc{k}_1 + \vc{k}_2 = \vc{k}_p$, то есть $\omega_1 \sim \omega_2$. Для I типа получается картина  концентрических колец с разными $\omega_{1,\, 2}$. 


% Второй тип может быть получен 
% ООЕ (отрицательные одноосные кристаллы)

% см кристаллы с периодической поляризация -- много чего можно сделать

Для формирования <<тонкой>> угловой струкутуры необходима фазовая синхронизация излучателей во всем объеме среды. О какой фазовой синхронизации двухчастотных пучков может идти речь?
Суммарное поле двух волн разных частот и направлений:
\begin{equation*}
    A \cos(\omega_1 t - \vc{k}_1\vc{r}) + A \cos(\omega_2 t - \vc{k}_2 \vc{r}) = 
    2 A \cos \left(\tfrac{\omega_1+\omega_2}{2} t - \tfrac{\smallvc{k}_1 + \smallvc{k}_2}{2}\vc{r}\right) \cos \left(\tfrac{\omega_1-\omega_2}{2} t - \tfrac{\smallvc{k}_1 - \smallvc{k}_2}{2}\vc{r}\right),
\end{equation*}
где второй множитель соответствует <<медленным>> осцилляциям. 

% квантовая электродинамика: возможности отклика одиночных частиц, на возбуждение квантами света.

% ещё один рисунок $\frac{1}{2}(\vc{k}_1 + \vc{k}_2)$. 



Рассмотрим, в частности, коллинейарное взаимодействие, и заданное поле накачки, также считаем $\Delta k = 0$, а тогда $e^{i \Delta k z} = 1$. Также считаем, что $A_1 (0) \neq 0$  и $A_2 (0) \neq 0$:
\begin{align*}
    \frac{d A_1}{d z} &= i \frac{2 \pi \omega_1}{c n_1} \chi^{(2)} A_p A_2^* (z) \\ 
    \frac{d A_2}{d z} &= i \frac{2 \pi \omega_2}{c n_2} \chi^{(2)} A_p A_1^* (z), 
\end{align*}
которые уже и модем решить. 

Решение для <<фотонных>> амплитуд:
\begin{equation*}
    \left.\begin{aligned}
        a_1 (z) &= \tfrac{1}{2} \left(a_{10} + i e^{i \varphi} a_{20}^*\right) e^{g |A_p| z} + \tfrac{1}{2} \left(a_{10} - i e^{i \varphi} a_{20}^*\right) e^{-g |A_p| z} \\
        a_2 (z) &= \tfrac{1}{2} \left(i e^{i \varphi} a_{10} +  a_{20}^*\right) e^{g |A_p| z} + \tfrac{1}{2} \left(- i e^{i \varphi} a_{10} + a_{20}^*\right) e^{-g |A_p| z} \\
    \end{aligned}\right.
\end{equation*}
где введено
\begin{equation*}
    a = \sqrt{\frac{8 \pi}{c n} \hbar \omega}\ A,
    \hspace{5 mm} 
    g = 4 \sqrt{2 \pi \hbar \frac{\pi^3 \omega_1 \omega_2 \omega_P}{c^3 n_1 n_2 n_p}}\ \chi^{(2)}.
\end{equation*}
Стоит заметить, что можно явно выделить затухающие и возрастающие слагаемые. 

% элементарный акт рождения фотонов -- распад фотона на два с меньшей энергией. 



\textbf{Диффузия фазы}. Допустим теперь, что нас интересует
\begin{equation*}
    f(t) = A \cos(\omega_1 t + \varphi_1 (t)) + A \cos(\omega_2 t + \varphi_2(t)).
\end{equation*}
При чём должно выполняться $\dot{\varphi}_1 + \dot{\varphi}_2 = 0$, а тогда $\varphi_2 = - \varphi_1$, так получаем
\begin{equation*}
    f(t) = 2 A \cos \left(\tfrac{\omega_P}{2}t\right) \cos \left(
        \ldots + \varphi(t)
    \right),
\end{equation*}
итого нули такой $f(t)$ будет строго периодично переходить через нули -- будет носить строго монохроматический характер, но это не означает возникновение монохроматической волны. 


%%%%%%%%%%%%%%%%%%%       конец про параметрику       %%%%%%%%%%%%%%%%%%%



% + табличка с некоторыми нецентросимметричными кристаллами $\chi^{(2)$.