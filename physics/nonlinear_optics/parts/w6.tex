% бить на сетку, брать по одной точке из каждого узла сетик.


% рабочие лошади:
    % неодимовый лазер
    % 1.06 мкр Nd^3+, -> 532 нм
    % Ti-Sa -- ионы титаны в матрице сапфира -- генерирует фемтосекундные импульсы, так как полоса усилиния покрывает диапазон 750-960 нм. -> импульсы до 50 фс ~ 20 \mu м, вылетающие с периодом ~ 10 нс. 
    % полупроводниковые лазеры почти во всем видимом диапазоне: 10-100 мВт. 


% детекция
% выбраем суммарную или разностуную частоту, 


\subsection{Применение нелинейной оптики}


\textbf{Детектирование}. 
Рассмотрим среду с $\chi^{(2)} \neq 0$, светим в нее ИК излучение $\omega_2 \ll \omega_1$, светим в видимом диапазоне, и на выходе получать что-то более подходящее для детектирования. 




\textbf{ТГц диапазон}. 
Пускаем в анизотропную среду лазерный импульс, на выходе имеем разностные частоты. Спектр пучка -- гаусс с границами на $\omega_1$ и $\omega_2$, где $\omega_1 - \omega_2 \ll \omega_{1,2}$.
% спектроскопия молекулярных веществ (просвечивание багажа)


% электронная нелинейность -- электроны реагируют на световые поля практически безинерцыонно 
% диполь Герца излучает в диапазоне частот, который заад
% прикольно создавать ТГц источники. 

\textbf{Измерение числа фотонов}. Аналогично пускаем лазерный импульс, и он проходит, почти не теряя в интенсивности, однако мы узнаем число фотонов: неразрушающее квантовое измерение числа фотонов с помощью оптческого выпрямления. 

Вообще можно орагнизовать аналогичную историю, просто посветив на зеркало -- давление света. 



% гравитационные детекторы -- интерферометры майкельсона
% оказалось, что одно из серьезнейших ограничений чувствительности -- давление света на зеркала
% => возникает шум, Пуассоновская статистика
% диапазон в области 100 Гц, 
% поднмаем чувствительность => растут шумы


\textbf{Измерение длительности пико- и фемтосекундных импульсов}. 
% оптическая связь -- 1нс~1ГГц, рекорд: 0.1нс~10ГГц
Строим коррелятор второй гармоники через неколлинеарный синхронизм. Делим короткие импульсы на два пучка, скрещиваем их под углом $\varphi$, тогда получается пучок, шириной $\tau c / \varphi$, что уже можно измерить при малых $\varphi$. 
% придумал Рудольф Гезолян -- армянский исследователь из СССР. 



\section{Кубичные нелинейные явления}


% ядерные нелинейности -- электрострикция (?), 
% показатель преломления становится функцией интенсивности света

К подходящим средам относятся все среды, включаи изотропные:
\begin{equation*}
    P = \chi^{(1)} E + \chi^{(2)} E^2 + \chi^{(3)} E^3 + \ldots,
\end{equation*}
так что рассмотрим некоторый, достаточно информативный список:
\vspace{-2mm}
\begin{itemize*}
    \item генерация третьей оптической гармоники;
    \item нелинейный показатель преломления;
    \item четырехволновое смещение (самодифракция излучения, обращение волнового фронта)
    \item генерация <<параметрических волн>>. 
\end{itemize*}


\textbf{Генераия третьей гармоники}.  В модели гармонического осциллятора, можем заметить, что при добавке, вида $U(x) \to U(X) + x^4 \ \Rightarrow \ \text{eq} \to \text{eq}+ x^3$, возникает \textit{самовоздействие} (изменение показателя преломления), и генерацич третьей гармоники. 


Рассматривая сумму от синфазных источников, можно получить условие фазового синхронизма: $3 \vc{k}_{\omega} = \vc{k}_{3 \omega}$, что равносильно коллинеарности генерируемой волны и волны накачки, а также $n_\omega = n_{3 \omega}$. 
% однако лучше два раза \chi^{(2)}


Заметим также, что самовоздействие не нуждается в фазовом синхронизме. Обычно вводят нелинейный покаатель преломления ($\varepsilon E = E + 4 \pi P$):
\begin{equation*}
    n^2 = (n_0 + \Delta n(I))^2 = 1 + 4 \pi \chi^{(1)} + 4 \pi \chi^{(3)} |E|^2,
    \hspace{0.5cm} \Rightarrow \hspace{0.5cm}
    \Delta n = \frac{3 \pi \chi^{(3)}}{2 n_0} |E|^2 \overset{\mathrm{def}}{=} n_2 I,
    \hspace{0.5cm} \Rightarrow \hspace{0.5cm}
    n_2 = \frac{12 \pi^2 \chi^{(3)}}{c n_0^2}.
\end{equation*}

Вообще, на возникновение $\chi^{(3)}$ влияет стрикция, ориентация молекул, тепловая нелинейность, плазменная нелинейность и так далее. 


В курсе общеё физики описывается электрострикция, когда диэлектрическая жидкость поднимается. 
% рая
% яна
% яся
Эффект не зависит от знака заряда. 
