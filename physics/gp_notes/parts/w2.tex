\textbf{Фотоэффект}. Максимальная кинетическая энергия, которой будут обладать электроны, вылетевшие при фотоэффекте определяется формулой Эйнштейна:
\begin{equation*}
    \frac{1}{2} \sub{m}{e} \sub{v}{max}^2 = \hbar \omega - A.
\end{equation*}

\textbf{Эффект Комптона}, -- изменение длины волны $\lambda' - \lambda$ в длинноволновую сторону спектра при рассеянии излучения. Смещение не зависит от состава тела и длины падающей волны,  но пропорционально $\sin^2 (\theta/2)$, где $\theta$ -- угол рассеяния. Рассмотрев упругое столкновение фотона и электрона, можем получить:
\begin{equation*}
    \frac{\sub{\E}{ph} \sub{\E}{ph}'}{c^2} + \frac{\sub{\E}{ph}' \sub{\E}{0}}{c^2} - 
    \frac{\sub{\E}{ph} \sub{\E}{0}}{c^2} - \sub{\vc{p}}{ph} \cdot \sub{\vc{p}}{ph}' = 0,
    \hspace{5 mm} \Leftrightarrow \hspace{5 mm} 
    1 - \cos \theta = \sub{m}{e} c \left(\frac{1}{\sub{p}{ph}'}-\frac{1}{\sub{p}{ph}}\right),
\end{equation*}
где $\theta$ -- угол рассеяния, т.е. угол между $\sub{\vc{p}}{pf}$ и $\sub{\vc{p}}{pf}$. Считая, что $\sub{p}{ph}' = h/\lambda'$ и $\sub{p}{ph} = h/\lambda$, находим
\begin{equation*}
    \lambda' - \lambda = \sub{\lambda}{K} (1- \cos \theta)  = 2 \sub{\lambda}{K} \sin^2 \frac{\theta}{2},
    \hspace{0.5cm} \Rightarrow \hspace{0.5cm}
    \sub{\lambda}{K} = \frac{h}{\sub{m}{e} c}  = 2.43 \cdot 10^{-3} \text{ нм},
\end{equation*}
так и находим комптоновскую длину\footnote{
    Формально, $\sub{\lambda}{K}$ можно рассматривать, как длину волны де Бройля, которой соответетствует величина импульса, равная инвариантной длине четырехмерного вектора энергии-импульса в пространстве Минковского. 
}  для электрона. Также можно встретить приведенную комптоновскую длину для электрона
\begin{equation*}
    \sub{\lambdabar}{K}  = \frac{\hbar}{\sub{m}{e} c} = \frac{\sub{\lambda}{K}}{2 \pi} = 3.86 \cdot 10^{-4} \text{ нм},
\end{equation*}
где электрон предполагается неподвижным. Движущийся электрон может передать свою энергию фотону, а сам остановиться -- обратный эффект Комптона.  Несмещенная компонента возникает из рассеяния на связанных электронах. 

Также можем посмотреть на направление вылета электрона отдачи:
\begin{equation*}
    \tg \varphi = \frac{\ctg(\theta /2)}{1 + \hbar \omega / (\sub{m}{e} c^2)}.
    \uparrow \darrowbar
\end{equation*}

