Постоянная Планка:
\begin{align*}
    &\hbar = 1.054 \times 10^{-34}\, \unm{Дж}{c},  &h &= 6.626 \times 10^{-34} \, \unm{Дж}{с}  \\
    &\hbar = 1.054 \times 10^{-27}\, \unm{эрг}{c}, &h &= 6.626 \times 10^{-27} \, \unm{эрг}{с} \\
    &\hbar = 6.582 \times 10^{-16}\, \unm{эВ}{c},  &h &= 4.136 \times 10^{-15} \, \unm{эВ}{с}  \\
\end{align*}
Для электрона:
\begin{equation*}
    \sub{m}{e} = 9.109 \times 10^{-31} \text{ кг},
    \hspace{5 mm} 
    |\bar{e}| = 1.602 \times  10^{-19} \text{ Кл} = 4.803 \times  10^{-10} \text{ ед. СГСЭ}.
\end{equation*}
Масса протона:
\begin{equation*}
    \sub{m}{p} = 1.673 \times  10^{-27} \text{ кг} = 1.836 \, \sub{m}{e}.
\end{equation*}
Постоянная Больцмана:
\begin{equation*}
    k = 1.381 \times  10^{-23} \, \unm{Дж}{К}^{-1} = 8.617 \times  10^{-5} \, \unm{эВ}{К}^{-1} = 
    1.381 \times  10^{-16} \, \unm{эрг}{К}^{-1}.
\end{equation*}
Энергия кванта видимого света:
\begin{equation*}
    \hbar \nu_{405 \text{нм}} \approx 2.3 \text{ эВ}, \hspace{5 mm} 
    \hbar \nu_{532 \text{нм}} \approx 2.3 \text{ эВ}, \hspace{5 mm} 
    \hbar \nu_{671 \text{нм}} \approx 1.8 \text{ эВ}.
\end{equation*}
Постоянная тонкой структуры:
\begin{equation*}
    \alpha = \frac{e^2}{\hbar c} = \frac{1}{137}.
\end{equation*}

Соотношения Крамера:
\begin{equation*}
    \langle r^{-1}\rangle_{s=0} = \frac{1}{a n^2},
    \hspace{5 mm} 
    \langle r\rangle_{s=1} = \frac{a}{2}\left[3 n^2 - l(l+1)\right],
    \hspace{5 mm} 
    \langle r^{2}\rangle_{s=2} = 
    \frac{a^2 n^2}{2} \left[5 n^2 + 1 - 3 l(l+1)\right].
\end{equation*}
где $a \equiv r_1$ -- Боровский радиус. 

Из соотношения Фенймана-Хеллмана:
\begin{equation*}
    \langle r^{-2}\rangle = \frac{2}{a^2} \frac{1}{n^3 (2l + 1)}.
\end{equation*}