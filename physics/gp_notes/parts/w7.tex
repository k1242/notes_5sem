
Итого, решение содержит три параметра: $n$ -- главное квантовое число:
\begin{equation*}
    % 4.17
    E_n = - \frac{m e^4 Z^2}{2 \hbar^2 n^2},
\end{equation*}
$l$ -- орбитальное квантовое число:
\begin{equation*}
    % 5.16
    M^2 = \hbar^2 l (l+1),\hspace{10 mm} 
    l = 0,\, 1,\, \ldots,\, n-1
\end{equation*}
$m_l$ -- магнитное квановое число:
\begin{equation*}
    % 5.12
    M_z = m_l \hbar, \hspace{10 mm} m_l = 0,\, \pm 1,\, \ldots, \pm l.
\end{equation*}
Колиство \textit{вырожденных состояний}: $n^2$. 

Для электрона спин $\in \pm 1/2$, тогда $M_s$:
\begin{equation*}
    M_s = \hbar \sqrt{s(s+1)},
    \hspace{5 mm} 
    s = 1/2,
\end{equation*}
где $s$ -- спиновое квантовое число, и
\begin{equation*}
    M_{sz} = \hbar m_s,
    \hspace{5 mm} 
    m_s = \pm s = \pm 1/2.
\end{equation*}
Тогда кратность вырождения увеличивается в 2 раза: $2 n^2$.
Для суммы орбитального и спинового угловых моментов верно, что
\begin{align*}
    M_j^2 &= \hbar^2 j (j+1),
    \hspace{5 mm} 
    j  =l+s = l + 1/2, \\
    M_{jz} &= \hbar m_j,
    \hspace{5 mm} m_j = j,\, j-1,\, \ldots,\, -j.
\end{align*}
Принято $l=0,1,2,3,4$ обозначать, как $s,\, p,\, d,\, f,\, g,\, h$. Состояния записывают, как
\begin{equation*}
    {}^{2s+1}L_j,
\end{equation*}
где $2s + 1$ -- мультиплетность, $L \in [s,\, p,\, \ldots]$.



Фотон имеет спин $\pm 1$, тогда момент импульса имеет проекции $\pm \hbar$. Электрон имеет спин $\pm 1/2$, собственные магнитные момент, равный магнетону Бора:
\begin{equation*}
    \mu_B = \frac{e}{2 m_e c} = 0.92740 \times  10^{-20} \text{ эрг}/\text{Гс}.
\end{equation*}

Для составляющей суммарного магнитного момента на направление суммарного углового момента вводится связь:
\begin{equation*}
    \vc{\mu}_j = - g_L \mu_B \vc{M}_j.
\end{equation*}
шде \textit{фактор Ланде}:
\begin{equation*}
    g_L = \frac{3}{2} + \frac{s(s+1) - l(l+1)}{2 j(j+1)}.
\end{equation*}
При  $L=0$, $S= 1/2$ и $J=1/2$ получаем $g_L = 2$. 
