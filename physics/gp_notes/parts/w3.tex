
\textbf{Гипотеза де Бройля}. Волны де Бройля:
\begin{equation*}
    \Psi = \Psi_0 e^{i(\smallvc{k} \cdot \smallvc{r} - \omega t)},
\end{equation*}
для которой верны следующие соотношения:
\begin{equation*}
    \E = \hbar \omega, 
    \hspace{5 mm} 
    \vc{p} = \hbar \vc{k},
    \hspace{5 mm} 
    \lambda = \frac{2\pi}{k} = \frac{2\pi \hbar}{p} = \frac{h}{p}.
\end{equation*}
Также можно получить выражения для фазовой и групповой скорости волн:
\begin{equation*}
    \sub{v}{ф} = \frac{\omega}{k} = \frac{\E}{p} \overset{v \sim c}{=} \frac{c^2}{v},
    \hspace{5 mm} 
    \sub{v}{гр} = \frac{d \omega}{d k}  = \frac{d \E}{d p} = v,
    \hspace{5 mm} 
    \sub{v}{ф} \sub{v}{гр} = c^2.
\end{equation*}

На всякий случай еще приведем условие Брэгга-Вульфа:
\begin{equation*}
    2 d \sin \varphi = m \lambda,
\end{equation*}
где $\varphi$ -- угол скольжкения, $d$ -- межплоскосной расстояние, $m \in \mathbb{N}$.

Для волн де Бройля случается дисперсия:
\begin{equation*}
    \left(\frac{\E}{c}\right)^2 - p^2 = (m_0 c)^2,
    \hspace{0.5cm} \Rightarrow \hspace{0.5cm}
    \left(\frac{\hbar \omega}{c}\right)^2 - (\hbar k)^2 = (m_0 c)^2. 
\end{equation*}



\textbf{Нерелятивистский случай}. Все так же
\begin{equation*}
    \E = \hbar \omega,
    \hspace{5 mm} 
    \vc{p} = \hbar \vc{k},
\end{equation*}
но тееперь (!) не учитывается $m_0 c^2$, а значит это $\omega$ отличается от $\omega$, что была выше на некоторую константу. В частности, теперь
\begin{equation*}
    \E = \frac{p^2}{2m}, \hspace{5 mm} 
    \omega = \frac{\hbar}{2m}k,
    \hspace{5 mm} 
    \sub{v}{ф} = \frac{\omega}{k} = \frac{p}{2m} = \frac{v}{2}.
\end{equation*}





\textbf{Соотношение неопределенностей}. Соотношение неопределенностей Гейзенберга:
\begin{equation*}
    \sqrt{\langle (\Delta q)^2\rangle} \cdot \sqrt{\langle (\Delta p)^2\rangle} \geq \frac{\hbar}{2}.
\end{equation*}
Вообще, для двух эрмитовых операторов, верно:
\begin{equation*}
    \sqrt{\langle (\Delta A)^2\rangle} \cdot \sqrt{\langle (\Delta B)^2\rangle} \geq \frac{\hbar}{2} 
    | \langle \left[A, B\right]\rangle|.
\end{equation*}
Также верно соотношение неопределенности Гейзенберга для времени и энергии:
\begin{equation*}
    \Delta t \cdot \Delta \E \geq \hbar.
\end{equation*}