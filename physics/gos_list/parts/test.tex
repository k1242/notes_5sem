\section*{Программа ГОСа по физике (I)}

\subsection*{Механика}
\begin{enumerate*}
\item Законы Ньютона. Движение тел в инерциальных и неинерциальных системах отсчета.
\item Принцип относительности Галилея и принцип относительности Эйнштейна. Преобразования Лоренца. Инвариантность интервала.
\item Законы сохранения энергии и импульса в классической механике. Упругие и неупругие столкновения.
\item Уравнение движения релятивистской частицы под действием внешней силы. Импульс и энергия
релятивистской частицы.
\item Закон всемирного тяготения и законы Кеплера. Движение тел в поле тяготения.
\item Закон сохранения момента импульса. Уравнение моментов. Вращение твердого тела вокруг неподвижной оси. Гироскопы.
\end{enumerate*}


\subsection*{МСС}
\begin{enumerate*}
\setcounter{enumi}{6}
\item Течение идеальной жидкости. Уравнение непрерывности. Уравнение Бернулли.
\item Закон вязкого течения жидкости. Формула Пуазейля. Число Рейнольдса, его физический смысл.
\item Упругие деформации. Модуль Юнга и коэффициент Пуассона. Энергия упругой деформации.

\end{enumerate*}


\subsection*{Термодинамика}
\begin{enumerate*}
\setcounter{enumi}{9}
\item Уравнение состояния идеального газа, его объяснение на основе молекулярно-кинетической теории. Уравнение неидеального газа Ван-дер-Ваальса.
\item Квазистатические процессы. Первое начало термодинамики. Количество теплоты и работа. Внутренняя энергия. Энтальпия.
\item Второе начало термодинамики. Цикл Карно. Энтропия и закон ее возрастания. Энтропия идеального газа. Статистический смысл энтропии.
\item Термодинамические потенциалы. Условия равновесия термодинамических систем.
\item Распределения Максвелла и Больцмана.
\item Теплоемкость. Закон равномерного распределения энергии по степеням свободы. Зависимость теплоемкости газов от температуры.
\item Фазовые переходы. Уравнение Клапейрона-Клаузиуса. Диаграммы состояний.
\item Явления переноса: диффузия, теплопроводность, вязкость. Коэффициенты переноса в газах. Уравнение стационарной теплопроводности.
\item Поверхностное натяжение. Формула Лапласа. Свободная энергия и внутренняя энергия поверхности.
\item Флуктуации в термодинамических системах.
\item Броуновское движение, закон Эйнштейна-Смолуховского. Связь диффузии и подвижности (соотношение Эйнштейна).
\end{enumerate*}

\newpage

\section*{Программа ГОСа по физике (II)}

\subsection*{Электричество и теория поля}
\begin{enumerate*}
\setcounter{enumi}{20}
\item Закон Кулона. Теорема Гаусса в дифференциальной и интегральной формах. Теорема о циркуляции для статического электрического поля. Потенциал. Уравнение Пуассона.
\item Электростатическое поле в веществе. Вектор поляризации, электрическая индукция. Граничные
условия для векторов $\vc{E}$ и $\vc{D}$.
\item Магнитное поле постоянных токов в вакууме. Основные уравнения магнитостатики в вакууме. Закон Био-Савара. Сила Ампера. Сила Лоренца.
\item Магнитное поле в веществе. Основные уравнения магнитостатики в веществе. Граничные условия
для векторов $\vc{B}$ и $\vc{H}$.
\item Закон Ома в цепи постоянного тока. Переходные процессы в электрических цепях.
\item Электромагнитная индукция в движущихся и неподвижных проводниках. ЭДС индукции. Само- и
взаимоиндукция. Теорема взаимности.
\item Система уравнений Максвелла в интегральной и дифференциальной формах. Ток смещения. Материальные уравнения.
\item Закон сохранения энергии для электромагнитного поля. Вектор Пойнтинга. Импульс электромагнитного поля.
\item Квазистационарные токи. Свободные и вынужденные колебания в электрических цепях. Явление
резонанса. Добротность колебательного контура, ее энергетический смысл.
\item Спектральное разложение электрических сигналов. Спектры колебаний, модулированных по амплитуде и фазе.
\item Электрические флуктуации. Дробовой и тепловой шумы. Предел чувствительности электроизмерительных приборов.
\item Электромагнитные волны. Волновое уравнение. Уравнение Гельмгольца.
\item Электромагнитные волны в волноводах. Критическая частота. Объемные резонаторы.
\item Плазма. Плазменная частота. Диэлектрическая проницаемость плазмы. Дебаевский радиус.
\end{enumerate*}


\subsection*{Оптика}
\begin{enumerate*}
\setcounter{enumi}{34}
\item Интерференция волн. Временная и пространственная когерентность. Соотношение неопределенностей.
\item Принцип Гюйгенса-Френеля. Зоны Френеля. Дифракция Френеля и Фраунгофера. Границы применимости геометрической оптики.
\item Спектральные приборы (призма, дифракционная решетка, интерферометр Фабри-Перо) и их основные характеристики.
\item Дифракционный предел разрешения оптических и спектральных приборов. Критерий Рэлея.
\item Пространственное фурье-преобразование в оптике. Дифракция на синусоидальных решетках. Теория Аббе формирования изображения.
\item Принципы голографии. Голограмма Габора. Голограмма с наклонным опорным пучком. Объемные
голограммы.
\item Волновой пакет. Фазовая и групповая скорости. Формула Рэлея. Классическая теория дисперсии.
Нормальная и аномальная дисперсии.
\item Поляризация света. Угол Брюстера. Оптические явления в одноосных кристаллах.
\item Дифракция рентгеновских лучей. Формула Брэгга-Вульфа. Показатель преломления вещества для
рентгеновских лучей.
\end{enumerate*}

\newpage

\section*{Программа ГОСа по физике (III)}

\subsection*{Квантовая механика}
\begin{enumerate*}
\setcounter{enumi}{43}
\item Квантовая природа света. Внешний фотоэффект. Уравнение Эйнштейна. Эффект Комптона.
\item Спонтанное и вынужденное излучение. Инверсная заселенность уровней. Принцип работы лазера.
\item Излучение абсолютно черного тела. Формула Планка, законы Вина и Стефана-Больцмана.
\item Корпускулярно-волновой дуализм. Волны де Бройля. Опыты Девиссона-Джермера и Томсона по
дифракции электронов.
\item Волновая функция. Операторы координаты и импульса. Средние значения физических величин.
Соотношение неопределенности для координаты и импульса. Уравнение Шредингера.
\item Строение водородоподобного атома. Уровни энергии и кратность их вырождения. Спектр излучения атома водорода.
\item Опыты Штерна и Герлаха. Спин электрона. Орбитальный и спиновый магнитные моменты электрона.
\item Тождественность частиц. Симметрия волновой функции относительно перестановки частиц. Бозоны и фермионы. Принцип Паули. Электронная структура атомов. Таблица Менделеева.
\item Тонкая и сверхтонкая структуры оптических спектров. Правила отбора при поглощении и испускании фотонов атомами.
\item Эффект Зеемана в слабых магнитных полях.
\item Эффект Зеемана в сильных магнитных полях.
\item Ядерный и электронный магнитный резонансы.
\item Виды распадов. Закон радиоактивного распада. Период полураспада и время жизни.
\item Туннелирование частиц сквозь потенциальный барьер. Альфа-распад. Закон Гейгера-Нэттола и его
объяснение.
\item Виды бета-распадов. Объяснение непрерывности энергетического спектра электронов распада.
Нейтрино.
\item Ядерные реакции. Составное ядро. Сечение нерезонансных реакций. Закон Бете.
\item Резонансные ядерные реакции, формула Брейта-Вигнера. Упругий и неупругие каналы реакции.
\item Деление ядер под действием нейтронов. Принцип работы ядерного реактора на тепловых нейтронах.
\item Соотношение неопределенностей для энергии и времени. Оценка времени жизни виртуальных частиц, радиусов сильного и слабого взаимодействий.
\item Фундаментальные взаимодействия и фундаментальные частицы (лептоны, кварки и переносчики
взаимодействий). Кварковая структура адронов
\end{enumerate*}