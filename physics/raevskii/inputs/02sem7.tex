Если бы в атоме гелия электроны взаимодействовали друг с другом, а только с ядром, то мы бы получили водородоподобный атом с $Z = 2$ и энергией отрыва одного электрона (однократная ионизация) была бы равна 
\begin{equation*}
	\frac{m e^4 Z^2}{2 \hbar^2} \frac{1}{1^2} = 13.6 \cdot 4 = 54.4 \text{ эВ}.
\end{equation*}
Экспериментальное значение $24.9$ эВ. Причина расхождения -- неучёт  кулоновского отталкивания электронов.

Будем считать кулоновское отталкивание "малым". Как видно из вышеприведенного примера это не так. Поэтому дальнейшее надо рассматривать, как в основном качественный подход.

В отсутствии взаимодействия основное состояние есть состояния $1s$ с энергией $-54.4$ эВ согласно принципу Паули мы можем поместить в это состояние 2 электрона только если их спины будут противоположны $\sout{\uparrow \downarrow}$